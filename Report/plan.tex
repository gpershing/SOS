\documentclass[main.tex]{subfiles}

\section{Project Plan}
SOS language was designed and developed as a project for the class "Programming Languages and Translators - 4115" taught by Prof. Stephen Edwards in Spring 2021 at Columbia University. We host our project in a Github repository to use Git as a version control system, with Project Kanban enabled to manage the project. We also had bi-weekly Zoom meetings on Friday to get updates from team members and ask questions, then, we held meetings twice a week before any important deadlines. Besides Zoom meetings, WhatsApp is our main instant messaging application used. We sent importatnt updates, bugs found and some funny jokes about our own language via WhatsApp. We also meet our TA Harry Choi one week before the due days to report our progress and ask clarification questions.


\subsection{Contributors}

The members of the project team and their roles are listed below:

\begin{itemize}
  \item Tojo Abella: Test Engineer
  \item Sitong Feng: Project Manager
  \item G Pershing: Language Guru
  \item Sheron Wang: Systems Architect
\end{itemize}

\subsection{Style Guide}
\paragraph{Ocaml:} TBD.
\paragraph{C:} Even though C only contains 5\% of our program, we still set up several basic style instructions such as 4 spaces as tab, left curly bracket on the same line but with one space before that, and new line for else. 
\paragraph{SOS:} As we are using our own language to write standard library, we set up a detailed style guide for SOS, please check SOS Style Guide after the tutorial section for more details. 

\subsection{Timeline}

 \renewcommand{\labelenumi}{$\blacksquare$}
 \renewcommand{\labelenumii}{$\square$}
 \begin{itemize}
   \item Feb 3 
   \begin{itemize}
     \item Project Proposal Complete
   \end{itemize}
 \end{itemize}
 
 
 \begin{itemize}
   \item Feb 24 
   \begin{itemize}
     \item Scanner, Parser, and LRM Complete
   \end{itemize}
 \end{itemize}

 \begin{itemize}
   \item Mar 24 
   \begin{itemize}
     \item Hello World milestone
        \begin{itemize}
        \item work entails codegen.ml, semant and more
     \end{itemize}
   \end{itemize}
 \end{itemize}
 
 
 \begin{itemize}
   \item April 15 
   \begin{itemize}
     \item Add third-party library support
        \begin{itemize}
        \item include \texttt{OpenGL} for graphics operation
     \end{itemize}
   \end{itemize}
 \end{itemize}

 \begin{itemize}
   \item April 26
   \begin{itemize}
     \item Language Project Complete
   \end{itemize}
 \end{itemize}

\subsection{Future Work}
The SOS language aims to develop more powerful and easy to use graphics features in the following aspects:

\begin{itemize}
  \item Incorporate more \texttt{OpenGL}  utilities such as varied line type
  \item Add basic built-in shapes for creating composite shapes
  \item Add stdlib function \texttt{drawCurve} for more smooth paths
  \item Add memory management
  \item Add function scope
\end{itemize}

More advanced features will include:
\begin{itemize}
  \item Add 3D Object/Plot Diagram support
  \item Link other third-party APIs
  \item Add animating function and gif file support
  \item Allow real-time interactivity
\end{itemize}

\subsection{Git Log}
Please check Appendix.



