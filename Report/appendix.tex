\documentclass[main.tex]{subfiles}

\begin{document}
	\section{Appendix}

\subsection{SOS Interpreter}	

\subsubsection{sos.ml}

\begin{lstlisting}
(* Top-level of the SOS compiler: scan & parse the input,
   check the resulting AST and generate an SAST from it, generate LLVM IR,
   and dump the module 
   Reference: The MicroC compiler *)
(* Written by Sheron *)
 
type action = Ast | Sast | LLVM_IR | Compile

let () =
  let action = ref Compile in
  let set_action a () = action := a in
  let speclist = [
    ("-a", Arg.Unit (set_action Ast), "Print the AST");
    ("-s", Arg.Unit (set_action Sast), "Print the SAST");
    ("-l", Arg.Unit (set_action LLVM_IR), "Print the generated LLVM IR");
    ("-c", Arg.Unit (set_action Compile),
      "Check and print the generated LLVM IR (default)");
  ] in  
  let usage_msg = "usage: ./sos.native [-a|-s|-l|-c] [file.sos]" in
  let channel = ref stdin in
  Arg.parse speclist (fun filename -> channel := open_in filename) usage_msg;
  
  let lexbuf = Lexing.from_channel !channel in
  let ast = Parser.program Scanner.token lexbuf in  
  match !action with
    Ast -> Astprint.basic_print ast
  | _ -> let sast = Semant.check ast in
    match !action with
      Ast     -> ()
    | Sast    -> Sastprint.basic_print sast
    | LLVM_IR -> print_string (Llvm.string_of_llmodule (Codegen.translate sast))
    | Compile -> let m = Codegen.translate sast in
	Llvm_analysis.assert_valid_module m;
	print_string (Llvm.string_of_llmodule m)
\end{lstlisting}

\subsubsection{scanner.mll}
\begin{lstlisting}
	
(* SOS Scanner *)
(* Written primarily by Sheron, later polish by G *)

{ open Parser 
  
  let find_file file = 
      if Sys.file_exists file then file
      else if Sys.file_exists ("lib/"^file) then ("lib/"^file)
      else raise (Failure ("Could not find file "^file))

  let import_table = Hashtbl.create 10
}

(* Definitions *)
let digit = ['0'-'9']
let digits = digit+

(* Rules *)

rule token = parse
  [' ' '\t' '\r' '\n'] { token lexbuf }
| "/*"     { comment 0 lexbuf }           (* Comments *)
| "//"     { single_comment lexbuf }            (* Single line comments *)
| "import " ([^'\n']+".sos" as file) { 
  let file = find_file file in
  if Hashtbl.mem import_table file then IMPORT [] (* Ignore *)
  else (
  Hashtbl.add import_table file ();
  let read = Lexing.from_channel (open_in file) in
  let parsed = Parser.program token read in
  IMPORT parsed ) } 
| '('      { LPAREN }
| ')'      { RPAREN }
| '{'      { LBRACE }
| '}'      { RBRACE }
| '['      { LBRACK }
| ']'      { RBRACK }
| '$'      { DOLLAR }
| ','      { COMMA }
| ':'      { COLON }
| '.'      { DOT }
| "->"     { TO }
| '+'      { ADD }
| '-'      { SUB }
| "**"     { MMUL }
| '*'      { MUL }
| '/'      { DIV }
| '%'      { MOD }
| '@'      { CONCAT }
| ';'      { SEQ }
| "=="     { EQEQ }
| "!="     { NEQ }
| '!'      { NOT }
| '='      { EQ }
| '<'      { LT }
| '>'      { GT }
| "<="     { LTEQ }
| ">="     { GTEQ }
| "&&"     { AND }
| "||"     { OR }
| "of"     { OF }
| "if"     { IF }
| "then"   { THEN }
| "else"   { ELSE }
| "struct" { STRUCT }
| "alias"  { ALIAS }
| "array"  { ARRAY }
| "func"   { FUNC }
| digits as lxm { INTLIT(int_of_string lxm) } 
| digits '.'  digit* ( ['e' 'E'] ['+' '-']? digits )? as lxm { FLOATLIT(lxm) }
| "true"   { BOOLLIT(true) }
| "false"  { BOOLLIT(false) }
| ['a'-'z' 'A'-'Z']['a'-'z' 'A'-'Z' '0'-'9' '_']*     as lxm { VAR(lxm) }
| eof { EOF }
| _ as char { raise (Failure("illegal character " ^ Char.escaped char)) }


and comment depth = parse
  "*/" { if depth==0 then token lexbuf else comment (depth-1) lexbuf }
| "/*" { comment (depth+1) lexbuf }
| _    { comment depth lexbuf }

and single_comment = parse
  '\n' { token lexbuf }
| _    { single_comment lexbuf }
\end{lstlisting}

\subsubsection{parser.mly}
\begin{lstlisting}

/* SOS Parser */
/* Written primarily by Sheron, later polish by G */

%{ open Ast %}

/* Declarations */

/* %token statements... */
%token ADD SUB MUL MMUL DIV MOD SEQ
%token NOT EQ LT GT LTEQ GTEQ EQEQ NEQ AND OR
%token CONCAT OF
%token DOT COMMA COLON DOLLAR
%token LPAREN RPAREN LBRACE RBRACE LBRACK RBRACK
%token IF THEN ELSE
%token STRUCT ALIAS ARRAY FUNC TO
%token <Ast.program> IMPORT
%token <int> INTLIT
%token <string> FLOATLIT
%token <bool> BOOLLIT
%token <string> VAR
%token EOF

%start program
%type <Ast.program> program

/* Associativity and Precedence */
%right VAR
%nonassoc IF THEN ELSE
%left COMMA 
%left SEQ
%right EQ
%left AND OR
%left EQEQ NEQ
%left LT GT LTEQ GTEQ
%left OF
%left CONCAT
%left ADD SUB
%right MMUL
%left MUL DIV MOD
%nonassoc LBRACK RBRACK LPAREN RPAREN LBRACE RBRACE
%right NOT
%left DOT


%%

/* rules */
typeid:
    VAR { TypeID($1) }
  | ARRAY typeid { ArrayTypeID($2) }
  | FUNC types TO typeid { FxnTypeID($2, $4) } 

value:
    VAR { Var ($1) }
  | value DOT VAR { StructField($1, $3) }
  | value LBRACK expr RBRACK { ArrayAccess($1, $3) }
  | DOLLAR LPAREN expr RPAREN { $3 }
  | fxn_app { $1 }

fxn_app:
    value LPAREN args RPAREN { FxnApp($1, $3) }

stexpr:
    VAR COLON typeid EQ expr { VarDef($3, $1, $5) }
  | VAR EQ expr { Assign ($1, $3) }
  | value DOT VAR EQ expr { AssignStruct($1, $3, $5) }
  | value LBRACK expr RBRACK EQ expr { AssignArray($1, $3, $6) }
  | IF expr THEN expr ELSE expr { IfElse($2,$4,$6) }
  | fxn_app { $1 }

expr:
    INTLIT { IntLit($1) }
  | FLOATLIT { FloatLit($1) }
  | BOOLLIT { BoolLit($1) }
  | NOT expr { Uop(Not,$2) }
  | SUB expr { Uop(Neg,$2) }
  | expr ADD expr { Binop($1,Add,$3) }
  | expr SUB expr { Binop($1,Sub,$3) }
  | expr MUL expr { Binop($1,Mul,$3) }
  | expr MMUL expr { Binop($1,MMul,$3) }
  | expr DIV expr { Binop($1,Div,$3) }
  | expr MOD expr { Binop($1,Mod,$3) }
  | expr EQEQ expr { Binop($1,Eq,$3) }
  | expr NEQ expr { Binop($1,Neq,$3) }
  | expr LT expr { Binop($1,Less,$3) }
  | expr GT expr { Binop($1,Greater,$3) }
  | expr LTEQ expr { Binop($1,LessEq,$3) }
  | expr GTEQ expr { Binop($1,GreaterEq,$3) }
  | expr AND expr { Binop($1,And,$3) }
  | expr OR expr { Binop($1,Or,$3) }
  | expr SEQ expr { Binop($1,Seq,$3) }
  | expr CONCAT expr {Binop($1,Concat,$3) }
  | expr OF expr { Binop($1,Of,$3) }
  | LPAREN expr RPAREN { $2 }
  | VAR LBRACE args RBRACE { NamedStruct($1, $3) }
  | LBRACE args RBRACE { AnonStruct($2) }
  | LBRACK args RBRACK { ArrayCon($2) }
  | VAR { Var ($1) }
  | value DOT VAR { StructField($1, $3) }
  | value LBRACK expr RBRACK {ArrayAccess($1, $3) }
  | stexpr { $1 }

fxn_args:
    /* nothing */ { [] }
  | fxn_args_list {List.rev $1}

fxn_args_list:
    VAR COLON typeid { [($3,$1)] }
  | fxn_args_list COMMA VAR COLON typeid { ($5,$3) :: $1 }

args:
    /* nothing */ { [] }
  | args_list {List.rev $1}

args_list:
    expr { [$1] }
  | args_list COMMA expr { $3 :: $1 }

types:
    /* nothing */ { [] }
  | rev_types {List.rev $1}

rev_types:
    typeid { [$1] }
  | rev_types COMMA typeid { $3 :: $1 }

typedef:
    ALIAS VAR EQ typeid { Alias($2,$4) }
  | STRUCT VAR EQ LBRACE fxn_args RBRACE { StructDef($2,$5) }

stmt:
    typedef { Typedef($1) }
  | VAR COLON LPAREN fxn_args RPAREN TO typeid EQ expr {FxnDef($7,$1,$4,$9)}
  | stexpr { Expression($1) }

stmts:
    stmt { [$1] }
  | stmts stmt { $2:: $1 }

program:
    stmts EOF { List.rev $1 }
  | IMPORT program { $1 @ $2 }
\end{lstlisting}

\subsubsection{ast.mli}
\begin{lstlisting}
(* Abstract syntax tree for SOS *)
(* Written by G *)

type operator = 
(*num operators*)
Add | Sub | Mul | Div | Mod | MMul
(*relational operators*)
| Eq | Neq | Less | Greater | LessEq | GreaterEq 
(*boolean operators*)
| And | Or 
(* array combination *)
| Concat | Of
(*sequencing*)
| Seq


type uop = Not | Neg

type id = string (* non-type id *)
type tid = (* type id *)
  TypeID of string
| ArrayTypeID of tid
| FxnTypeID of tid list * tid
type import = string

(* type name pair *)
type argtype = tid * id

(* all possible expression statements, found in LRM sec 4 *)
and expr = 
  VarDef of tid * id * expr                (* type name = val *)
| Assign of id * expr                      (* id = val *)
| AssignStruct of expr * id * expr         (* struct.field = val *)
| AssignArray of expr * expr * expr          (* id[expr] = expr *)
| Uop of uop * expr                        (* uop expr *)
| Binop of expr * operator * expr          (* expr op expr *)
| FxnApp of expr * expr list
| IfElse of expr * expr * expr             (* if expr then expr else expr *)
| ArrayCon of expr list                    (* [expr, ...] *)
| AnonStruct of expr list                  (* {expr, ...} *)
| NamedStruct of id * expr list           (* name{expr, ...} *)
| Var of id                                (* name *)
| ArrayAccess of expr * expr                 (* name[expr] *)
| StructField of expr * id                 (* struct.id *)
| IntLit of int                            (* int *)
| FloatLit of string                       (* float *)
| BoolLit of bool                          (* bool *)

type typedef = 
  Alias of id * tid                       (* alias name = type *)
| StructDef of id * argtype list           (* struct name = {type name, ...} *)

type stmt = 
  Typedef of typedef
| Expression of expr
| FxnDef of tid * id * argtype list * expr

type program = stmt list
\end{lstlisting}

\subsubsection{astprint.ml}
\begin{lstlisting}
(* Very basic """pretty""" printer for the AST *)
(* Written by G *)
open Ast

let rec comma_list_str f l = match l with
  [] -> ""
| hd :: tl -> match tl with
    [] -> f hd
  | _ -> f hd ^ ", " ^ comma_list_str f tl

let rec typeid_str t = match t with
  TypeID(s) -> s
| ArrayTypeID(p) -> "array " ^ typeid_str p
| FxnTypeID(l, t) -> "func " ^ comma_list_str typeid_str l ^" -> "^typeid_str t

let basic_print prog = 
  let rec print_stmt = function
    Typedef(t) -> let print_tdef = function
      Alias(a, b) -> print_endline ("alias " ^ a ^ " " ^ typeid_str(b))
    | StructDef(a, b) -> print_endline ("struct " ^ a ^ " = {" ^ comma_list_str (fun (a, b) -> typeid_str(a) ^ " " ^ b) b ^ "}")
    in print_tdef t
  | FxnDef(a, b, c, d) -> print_endline (typeid_str(a) ^ " " ^ b ^ "(" ^ comma_list_str (fun (a, b) -> typeid_str(a) ^ " " ^ b) c ^ ") = "); print_stmt (Expression(d))
  | Expression(e) -> let rec expr_str = function
      VarDef(a, b, c) -> typeid_str(a) ^ " " ^ b ^ " = " ^ expr_str c
    | Assign(a, b) -> a ^ " = " ^ expr_str b
    | AssignStruct(a, b, c) -> expr_str a ^ "." ^ b ^ " = " ^ expr_str c
    | AssignArray(a, b, c) -> expr_str a^"["^expr_str b ^"] = "^ expr_str c
    | ArrayAccess(nm, idx) -> expr_str nm^"["^expr_str idx^"]"
    | Uop(a, b) -> let uoperator_str = function Not -> "!" | Neg -> "-" in uoperator_str a ^ expr_str b
    | Binop(a, b, c) -> let operator_str = function
        Add -> "+"
      | Sub -> "-"
      | Mul -> "*"
      | MMul -> "**"
      | Div -> "/"
      | Mod -> "%"
      | Eq -> "="
      | Neq -> "!="
      | Less -> "<"
      | Greater -> ">"
      | LessEq -> "<="
      | GreaterEq -> ">="
      | And -> "&&"
      | Or -> "||"
      | Of -> "of"
      | Concat -> "@"
      | Seq -> ";" in
      "(" ^ expr_str a ^ " " ^ operator_str b ^ " " ^ expr_str c ^ ")"
    | FxnApp(a, b) -> 
        expr_str a ^ "(" ^ comma_list_str expr_str b ^ ")"
    | IfElse(a, b, c) -> "if (" ^ expr_str a ^ ")\n then (" ^ expr_str b ^ ")\n else (" ^ expr_str c ^")\n"
    | ArrayCon(a) -> "[" ^ comma_list_str expr_str a ^ "]"
    | AnonStruct(a) -> "{" ^ comma_list_str expr_str a ^ "}"
    | NamedStruct(a, b) -> a ^ "{" ^ comma_list_str expr_str b ^ "}"
    | Var(a) -> a
    | StructField(a, b) -> expr_str a ^ "." ^ b
    | IntLit(i) -> string_of_int i
    | FloatLit(f) -> f
    | BoolLit(true) -> "true"
    | BoolLit(false) -> "false"
    in print_endline(expr_str e)
  in
  List.iter print_stmt prog

(*let _ =
  let lexbuf = Lexing.from_channel stdin in
  let prog = Parser.program Scanner.token lexbuf in
  basic_print prog *)
\end{lstlisting}

\subsubsection{semant.ml}
\begin{lstlisting}
(* Semantic checking for the SOS compiler *)
(* Written primarily by G *)

open Ast
open Sast

(* import map for global variables (VarDef, FxnDef, Alias, StructDef) *)
(* we don't have scope defined so a string * tid is enough? *)
module StringMap = Map.Make(String)
module StringSet = Set.Make(String)

(* Semantic checking of the AST. Returns an SAST if successful,
   throws an exception if something is wrong.
   Check each statement *)

(* Environment type for holding all the bindings currently in scope *)
type environment = {
  typemap : typeid StringMap.t;
  fxnnames : StringSet.t;
  varmap : typeid StringMap.t;
}

(* External function signatures *)
(* This is re-used in Codegen *)
(* type, name that to be called in SOS, name in c file *)
let external_functions : (typeid * string) list =
[ Func([Float], Float), "sqrtf" ;
  Func([Float], Float), "sinf" ;
  Func([Float], Float), "cosf" ;
  Func([Float], Float), "tanf" ;
  Func([Float], Float), "asinf" ;
  Func([Float], Float), "acosf" ;
  Func([Float], Float), "atanf" ;
  Func([Float], Float), "toradiansf" ;
  Func([Int; Int], Void), "gl_startRendering" ;
  Func([Int; Int; Int], Void), "gl_endRendering" ;
  Func([Array(Float); Array(Float); Int], Void), "gl_drawCurve" ;
  Func([Array(Float); Array(Float); Int; Int], Void), "gl_drawShape" ;
  Func([Array(Float); Array(Float); Int], Void), "gl_drawPoint" ;
]


let raisestr s = raise (Failure s) 

let check prog =

  (* add built-in function such as basic printing *)
  let built_in_decls =
    let add_bind map (name, ty) = StringMap.add name (Func([ty], Void)) map 
    in List.fold_left add_bind StringMap.empty [ ("print", Int);
                                                 ("printf", Float) ]
  in
  (* add external functions *)
  let built_in_decls = List.fold_left
    (fun map (decl, nm) -> StringMap.add nm decl map)
    built_in_decls external_functions
  in
  let starting_fxns = List.fold_left
    (fun map (_, nm) -> StringSet.add nm map)
    StringSet.empty external_functions
  in
  let starting_fxns = StringSet.add "print" starting_fxns in
  let starting_fxns = StringSet.add "printf" starting_fxns in
(*  (* add math functions *)
  let built_in_decls = List.fold_left
    (fun map (decl, nm) -> StringMap.add nm decl map)
    built_in_decls math_functions
  in *)

  (* add built-in types such as int, float *)
  let built_in_types = (
    let add_type map (name, ty) = StringMap.add name ty map
    in List.fold_left add_type StringMap.empty [("int", Int); ("bool", Bool); ("float", Float); ("void", Void)] )
  in

  (* Void id *)
  let built_in_decls = StringMap.add "void" Void built_in_decls in

  (* Initial environment containing built-in types and functions *)
  let global_env = { typemap = built_in_types; varmap = built_in_decls 
  ; fxnnames = starting_fxns }
  in

  (* resolve the type of a tid to a typeid *)
  let rec resolve_typeid t map = match t with
    TypeID(s) -> if StringMap.mem s map
      then StringMap.find s map
      else raisestr ("Could not resolve type id "^s)
  | ArrayTypeID(s) -> Array(resolve_typeid s map)
  | FxnTypeID(l, r) -> Func(List.map (fun tt -> resolve_typeid tt map) l,
          resolve_typeid r map)
  in

  (* should add a function to add three things above dynamically *)
  (* let add_id_type = ()
  in *)

  (* function to lookup *)
  let type_of_id s map = 
    if StringMap.mem s map then StringMap.find s map
    else raisestr ("Unknown variable name "^s)
  in

  (* function to lookup the type of a struct field *)
  let type_of_field stype f = 
    match stype with
      Struct(sargs) -> 
      let rec find_field f = function
        (ft, fn) :: tl -> if fn = f then ft else find_field f tl
        | _ -> raisestr ("Could not find field "^f)
      in find_field f sargs
     | _ -> raisestr ("Cannot access fields for a non-struct variable")
  in

  let add_typedef td map =
    match td with
      Alias(nm, t) -> if StringMap.mem nm map
        then raisestr ("Cannot create an alias with preexisting name " ^ nm)
        else StringMap.add nm (resolve_typeid t map) map
    | StructDef(nm, l) -> let sargl = List.map (fun (t, i) -> (resolve_typeid t map, i)) l in
      StringMap.add nm (Struct(sargl)) map
  in

  let rec add_formals args vmap tmap = match args with
    (typ, nm) :: tl -> add_formals tl (StringMap.add nm (resolve_typeid typ tmap) vmap) tmap
    | _ -> vmap
  in

  (* Matches a struct type component-wise without names *)
  (* Can also work within arrays or other structs *)
  let rec match_str_type t1 t2 =
    match (t1, t2) with
      (Int, Int) -> true
    | (Float, Float) -> true
    | (Bool, Bool) -> true
    | (Void, Void) -> true
    | (Array(a1), Array(a2)) -> match_str_type a1 a2
    | (Struct(s1), Struct(s2)) -> 
        if List.length s1 != List.length s2 then false
        else
        List.fold_left2
          (fun b (st1, _) (st2, _) -> if match_str_type st1 st2 then
             b else false) true s1 s2
    | _ -> false
  in

  (* Returns a sexp that casts sexp to typ, if possible. *)
  (* Returns sexp if no cast is required *)
  let cast_to typ sexp err_str = 
   let (expt, sx) = sexp in
   if expt=typ then sexp else
   if match_str_type expt typ then (typ, sx) else
   if expt=EmptyArray then
    match typ with
      Array(t) -> (Array(t), SArrayCon([]))
    | _ -> raisestr ("Cannot cast empty array to non-array type")
   else
   (
   (match (typ, expt) with
     (Int, Float)   -> ()
   | (Int, Bool)    -> ()
   | (Bool, Int)    -> ()
   | (Bool, Float)  -> ()
   | (Float, Int)   -> ()
   | (Void, _)      -> ()
   | _ -> raisestr err_str );
   (typ, SCast(sexp)))
  in

  (* Function with the same signature as cast_to
   * Used to ignore casting checks *)
  let no_cast typ sexp err_str = 
    ignore(err_str);
    let (expt, _) = sexp in
    if expt=typ then sexp else
    raisestr ("No type casting allowed within arrays")
  in
  
  (* Converts a type to a string *)
  let rec type_str = function
    Int        -> "int"
  | Float      -> "float"
  | Bool       -> "bool"
  | Void       -> "void"
  | Array(t)   -> "array "^type_str t
  | Struct(sl) -> "struct {"^
    (let rec struct_typ_str = function
      (hdt, _) :: (h::t) -> type_str hdt ^ ", " ^struct_typ_str (h::t) 
    | (hdt, _) :: _ -> type_str hdt
    | _ -> ""
    in struct_typ_str sl)^"}"
  | Func(al, rt) -> "func "^
    (let rec func_typ_str = function
      (hdt) :: (h::t) -> type_str hdt ^", "^func_typ_str (h::t)
    | (hdt) :: _ -> type_str hdt
    | _ -> "" in func_typ_str al)^" -> "^type_str rt
  | EmptyArray -> "[]"
  in

  (* Identifies structs with only ints or only floats *)
  let arith_struct t1 at = 
      match t1 with
        Struct(l) -> 
          List.fold_left (fun b (ft, _) -> if ft=at then b else false)
            true l
      | _ -> false
  in
  let either_struct t1 t2 = 
    match t1 with Struct(_) -> true
    | _ -> match t2 with Struct(_) -> true
    | _ -> false
  in
  let is_struct t1 = match t1 with Struct(_) -> true | _ -> false
  in

  let assert_arith t1 = 
    if arith_struct t1 Float then () else
    if arith_struct t1 Int then () else
    raisestr ("Can only operate on arithmetic structs")
  in

  let sop_type t1 =
    if arith_struct t1 Float then Float else
    if arith_struct t1 Int then Int else
    Void
  in

  let addsub_expr env exp1 op exp2 cast = 
      let (t1, _) = exp1 in let (t2, _) = exp2 in
      (* Can add structs component-wise *)
      if either_struct t1 t2 then
        if match_str_type t1 t2 then
          (assert_arith t1 ;
          (t1, SBinop(exp1, op, exp2)), env)
        else
        raisestr ("Can only add or subtract structs of matching type")

      else
      let err = "Cannot add or subtract "^type_str t1^" and "^type_str t2 in
      match (t1, t2) with
        (Float, _) -> (Float, SBinop(exp1, op, cast Float exp2 err)), env
      | (_, Float) -> (Float, SBinop(cast Float exp1 err, op, exp2)), env
      | (Int, _)   -> (Int,   SBinop(exp1, op, cast Int   exp2 err)), env
      | (_, Int)   -> (Int,   SBinop(cast Int exp1 err, op,   exp2)), env
      | _ -> raisestr (err)
  in

  let mul_expr env exp1 op exp2 cast = 
    let (t1, _) = exp1 in let (t2, _) = exp2 in
    (* Can scale structs and take the dot product *)
    if either_struct t1 t2 then
      if match_str_type t1 t2 then
        (assert_arith t1 ;
        (sop_type t1, SBinop(exp1, op, exp2)), env)

      else if is_struct t1 then
        (assert_arith t1 ;
        (t1, SBinop(exp1, op, cast (sop_type t1) exp2
          "Cannot scale a struct by a non-scalar")), env)
      else
        (assert_arith t2 ;
        (t2, SBinop(exp2, op, cast (sop_type t2) exp1
          "Cannot scale a struct by a non-scalar")), env)
    else
    let err = "Cannot multiply "^type_str t1^" and "^type_str t2 in
    match (t1, t2) with
      (Float, _) -> (Float, SBinop(exp1, op, cast Float exp2 err)), env
    | (_, Float) -> (Float, SBinop(cast Float exp1 err, op, exp2)), env
    | (Int, _)   -> (Int,   SBinop(exp1, op, cast Int   exp2 err)), env
    | (_, Int)   -> (Int,   SBinop(cast Int exp1 err, op,   exp2)), env
    | _ -> raisestr ("Cannot multiply "^type_str t1^" and "^type_str t2)
  in

  let mmul_expr env exp1 op exp2 cast =
    ignore(cast); 
    let (t1, _) = exp1 in let (t2, _) = exp2 in
    (* Can multiply two n * n matrices OR
       Can multiply an n*n matrix with an n*1 vector *)
    match (t1, t2) with
      (Struct(l1), Struct(l2)) ->
      let n1 = List.length l1 in let n2 = List.length l2 in
      let int_sqrt n =
        let rec int_sqrt_inner n m = 
          if m * m = n then Some(m)
          else if m * m < n then int_sqrt_inner n (m+1)
          else None
        in int_sqrt_inner n 1
      in
      let sq1 = int_sqrt n1 in (match sq1 with
        | Some(m1) -> 
          if n1 = n2 then
            (Struct(l1), SBinop(exp1, op, exp2)), env

          else if m1 = n2 then
            (Struct(l2), SBinop(exp1, op, exp2)), env

          else raisestr ("Can only multiply a "^string_of_int m1^" by "^string_of_int m1^" matrix with a square matrix or vector of the same height")
        | None -> raisestr ("Can only multiply square matrices")
      )

    | _ -> raisestr ("Cannot matrix multiply non-structs")
  in

  let div_expr env exp1 op exp2 cast = 
      let (t1, _) = exp1 in let (t2, _) = exp2 in
      (* Can scale structs *)
      if is_struct t1 then
        (assert_arith t1 ;
        (t1, SBinop(exp1, op, cast (sop_type t1) exp2
          "Cannot scale a struct by a non-scalar")), env)
      else
      let err = "Cannot divide "^type_str t1^" and "^type_str t2 in
      match (t1, t2) with
        (Float, _) -> (Float, SBinop(exp1, op, cast Float exp2 err)), env
      | (_, Float) -> (Float, SBinop(cast Float exp1 err, op, exp2)), env
      | (Int, _)   -> (Int,   SBinop(exp1, op, cast Int   exp2 err)), env
      | (_, Int)   -> (Int,   SBinop(cast Int exp1 err, op,   exp2)), env
      | _ -> raisestr ("Cannot divide "^type_str t1^" and "^type_str t2)
  in

  let mod_expr env exp1 op exp2 cast = 
      ignore (cast); 
      let (t1, _) = exp1 in let (t2, _) = exp2 in
      match (t1, t2) with
        (Int, Int) -> (Int, SBinop(exp1, op, exp2)), env
      | _ -> raisestr ("Can only take the modulo with integers")
  in

  let eq_expr env exp1 op exp2 cast = 
      let (t1, _) = exp1 in let (t2, _) = exp2 in
      (* Can equate arith structs *)
      if either_struct t1 t2 then
        if match_str_type t1 t2 then
          (assert_arith t1 ;
          (Bool, SBinop(exp1, op, exp2)), env)
        else
        raisestr ("Can only equate structs of matching type")

      else
      let err = "Cannot equate "^type_str t1^" and "^type_str t2 in
      match (t1, t2) with
        (Float, _) -> (Bool, SBinop(exp1, op, cast Float exp2 err)), env
      | (_, Float) -> (Bool, SBinop(cast Float exp1 err, op, exp2)), env
      | (Int, _)   -> (Bool, SBinop(exp1, op, cast Int   exp2 err)), env
      | (_, Int)   -> (Bool, SBinop(cast Int exp1 err, op,   exp2)), env
      | _ -> raisestr (err)
  in

  let comp_expr env exp1 op exp2 cast = 
      let (t1, _) = exp1 in let (t2, _) = exp2 in
      let err = "Cannot compare "^type_str t1^" and "^type_str t2 in
      match (t1, t2) with
        (Float, _) -> (Bool, SBinop(exp1, op, cast Float exp2 err)), env
      | (_, Float) -> (Bool, SBinop(cast Float exp1 err, op, exp2)), env
      | (Int, _)   -> (Bool, SBinop(exp1, op, cast Int exp2 err  )), env
      | (_, Int)  -> (Bool, SBinop(cast Int exp1 err, op, exp2  )), env
      | _ -> raisestr ("Cannot compare "^type_str t1^" and "^type_str t2)
  in

  let logic_expr env exp1 op exp2 cast = 
      let err_str = "Could not resolve boolean operands to boolean values" in
      (Bool, SBinop(cast Bool exp1 err_str, op, cast Bool exp2 err_str)), env
  in

  let array_expr env exp1 op exp2 = 
      let (t1, _) = exp1 in let (t2, _) = exp2 in
      (match t2 with Array(_) -> ()
       | _ -> raisestr ("Cannot perform array operations on non-array type "^type_str t2)      ) ;
      match op with
        Concat     -> if t1 = t2 then (t2, SBinop(exp1, op, exp2)), env
        else raisestr ("Cannot concatenate arrays of different types")
      | _ (* Of *) -> (t2, SBinop((cast_to Int exp1
                    "First operand of of operator must be an int"),
                  op, exp2)), env
  in

  let rec binop_expr env exp1 op exp2 cast = 
    if op = Of || op = Concat then array_expr env exp1 op exp2
    else
    let e = SVar("empty") in
    let t1, _ = exp1 in
    match t1 with Array(t) ->
      let (ot, _), _ = binop_expr env (t, e) op exp2 no_cast in
      (Array(ot), SBinop(exp1, op, exp2)), env
    | _ ->
    let t2, _ = exp2 in
    match t2 with Array(t) ->
      let (ot, _), _ = binop_expr env exp1 op (t, e) no_cast in
      (Array(ot), SBinop(exp1, op, exp2)), env
    | _ ->
    match op with
      Add -> addsub_expr env exp1 op exp2 cast
    | Sub -> addsub_expr env exp1 op exp2 cast
    | Mul -> mul_expr    env exp1 op exp2 cast
    | MMul-> mmul_expr   env exp1 op exp2 cast
    | Div -> div_expr    env exp1 op exp2 cast
    | Mod -> mod_expr    env exp1 op exp2 cast
    | Eq  -> eq_expr     env exp1 op exp2 cast
    | Neq -> eq_expr     env exp1 op exp2 cast
    | Less      -> comp_expr env exp1 op exp2 cast
    | Greater   -> comp_expr env exp1 op exp2 cast
    | LessEq    -> comp_expr env exp1 op exp2 cast
    | GreaterEq -> comp_expr env exp1 op exp2 cast
    | Or ->  logic_expr env exp1 op exp2 cast
    | And -> logic_expr env exp1 op exp2 cast
    | _ -> raisestr ("Special case, this should never happen")
  in

  (* Takes a pair of sexprs and makes their types agree by adding casts,
   if possible. *)
  let agree_type e1 e2 err_str =
    let ((t1, _), (t2, _)) = (e1, e2) in
    if t1=t2 then (e1, e2) else
    (match (t1, t2) with
     (* Priority is Float -> Int -> Bool *)
      (Void, _)  -> (e1, cast_to t1 e2 err_str)
    | (_, Void)  -> (cast_to t2 e1 err_str, e2)
    | (Float, _) -> (e1, cast_to t1 e2 err_str)
    | (_, Float) -> (cast_to t2 e1 err_str, e2)
    | (Int, _)   -> (e1, cast_to t1 e2 err_str)
    | (_, Int)   -> (cast_to t2 e1 err_str, e2)
    | (Bool, _)  -> (e1, cast_to t1 e2 err_str)
    | (_, Bool)  -> (cast_to t2 e1 err_str, e2)
    | _ -> raisestr err_str )
  in

  let rec assert_nonvoid = function
    Void -> raisestr ("Cannot use a void type in this context")
  | Array(t) -> assert_nonvoid t
  | _ -> ()
  in

  let assert_non_reserved env name =
    if name="copy" || name="free" then
    raisestr ("Cannot create an identifier with reserved name "^name)
    else
    if StringSet.mem name env.fxnnames then
    raisestr ("Cannot create an identifier with defined function name "^name)
    else ()
  in

  let rec expr env = function
     
      VarDef (tstr, name, exp) -> 
        assert_non_reserved env name;
        let (sexp, _) = expr env exp in
        let t = resolve_typeid tstr env.typemap in
        assert_nonvoid t ;
        let (exptype, _) = sexp in
        ((t, SVarDef(t, name, cast_to t sexp
                     ("Could not resolve type when defining "^name^
                      "(Found "^type_str exptype^", expected "^type_str t^")"))),
         { env with varmap = StringMap.add name t env.varmap } )

    | Assign (name, exp) ->
         if StringSet.mem name env.fxnnames then
         raisestr ("Cannot assign a defined (non-variable) function")
         else
         let ((exptype, sexp), _) = expr env exp in
         let t = type_of_id name env.varmap in
         ((t, SAssign(name, cast_to t (exptype, sexp)
                  ("Could not match type when assigning variable "^name^
                  " (Found "^type_str exptype^", expected "^type_str t^")"))),env)

    | AssignStruct (struct_exp, field, exp) ->
         let ((exptype, sexp), env) = expr env exp in
         let (struct_sexp, env)  = expr env struct_exp in 
         let (strt, _) = struct_sexp in
         let t = type_of_field strt field in
         ((t, SAssignStruct(struct_sexp, field, cast_to t (exptype, sexp)
                ("Could not match type when assigning field "^field^
                 " (Found "^type_str exptype^", expected "^type_str t^")"))), env)

    | AssignArray (array_exp, idx, exp) ->
        let ((exptype, sexp), env) = expr env exp in
        let (array_sexp, env) = expr env array_exp in
        let (arrt, _) = array_sexp in
        let (sidx, _) = expr env idx in
        (match sidx with (Int, _) -> () | _ ->
           raisestr ("Array index must be an integer ") );
        let eltype = match arrt with Array(el) -> el | _ -> Void in
        ((eltype, SAssignArray(array_sexp, sidx, cast_to eltype (exptype, sexp)
    ("Could not match type when assigning array "^
    "(Found "^type_str exptype^", expected"^type_str eltype^")"))), env)

    | Uop(op, exp) ->
        let (sexp, env) = expr env exp in (
        match op with
          Not -> (Bool, SUop(op, cast_to Bool sexp 
            "Could not resolve expression to bool")), env
        | Neg -> let (t, _) = sexp in
          (match t with
            Int -> ()
          | Float -> ()
          | _ -> raisestr "Cannot negate non-arithmetic types" );
          (t, SUop(op, sexp)), env )

    | Binop(exp1, op, exp2) -> 
      if op = Seq then
         (* Need to pass new environments *)
         (* jk this happens anyways. but seq still gets to feel special *)
         let (e1, env) = expr env exp1 in
         let (e2, env) = expr env exp2 in
         let (t, _) = e2 in
         (t, SBinop(e1, Seq, e2)), env
      else
         let (e1, env) = expr env exp1 in
         let (e2, env) = expr env exp2 in
         binop_expr env e1 op e2 cast_to

    | FxnApp (exp, args) -> 
         if exp=Var("copy") then (* Copy constructor *)
         (match args with
           [ex] ->
            let (sexp, _) = expr env ex in
            let (t, _) = sexp in
            (match t with
              Array(_) -> ()
            | Struct(_) -> ()
            | _ -> raisestr ("Can only use Copy constructor on reference types"));
             ((t, SFxnApp((Func([t], t), SVar("copy")), [sexp])), env)

         | _ -> raisestr ("Too many arguments for Copy constructor")
         )

         else if exp=Var("free") then (* Free instr *)
         (match args with
           [ex] ->
             let (sexp, _) = expr env ex in
             let (t, _) = sexp in
             (match t with
               Array(_) -> ()
             | Struct(_) -> ()
             | _ -> raisestr ("Can only free memory of struct and array types")) ;
             ((Void, SFxnApp((Func([t], t), SVar("free")), [sexp])), env)
          | _ -> raisestr ("Too many arguments for free()")
          )

         else (* All other functions *)
         let fxn, env = expr env exp in
         let sargs, base_rt = match fxn with (Func(l, t), _) -> l, t
          | _ -> raisestr ("Could not resolve expression to a function") in

         let check_args sigl expl env =
           if (List.length sigl) != (List.length expl) then
           raisestr ("Incorrect number of arguments for function")
           else
           let (l, b) = List.fold_left2 
           (fun (l, arr) typ e ->
            let ((exptype, sexp), _) = expr env e in
            if exptype = Array(typ) then (exptype, sexp) :: l, true
            else (cast_to typ (exptype, sexp) 
                ("Could not match type of argument")) :: l, arr ) 
            ([], false) sigl expl
           in (List.rev l, b)
         in
         let cargs, arrmode = check_args sargs args env in
         if arrmode then
         (( (if base_rt=Void then Void else Array(base_rt)), SIterFxnApp(fxn, cargs)), env)
         else ((base_rt, SFxnApp(fxn, cargs)), env)

    | IfElse (eif, ethen, eelse) -> 
        let (sif, env) = expr env eif in
        let scif = cast_to Bool sif
         "Could not resolve if condition to a bool" in
        let (sthen, _) = expr env ethen in
        let (selse, _)  = expr env eelse in
        let (scthen, scelse) = agree_type sthen selse
          ("Could not reconcile types of then and else clauses ("^
         (let (t,_) = sthen in type_str t)^", "^
         (let (t,_) = selse in type_str t)^")") in
        let (t, _) = scthen in
        ((t, SIfElse(scif, scthen, scelse)), env)

    | ArrayCon l -> (match l with
      hd :: tl ->
        let ((exptype, sexp), env) = expr env hd in 
        assert_nonvoid exptype ;
        let rev_sexprs, env = List.fold_left 
          (fun (l, env) ex -> let (se, env) = expr env ex in
           (cast_to exptype se
             "Could not agree types of array literal") :: l, env)
          ([(exptype, sexp)], env) tl in
        ((Array(exptype), SArrayCon(List.rev rev_sexprs)), env)
      | [] -> ((EmptyArray, SArrayCon([])), env) 
      )

    | AnonStruct l -> 
      let rec create_anon_struct env n = function
        e :: tl -> let ((exptype, sexp), env) = expr env e in
          let (typel, expl), env = create_anon_struct env (n+1) tl in
          ((exptype, "x"^string_of_int n) :: typel, (exptype, sexp) :: expl), env
      | _ -> ([], []), env
      in
      let (typel, expl), env = create_anon_struct env 1 l in
      ((Struct(typel), SStruct("anon", expl)), env)

    | NamedStruct (name, l)  ->
      let st = resolve_typeid (TypeID(name)) env.typemap in
      (match st with
        Struct(sargs) -> 
        let rec create_named_struct env argl = function
          e :: tl -> let (sexp, env) = expr env e in
            (match argl with (t, nm) :: argtl ->
              let stl, env = create_named_struct env argtl tl in
              cast_to t sexp
                ("Could not resolve type of struct field "^nm)
               :: stl, env
            | _ -> raisestr ("Too many arguments for struct "^name))
          | [] -> (match argl with
            [] -> [], env
           | _ -> raisestr ("Not enough arguments for struct "^name))
        in
        let sexprs, env = create_named_struct env sargs l
        in ((st, SStruct(name, sexprs)), env)
      | _ -> raisestr ("Cannot resolve the struct name "^name)
     )

    | Var i -> ((type_of_id i env.varmap, SVar(i)), env)
    | ArrayAccess (arr, idx) -> 
      let (sarr, env) = expr env arr in
      let (sidx, env) = expr env idx in
      let (t, _) = sarr in
      let el_t = (match t with Array(e) -> e
        | _ -> raisestr ("Cannot access elements of non-array variable"))
      in
      ((el_t, SArrayAccess(sarr, cast_to Int sidx
         "Could not cast array index to an integer")), env)
    | StructField (str, fl) -> 
       let (sstr, env) = expr env str in
       let (t, _) = sstr in
       (match t with
         Struct(_) ->
          ((type_of_field t fl, SStructField(sstr ,fl)), env)
       | Array(_) -> if fl="length" then
          (Int, SArrayLength(sstr)), env
          else raisestr ("Cannot access fields for a non-struct variable")
       | _ -> raisestr ("Cannot access fields for a non-struct variable")
      )

    | IntLit i -> ((Int, SIntLit i), env)
    | FloatLit f -> ((Float, SFloatLit f), env)
    | BoolLit b -> ((Bool, SBoolLit b), env)
  in

  let make_stypedef env = function
      Alias(nm, tp) -> SAlias(nm, resolve_typeid tp env.typemap)
    | StructDef(nm , l) -> SStructDef(nm, List.map (fun (t, i) -> let tt = resolve_typeid t env.typemap in assert_nonvoid tt; (tt, i)) l)
  in

  (* check a single statement and update the environment *)
  let stmt env = function
      Expression(e) -> let (se, en) = expr env e in (SExpression (se), en)
    | Typedef(td) -> (STypeDef(make_stypedef env td), {env with typemap = add_typedef td env.typemap})
    | FxnDef (tstr, name, args, exp) ->
        assert_non_reserved env name ;
        let t = resolve_typeid tstr env.typemap in
        let sargs = List.map (fun (tp, nm) -> resolve_typeid tp env.typemap, nm)
          args in
        let argtypes = List.map (fun (tp, _) -> assert_nonvoid tp; tp) sargs in
        let newvarmap = StringMap.add name (Func(argtypes, t)) env.varmap in
        let env = { env with varmap = newvarmap; fxnnames = StringSet.add
            name env.fxnnames } in
        let ((exptype, sx), _) = expr { env with
           varmap = add_formals args env.varmap env.typemap; } exp
        in
        (SFxnDef(t, name, sargs, cast_to t (exptype, sx)
                     ("Incorrect return type for function "^name
                     ^" (Found "^type_str exptype^", expected "^type_str t^")"))),
         env

  in

  let rec stmts env = function 
    hd :: tl -> let (st, en) = stmt env hd in st :: stmts en tl
  | _ -> []
  in stmts global_env prog
\end{lstlisting}

\subsubsection{sast.mli}
\begin{lstlisting}
(* Semantically-checked AST *)

open Ast

(* Detailed type meaning *)
type typeid =
  Int
| Float
| Bool
| Void
| Array of typeid
| Struct of sargtype list
| Func of typeid list * typeid
| EmptyArray (* The empty array constructor, [] *)
and sargtype = typeid * id

(* Detailed function binding *)
type func_bind = {
  ftype : typeid;
  formals : sargtype list;
}

type sexpr = typeid * sx
and sx = 
  SVarDef of typeid * id * sexpr                (* type name = val *)
| SAssign of id * sexpr                      (* id = val *)
| SAssignStruct of sexpr * id * sexpr           (* id.field = val *)
| SAssignArray of sexpr * sexpr * sexpr          (* id[expr] = expr *)
| SUop of uop * sexpr                        (* uop expr *)
| SBinop of sexpr * operator * sexpr          (* expr op expr *)
| SFxnApp of sexpr * sexpr list
| SIterFxnApp of sexpr * sexpr list
| SIfElse of sexpr * sexpr * sexpr             (* if expr then expr else expr *)
| SArrayCon of sexpr list                    (* [expr, ...] *)
(* | SAnonStruct of sexpr list                  (* {expr, ...} *)
| SNamedStruct of id * sexpr list           (* name{expr, ...} *) *)
| SStruct of id * sexpr list
| SVar of id                                (* name *)
| SArrayAccess of sexpr * sexpr                (* name[expr] *)
| SArrayLength of sexpr                        (* name.length *)
| SStructField of sexpr * id                   (* name.id *)
| SIntLit of int                            (* int *)
| SFloatLit of string                       (* float *)
| SBoolLit of bool                          (* bool *)
| SCast of sexpr                            (* type casting *)

type stypedef = 
  SAlias of id * typeid
| SStructDef of id * sargtype list

type sstmt = 
  STypeDef of stypedef
| SExpression of sexpr
| SFxnDef of typeid * id * sargtype list * sexpr (* type id (type name, ...) = val *)

type sprogram = sstmt list
\end{lstlisting}

\subsubsection{sastprint.ml}
\begin{lstlisting}
(* Very basic """pretty""" printer for SAST *)
(* Written by G *)
open Sast
open Ast

let basic_print sast = 
  let rec sargl_string l =
    List.fold_left (fun str (t, id) -> (if str = "" then "" else str^", ")^typeid_string t ^" "^id) "" l
  and typeid_string = function
    Int -> "int"
  | Float -> "float"
  | Bool -> "bool"
  | Void -> "void"
  | Array(t) -> "array "^typeid_string t
  | Struct(l) -> "{"^sargl_string l^"}"
  | Func(_) -> "func"
  | EmptyArray -> "[]"
  in

  let print_typedef = function
    SAlias(nm, al) -> print_string("alias "^nm^" = "^typeid_string al^"\n")
  | SStructDef(nm,  l) -> print_string("struct "^nm^" = {"^sargl_string l^"}\n")
  in

  let rec explstr l = List.fold_left
    (fun s e -> (if s="" then s else s^", ")^sexp_string e) "" l

  and sexp_string (t, e) = match e with
    SVarDef(_, var, exp) -> typeid_string t ^" "^var^" = "^sexp_string exp^"\n"
  | SAssign(var, exp) -> "("^typeid_string t^") "^var^" = "^sexp_string exp^"\n"
  | SAssignStruct(var, f, exp) -> "("^typeid_string t^") "^sexp_string var^"."^f^" = "^sexp_string exp^"\n"
  | SAssignArray(var, e1, e2) -> "("^typeid_string t^") "^sexp_string var^"["^sexp_string e1^"] = "^sexp_string e2^"\n"
  | SUop(op, exp) -> "("^typeid_string t^")"^
    (match op with Not -> "!" | Neg -> "-")^sexp_string exp
  | SBinop (e1, op, e2) -> let opstr  = match op with
      Add -> "+" | Sub -> "-" | Mul -> "*" | Div -> "/" | Mod -> "%" |
      MMul -> "**" |
      Eq -> "==" | Neq -> "!=" | Less -> "<" | Greater -> ">" | LessEq -> "<=" |
      GreaterEq -> ">=" | And -> "&&" | Or -> "||" | Seq -> ";" |
      Of -> "of" | Concat -> "@" in
    "("^typeid_string t^") ("^sexp_string e1^" "^opstr^" "^sexp_string e2^")"
  | SIterFxnApp (fe, expl)
  | SFxnApp (fe, expl) -> "("^typeid_string t^") "^sexp_string fe^"("^
    (explstr expl)^")\n"
  | SIfElse(e1, e2, e3) -> "("^typeid_string t^") if "^sexp_string e1^"\nthen "^sexp_string e2^"\nelse "^sexp_string e3^"\n"
  | SArrayCon(expl) -> 
      "("^typeid_string t^") ["^explstr expl^"]"
  | SStruct(nm, expl) -> 
      "("^typeid_string t^") "^nm^"{"^explstr expl^"}"
  | SVar(id) -> "("^typeid_string t^") "^id
  | SStructField(id, f) -> "("^typeid_string t^") "^sexp_string id^"."^f
  | SIntLit(i) -> string_of_int i
  | SFloatLit(f) -> f
  | SBoolLit(b) -> string_of_bool b
  | SArrayAccess(nm, idx) -> "("^typeid_string t^") "^sexp_string nm^"["^sexp_string idx^"]"
  | SArrayLength(nm) -> sexp_string nm^".length"
  | SCast(exp) -> sexp_string exp
  in

  let print_sstmt = function
    STypeDef(td) -> print_typedef td
  | SExpression(sexpr) -> print_string(sexp_string sexpr)
  | SFxnDef(_, var, sargl, exp) ->
      print_string(var^"("^sargl_string sargl^") = "^sexp_string exp^"\n")

  in List.iter print_sstmt sast

(*
let _ =
  let lexbuf = Lexing.from_channel stdin in
  let ast = Parser.program Scanner.token lexbuf in
  let sast = Semant.check ast in 
  basic_print sast *)
\end{lstlisting}

\subsubsection{codegen.ml}
\begin{lstlisting}
(* Code generation: translate takes a semantically checked AST and
produces LLVM IR *)
(* Written primarily by G, some initial setup by Sitong & Sheron *)

module L = Llvm
open Ast
open Sast 

module StringMap = Map.Make(String)
module StringSet = Set.Make(String)

type environment = {
ebuilder : L.llbuilder;
evars : L.llvalue StringMap.t; (* The storage associated with a given var *)
efxns : StringSet.t; (* Which names are original fxn definitions *)
esfxns : (L.llvalue * func_bind) StringMap.t; (* Decls for struct op fxns *)
ecurrent_fxn : L.llvalue; (* The current function *)
}

(* translate : Sast.program -> Llvm.module *)
let translate prog =
  let context    = L.global_context () in
  
  (* Create the LLVM compilation module into which
     we will generate code *)
  let the_module = L.create_module context "SOS" in

  (* Get types from the context *)
  let i32_t      = L.i32_type     context
  and i8_t       = L.i8_type      context
  and i1_t       = L.i1_type      context
  and float_t    = L.float_type   context
  and void_t     = L.void_type    context 
  and ptr_t      = L.pointer_type 
  and struct_t   = L.struct_type  context in

  (* Convenient notation for GEP instructions, etc *)
  let l0         = L.const_int i32_t 0 in
  let l1         = L.const_int i32_t 1 in

  (* Return the LLVM type for a SOS type *)
  let rec ltype_of_typ = function
      Int      -> i32_t
    | Bool     -> i1_t
    | Float    -> float_t
    | Void     -> void_t
    (* An array is a pointer to a struct containing an array (as a pointer)
      and its length, an int *)
    | Array(t) -> ptr_t (struct_t [|ptr_t (ltype_of_typ t); i32_t|])
    | Struct(l) -> ptr_t (struct_t
       (Array.of_list (List.map (fun (tid, _) -> ltype_of_typ tid) l)))
    | Func(l, r) -> ptr_t (L.function_type (ltype_of_typ r) (Array.of_list (List.map ltype_of_typ l)))
    | EmptyArray-> raise (Failure "Unexpected empty array")

  in

  let printf_t : L.lltype = 
      L.var_arg_function_type i32_t [| L.pointer_type i8_t |] in
  let printf_func : L.llvalue = 
      L.declare_function "printf" printf_t the_module in

  (* External Functions *)
  let add_external_fxn env (decl, name) =
      let formals, rt = match decl with Func(formals, rt) -> formals, rt
        | _ -> raise (Failure "Unexpected external function decl") in
      let ftype = L.function_type (ltype_of_typ rt)
        (Array.of_list (List.map (fun t -> ltype_of_typ t) formals)) in
      let lldecl = L.declare_function name ftype the_module in
      { env with evars = StringMap.add name lldecl env.evars ;
          efxns = StringSet.add name env.efxns }
  in

  (* Setup main function *)
  let main =
      L.define_function "main" (L.function_type i32_t [||]) the_module
  in

  (* Add a variable llvalue to environment.evars *)
  let add_variable env nm lv =
       {env with evars = StringMap.add nm lv env.evars }
  in
  
  (* Get a variable's llvalue from environment.evars *)
  let get_variable env nm = if StringMap.mem nm env.evars then StringMap.find nm env.evars
    else raise (Failure ("Unexpected variable name "^nm))
  in

  (* Add a function declaration to environment.efxns *)
  let add_function env nm llv = 
      {env with evars = StringMap.add nm llv env.evars;
       efxns = StringSet.add nm env.efxns }
  in

  (* Add a formal argument llvalue to environment.evars *)
  let add_formal env (ty, nm) param =
    L.set_value_name nm param;
    let local = L.build_alloca (ltype_of_typ ty) nm env.ebuilder in
    ignore (L.build_store param local env.ebuilder);
    add_variable env nm local
  in

   (* Operator Maps *)
   let opstr = function
     Add -> "Add"
   | Sub -> "Sub"
   | Mul -> "Mul"
   | MMul-> "MMul"
   | Div -> "Div"
   | Mod -> "Mod"
   | Eq -> "Eq"
   | Neq -> "Neq"
   | Less -> "Less"
   | Greater -> "Greater"
   | LessEq -> "LessEq"
   | GreaterEq -> "GreaterEq"
   | And -> "And"
   | Or -> "Or" 
   | Of -> "Of"
   | Concat -> "Concat"
   | Seq -> "Seq"
   in

   let make_opmap l =
     List.fold_left (fun map (op, fxn) -> StringMap.add (opstr op) fxn map)
      StringMap.empty l
   in

   let int_map = make_opmap
   [(Add, L.build_add); (Sub, L.build_sub);
    (Mul, L.build_mul); (Div, L.build_sdiv);
    (Eq, L.build_icmp L.Icmp.Eq); (Neq, L.build_icmp L.Icmp.Ne);
    (Mod, L.build_srem);
    (Less, L.build_icmp L.Icmp.Slt); (Greater, L.build_icmp L.Icmp.Sgt);
    (LessEq, L.build_icmp L.Icmp.Sle); (GreaterEq, L.build_icmp L.Icmp.Sge)
     ]
   in
   
   let float_map = make_opmap
   [(Add, L.build_fadd); (Sub, L.build_fsub);
    (Mul, L.build_fmul); (Div, L.build_fdiv);
    (Eq, L.build_fcmp L.Fcmp.Oeq); (Neq, L.build_fcmp L.Fcmp.One);
    (Less, L.build_fcmp L.Fcmp.Olt); (Greater, L.build_fcmp L.Fcmp.Ogt);
    (LessEq, L.build_fcmp L.Fcmp.Ole); (GreaterEq, L.build_fcmp L.Fcmp.Oge)
     ]
   in

   (* Creates a loop that increments i by 1 each iteration,
    * and branches if i >= length. 
    * i_addr : the address of the int to be iterated on
    * length : the value of i to branch at
    * nm     : the name of the iterated variable, to make the LL readable
    * build  : a llvalue -> llbuilder -> llbuilder that builds all the statements using i
    * loop_bb: the basic block to build in
    * end_bb : the basic block to go to *)
   let build_loop i_addr length nm loop_bb end_bb build = 
     let builder = L.builder_at_end context loop_bb in
     let i = L.build_load i_addr "i" builder in
     let builder = build i builder in
     ignore(L.build_store (L.build_add i l1 nm builder) i_addr builder);
     let i = L.build_load i_addr nm builder in
     ignore (L.build_cond_br (L.build_icmp L.Icmp.Slt i length "tmp" builder)
       loop_bb end_bb builder);
   in

   (* General shorthand *)
   let build_zero builder nm = 
     let addr = L.build_alloca i32_t nm builder in
     ignore(L.build_store l0 addr builder);
     addr
   in
   
   let build_param builder decl typ n nm =
       let param = (Array.get (L.params decl) n) in
       L.set_value_name nm param ;
       let local = L.build_alloca typ nm builder in
       ignore (L.build_store param local builder);
       L.build_load local nm builder
    in

   (* Array operataions *)
   let build_array_load data idx nm builder =
     let elref = L.build_gep data [|idx|] (nm^"ref") builder in
     L.build_load elref nm builder
   in

   let build_array_store data idx llv builder = 
     let ref = L.build_gep data [|idx|] "storeref" builder in
     ignore (L.build_store llv ref builder)
   in

   let build_array_struct lltyp data length nm env = 
     let arr_struct = L.build_malloc (L.element_type lltyp) nm env.ebuilder in
     let data_addr = L.build_gep arr_struct [|l0; l0|] (nm^"data") env.ebuilder in
     let len_addr  = L.build_gep arr_struct [|l0; l1|] (nm^"len")  env.ebuilder in
     ignore (L.build_store data  data_addr env.ebuilder);
     ignore (L.build_store length len_addr env.ebuilder);
     arr_struct
   in

   let build_array_data builder lv nm =
     let data_ref = L.build_gep lv [|l0; l0|] (nm^"ref") builder in
     L.build_load data_ref nm builder
   in

   let build_array_len builder lv nm =
     let lenref = L.build_gep lv [|l0; l1|]
      (nm^"ref") builder in
     L.build_load lenref nm  builder
   in

   let build_of t2 ll1 ll2 env = 
     (* Get ll2's length *)
     let len = build_array_len env.ebuilder ll2 "len" in
    
     (* Compute new length *)
     let n = L.build_mul ll1 len "oflen" env.ebuilder in
     (* Pre-GEP the array *)
     let old_data = build_array_data env.ebuilder ll2 "olddata" in
     
     (* Create a new array *)
     let el_typ = match t2 with Array(et) -> et | _ -> Void in
     let data = L.build_array_malloc (ltype_of_typ el_typ) n
       "arrdata" env.ebuilder in
     (* Set up loop *)
     let i_addr = build_zero env.ebuilder "i" in
     let j_addr = build_zero env.ebuilder "j" in
     let loop_bb = L.append_block context "loop" env.ecurrent_fxn in
     let inner_bb = L.append_block context "inner" env.ecurrent_fxn in
     let continue_bb = L.append_block context "continue" env.ecurrent_fxn in
     ignore (L.build_br inner_bb env.ebuilder);

     (* Inner loop *)
     build_loop j_addr len "j" inner_bb loop_bb
       (fun j builder -> 
        let i = L.build_load i_addr "i" builder in
        let el = build_array_load old_data j "el" builder in
        build_array_store data i el builder ;
        ignore(L.build_store (L.build_add i l1 "i" builder) i_addr builder);
        builder 
       ) ;

     (* Outer loop *)
     let builder = L.builder_at_end context loop_bb in
     let i = L.build_load i_addr "i" builder in
     ignore (L.build_store l0 j_addr builder);
     ignore (L.build_cond_br (L.build_icmp L.Icmp.Slt i n "tmp" builder)
       inner_bb continue_bb builder);

     (* Continue *)
     let builder = L.builder_at_end context continue_bb in
     let env = { env with ebuilder = builder } in

     (* Create array struct *)
     let arr_struct = build_array_struct (ltype_of_typ t2) data n "arr" env in
     arr_struct, env
   in

   let build_concat t2 ll1 ll2 env =
     (* Get lengths *)
     let len1 = build_array_len env.ebuilder ll1 "len1" in
     let len2 = build_array_len env.ebuilder ll2 "len2" in
     let n = L.build_add len1 len2 "n" env.ebuilder in

     (* Pre-GEP the arrays *)
     let data1 = build_array_data env.ebuilder ll1 "data1" in
     let data2 = build_array_data env.ebuilder ll2 "data2" in

     (* Create a new array *)
     let el_typ = match t2 with Array(et) -> et | _ -> Void in
     let data = L.build_array_malloc (ltype_of_typ el_typ) n
       "data" env.ebuilder in 
     (* Set up loop *)
     let i_addr = build_zero env.ebuilder "i" in
     let j_addr = build_zero env.ebuilder "j" in

     let loop1 = L.append_block context "loop1" env.ecurrent_fxn in
     let inbtw = L.append_block context "inbtw" env.ecurrent_fxn in
     let loop2 = L.append_block context "loop2" env.ecurrent_fxn in
     let contb = L.append_block context "contb" env.ecurrent_fxn in
     ignore (L.build_br loop1 env.ebuilder);

     let make_concat_loop sbb tbb from_data len =
       build_loop j_addr len "j" sbb tbb
       (fun j builder ->
       let i = L.build_load i_addr "i" builder in
       let el = build_array_load from_data j "el" builder in
       build_array_store data i el builder ;
       ignore (L.build_store (L.build_add i l1 "tmp" builder)
               i_addr builder) ;
       builder )
     in
     (* Loop 1 *)
     make_concat_loop loop1 inbtw data1 len1 ;
     
     let builder = L.builder_at_end context inbtw in
     ignore(L.build_store l0 j_addr builder);
     ignore(L.build_br loop2 builder);
     (* Loop 2 *)
     make_concat_loop loop2 contb data2 len2 ;

     (* Continue *)
     let builder = L.builder_at_end context contb in
     let env = { env with ebuilder = builder } in
     (* Create array struct *)
     let arr_struct = build_array_struct (ltype_of_typ t2) data n "arr" env in
     arr_struct, env
   in

   (* Struct ops *)
   (* Finds the integer field index *)
   let find_field struct_typ field = 
       let sargl = match struct_typ with Struct(l) -> l | _  -> [] in
       let rec find_field_inner sargl field n = match sargl with sarg :: tl ->
         let (_, nm) = sarg in
         if nm = field then n else find_field_inner tl field n+1
       | _ -> raise (Failure "Field not found")
       in
       find_field_inner sargl field 0
   in

   let build_struct_field builder lv n nm =
     let ref = L.build_gep lv [|l0; L.const_int i32_t n|] (nm^"ref") builder in
     L.build_load ref nm builder
   in     

   let build_struct_store builder lv s_lv n = 
     let ref = L.build_gep s_lv [|l0; L.const_int i32_t n|] "ref" builder in
     ignore (L.build_store lv ref builder)
   in

   let dot_product stype slist env =
     let atype = match slist with
       (hd, _) :: _ -> if hd = Float then Float else Int
     | _ -> Float in
     let len = List.length slist in
     
     let name = "__dot"^(if atype=Float then "f" else "i")^(string_of_int len) in

     if StringMap.mem name env.esfxns then
       StringMap.find name env.esfxns, env
     else (* Make new function *)
       let ltype = ltype_of_typ stype in
       let formals = [|ltype; ltype|] in
       let ftype = L.function_type (ltype_of_typ atype) formals in
       let decl = L.define_function name ftype the_module in
       let builder = L.builder_at_end context (L.entry_block decl) in

       let bind = { ftype = atype; formals = [stype, "a"; stype, "b"] } in

       let a = build_param builder decl ltype 0 "a" in
       let b = build_param builder decl ltype 1 "b" in

       let res = L.build_alloca (ltype_of_typ atype) "dot" builder in
       let tmp = L.build_alloca (ltype_of_typ atype) "tmp" builder in
       ignore(L.build_store (if atype=Float then L.const_float (ltype_of_typ Float) 0.0 else L.const_int i32_t 0) res builder) ;
       let map = (if atype=Float then float_map else int_map) in
       (if StringMap.mem "Mul" map && StringMap.mem "Add" map then () else raise (Failure("Could not find operator in dot_product"))) ;
       let opmul = StringMap.find "Mul" map in
       let opadd = StringMap.find "Add" map in
       let rec dot_prod n = 
         if n < len then 
           let aval = build_struct_field builder a n "aval" in
           let bval = build_struct_field builder b n "bval" in
           ignore (L.build_store (opmul aval bval "tmp" builder) tmp builder);
           let tmpv = L.build_load tmp "tmp" builder in
           let resv = L.build_load res "res" builder in
           ignore (L.build_store (opadd tmpv resv "tmp" builder) res builder);
           dot_prod (n+1)
         else ()
       in
       dot_prod 0 ;

       (* Return *)
       let resv = L.build_load res "res" builder in
       ignore (L.build_ret resv builder) ;
       (decl, bind), { env with esfxns = StringMap.add name (decl, bind) env.esfxns }
   in

   let struct_sum stype slist op env =
     let atype = match slist with
       (hd, _) :: _ -> if hd = Float then Float else Int
     | _ -> Float in
     let len = List.length slist in
     
     let name = "__"^(if op=Add then "add" else "sub")^
      (if atype=Float then "f" else "i")^(string_of_int len) in

     if StringMap.mem name env.esfxns then
       StringMap.find name env.esfxns, env
     else (* Make new function *)
       let ltype = ltype_of_typ stype in
       let formals = [|ltype; ltype|] in
       let ftype = L.function_type ltype formals in
       let decl = L.define_function name ftype the_module in
       let builder = L.builder_at_end context (L.entry_block decl) in

       let bind = { ftype = stype; formals = [stype, "a"; stype, "b"] } in

       let a = build_param builder decl ltype 0 "a" in
       let b = build_param builder decl ltype 1 "b" in

       let struc = L.build_malloc (L.element_type ltype) "ret" builder in
       let map = (if atype=Float then float_map else int_map) in
       (if StringMap.mem (opstr op) map then () else raise (Failure("Could not find op "^(opstr op)^" in struct_sum map")) );
       let sumop = StringMap.find (opstr op) map in
       let rec strsum n = 
         if n < len then 
           let aval = build_struct_field builder a n "aval" in
           let bval = build_struct_field builder b n "bval" in
           build_struct_store builder (sumop aval bval "tmp" builder) struc n;
           strsum (n+1)
         else ()
       in
       strsum 0 ;

       (* Return *)
       ignore (L.build_ret struc builder) ;
       (decl, bind), { env with esfxns = StringMap.add name (decl, bind) env.esfxns }
   in

   let struct_scale stype slist op env =
     let atype = match slist with
       (hd, _) :: _ -> if hd = Float then Float else Int
     | _ -> Float in
     let len = List.length slist in
     
     let name = "__"^(if op=Mul then "mul" else "div")^
      (if atype=Float then "f" else "i")^(string_of_int len) in

     if StringMap.mem name env.esfxns then
       StringMap.find name env.esfxns, env
     else (* Make new function *)
       let ltype = ltype_of_typ stype in
       let altype = ltype_of_typ atype in
       let formals = [|ltype; altype|] in
       let ftype = L.function_type ltype formals in
       let decl = L.define_function name ftype the_module in
       let builder = L.builder_at_end context (L.entry_block decl) in

       let bind = { ftype = stype; formals = [stype, "a"; stype, "b"] } in

       let a = build_param builder decl ltype 0 "a" in
       let b = build_param builder decl altype 1 "b" in

       let struc = L.build_malloc (L.element_type ltype) "ret" builder in
       let map = (if atype=Float then float_map else int_map) in
       (if StringMap.mem (opstr op) map then () else raise (Failure("Could not find op "^(opstr op)^" in struct_scale map")) );
       let sumop = StringMap.find (opstr op) map in
       let rec strscl n = 
         if n < len then 
           let aref = L.build_gep a [|l0; L.const_int i32_t n|] "aref" builder in
           let aval = L.build_load aref "aval" builder in
           build_struct_store builder (sumop aval b "tmp" builder) struc n;
           strscl (n+1)
         else ()
       in
       strscl 0 ;

       (* Return *)
       ignore (L.build_ret struc builder) ;
       (decl, bind), { env with esfxns = StringMap.add name (decl, bind) env.esfxns }
   in
   
   let struct_eq stype slist op env =
     let atype = match slist with
       (hd, _) :: _ -> if hd = Float then Float else Int
     | _ -> Float in
     let len = List.length slist in

     let name = "__"^(if op=Eq then "eq" else "neq")^
       (if atype=Float then "f" else "i")^(string_of_int len) in

     if StringMap.mem name env.esfxns then
       StringMap.find name env.esfxns, env
     else (* Make new function *)
       let ltype = ltype_of_typ stype in
       let formals = [|ltype; ltype|] in
       let ftype = L.function_type (ltype_of_typ Bool) formals in
       let decl = L.define_function name ftype the_module in
       let builder = L.builder_at_end context (L.entry_block decl) in

       let bind = { ftype = Bool; formals =
           [stype, "a"; stype, "b"] } in

       let a = build_param builder decl ltype 0 "a" in
       let b = build_param builder decl ltype 1 "b" in

       let ret = L.build_alloca (ltype_of_typ Bool) "ret" builder in
       let eq = StringMap.find (opstr op) (if atype=Float then float_map
          else int_map) in
       let combine = if op=Eq then L.build_and else L.build_or in
       ignore(L.build_store (L.const_int i1_t (if op=Eq then 1 else 0)) ret builder);
       let rec streq n = 
         if n < len then
           let aval = build_struct_field builder a n "aval" in
           let bval = build_struct_field builder b n "bval" in
           let rval = L.build_load ret "rval" builder in
           ignore(L.build_store
             (combine rval (eq aval bval "eq" builder) "ret" builder) ret builder);
           streq (n+1)
         else ()
       in streq 0 ;

       (* Return *)
       let rv = L.build_load ret "rval" builder in
       ignore(L.build_ret rv builder) ;
       (decl, bind), { env with esfxns = StringMap.add name (decl, bind) env.esfxns }
   in

   let mat_mul rtype slist1 slist2 env =
     let atype = match slist1 with
       (hd, _) :: _ -> if hd = Float then Float else Int
     | _ -> Float in
     let size = List.length slist1 in
     let int_sqrt n = 
       let rec int_sqrt_inner n m = 
         if m * m = n then m
         else if m * m < n then int_sqrt_inner n (m+1)
         else raise (Failure "Unexpected struct size")
       in int_sqrt_inner n 1
     in
     let n = int_sqrt size in
     let m = List.length slist2 in

     let name = "__"^(if m=n then "vec" else "mat")^
       (if atype=Float then "f" else "i")^(string_of_int n) in
     if StringMap.mem name env.esfxns then
       StringMap.find name env.esfxns, env
     else (* Make new function *)
       let rltype = ltype_of_typ rtype in
       let ltype1 = ltype_of_typ (Struct(slist1)) in
       let ltype2 = ltype_of_typ (Struct(slist2)) in
       let altype = ltype_of_typ atype in
       let formals = [|ltype1; ltype2|] in
       let ftype = L.function_type rltype formals in
       let decl = L.define_function name ftype the_module in
       let builder = L.builder_at_end context (L.entry_block decl) in

       let bind =
        { ftype = rtype; formals = [Struct(slist1), "A"; Struct(slist2), "B"] } in
       
       let a = build_param builder decl ltype1 0 "A" in
       let b = build_param builder decl ltype2 1 "B" in

       let struc = L.build_malloc (L.element_type rltype) "ret" builder in
       let map = (if atype=Float then float_map else int_map) in
       let sumop = StringMap.find (opstr Add) map in
       let mulop = StringMap.find (opstr Mul) map in
       let height = n in
       let width = (if m=n then 1 else n) in
       let tmp = L.build_alloca altype "tmp" builder in
       let rec mmul i j =
         if i < width then
         if j < height then (
         ignore(L.build_store (if atype=Float then L.const_float altype 0.0 else L.const_int altype 0) tmp builder) ;
         let rec mmul_inner k = 
           if k < height then (
           let aval = build_struct_field builder a (j+k*n) "aval" in
           let bval = build_struct_field builder b (k+i*n) "bval" in
           let mval = mulop aval bval "tmp2" builder in
           let tval = L.build_load tmp "tval" builder in
           ignore (L.build_store (sumop mval tval "tmp" builder) tmp builder);
           mmul_inner (k+1))
           else ()
         in mmul_inner 0 ;
         let tval = L.build_load tmp "tval" builder in
         build_struct_store builder tval struc (j+i*n);
         mmul (i+1) j )
         else (* j >= height *) ()
         else (* i >= width *)  mmul 0 (j+1)
       in
       mmul 0 0 ;

       (* Return *)
       ignore (L.build_ret struc builder) ;
       (decl, bind), { env with esfxns = StringMap.add name (decl, bind) env.esfxns }
   in

   (* Binops *)
   let rec binop op rt t1 t2 ll1 ll2 env = 
     if op = Of then build_of t2 ll1 ll2 env
     else if op = Concat then build_concat t2 ll1 ll2 env
     else
     (match (t1, t2) with
       (Bool, Bool) ->
         let llop = match op with
           Or -> L.build_or
         | And -> L.build_and
         | _ -> raise(Failure "Unexpected boolean operator") in
         llop ll1 ll2 "tmp" env.ebuilder, env
     | (Int, Int) -> StringMap.find (opstr op) int_map ll1 ll2 "tmp"
         env.ebuilder, env
     | (Float, Float) -> StringMap.find (opstr op) float_map ll1 ll2
         "tmp" env.ebuilder, env
     | (Struct(l1), Struct(l2)) -> let (decl, _), env = 
       (match op with
          Mul -> dot_product t1 l1 env
        | MMul-> mat_mul rt l1 l2 env
        | Eq  -> struct_eq t1 l1 op env
        | Neq -> struct_eq t1 l1 op env
        | Add -> struct_sum t1 l1 op env
        | Sub -> struct_sum t1 l1 op env
        | _ -> raise (Failure("Unexpected struct operator "^(opstr op))) ) in
       L.build_call decl [|ll1; ll2|] "result" env.ebuilder, env
     | (Array(el_typ), _) ->
       let len = build_array_len env.ebuilder ll1 "len" in
      
       (* Pre-GEP the array *)
       let argdata = build_array_data env.ebuilder ll1 "argdata" in
     
       (* Create a new array *)
       let rtel_typ = match rt with Array(et) -> et | _ -> Void in
       let data = L.build_array_malloc (ltype_of_typ rtel_typ) len
         "arrdata" env.ebuilder in
       (* Set up loop *)
       let i_addr = build_zero env.ebuilder "i" in
       let loop_bb = L.append_block context "loop" env.ecurrent_fxn in
       let cont_bb = L.append_block context "cont" env.ecurrent_fxn in
       ignore (L.build_br loop_bb env.ebuilder);
       (* Build loop *)
       build_loop i_addr len "i" loop_bb cont_bb (
         fun i builder -> 
         let v = build_array_load argdata i "v" builder in
         let fenv = { env with ebuilder = builder } in
         let llv, fenv = binop op rtel_typ el_typ t2 v ll2 fenv in
         let builder = fenv.ebuilder in
         build_array_store data i llv builder;
         builder ) ;
       (* Continue *)
       let builder = L.builder_at_end context cont_bb in
       let env = { env with ebuilder = builder } in
       (* Create array struct *)
       let arr_struct = build_array_struct (ltype_of_typ rt) data len "arr" env in
       arr_struct, env

     | (_, Array(el_typ)) -> 
       let len = build_array_len env.ebuilder ll2 "len" in
      
       (* Pre-GEP the array *)
       let argdata = build_array_data env.ebuilder ll2 "argdata" in
     
       (* Create a new array *)
       let rtel_typ = match rt with Array(et) -> et | _ -> Void in
       let data = L.build_array_malloc (ltype_of_typ rtel_typ) len
         "arrdata" env.ebuilder in
       (* Set up loop *)
       let i_addr = build_zero env.ebuilder "i" in
       let loop_bb = L.append_block context "loop" env.ecurrent_fxn in
       let cont_bb = L.append_block context "cont" env.ecurrent_fxn in
       ignore (L.build_br loop_bb env.ebuilder);
       (* Build loop *)
       build_loop i_addr len "i" loop_bb cont_bb (
         fun i builder -> 
         let v = build_array_load argdata i "v" builder in
         let fenv = { env with ebuilder = builder } in
         let llv, fenv = binop op rtel_typ t1 el_typ ll1 v fenv in
         let builder = fenv.ebuilder in
         build_array_store data i llv builder;
         builder ) ;
       (* Continue *)
       let builder = L.builder_at_end context cont_bb in
       let env = { env with ebuilder = builder } in
       (* Create array struct *)
       let arr_struct = build_array_struct (ltype_of_typ rt) data len "arr" env in
       arr_struct, env
     | (Struct(l1), _) -> let (decl, _), env = struct_scale t1 l1 op env 
       in L.build_call decl [|ll1; ll2|] "result" env.ebuilder, env
     | _ -> raise (Failure "Unsupported operation")
     )
   in

   (* Construct code for an expression
      Return its llvalue and the updated builder *)
   let rec expr env sexpr = 
     let (t, e) = sexpr in match e with
     (* Literals *)
     SIntLit(i) -> L.const_int i32_t i, env
   | SFloatLit(f) -> L.const_float_of_string float_t f, env
   | SBoolLit(b) -> L.const_int i1_t (if b then 1 else 0), env
   | SArrayCon(expl) -> 
     let n = List.length expl in
     let el_typ = match t with Array(et) -> et | _ -> Void in
     (* Create data *)
     let data = L.build_array_malloc
       (ltype_of_typ el_typ)
       (L.const_int i32_t n) "arrdata" env.ebuilder in 
     let (_, env) = List.fold_left
       (fun (n, env) sx -> 
         let (lv, env) = expr env sx in
         build_array_store data (L.const_int i32_t n) lv env.ebuilder;
         (n+1, env) ) (0, env) expl in
     (* Create struct *)
     let arr_struct = build_array_struct (ltype_of_typ t) data
       (L.const_int i32_t n) "arr" env in
     arr_struct, env

   | SStruct(nm, expl) ->
     let struc = L.build_malloc (L.element_type (ltype_of_typ t)) nm env.ebuilder in
     let rec set_fields env n = function
       exp :: tl -> 
         let fieldaddr = L.build_gep struc [|l0; L.const_int i32_t n|]
           "fieldaddr" env.ebuilder in 
         let (lv, env) = expr env exp in
         ignore(L.build_store lv fieldaddr env.ebuilder);
         set_fields env (n+1) tl
       | [] -> env
     in
     let env = set_fields env 0 expl in
     struc, env

     (* Access *)
   | SVar(nm) -> 
     if nm="void" then l0, env else (* Void id *)
     if StringSet.mem nm env.efxns then (* Global fxn name, don't load *)
     get_variable env nm, env
     else (* All other variables *) 
     (L.build_load (get_variable env nm) nm env.ebuilder),env

   | SArrayAccess(arr_exp, idx) -> 
     let (arr, env) = expr env arr_exp in
     let (idx_lv, env) = expr env idx in
     (* Access the struct pointer, then the field *)
     let elref = L.build_gep arr [|l0; l0|]
       ("dataref") env.ebuilder in
     (* Access data *)
     let d = L.build_load elref ("data") env.ebuilder in
     build_array_load d idx_lv "el" env.ebuilder, env

   | SArrayLength (arr_exp) -> let (arr, env) = expr env arr_exp in
     build_array_len env.ebuilder arr "len", env

   | SStructField (str_exp, fl) ->
       let (stype, _) = str_exp in
       let (struc, env) = expr env str_exp in
       let idx = find_field stype fl in
       let adr = L.build_gep struc [|l0; L.const_int i32_t idx|]
         "fieldadr" env.ebuilder in
         L.build_load adr (fl) env.ebuilder, env

     (* Definitions *)
   | SVarDef(ty, nm, ex) ->  
       let var = L.build_alloca (ltype_of_typ ty) nm env.ebuilder in
       let env = add_variable env nm var in
       expr env (t, SAssign(nm, ex)) (* Bootstrap off Assign *)

     (* Assignments *)
   | SAssign(nm, ex) ->
       let ex' = expr env ex in
       let (lv, env) = ex' in
       ignore(L.build_store lv (get_variable env nm) env.ebuilder); ex'

   | SAssignStruct (str_exp, fl, ex) ->
       let (stype, _) = str_exp in
       let (struc, env) = expr env str_exp in
       let ex' = expr env ex in
       let (lv, env) = ex' in
       (* Find field index *)
       let idx = find_field stype fl in
       build_struct_store env.ebuilder lv struc idx; ex'

   | SAssignArray (arr_exp, idx, ex) ->
       let (arr, env) = expr env arr_exp in
       let data = build_array_data env.ebuilder arr "dataref" in
       let (i, env) = expr env idx in
       let (el, env) = expr env ex in
       build_array_store data i el env.ebuilder; (el, env)

     (* Operators *)
   | SUop (op, exp) ->
       let (l, env) = expr env exp in
       (match op with
         Neg when t = Float -> L.build_fneg
       | Neg                -> L.build_neg
       | Not                -> L.build_not) l "tmp" env.ebuilder, env

   | SBinop(exp1, op, exp2) ->
       (match op with
         Seq ->
           let (_, env) = expr env exp1 in
           expr env exp2 
       | _ -> 
         let (t1, _) = exp1 in let (t2, _) = exp2 in
         let (ll1, env) = expr env exp1 in
         let (ll2, env) = expr env exp2 in
         binop op t t1 t2 ll1 ll2 env       
       )

     (* Function application *)
    (* Special functions *)
   | SFxnApp((_, SVar("printf")), [e]) -> 
      let float_format_str =
       L.build_global_stringptr "%g\n" "fmt" env.ebuilder in
      let arg, env  = expr env e in 
      let dval = L.build_fpext arg (L.double_type context) "fmtp" env.ebuilder in
      L.build_call printf_func [| float_format_str; dval |]
        "printf" env.ebuilder, env
   | SFxnApp((_, SVar("print")), [e]) ->
      let int_format_str = 
       L.build_global_stringptr "%d\n" "fmt" env.ebuilder in
      let arg, env = expr env e in
      L.build_call printf_func [| int_format_str ; arg |]
        "printf" env.ebuilder, env


   | SFxnApp((_, SVar("copy")), [e]) ->
      let (ctype, _) = e in
      let arg, env = expr env e in
      (
      match ctype with
        Array(atype)    -> 
          let n = build_array_len env.ebuilder arg "len"in
          (* Pre-GEP the array *)
          let cdata = build_array_data env.ebuilder arg "cdata" in
     
          (* Create a new array *)
          let data = L.build_array_malloc (ltype_of_typ atype) n
            "arrdata" env.ebuilder in
          (* Set up loop *)
          let i_addr = build_zero env.ebuilder "i" in
          let loop_bb = L.append_block context "loop" env.ecurrent_fxn in
          let continue_bb = L.append_block context "continue" env.ecurrent_fxn in
          ignore (L.build_br loop_bb env.ebuilder);

          (* Loop *)
          build_loop i_addr n "i" loop_bb continue_bb
            (fun i builder ->
             let el = build_array_load cdata i "el" builder in
             build_array_store data i el builder ; builder ) ;

         (* Continue *)
         let builder = L.builder_at_end context continue_bb in
         let env = { env with ebuilder = builder } in
         (* Create array struct *)
         let arr_struct = build_array_struct (ltype_of_typ t) data n
           "arr" env in
         arr_struct, env
          
      | Struct(sfields) ->
         let len = List.length sfields in
         let name = "__copy"^(string_of_int len) in

         let (decl, _), env = if StringMap.mem name env.esfxns
         then StringMap.find name env.esfxns, env
         else (* Make a new copy fxns *)
           let formals = [|ltype_of_typ ctype|] in
           let ftype = L.function_type (ltype_of_typ ctype) formals in
           let decl = L.define_function name ftype the_module in
           let builder = L.builder_at_end context (L.entry_block decl) in

           let bind = { ftype = ctype; formals = [ctype, "to_copy"] } in

           let tocopy = build_param builder decl (ltype_of_typ ctype) 0 "to_copy" in

           (* Create a new struct *)
           let struc = L.build_malloc (L.element_type (ltype_of_typ ctype))
             "struct" builder in
           let rec copy_struct n =
             if n < len then
             let fl = build_struct_field builder tocopy n "fl" in
             build_struct_store builder fl struc n;
             copy_struct (n+1)
             else ()
           in
           copy_struct 0 ;

           (* Return *)
           ignore (L.build_ret struc builder) ;
           (decl, bind), { env with esfxns = StringMap.add name (decl, bind) env.esfxns }
         in
         L.build_call decl [|arg|] "copied" env.ebuilder, env


      | _ -> raise (Failure "Copy constructor only works on reference types")
      )

   (* Free instruction *)
   | SFxnApp((_, SVar("free")), [e]) ->
      let (ctype, _) = e in
      let arg, env = expr env e in
      (
      match ctype with
        Array(_)    -> 
         (* Need to free data as well as the structure *)
         let dataref = build_array_data env.ebuilder arg "data" in
         ignore(L.build_free dataref env.ebuilder);
      | _ -> () ) ;
      (* Free structure *)
      ignore(L.build_free arg env.ebuilder) ;
      l0, env

    (* General functions *)
   | SFxnApp(fexp, args) ->  
      let fdef, env = expr env fexp in
      let (fxntype, _) = fexp in
      let _, rt = match fxntype with Func(l, t) -> l, t | _ -> 
        raise (Failure "Unexpected function type") in

      (* Get llvalues of args and accumualte env *)
      let (llargs_rev, env) = List.fold_left
        (fun (l, env) a -> let (ll, e) = expr env a in (ll::l, e))
        ([], env) args in
      let llargs = List.rev llargs_rev in
      let result = (match rt with
                      Void -> ""
                    | _ -> "fxn_result") in
      (* Normal function application *)
      L.build_call fdef (Array.of_list llargs) result env.ebuilder, env

    | SIterFxnApp(fexp, args) ->
      let fdef, env = expr env fexp in
      let (fxntype, _) = fexp in
      let fargs, rt = match fxntype with Func(l, t) -> l, t | _ -> 
        raise (Failure "Unexpected function type") in

      (* Get llvalues of args and accumualte env *)
      let (llargs_rev, env) = List.fold_left
        (fun (l, env) a -> let (ll, e) = expr env a in (ll::l, e))
        ([], env) args in
      let llargs = List.rev llargs_rev in
      let result = (match rt with
                      Void -> ""
                    | _ -> "fxn_result") in

      (* Iterated fxn application *)
      let arr_args = List.map2 (fun (ty, _) fty -> ty=Array(fty) )
           args fargs in
      let rec find_first bools = function 
        (hd :: tl) -> if List.hd bools then hd else find_first (List.tl bools) tl
      | _ -> raise (Failure "Unexpected arguments")
      in let first = find_first arr_args llargs in
      let len = build_array_len env.ebuilder first "len" in
      (* Pre-GEP all the arrays *)
      let datalist = List.map2
        (fun llarg b -> if b then 
          Some(build_array_data env.ebuilder llarg "data")
         else None) llargs arr_args in

      (* Create a new array *)
      let data = if t=Void then None else 
        Some(L.build_array_malloc (ltype_of_typ rt) len
       "arrdata" env.ebuilder) in
      (* Set up loop *)
      let i_addr = build_zero env.ebuilder "i" in
      let loop_bb = L.append_block context "loop" env.ecurrent_fxn in
      let cont_bb = L.append_block context "continue" env.ecurrent_fxn in
      ignore (L.build_br loop_bb env.ebuilder);

      (* Loop *)
      build_loop i_addr len "i" loop_bb cont_bb
        (fun i builder ->
          let llargs_i = List.map2
           (fun llarg data_opt -> match data_opt with
             Some(data) -> build_array_load data i "el" builder
           | None -> llarg ) llargs datalist in
          let ret = L.build_call fdef (Array.of_list llargs_i) result builder in
          (match data with 
           Some(d) -> build_array_store d i ret builder |_->()) ; builder ) ;    

     (* Continue *)
     let builder = L.builder_at_end context cont_bb in
     let env = { env with ebuilder = builder } in
     (* Create array struct *)
     (match data with Some(d) ->
     let arr_struct = build_array_struct (ltype_of_typ t) d len
       "arr" env in
     arr_struct, env
     | _ -> l0, env )

    (* Control flow *)
   | SIfElse (eif, ethen, eelse) ->
      let (cond, env) = expr env eif in
      (* Memory to store the value of this expression *)
      let ret = (if t != Void then
        Some(L.build_alloca (ltype_of_typ t) "if_tmp" env.ebuilder)
        else None) in
      let merge_bb = L.append_block context "merge" env.ecurrent_fxn in
      let then_bb = L.append_block context "then" env.ecurrent_fxn in
      let else_bb = L.append_block context "else" env.ecurrent_fxn in
      ignore (L.build_cond_br cond then_bb else_bb env.ebuilder);

      let (thenv, then_env) = expr {env with ebuilder=(L.builder_at_end context then_bb)} ethen in
      (match ret with Some(rv) ->
        ignore (L.build_store thenv rv then_env.ebuilder)
        | None -> () );
      ignore (L.build_br merge_bb then_env.ebuilder);

      let (elsev, else_env) = expr {env with ebuilder=(L.builder_at_end context else_bb)} eelse in
      (match ret with Some(rv) ->
        ignore (L.build_store elsev rv else_env.ebuilder)
        | None -> () );
      ignore (L.build_br merge_bb else_env.ebuilder);

      let env ={env with ebuilder=(L.builder_at_end context merge_bb)} in
      (match ret with
        Some(rv) -> (L.build_load rv "if_tmp" env.ebuilder), env
      | None -> (L.const_int (ltype_of_typ Bool) 0), env)

    (* Type casting *)
   | SCast (ex) -> let t_to = t in let (t_from, _) = ex in 
      let normal_cast command = 
        let (lv, env) = expr env ex in
        (command lv (ltype_of_typ t_to) "cast" env.ebuilder, env)
      in
      let il i = (Int, SIntLit(i)) in
      let fl f = (Float, SFloatLit(f)) in
      (
      match (t_to, t_from) with
        (Int, Float)  -> normal_cast L.build_fptosi 
      | (Int, Bool)   -> expr env (Int, SIfElse(ex, il 1, il 0))
      | (Float, Int)  -> normal_cast L.build_sitofp
      | (Bool, Int)   -> expr env (Bool, SBinop(ex, Neq, il 0))
      | (Bool, Float) -> expr env (Bool, SBinop(ex, Neq, fl "0"))
      | (Void, _)     -> expr env ex
      | _             -> raise (Failure "Unknown type cast")
      )
   
   in

   (* Builds an SOS statement and returns the updated environment *)
   let build_stmt env = function
     STypeDef(_) -> env (* Everything handled in semant *)
   | SExpression(ex) -> let (_, env) = expr env ex 
     in env
   | SFxnDef(ty, nm, args, ex) ->
     let formal_types =
         Array.of_list (List.map (fun (t, _) -> ltype_of_typ t) args) in
     let ftype = L.function_type (ltype_of_typ ty) formal_types in
     let decl = L.define_function nm ftype the_module in
     let new_builder = L.builder_at_end context (L.entry_block decl) in

     (* Add function to fxns set *)
     let env = add_function env nm decl in

     let new_env = { env with ebuilder=new_builder } in
     let new_env = List.fold_left2 add_formal new_env args (
            Array.to_list (L.params decl)) in
     let new_env = {new_env with ebuilder = new_builder; ecurrent_fxn=decl} in
     let (lv, ret_env) = expr new_env ex in
     
     (* End with a return statement *)
     ( if ty=Void then
     ignore (L.build_ret_void ret_env.ebuilder)
     else
     ignore (L.build_ret lv ret_env.ebuilder) );

     env
   in
  
   (* Build the main function, the entry point for the whole program *)
   let build_main stmts = 
     (* Init the builder at the beginning of main() *)
     let builder = L.builder_at_end context (L.entry_block main) in
     (* Init the starting environment from exeternal functions *)
     let start_env = { ebuilder = builder; evars = StringMap.empty;
       efxns = StringSet.empty; esfxns = StringMap.empty; 
       ecurrent_fxn = main } in
     let start_env = List.fold_left add_external_fxn start_env
       Semant.external_functions in
     (* Build the program *)
     let end_env = List.fold_left build_stmt start_env stmts in
     (* Add a return statement *)
     L.build_ret (L.const_int i32_t 0) end_env.ebuilder    

   in
   ignore(build_main prog);
   the_module
\end{lstlisting}

\subsubsection{util\_math.c}
\begin{lstlisting}
#include <math.h>

float toradiansf(float x) {
    return (float) x * (M_PI / 180.0);
} 

#ifdef BUILD_TEST
int main() {
    toradiansf(12.5);
    return 0;
}
#endif
\end{lstlisting}

\subsubsection{util\_opengl.c}
\begin{lstlisting}
#include <stdio.h>
#include <stdlib.h>
#include <string.h>
#include "GL/osmesa.h"

/* reference: 
   https://github.com/freedesktop/mesa-demos/blob/master/src/osdemos/osdemo.c
   (mainly for gl_startRendering, write_ppm and gl_endRendering)
*/

#define maxpoints 50000

struct array {
    float *arr;
    int length;
};

OSMesaContext ctx;
void *buffer;

static void rendering_helper_init() {
   glMatrixMode(GL_PROJECTION);
   glLoadIdentity();
   glMatrixMode(GL_MODELVIEW);
   glClear(GL_COLOR_BUFFER_BIT);
   glPushMatrix();
   glEnableClientState(GL_VERTEX_ARRAY);
   glEnableClientState(GL_COLOR_ARRAY);
   glColor4f(1.0, 1.0, 1.0, 1.0); //initalize color as white
}

static void rendering_helper_close() {
    glFinish();
}

/*
 * startRendering: an initalization that must be called before drawing
 * any image. Creates Mesa and OpenGL contexts and image buffer.
 */
void gl_startRendering(int width, int height) {
    ctx = OSMesaCreateContextExt(OSMESA_RGBA, 16, 0, 0, NULL);
    if (!ctx) {
        printf("OSMesaCreateContext failed!\n");
    }

    buffer = malloc( width * height * 4 * sizeof(GLubyte) );
    if (!buffer) {
        printf("Alloc image buffer failed!\n");
    }

    // Bind the buffer to the context and make it current
    if (!OSMesaMakeCurrent(ctx, buffer, GL_UNSIGNED_BYTE, width, height)) {
        printf("OSMesaMakeCurrent failed!\n");
    }
    printf("startRendering...\n");
    rendering_helper_init();
}

/*
 * drawCurve: draws segments between a list of points,
 * meaning n-1 segments for n points
 *
 * points: an array of points, with a point (x,y) located at [2i, 2i+1]
 * colors: an array of colors, with a the RGBA values of a point located at [4i, 4i+1, 4i+2, 4i+3]
 * size_arr: the number of points
 * color_mode: 0 -> between points i and i+1, the color of the segment is the color of point i+1
 *             1 -> each point has its own color. The segment between each point is a gradient between point colors
 */
void gl_drawCurve(struct array *spoints, struct array *scolors, int color_mode) {
    if (spoints->length!=scolors->length/2) {
        fprintf( stderr, "Unable to draw: The length of points array and colors array mismatched!\n" );
        return;
    }
    glPushMatrix();
    
    if (color_mode == 0) {
        glShadeModel(GL_FLAT);
    }
    else {
        glShadeModel(GL_SMOOTH);
    }

    float points[maxpoints];
    memcpy(points, spoints->arr, sizeof(float)*spoints->length);
    float colors[maxpoints];
    memcpy(colors, scolors->arr, sizeof(float)*scolors->length);
    int size_arr = spoints->length;
    size_arr = size_arr/2;

    glVertexPointer(2, GL_FLOAT, 0, points);
    glColorPointer(4, GL_FLOAT, 0, colors);
    glDrawArrays(GL_LINE_STRIP, 0, size_arr);
    
    glPopMatrix();
}

/*
 * drawShape: draws segments between a list of points,
 * including the segment connecting the first and last point
 *
 * points: an array of points, with a point (x,y) located at [2i, 2i+1]
 * colors: an array of colors, with a the RGBA values of a point located at [4i, 4i+1, 4i+2, 4i+3]
 * size_arr: the number of points
 * color_mode: 0 -> between points i and i+1, the color of the segment is the color of point i+1
 *             1 -> each point has its own color. The segment between each point is a gradient between point colors
 * filed: 0 -> shape is not filled with color
 *        1 -> shape will be filled with color
 */
void gl_drawShape(struct array *spoints, struct array *scolors, int color_mode, int filled) {
    if (spoints->length!=scolors->length/2) {
        printf("%d %d", spoints->length, scolors->length);
        fprintf( stderr, "Unable to draw: The length of points array and colors array mismatched!\n" );
        return;
    }
    glPushMatrix();

    if (color_mode == 0) {
        glShadeModel(GL_FLAT);
    }
    else {
        glShadeModel(GL_SMOOTH);
    }

    float points[maxpoints];
    memcpy(points, spoints->arr, sizeof(float)*spoints->length);
    float colors[maxpoints];
    memcpy(colors, scolors->arr, sizeof(float)*spoints->length);
    int size_arr = spoints->length;;
    size_arr = size_arr/2;

    glVertexPointer(2, GL_FLOAT, 0, points);
    glColorPointer(3, GL_FLOAT, 0, colors);

    if (filled==1) {
        glDrawArrays(GL_POLYGON, 0, size_arr);
    }
    else {
        glDrawArrays(GL_LINE_LOOP, 0, size_arr);
    }

    glPopMatrix();
}

/*
 * drawPoint: draws all points without creating segments
 *
 * points: an array of points, with a point (x,y) located at [2i, 2i+1]
 * colors: an array of colors, with a the RGBA values of a point located at [4i, 4i+1, 4i+2, 4i+3]
 * size_arr: the number of points
 * point_size: the size of each point
 */
void gl_drawPoint(struct array *spoints, struct array *scolors, int point_size) {
    if (spoints->length!=scolors->length/2) {
        fprintf( stderr, "Unable to draw: The length of points array and colors array mismatched!\n" );
        return;
    }
    glPushMatrix();
    
    float points[maxpoints];
    memcpy(points, spoints->arr, sizeof(float)*spoints->length);
    float colors[maxpoints];
    memcpy(colors, scolors->arr, sizeof(float)*spoints->length);
    int size_arr = spoints->length;
    size_arr = size_arr/2;

    glVertexPointer(2, GL_FLOAT, 0, points);
    glColorPointer(3, GL_FLOAT, 0, colors);
    glPointSize(point_size);
    glDrawArrays(GL_POINTS, 0, size_arr);
    glPopMatrix();
}

static void gl_clearCanvas() {
    glMatrixMode(GL_MODELVIEW);
    glClear(GL_COLOR_BUFFER_BIT);
}

/*
 * write_ppm: saves drawing
 *
 * filename: file name
 * buffer: 
 * width: canvas width
 * height: canvas height
 */
static void write_ppm(int fileNumber, const GLubyte *buffer, int width, int height) {
    char filename[50];
    char filenumasstr[50];
    sprintf(filenumasstr, "%d.ppm", fileNumber);
    strcpy(filename, "pic");
    strcat(filename, filenumasstr);
    const int binary = 0;
    FILE *f = fopen( filename, "w" );
    if (f) {
       int i, x, y;
       const GLubyte *ptr = buffer;
       if (binary) {
          fprintf(f,"P6\n");
          fprintf(f,"# ppm-file created by util_opengl.c\n");
          fprintf(f,"%i %i\n", width,height);
          fprintf(f,"255\n");
          fclose(f);
          f = fopen( filename, "ab" );  /* reopen in binary append mode */
          for (y=height-1; y>=0; y--) {
             for (x=0; x<width; x++) {
                i = (y*width + x) * 4;
                fputc(ptr[i], f);   /* write red */
                fputc(ptr[i+1], f); /* write green */
                fputc(ptr[i+2], f); /* write blue */
             }
          }
       }
       else {
          /*ASCII*/
          int counter = 0;
          fprintf(f,"P3\n");
          fprintf(f,"# ascii ppm file created by util_opengl.c\n");
          fprintf(f,"%i %i\n", width, height);
          fprintf(f,"255\n");
          for (y=height-1; y>=0; y--) {
             for (x=0; x<width; x++) {
                i = (y*width + x) * 4;
                fprintf(f, " %3d %3d %3d", ptr[i], ptr[i+1], ptr[i+2]);
                counter++;
                if (counter % 5 == 0)
                   fprintf(f, "\n");
             }
          }
       }
       fclose(f);
    }
}

/*
 * endRendering: closes OpenGL and Mesa contexts and saves drawing
 * by calling write_ppm
 */
void gl_endRendering(int width, int height, int fileNumber) {
    rendering_helper_close();
    write_ppm(fileNumber, buffer, width, height);
    free(buffer);
    OSMesaDestroyContext(ctx);
    printf("endRendering...\n");
}
/*
//sample program
//#ifdef BUILD_TEST
int main(int argc, char *argv[]) {
    startRendering();
    
    float p[]= {-.5, 0, .5, 0, 0, .5};
    struct array points;
    memcpy(points.arr, p, sizeof(p));
    points.length = 6;

    float c[] = {1.0, 0.5, 1.0, 1.0, 1.0, 0.5, 0, 1.0,  0.5, 1.0, 1.0, 1.0};
    struct array colors;
    memcpy(colors.arr, c, sizeof(c));
    colors.length = 12;

    int fileNumber = 1;

    startRendering();

    drawCurve(points, colors, 1);
    
    glTranslatef(-.2,-.2,0);
    drawCurve(points, colors, 0);

    glTranslatef(-.2, -.2, 0);
    drawShape(points, colors, 1, 1);

    glTranslatef(.6, 0 , 0);
    drawShape(points, colors, 0, 0);

    glTranslatef(-.3, -.2, 0);
    drawPoint(points, colors, 10);

    endRendering(1);

    return 0;
}
//#endif
*/
\end{lstlisting}

\subsubsection{Makefile}
\begin{lstlisting}
.PHONY : all
all: sos.native util_math.o util_opengl.o

# Creates the main compiler
sos.native:
	opam config exec -- \
	ocamlbuild -use-ocamlfind sos.native

util_math: util_math.c
	cc -lm -o util_math -DBUILD_TEST util_math.c

util_opengl: util_opengl.c
	cc -o util_opengl -DBUILD_TEST util_opengl.c -I/usr/local/include/ -L/user/local/lib/ -lOSMesa -lGLU -lm

.PHONY : clean

test: 
	./testall.sh

clean : 
	ocamlbuild -clean
	rm util_math.o util_opengl.o

\end{lstlisting}

\subsubsection{Dockerfile}
\begin{lstlisting}
# Sheron Wang
# Based on 20.04 LTS
FROM ubuntu:focal

# Set timezone:
RUN ln -snf /usr/share/zoneinfo/$CONTAINER_TIMEZONE /etc/localtime && echo $CONTAINER_TIMEZONE > /etc/timezone

RUN apt-get -yq update && \
    apt-get -y upgrade && \
    apt-get -yq --no-install-suggests --no-install-recommends install \
    ocaml \
    menhir \
    llvm-10 \
    llvm-10-dev \
    m4 \
    git \
    aspcud \
    ca-certificates \
    python \
    pkg-config \
    cmake \
    opam \
    python3 \
    python3-distutils \
    ninja-build

##################################################################
# for building MESA
##################################################################
# add environment variable
RUN export PATH="/usr/bin/python:$PATH"

# add Mako and meson dependency from python
RUN wget https://bootstrap.pypa.io/get-pip.py
RUN python3 get-pip.py
RUN pip install Mako
RUN pip install meson
RUN apt-get install libpciaccess-dev -y

# download and install newest libdrm
RUN wget https://dri.freedesktop.org/libdrm/libdrm-2.4.105.tar.xz
RUN tar xf libdrm-2.4.105.tar.xz && rm libdrm-2.4.105.tar.xz
WORKDIR libdrm-2.4.105/
RUN meson build/ && cd build && ninja && ninja install
WORKDIR ../
RUN rm -r libdrm-2.4.105/

# download mesa
RUN wget https://archive.mesa3d.org//mesa-20.3.5.tar.xz
RUN tar xf mesa-20.3.5.tar.xz && rm mesa-20.3.5.tar.xz
WORKDIR mesa-20.3.5

# add things to sources.list
RUN cp /etc/apt/sources.list /etc/apt/sources.list~
RUN sed -Ei 's/^# deb-src /deb-src /' /etc/apt/sources.list
RUN apt-get update

# add dependencies
RUN apt-get install -y libdrm-dev libxxf86vm-dev libxt-dev xutils-dev flex bison xcb libx11-xcb-dev libxcb-glx0 libxcb-glx0-dev xorg-dev libxcb-dri2-0-dev
RUN apt-get install -y libelf-dev libunwind-dev valgrind libwayland-dev wayland-protocols libwayland-egl-backend-dev
RUN apt-get install -y libxcb-shm0-dev libxcb-dri3-dev libxcb-present-dev libxshmfence-dev
RUN apt-get build-dep mesa -y

# build, compile and install
RUN meson build/ -Dosmesa=classic && ninja -C build/ && ninja -C build/ install
WORKDIR ../

# add OSMesa to path
RUN echo "export LD_LIBRARY_PATH=/usr/local/lib/x86_64-linux-gnu/:$LD_LIBRARY_PATH" >> ~/.bashrc

# install MESA GLU if you need more advanced features
# RUN git clone https://gitlab.freedesktop.org/mesa/glu.git
# RUN cd glu && ./autogen.sh && ./configure --enable-osmesa --prefix=/usr/local/ && make && make install
# RUN rm -r glu mesa-20.3.5 get-pip.py

# add GLU to path
# RUN echo "export LD_LIBRARY_PATH=/usr/local/lib/:$LD_LIBRARY_PATH" >> ~/.bashrc

# install vim for testing
# RUN apt-get install vim -y

##################################################################
# for building LLVM & others
##################################################################
RUN ln -s /usr/bin/lli-10 /usr/bin/lli
RUN ln -s /usr/bin/llc-10 /usr/bin/llc

RUN opam init --disable-sandboxing -y
RUN opam install \
    llvm.10.0.0 \
    ocamlfind \
    ocamlbuild -y
RUN eval `opam config env`

WORKDIR /root

ENTRYPOINT ["opam", "config", "exec", "--"]

CMD ["bash"]
\end{lstlisting}

\subsubsection{compile\_exec.sh}
\begin{lstlisting}
#!/bin/sh

# Builds a .sos file to an executable file
# Requires that ./sos.native has been built

# Path to the LLVM interpreter
LLI="lli"

# Path to the LLVM compiler
LLC="llc"

# Path to the C compiler
CC="cc"

# Path to the SOS compiler
SOS="./sos.native"

if [ ! -f util_math.o ]
then
    echo "Could not find util_math.o"
    echo "Try \"make util_math.o\""
    exit 1
fi

filename=$1
basename=${filename%.sos}

$SOS $filename >${basename}.ll
$LLC -relocation-model=pic ${basename}.ll >${basename}.s
$CC -o ${basename}.exe ${basename}.s util_math.o -lm util_opengl.o -I/usr/local/include/ -L/user/local/lib/ -lOSMesa

rm ${basename}.ll ${basename}.s

./${basename}.exe
\end{lstlisting}

\subsubsection{docker\_connect.sh}
\begin{lstlisting}
#!/bin/sh

docker run --rm -it -v `pwd`:/home/sos -w=/home/sos sheronw1174/sos-env:latest
\end{lstlisting}

\subsubsection{docker\_image\_fetching.sh}
\begin{lstlisting}
#!/bin/sh

if [ "$1" = "build" ]
then
    docker build . -t sheronw1174/sos-env
elif [ "$1" = "pull" ]
then
    docker pull sheronw1174/sos-env
else
    echo "usage: $0 [build|pull]"
fi
\end{lstlisting}

\subsubsection{testall.sh}
\begin{lstlisting}
#!/bin/sh

# Regression testing script for SOS
# Step through a list of files
#  Compile, run, and check the output of each expected-to-work test
#  Compile and check the error of each expected-to-fail test

# Path to the LLVM interpreter
#LLI="/usr/local/Cellar/llvm/9.0.1_2/bin/lli"
LLI="lli"

# Path to the LLVM compiler
#LLC="/usr/local/Cellar/llvm/9.0.1_2/bin/lli"
LLC="llc"

# Path to the C compiler
CC="cc"

# Path to the SOS compiler.  Usually "./sos.native"
# Try "_build/sos.native" if ocamlbuild was unable to create a symbolic link.
SOS="./sos.native"

OPENGL_FLAGS="-lOSMesa -lm"

# Set time limit for all operations
ulimit -t 30

globallog=testall.log
rm -f $globallog
error=0
globalerror=0

keep=0

Usage() {
    echo "Usage: testall.sh [options] [.sos files]"
    echo "-k    Keep intermediate files"
    echo "-h    Print this help"
    exit 1
}

SignalError() {
    if [ $error -eq 0 ] ; then
	echo "FAILED"
	error=1
    fi
    echo "  $1"
}

# Compare <outfile> <reffile> <difffile>
# Compares the outfile with reffile.  Differences, if any, written to difffile
Compare() {
    generatedfiles="$generatedfiles $3"
    echo diff -b $1 $2 ">" $3 1>&2
    diff -b "$1" "$2" > "$3" 2>&1 || {
	SignalError "$1 differs"
	echo "FAILED $1 differs from $2" 1>&2
    }
}

# Run <args>
# Report the command, run it, and report any errors
Run() {
    echo $* 1>&2
    eval $* || {
	SignalError "$1 failed on $*"
	return 1
    }
}

# RunFail <args>
# Report the command, run it, and expect an error
RunFail() {
    echo $* 1>&2
    eval $* && {
	SignalError "failed: $* did not report an error"
	return 1
    }
    return 0
}

Check() {
    error=0
    basename=`echo $1 | sed 's/.*\\///
                             s/.sos//'`
    reffile=`echo $1 | sed 's/.sos$//'`
    basedir="`echo $1 | sed 's/\/[^\/]*$//'`/."

    echo -n "$basename..."

    echo 1>&2
    echo "###### Testing $basename" 1>&2

    generatedfiles=""
    
    generatedfiles="$generatedfiles ${basename}.ll ${basename}.s ${basename}.exe ${basename}.out" &&
    Run "$SOS" "$1" ">" "${basename}.ll" &&
    Run "$LLC" "-relocation-model=pic" "${basename}.ll" ">" "${basename}.s" &&
    Run "$CC" "-o" "${basename}.exe" "${basename}.s" "util_opengl.o" "util_math.o" "$OPENGL_FLAGS" &&
    Run "./${basename}.exe" > "${basename}.out" &&
    Compare ${basename}.out ${reffile}.out ${basename}.diff

    # Report the status and clean up the generated files

    if [ $error -eq 0 ] ; then
	if [ $keep -eq 0 ] ; then
	    rm -f $generatedfiles
	fi
	echo "OK"
	echo "###### SUCCESS" 1>&2
    else
	echo "###### FAILED" 1>&2
	globalerror=$error
    fi
}

CheckFail() {
    error=0
    basename=`echo $1 | sed 's/.*\\///
                             s/.sos//'`
    reffile=`echo $1 | sed 's/.sos$//'`
    basedir="`echo $1 | sed 's/\/[^\/]*$//'`/."

    echo -n "$basename..."

    echo 1>&2
    echo "###### Testing $basename" 1>&2

    generatedfiles=""

    generatedfiles="$generatedfiles ${basename}.err ${basename}.diff" &&
    RunFail "$SOS" "<" $1 "2>" "${basename}.err" ">>" $globallog &&
    Compare ${basename}.err ${reffile}.err ${basename}.diff

    # Report the status and clean up the generated files

    if [ $error -eq 0 ] ; then
	if [ $keep -eq 0 ] ; then
	    rm -f $generatedfiles
	fi
	echo "OK"
	echo "###### SUCCESS" 1>&2
    else
	echo "###### FAILED" 1>&2
	globalerror=$error
    fi
}

while getopts kdpsh c; do
    case $c in
	k) # Keep intermediate files
	    keep=1
	    ;;
	h) # Help
	    Usage
	    ;;
    esac
done

shift `expr $OPTIND - 1`

LLIFail() {
  echo "Could not find the LLVM interpreter \"$LLI\"."
  echo "Check your LLVM installation and/or modify the LLI variable in testall.sh"
  exit 1
}

which "$LLI" >> $globallog || LLIFail

if [ ! -f util_opengl.o ]
then
    echo "Could not find util_opengl.o"
    echo "Try \"make util_opengl.o\""
    exit 1
fi

if [ $# -ge 1 ]
then
    files=$@
else
    files="tests/test-*.sos tests/fail-*.sos"
fi

for file in $files
do
    case $file in
	*test-*)
	    Check $file 2>> $globallog
	    ;;
	*fail-*)
	    CheckFail $file 2>> $globallog
	    ;;
	*)
	    echo "unknown file type $file"
	    globalerror=1
	    ;;
    esac
done

exit $globalerror
\end{lstlisting}

\subsubsection{helloworld.sos}
\begin{lstlisting}
import renderer.sos

a: int = 5
print(a)

p1: point = {-0.9, -0.9}
p2: point = {-0.9, -0.7}
p3: point = {-0.7, -0.7}
p4: point = {-0.7, -0.9}

point_arr : path = [p1, p2, p3, p4]

c1 : color = {255.0, 0.0, 0.0, 0.8}
c2 : color = {0.0, 255.0, 0.0, 0.8}
c3 : color = {0.0, 0.0, 255.0, 0.8}
c4 : color = {100.0, 100.0, 0.0, 0.8}
color_arr : colors = [c1, c2, c3, c4]

my_canvas : canvas = {400, 400, 2}

startCanvas(my_canvas)
drawShape(point_arr, color_arr, 0, 1)
endCanvas(my_canvas)
\end{lstlisting}

\subsection{Library Programs}

\subsubsection{lib/affine.sos}

\begin{lstlisting}

import point.sos

struct mat2 = {a11: float, a21: float, a12: float, a22: float}
struct mat3 = {a11: float, a21: float, a31: float,
               a12: float, a22: float, a32: float,
               a13: float, a23: float, a33: float}
alias affine = mat3

affine_mul : (A: affine, v: point, w: float) -> point =
    to: point3 = {v.x, v.y, w};
    res: point3 = A ** to;
    free(to);
    if w == 0 || res.z == 1 then
    ret: point = {res.x, res.y};
    free(res) ; ret
    else
    ret: point = {res.x/res.z, res.y/res.z};
    free(res) ; ret

rotation : (r: float) -> mat2 = 
    c: float = cos(r) ; s: float = sin(r);
    {c,s,-s,c}

scale : (sx: float, sy: float) -> mat2 = 
    {sx,0.0,0.0,sy}

translate : (dx: float, dy: float) -> affine =
    {1.0,0.0,0.0, 0.0,1.0,0.0, dx,dy,1.0}

// For internal use only
// Functions not normally free their arguments like this
make_affine__ : (m: mat2) -> affine = 
    ret: affine = {m.a11, m.a21, 0.0,
                   m.a12, m.a22, 0.0,
                   0.0, 0.0, 1.0};
    free(m) ; ret

rotation_aff : (r: float) -> affine = make_affine__(rotation(r))
scale_aff : (sx: float, sy: float) -> affine = make_affine__(scale(sx,sy))
\end{lstlisting}

\subsubsection{lib/array.sos}

\begin{lstlisting}


fill_ints : (a: array int, i: int) -> void = 
    if i < a.length then a[i] = i; fill_ints(a, i+1)
    else 0

// A very useful function that makes an array of consecutive ints
// Use with implicit iteration for very nice results
ints : (n : int) -> array int = 
  if n <= 0 then (a: array int = [])
  else arr : array int = n of [0];
  fill_ints(arr, 0) ; arr
\end{lstlisting}

\subsubsection{lib/color.sos}

\begin{lstlisting}

import math.sos

struct color = {r: float, g: float, b: float, a: float}

alias colors = array color

// Standard: r,g,b in [0, 1]
rgb: (r: float, g: float, b: float) -> color = {r,g,b,1.0}

// Hue/saturation/value: hsv in [0, 1]
hsv: (h: float, s: float, v: float) -> color =
   c: float = v * s ;
   hfac: float = modf(h*6.0, 2.0) ;
   x: float = c * (1.0-abs(hfac-1.0)) ;
   m: float = v - c ;
   hh: float = h*6.0 ;

   if hh < 1.0 then rgb(v,x+m,m) else
   if hh < 2.0 then rgb(x+m,v,m) else
   if hh < 3.0 then rgb(m,v,x+m) else
   if hh < 4.0 then rgb(m,x+m,v) else
   if hh < 5.0 then rgb(x+m,m,v) else
                    rgb(v,m,x+m)
    
\end{lstlisting}

\subsubsection{lib/math.sos}

\begin{lstlisting}

// Some floating point ops

floor : (x: float) -> float = 
    z: float = (y: int = x);
    if z <= x then z
    else z - 1.0

ceil : (x: float) -> float = -floor(-x)

frac : (x: float) -> float = x - floor(x) 

max : (a: float, b: float) -> float = if a<b then b else a
min : (a: float, b: float) -> float = if a<b then a else b

clamp : (x: float, m: float, M: float) -> float = min(M, max(x, m))

abs : (x: float) -> float = if x < 0 then -x else x

// Returns the value y between 0 and m such that y = x+mn for an integer n
modf : (x: float, m: float) -> float = m * frac(x/m)

sin : (x: float) -> float = sinf(x)
cos : (x: float) -> float = cosf(x)
tan : (x: float) -> float = tanf(x)
asin : (x: float) -> float = asinf(x)
acos : (x: float) -> float = acosf(x)
atan : (x: float) -> float = atanf(x)
sqrt : (x: float) -> float = sqrtf(x)
toradians : (x: float) -> float = toradiansf(x)
\end{lstlisting}

\subsubsection{lib/point.sos}

\begin{lstlisting}

import math.sos

struct point = {x: float, y: float}
struct point3 = {x: float, y: float, z: float}

sqrMagnitude : (p: point) -> float = p * p
magnitude : (p: point) -> float = sqrt(sqrMagnitude(p))

sqrDistance : (a: point, b: point) -> float = 
   p: point = a - b;
   d: float = sqrMagnitude(p);
   free(p) ; d
distance : (a: point, b: point) -> float = sqrt(sqrDistance(a,b))
\end{lstlisting}

\subsubsection{lib/random.sos}

\begin{lstlisting}

// Wichmann-Hill PRNG
// en.wikipedia.org/wiki/Wichmann-Hill
import math.sos

struct rng = {s1: int, s2: int, s3: int}

randf : (r: rng) -> float =
   r.s1 = 171 * r.s1 % 30269;
   r.s2 = 172 * r.s2 % 30307;
   r.s3 = 170 * r.s3 % 30323;
   frac(r.s1/30269.0 + r.s2/30307.0 + r.s3/30323.0)
\end{lstlisting}

\subsubsection{lib/renderer.sos}

\begin{lstlisting}

import shape.sos
import color.sos

struct canvas = {width: int, height: int, file_number: int}

startCanvas: (c : canvas) -> void = gl_startRendering(c.width, c.height)

cvoid : () -> void = 0

drawHelper : (point_structs: path, color_structs: array color, numOfPoints: int, i: int,  points: array float, colors: array float) -> void =
    if i >= numOfPoints /*i>=numOfPoints-1?*/
    then cvoid()
    else
        px : float = point_structs[i].x;
        py : float= point_structs[i].y;
        points[2*i] = px;
        points[2*i+1] = py;

        cr : float = color_structs[i].r;
        cg : float = color_structs[i].g;
        cb : float= color_structs[i].b;
        ca : float= color_structs[i].a;
        colors[4*i] = cr;
        colors[4*i+1] = cg;
        colors[4*i+2] = cb;
        colors[4*i+3] = ca;

        drawHelper(point_structs, color_structs, numOfPoints, i + 1, points, colors)

drawPoints : (point_structs : path, color_structs : array color) -> void = 
        numOfPoints : int = point_structs.length;
        points : array float = numOfPoints*2 of [0.0];
        colors : array float = numOfPoints*4 of [0.0];
        
        drawHelper(point_structs, color_structs, numOfPoints, 0,  points, colors);

        gl_drawPoint(points, colors, 2)

drawPath : (point_structs : path, color_structs : array color,  colorMode : int) -> void = 
        numOfPoints : int= point_structs.length;
        points : array float= numOfPoints*2 of [0.0];
        colors : array float= numOfPoints*4 of [0.0];

        drawHelper(point_structs, color_structs, numOfPoints, 0,  points, colors);
        
        gl_drawCurve(points, colors, colorMode)


drawShape : (point_structs : path, color_structs : array color,  colorMode : int, filled : int) ->
    void = 
        numOfPoints : int = point_structs.length;
        points : array float = numOfPoints*2 of [0.0];
        colors : array float = numOfPoints*4 of [0.0];

        drawHelper(point_structs, color_structs, numOfPoints, 0,  points, colors);

        gl_drawShape(points, colors, colorMode, filled)


endCanvas : (c : canvas) -> void = gl_endRendering(c.width, c.height, c.file_number)
\end{lstlisting}

\subsubsection{lib/shape.sos}

\begin{lstlisting}

import point.sos

alias path = array point
alias shape = array point

// Wrappers to enable array iteration
copy_point : (p: point) -> point = copy(p)
free_point : (p: point) -> void = free(p)

// More convenient names
copy_path : (p: path) -> path = copy_point(p)
free_path : (p: path) -> void = free_point(p)

appendhelp_copyin : (in: path, from: path, i: int) -> void = 
    if i<in.length then
    in[i] = copy(from[i+1]);
    appendhelp_copyin (in, from, i+1)
    else 0

appendhelp_tail : (p: path) -> path = 
    tail: path = p.length-1 of [{0.0, 0.0}];
    appendhelp_copyin(tail, p, 0); tail

// Appends two paths, merging them at their endpoints, if needed
// Epsilon is the max distance that can be merged
append : (p1: path, p2: path, epsilon: float) -> path = 
    if p1.length == 0 then copy_path(p2)
    else if p2.length == 0 then copy_path(p1)
    else 
    merge: bool = sqrDistance(p1[p1.length-1], p2[0]) < epsilon*epsilon;
    p2c: path = (if merge then appendhelp_tail(p2) else p2);
    ret: path = copy_path(p1) @ copy_path(p2c);
    ret

reversedhelp: (in: path, from: path, i: int) -> void = 
    if i < in.length then
    in[i].x = from[in.length-1-i].x;
    in[i].y = from[in.length-1-i].y;
    reversedhelp(in, from, i+1)
    else 0

// Creates a new array that is p reversed
reversed: (p : path) -> path =
    newpath : path = p.length of [{0.0, 0.0}];
    reversedhelp(newpath, p, 0);
    newpath

reversehelp : (p: path, i: int) -> void =
    if i < p.length/2 then
    q: point = p[i];
    p[i] = p[p.length-1-i];
    p[p.length-1-i] = q;
    reversehelp (p, i+1)
    else 0

// Reverses p in-place
reverse: (p: path) -> void = reversehelp(p, 0)
    
\end{lstlisting}

\subsubsection{lib/size.sos}

\begin{lstlisting}

import shape.sos

struct minPoint = {float x, float y}

// scale rightward downward
// should based on a rectangular box aound the object and get that vertex? or just use opengl

min(a: point, b: minPoint) -> void =
    if a.x < b.x and a.y < b.y then b.x = a.x and b.y = a.y
    else 0

// shape size(float multiple) =
    

\end{lstlisting}

\subsubsection{lib/std.sos}

\begin{lstlisting}

import vector.sos
import affine.sos
import renderer.sos
\end{lstlisting}

\subsubsection{lib/transform.sos}

\begin{lstlisting}

import shape.sos
import math.sos
import vector.sos

// Rotates the given point by angle radians about the given point
// Either rotate clockwise (1) or counterclockwise (-1)
rotate : (p: point, angle: float, direction: int, about: point) -> void =
    //shifted
    px : float = p.x - about.x;
    py : float = p.y - about.y;

    if direction == -1
    then //counterclockwise
        p.x = (px*cos(angle) - py*sin(angle)) + about.x;
        p.y = (px*sin(angle) + py*cos(angle)) + about.y
    else if direction == 1
    then //clockwise
        p.x = (px*cos(angle) + py*sin(angle)) + about.x;
        p.y = (-px*sin(angle)+ py*cos(angle)) + about.y
    else //do no rotation
        p.x = p.x;
        p.y = p.y 

// Translates the given point by the given vector
trans : (p: point, direction: vector) -> void = 
    p.x = p.x + direction.x ; p.y = p.y + direction.y

// Scales the point by (sx, sy)
scale : (p: point, sx: float, sy: float) -> void =
    p.x = p.x * sx ; p.y = p.y * sy

// Performs rotate() on a new point
rotated : (p: point, angle: float, direction: int, about: point) -> point =
    q: point = copy(p) ;
    rotate(q, angle, direction, about) ;
    q

// Performs translate() on a new point
translated : (p: point, direction: vector) -> point =
    q: point = copy(p) ;
    trans(q, direction) ;
    q

scaled : (p: point, sx: float, sy: float) -> point = 
    q: point = copy(p) ; 
    scale(p, sx, sy) ;
    q
\end{lstlisting}

\subsubsection{lib/vector.sos}

\begin{lstlisting}

struct vector = {x: float, y: float}

\end{lstlisting}

\subsection{Example Programs}

\subsubsection{sample\_programs/dragon.sos}

\begin{lstlisting}

import renderer.sos
import vector.sos
import transform.sos
import array.sos
import math.sos

// Creates a dragon curve of depth n
dragon: (n: int) -> path = 
    if n == 0 // Base case
    then [point{0.0, 0.0}, point{1.0, 0.0}]
    else
    // Create two copies of the previous depths
    d1: path = dragon(n-1) ; 
    d2: path = copy_path(d1) ;

    // Position d1
    s: float = sqrt(2.0)/2.0 ;
    rotate(d1, toradians(45.0), -1, {0.0, 0.0}) ; 
    scale(d1, s, s) ; 

    // Position d2
    rotate(d2, toradians(135.0), -1, {0., 0.}) ;
    scale(d2,s,s) ;
    trans(d2, {1., 0.}) ;
    reverse(d2) ;

    // Merge the paths
    r: path = append(d1, d2, 1.0) ;
    free_path(d1); free_path(d2); r

// Creates a rainbow color effect
rainbow: (r: int, len: int) -> color = 
    h: float = (1.0*r)/len ;
    hsv(h, 0.8, 0.8)

// Render a 400px by 400px canvas, name the image pic0
my_canvas: canvas = {400, 400, 0}

// Start render
startCanvas(my_canvas)
d: path = dragon(7)
// Position the curve (0.4, 0.2 is approximately the center of mass of the curve for large n)
trans(d, {-0.4, -0.2})

// Draw it
drawPath(d, rainbow(ints(d.length), d.length), 0)
endCanvas(my_canvas)
\end{lstlisting}

\subsubsection{sample\_programs/drunk.sos}

\begin{lstlisting}

import renderer.sos
import random.sos

n: int = 100
p: path = n of [{0.0, 0.0}]
c: colors = n of [{0.5, 0.5, 0.5, 0.5}]
r: rng = {1,2,3}

drunk_walk : (i: int, p: path, c: colors, r: rng) -> void = 
    if i < p.length then
      theta: float = randf(r) * 6.28319;
      d: float = randf(r)*0.1 + 0.02;
      dx: float = cos(theta)*d ; dy: float = sin(theta)*d ;
      p[i] = {p[i-1].x+dx, p[i-1].y+dy} ;
      dc: float = 0.1 ; 
      dr: float = (randf(r) - 0.5) * dc ;
      dg: float = (randf(r) - 0.5) * dc ;
      db: float = (randf(r) - 0.5) * dc ;
      c[i] = rgb(c[i-1].r + dr, c[i-1].g+dg, c[i-1].b+db) ;
      drunk_walk(i+1,p,c,r)

    else void


my_canvas: canvas = {400, 400, 0}
startCanvas(my_canvas)

draw_walks : (count: int, p: path, c: colors, r: rng) -> void =
    if count > 0 then
    p[0] = {randf(r)*0.5-0.25, randf(r)*0.5-0.25};
    c[0] = hsv(randf(r), 0.8, 0.8) ;
    drunk_walk(1, p, c, r) ;
    drawPath(p,c,0) ;
    draw_walks(count-1,p,c,r)
    else void

draw_walks(20, p, c, r)

endCanvas(my_canvas)

\end{lstlisting}

\subsubsection{sample\_programs/lorenz.sos}

\begin{lstlisting}

import renderer.sos
import array.sos
alias path3 = array point3

create_lorenz : (p: path3, i: int, sigma: float, rho: float, beta: float) -> void = 
    if i == 0 then p[i] = {0.1,0.1,0.1}; create_lorenz(p,i+1,sigma,rho,beta)
    else if i < p.length then
      q: point3 = p[i-1] ;
      dx: float = sigma * (q.y - q.x) ;
      dy: float = q.x * (rho - q.z) - q.y ;
      dz: float = q.x*q.y - beta * q.z ;
      dt: float = 0.005 ;
      p[i] = {q.x + dx*dt, q.y + dy*dt, q.z + dz*dt} ;
      create_lorenz(p,i+1,sigma,rho,beta)
    else void

len: int = 5000
l3: path3 = len of [{0.0,0.0,0.0}]
create_lorenz (l3, 0, 10, 28, 8.0/3)
rainbow: (r: int, len: int) -> color = 
    h: float = (1.0*r)/len ;
    hsv(h, 0.8, 0.8)
c: colors = rainbow(ints(len), len+1000)


reduce : (p: point3) -> point = {p.x/30.0, p.y/30.0}

l: path = reduce(l3)

my_canvas : canvas = {400,400,1}

startCanvas(my_canvas)
drawPath(l, c, 0)
endCanvas(my_canvas)

\end{lstlisting}

\subsubsection{sample\_programs/square.sos}
\begin{lstlisting}
import renderer.sos

a: int = 5 
print(a)

p1: point = {-0.5, -0.5}
p2: point = {-0.5, 0.5}
p3: point = {0.5, 0.5}
p4: point = {0.5, -0.5}
point_arr : path = [p1, p2, p3, p4] 

c1 : color = {255.0, 0.0, 0.0, 0.8}
c2 : color = {0.0, 255.0, 0.0, 0.8}
c3 : color = {0.0, 0.0, 255.0, 0.8}
c4 : color = {100.0, 100.0, 0.0, 0.8}
color_arr : colors = [c1, c2, c3, c4] 

canvas1 : canvas = {400, 400, 0}

startCanvas(canvas1)
drawShape(point_arr, color_arr, 0, 1)
endCanvas(canvas1)

canvas2 : canvas = {400, 400, 1}

startCanvas(canvas2)
drawShape(point_arr, color_arr, 0, 1)
endCanvas(canvas2)
\end{lstlisting}


\subsubsection{sample\_programs/tree.sos}

\begin{lstlisting}

import affine.sos
import renderer.sos

branchHelper : (p: path, i: int, A: affine) -> void = 
    if i < p.length then
    p[i] = affine_mul(A, p[i-1]-p[i-2], 0) + p[i-1];
    branchHelper(p,i+1,A)
    else void

makeBranch : (n: int, A: affine) -> path =
    p: path = n of [{0.0,0.0}];
    p[1] = {0.0, 0.2};
    branchHelper(p, 2, A); p

powersHelper : (arr: array affine, i: int, A: affine) -> void = 
    if i < arr.length then
    arr[i] = A ** arr[i-1];
    powersHelper(arr,i+1,A)
    else void

powers : (n: int, A: affine) -> array affine = 
    arr: array affine = n of [copy(A)];
    powersHelper(arr, 1, A) ; arr

A: affine = translate(0, 0.2) ** rotation_aff(0.13) ** scale_aff(0.8, 0.8)
As: array affine = powers(8, A)
L: affine = scale_aff(0.36, 0.36) ** rotation_aff(0.66)
R: affine = scale_aff(0.49,0.49) ** rotation_aff(-0.78)
base: path = makeBranch(10, A)

white: colors = 10 of [rgb(1.0,1.0,1.0)]
green: colors = 10 of [rgb(0.0,0.8,0.0)]

my_canvas: canvas = {400,400,0}
startCanvas(my_canvas)

drawTree : (base: path, T: affine, A: affine, As: array affine, L: affine, R: affine, depth: int, c1: colors, c2: colors)
  -> void = 
  if depth==0 then drawPath(affine_mul(T,base,1), c2, 0) else
  drawPath(affine_mul(T,base,1), c1, 0) ;
  drawTree(base, T ** As ** L, A, As, L, R, depth-1, c1, c2) ;
  drawTree(base, T ** As ** R, A, As, L, R, depth-1, c1, c2)

T: affine = {1.8,0.0,0.0,0.0,1.8,0.0,0.0,-0.8,1.0}
drawTree (base, T, A, As, L, R, 3, white, green)

endCanvas(my_canvas)

\end{lstlisting}

\subsubsection{sample\_programs/web.sos}

\begin{lstlisting}

import renderer.sos
import random.sos
import array.sos
import math.sos

build_ring : (i: int, r: int, radius: float) -> point = 
    angle: float = (i*6.2831852)/r  + 0.134;
    {cos(angle) * radius, sin(angle) * radius}

build_rings : (n: int, r: int, radius_interval: float, rr: rng) -> array point =
    build_ring(ints(r), r, radius_interval * n * (randf(rr)*0.18+0.91))

build_points : (n : int, r : int, max_radius: float, rr: rng) -> array array point = 
    build_rings (ints(n), r, max_radius/(n-1), rr) 

connect_ring_inner : (i: int, a: array point, c: colors) -> void =
    if  i==a.length-1 then drawPath([a[i], a[0]], c, 0)
    else drawPath([a[i], a[i+1]], c, 0); connect_ring_inner(i+1,a,c)

connect_ring : (a: array point, c: colors) -> void =
    connect_ring_inner(0, a, c) 

ring_iter : (w: array array point, c: colors, i: int, j: int, r: rng,
        f: func array array point, colors, int, int, rng -> void) -> void = 
    if i < w.length then if j < w[i].length then
    f(w,c,i,j,r) ; ring_iter(w,c,i,j+1,r,f)
    else ring_iter(w,c,i+1,0,r,f) else void

connect_lines_in : (w: array array point, c: colors, i: int, j: int, r: rng) -> void = 
    if i < w.length - 1 then
    drawPath([w[i][j], w[i+1][j]], c, 0)
    else drawPath([w[i][j], 5*w[i][j]], c, 0)

connect_lines : (w: array array point, c: colors, r: rng) -> void = 
    ring_iter(w,c,0,0,r, connect_lines_in)

random_connections_in : (w: array array point, c: colors, i: int, j: int, r: rng) -> void =
    if i < w.length - 1 then
    (if randf(r) > 0.65 then drawPath([w[i][j], w[i+1][(j+1)%w[0].length]], c, 0) else void); 
    if randf(r) > 0.65 then
      drawPath([w[i][j], w[i+1][(j-1+w[0].length)%w[0].length]], c, 0) else void
    else void

random_connections : (w: array array point, c: colors, r: rng) -> void = 
   ring_iter(w,c,1,0,r, random_connections_in)

perturb_in : (w: array array point, c: colors, i: int, j: int, r: rng) -> void =
    s: float = 0.1 ;
    dx: float = randf(r) * s + 1.0 - s/2;
    dy: float = randf(r) * s + 1.0 - s/2;
    w[i][j].x = w[i][j].x * dx ;
    w[i][j].y = w[i][j].y * dy

perturb : (w: array array point, c: colors, r: rng) -> void = 
   ring_iter(w,c,1,0,r, perturb_in)

my_canvas: canvas = {400, 400, 3}
r: rng = {35,62,21}
startCanvas(my_canvas)

w: array array point = build_points(7, 7, 0.95, r)
perturb(w, [], r)
connect_ring(w, [rgb(1,1,1), rgb(1,1,1)])
connect_lines(w, [rgb(1,1,1), rgb(1,1,1)], r)
random_connections(w, [rgb(1,1,1), rgb(1,1,1)], r)

endCanvas(my_canvas)
\end{lstlisting}

\subsection{Test Programs}

\subsubsection{tests/fail-alias.err}

\begin{lstlisting}

Fatal error: exception Failure("Cannot create an alias with preexisting name int")
\end{lstlisting}

\subsubsection{tests/fail-alias.sos}

\begin{lstlisting}

//aliasing type id is already defined. could have problems if aliasing float as int, etc.
alias int = int
\end{lstlisting}

\subsubsection{tests/fail-alias2.err}

\begin{lstlisting}

Fatal error: exception Stdlib.Parsing.Parse_error
\end{lstlisting}

\subsubsection{tests/fail-alias2.sos}

\begin{lstlisting}

//an alias of an undefined type id

alias x: notatype
\end{lstlisting}

\subsubsection{tests/fail-array-construction2.err}

\begin{lstlisting}

Fatal error: exception Failure("Could not resolve type when defining a(Found array float, expected array int)")
\end{lstlisting}

\subsubsection{tests/fail-array-construction2.sos}

\begin{lstlisting}

//if first element in array construction does not match array type, automatic type conversion does not take place
a: array int = [2.2, 1]
\end{lstlisting}

\subsubsection{tests/fail-array-of.err}

\begin{lstlisting}

Fatal error: exception Failure("First operand of of operator must be an int")
\end{lstlisting}

\subsubsection{tests/fail-array-of.sos}

\begin{lstlisting}

struct point = {x: float, y: float}
p1: point = {1.2, 2.3}

a: array float = p1 of [0.0]
\end{lstlisting}

\subsubsection{tests/fail-array-void.err}

\begin{lstlisting}

Fatal error: exception Stdlib.Parsing.Parse_error
\end{lstlisting}

\subsubsection{tests/fail-array-void.sos}

\begin{lstlisting}

//arrays cannot be of type void
array void a = []
\end{lstlisting}

\subsubsection{tests/fail-bool-arith.err}

\begin{lstlisting}

Fatal error: exception Failure("Cannot add or subtract bool and bool")
\end{lstlisting}

\subsubsection{tests/fail-bool-arith.sos}

\begin{lstlisting}

t: bool = true
f: bool = false

sum: bool = t + f
\end{lstlisting}

\subsubsection{tests/fail-bool-comparison.err}

\begin{lstlisting}

Fatal error: exception Failure("Cannot equate bool and bool")
\end{lstlisting}

\subsubsection{tests/fail-bool-comparison.sos}

\begin{lstlisting}

b: bool = true == true
\end{lstlisting}

\subsubsection{tests/fail-bool.err}

\begin{lstlisting}

Fatal error: exception Failure("Cannot equate bool and bool")
\end{lstlisting}

\subsubsection{tests/fail-bool.sos}

\begin{lstlisting}

b: bool = true == true
\end{lstlisting}

\subsubsection{tests/fail-comparison-bool-float.err}

\begin{lstlisting}

Fatal error: exception Failure("Cannot compare bool and float")
\end{lstlisting}

\subsubsection{tests/fail-comparison-bool-float.sos}

\begin{lstlisting}

b: bool = true > 0.1
\end{lstlisting}

\subsubsection{tests/fail-definitions-in-definitions.err}

\begin{lstlisting}

Fatal error: exception Stdlib.Parsing.Parse_error
\end{lstlisting}

\subsubsection{tests/fail-definitions-in-definitions.sos}

\begin{lstlisting}

//struct defined in function definition
func2: (x: int) -> int = 
    struct s = {field: int};
    temp: s = {x};
    s.field
\end{lstlisting}

\subsubsection{tests/fail-definitions-in-definitions2.err}

\begin{lstlisting}

Fatal error: exception Stdlib.Parsing.Parse_error
\end{lstlisting}

\subsubsection{tests/fail-definitions-in-definitions2.sos}

\begin{lstlisting}

//defining function in function
bar: () -> int = 5
baruser: (f: func  -> int) -> int = 
    sum: (x: int, y: int) -> int = x+y; temp: int = f(); sum(temp, 5)

barvar: func -> int = bar
print(baruser(barvar))
\end{lstlisting}

\subsubsection{tests/fail-derived-comparison.err}

\begin{lstlisting}

Fatal error: exception Failure("Can only equate structs of matching type")
\end{lstlisting}

\subsubsection{tests/fail-derived-comparison.sos}

\begin{lstlisting}

x: bool = {1,2} == {3,4,5}
\end{lstlisting}

\subsubsection{tests/fail-float.err}

\begin{lstlisting}

Fatal error: exception Stdlib.Parsing.Parse_error
\end{lstlisting}

\subsubsection{tests/fail-float.sos}

\begin{lstlisting}

f: float = .1
\end{lstlisting}

\subsubsection{tests/fail-func-param-type.err}

\begin{lstlisting}

Fatal error: exception Failure("Could not resolve type when defining p(Found struct {int, int}, expected struct {float, float})")
\end{lstlisting}

\subsubsection{tests/fail-func-param-type.sos}

\begin{lstlisting}

struct point = {x: float, y: float}
p: point = {1,2}

sum: (x: int, y: int) -> int = x+y

sum(3, p)
\end{lstlisting}

\subsubsection{tests/fail-func-return-type.err}

\begin{lstlisting}

Fatal error: exception Failure("Incorrect return type for function f (Found int, expected struct {float, float})")
\end{lstlisting}

\subsubsection{tests/fail-func-return-type.sos}

\begin{lstlisting}

struct point = {x: float, y: float}

f: (x: int) -> point = if x == 1 then 1 else 2
\end{lstlisting}

\subsubsection{tests/fail-func.err}

\begin{lstlisting}

Fatal error: exception Failure("Unknown variable name sum")
\end{lstlisting}

\subsubsection{tests/fail-func.sos}

\begin{lstlisting}

//undefined func

x: int = sum(1, 2)
\end{lstlisting}

\subsubsection{tests/fail-if-elseless.err}

\begin{lstlisting}

Fatal error: exception Stdlib.Parsing.Parse_error
\end{lstlisting}

\subsubsection{tests/fail-if-elseless.sos}

\begin{lstlisting}

x: int = if 1==2 then 1
\end{lstlisting}

\subsubsection{tests/fail-if-expr-type.err}

\begin{lstlisting}

Fatal error: exception Failure("Could not reconcile types of then and else clauses (int, array int)")
\end{lstlisting}

\subsubsection{tests/fail-if-expr-type.sos}

\begin{lstlisting}

x: int = 
if 1 == 1 
then 1
else [2]
\end{lstlisting}

\subsubsection{tests/fail-memory.err}

\begin{lstlisting}

Fatal error: exception Failure("Can only free memory of struct and array types")
\end{lstlisting}

\subsubsection{tests/fail-memory.sos}

\begin{lstlisting}

add: (x: int, y: int) -> int = 
    x + 7

squareafteradd: (x: int, y: int) -> int =
    temp: int = add(x, y);
    ret: int = temp*temp;
    free(temp);
    ret
\end{lstlisting}

\subsubsection{tests/fail-negate.err}

\begin{lstlisting}

Fatal error: exception Failure("Cannot negate non-arithmetic types")
\end{lstlisting}

\subsubsection{tests/fail-negate.sos}

\begin{lstlisting}

x: int = -true
\end{lstlisting}

\subsubsection{tests/fail-struct-arith.err}

\begin{lstlisting}

Fatal error: exception Failure("Can only add or subtract structs of matching type")
\end{lstlisting}

\subsubsection{tests/fail-struct-arith.out}

\begin{lstlisting}

Can only add or subtract structs of matching type
\end{lstlisting}

\subsubsection{tests/fail-struct-arith.sos}

\begin{lstlisting}

struct s1 = {a: int, b: float}

struct s2 = {a: int, b: int}

a: s1 = {1, 1.0}
b: s2 = {1, 1}

c: s3 = a+b
\end{lstlisting}

\subsubsection{tests/fail-struct-arith2.err}

\begin{lstlisting}

Fatal error: exception Failure("Can only add or subtract structs of matching type")
\end{lstlisting}

\subsubsection{tests/fail-struct-arith2.out}

\begin{lstlisting}

Can only operate on arithmetic structs
\end{lstlisting}

\subsubsection{tests/fail-struct-arith2.sos}

\begin{lstlisting}

//arithmetic on structs that do not match types

struct s1 = {a: float, b: float}

struct s2 = {a: float, b: bool}

a: s1 = {1.4, 2.3}
b: s2 = {1.0, false}

c: s1 = a + b

\end{lstlisting}

\subsubsection{tests/fail-struct-construction.err}

\begin{lstlisting}

Fatal error: exception Failure("Could not resolve type when defining v2(Found struct {float, int, int}, expected struct {float, float})")
\end{lstlisting}

\subsubsection{tests/fail-struct-construction.sos}

\begin{lstlisting}

struct vector = {x: float, y: float}
v2: vector = {1.0, 2, 0}
\end{lstlisting}

\subsubsection{tests/fail-struct-field.err}

\begin{lstlisting}

Fatal error: exception Failure("Could not find field y")
\end{lstlisting}

\subsubsection{tests/fail-struct-field.sos}

\begin{lstlisting}

//accessing an undefined struct field

struct a = {x: int}

s: a = {1}

b: int = s.y
\end{lstlisting}

\subsubsection{tests/fail-struct-field2.err}

\begin{lstlisting}

Fatal error: exception Failure("Cannot access fields for a non-struct variable")
\end{lstlisting}

\subsubsection{tests/fail-struct-field2.out}

\begin{lstlisting}

Cannot access fields for a non-struct variable
\end{lstlisting}

\subsubsection{tests/fail-struct-field2.sos}

\begin{lstlisting}

//attempting to access a field of a non-struct

struct s = {x: int}
b: int = 1
y: int = b.x
\end{lstlisting}

\subsubsection{tests/fail-type-conversions.err}

\begin{lstlisting}

Fatal error: exception Failure("Could not resolve type when defining p2(Found struct {float, float}, expected struct {float, float, float})")
\end{lstlisting}

\subsubsection{tests/fail-type-conversions.sos}

\begin{lstlisting}

struct point = {x: float, y: float}

struct point3d = {x: float, y: float, z: float}

p1: point = {1.0, 2.0}

p2: point3d = p1
\end{lstlisting}

\subsubsection{tests/fail-unop.err}

\begin{lstlisting}

Fatal error: exception Failure("Cannot negate non-arithmetic types")
\end{lstlisting}

\subsubsection{tests/fail-unop.sos}

\begin{lstlisting}

//fixed b to bool type. no type conversion

b: bool = -true
\end{lstlisting}

\subsubsection{tests/fail-unop2.err}

\begin{lstlisting}

Fatal error: exception Failure("Cannot negate non-arithmetic types")
\end{lstlisting}

\subsubsection{tests/fail-unop2.sos}

\begin{lstlisting}

x: int= -true
print(x)
\end{lstlisting}

\subsubsection{tests/fail-value-access-modifier.err}

\begin{lstlisting}

Fatal error: exception Failure("Could not find field c")
\end{lstlisting}

\subsubsection{tests/fail-value-access-modifier.sos}

\begin{lstlisting}

import color.sos
printf($(c : color = {0.0, 255.0, 255.0, 0.8}).c)\end{lstlisting}

\subsubsection{tests/fail-vardecl-as-expr-scoping.err}

\begin{lstlisting}

Fatal error: exception Failure("Unknown variable name x")
\end{lstlisting}

\subsubsection{tests/fail-vardecl-as-expr-scoping.sos}

\begin{lstlisting}

y: int = 5 
y= (x: int = 5)

print(y) //5 
print(x) //5 
\end{lstlisting}

\subsubsection{tests/fail-vardef.err}

\begin{lstlisting}

Fatal error: exception Failure("Could not resolve type id notatype")
\end{lstlisting}

\subsubsection{tests/fail-vardef.sos}

\begin{lstlisting}

//defining a variable with an undefined type id

x: notatype = 10
\end{lstlisting}

\subsubsection{tests/fail-vardef2.err}

\begin{lstlisting}

Fatal error: exception Failure("Unknown variable name x")
\end{lstlisting}

\subsubsection{tests/fail-vardef2.sos}

\begin{lstlisting}

//assignment before variable declaration

x = 10
\end{lstlisting}

\subsubsection{tests/fail-vardef3.err}

\begin{lstlisting}

Fatal error: exception Stdlib.Parsing.Parse_error
\end{lstlisting}

\subsubsection{tests/fail-vardef3.sos}

\begin{lstlisting}

//variable declaration without an initialization to a value
x: int
x = 5
\end{lstlisting}

\subsubsection{tests/fail-void.err}

\begin{lstlisting}

Fatal error: exception Failure("Cannot use a void type in this context")
\end{lstlisting}

\subsubsection{tests/fail-void.sos}

\begin{lstlisting}

x: void = 0
\end{lstlisting}

\subsubsection{tests/test-E.out}

\begin{lstlisting}

200
\end{lstlisting}

\subsubsection{tests/test-E.sos}

\begin{lstlisting}

x: float = 2.0E2
print(x)
\end{lstlisting}

\subsubsection{tests/test-alias.out}

\begin{lstlisting}

2021
3
20
2022
2022
2024
2022
4042
2024
2041
56
1
1
1
1
2021
3
20
1
1
0
1.2
87
90
1
0
90
4
\end{lstlisting}

\subsubsection{tests/test-alias.sos}

\begin{lstlisting}

/*alias of int */
alias year = int
alias month = int

//alias of alias
alias day = month 

present_year: year = 2021
present_month: month = 3
present_day: day = 20
print(present_year) //2021
print(present_month) //3
print(present_day) //20


/*alias operations */

//alias operator primitive -> alias
//year + int -> year
test: year = present_year + 1
print(test) //2021+1=2022

//alias operator primitive -> alias2
//year + int -> month
test2: month = present_year + 1
print(test2) //2021+1=2022

//alias operator primitive -> aliasofalias
//year + int -> day
test3: day = present_year + 3
print(test3) //2021+1=2022

//alias operator primitive -> primitive
//year + int -> int
test4: int = present_year + 1
print(test4) //2021+1=2022

//alias operator alias -> primitive
//year + year -> int
test5: int = present_year + present_year
print(test5) //2021+2021=4042

//alias operator alias2 -> primitive
//year + month -> int
test6: int = present_year + present_month
print(test6) //2021+3=2024

//alias operator aliasofalias -> primitive
//year + month -> month
test7: int = present_year + present_day
print(test7) //2021+20=2041

//mixing int operators and aliases
test8: int = present_month*present_day-present_year/10%9
print(test8) //3*20-(2021/10)%9=56

/*alias comparison */
//primitive vs alias
i1: int = 1
y1: year = 1
result: bool = i1 == y1
print(result) //True

//alias vs alias
y2: year = 1
result2: bool = y1 == y2
print(result2) //True

//alias vs alias2
m1: month = 1
result3: bool = y1 == m1
print(result3) //True

//alias vs aliasofalias
d1: day = 1
result4: bool = y1 == d1
print(result4) //True

/*alias of struct*/
struct date = {day: int, yr: year, mth: month}
present_day: date = {20, present_year, present_month}
print(present_day.yr) //2021

//alias of struct field
present_day.yr = present_month
print(present_day.yr) //3

//alias of struct
alias new_date = date
newer: new_date = present_day
print(newer.day) //20

newer2: new_date = {1, 2, 3}
print(newer2.day) //1

result5: bool = newer == present_day
print(result5) //True (pass by reference)
result6: bool = newer == newer2 
print(result6) //False

/*alias of float*/
alias x_dist = float
x: x_dist = 1.2
printf(x) //1.2

/*alias array*/
//alias of array int
alias scores = array int
class1: scores = [87,93,70]
print(class1[0]) //87
class1 = class1 + 3
print(class1[0]) //90

//alias of alias of array int
alias marks = scores
class2: marks = class1
class3: marks = [87, 93, 70]
result7: array array bool = class2 == class1
print(result7[0][0]) //True
result8: array array bool = class3 == class1
print(result8[0][0]) //False

//alias of array of alias of array int: array array int
alias class_scores = array scores
highschool1: class_scores = [class1, [1,2,3]]
print(highschool1[0][0]) //90

//alias of array array array int
alias school_scores = array array array int
district1: school_scores = [highschool1, [[4,5,6], [7,8,9]], [[10,11], [12,13,14,15]]]
print(district1[1][0][0]) //4
\end{lstlisting}

\subsubsection{tests/test-array-access.out}

\begin{lstlisting}

1
2
3
1
2
2
4
6
\end{lstlisting}

\subsubsection{tests/test-array-access.sos}

\begin{lstlisting}

mult_2 : (i: int, arr: array int) -> array int = 
    if i == -1
    then arr
    //array access and assignment
    else 
        arr[i] = arr[i]*2; 
        mult_2(i - 1, arr)

printingarray : (arr: array int, length: int, current: int) -> void =
    if current == length
    then void
    else 
        print(arr[current]);
        printingarray(arr, length, current+1)

a: array int = [1,2,3]
n: int = a[0]
b: array int = copy(a)
b = mult_2(3, b)

printingarray(a, 3, 0) //1, 2, 3
print(n) //1
print(a[0+1]) //2
printingarray(b, 3, 0) //2, 4, 6
\end{lstlisting}

\subsubsection{tests/test-array-arith-operators.out}

\begin{lstlisting}

1
2
3
4
5
6
7
8
9
10
-1
0
1
2
3
2
4
6
8
10
0
1
1
2
2
\end{lstlisting}

\subsubsection{tests/test-array-arith-operators.sos}

\begin{lstlisting}

printing: (x: int) -> void = print(x)

arr: array int = [1,2,3,4,5]
printing(arr)

arr2: array int = 5 + arr
printing(arr2)

arr3: array int = arr - 2
printing(arr3)

arr4: array int = 2 * arr
printing(arr4)

arr5: array int = arr/2
printing(arr5)
\end{lstlisting}

\subsubsection{tests/test-array-concat.out}

\begin{lstlisting}

1
2
3
4
\end{lstlisting}

\subsubsection{tests/test-array-concat.sos}

\begin{lstlisting}

printing: (x: int) -> void = print(x)

x: array int = [1, 2]
y: array int = [3, 4]

z: array int = x @ y

printing(z)
\end{lstlisting}

\subsubsection{tests/test-array-construction.out}

\begin{lstlisting}

1
2.4
0
1
2
1
3
1
\end{lstlisting}

\subsubsection{tests/test-array-construction.sos}

\begin{lstlisting}

/*int array*/
arrr: array int = [1]
print(arrr[0]) //1

/*float array*/
arrr2: array float = [1.2, 2.4]
printf(arrr2[1]) //2.4

/*bool array*/
arrr3: array bool = [false]
print(arrr3[0]) //False

/*struct array in test-array-of-struct.sos*/

/*nested arrays*/
nested: array array int = [[1], [2,3], [4,5,6]]
print(nested[0][0]) //1

/*func arrays*/
add1: (k: int) -> int = k+1
minus1: (k: int) -> int = k-1
f1: func int -> int = add1
f2: func int -> int = minus1

f_usage: (i: int, f: func int -> int) -> int =
    f(i)
arr_name: array func int -> int = [f1, f2]
print(f_usage(1, arr_name[0]))

/*construction with automatic type conversion*/
printing: (x: int) -> void = print(x)
a: array int = [1, 3.4, true]
printing(a)
\end{lstlisting}

\subsubsection{tests/test-array-iteration.out}

\begin{lstlisting}

1
4
3
6
3
8\end{lstlisting}

\subsubsection{tests/test-array-iteration.sos}

\begin{lstlisting}

x: int = 1
y: int = 3
z: int = 8

many_x: array int = [x, 4, y, 6, 3, z]

printing: (x: int) -> void = print(x)

printing(many_x)
\end{lstlisting}

\subsubsection{tests/test-array-of-struct.out}

\begin{lstlisting}

5
10
1
6
1
2
\end{lstlisting}

\subsubsection{tests/test-array-of-struct.sos}

\begin{lstlisting}

//The structs within array of structs only need to have the same property types

struct x = {i: int}
struct y = {i: int}

distance1_x: x = {5}
distance2_x: x = {6}
distance1_y: y = {1}

b: array x = [distance1_x, {10}, distance1_y]

print(b[0].i) //5
print(b[1].i) //10
print(b[2].i) //1

b[0] = distance2_x
print(b[0].i) //6

b[0] = distance1_y
print(b[0].i) //1

b[0].i = 2
print(b[0].i) //2
\end{lstlisting}

\subsubsection{tests/test-assign.out}

\begin{lstlisting}

5
6
1
2.5
1
0
1
2
4
4.4
1
3.3
5
\end{lstlisting}

\subsubsection{tests/test-assign.sos}

\begin{lstlisting}

//int
a: int = 5
print(a) //5
a = 6
print(a) //6


//float
b: float = 1.0
printf(b) //1.0
b = 2.5
printf(b) //2.5


//bool
c: bool = true
print(c) //True
c = false
print(c) //False


//array
d: array int = [1, 2]
print(d[0]) //1
temp: int = d[1]
d[1] = d[0]
d[0] = temp
print(d[0]) //2


//struct
struct point = {x: float, y: float}
struct e = {v: int, w: float, x: bool, y: point, z: array int }

es: e = {4, 4.4, true, {3.3, 4.5}, [5, 6, 7]}
print(es.v) //4
printf(es.w) //4.4
print(es.x) //True
printf(es.y.x) //3.3
print(es.z[0]) //5
\end{lstlisting}

\subsubsection{tests/test-associativity.out}

\begin{lstlisting}

1
2
2
1
0
1
1
1
3
2
2
50
\end{lstlisting}

\subsubsection{tests/test-associativity.sos}

\begin{lstlisting}

/* 
right associative: =, !, **
left associative: ., *, /, %, +, -, @, of, (comparison), (boolean), ",", ;
*/

/*right*/
//=
a: int = 1
print(a) //1
b: int = a = 2 //b = (a=2) = 2
print(a) //2
print(b) //2

//!
b: bool = true
print(b) //1
print(!b) //0
print(!!b) //1

//left
print(5 - 3 - 1) //2-1 = 1
print((5 - 3) - 1) //2-1 = 1
print(5 - (3 - 1)) //5-2=3

print(100/10/5) // 10/5 = 2
print((100/10)/5) //10/5 = 2
print(100/(10/5)) //100/2 = 50
\end{lstlisting}

\subsubsection{tests/test-conditional.out}

\begin{lstlisting}

1
0
\end{lstlisting}

\subsubsection{tests/test-conditional.sos}

\begin{lstlisting}

x: int = 1

//if then
y: int = if x == 1 then 1 else 0
print(y) //1

//else
z: int = if y != 1 then 1 else 0
print(z) //0
\end{lstlisting}

\subsubsection{tests/test-derived-comparison.out}

\begin{lstlisting}

1
0
1
1
0
\end{lstlisting}

\subsubsection{tests/test-derived-comparison.sos}

\begin{lstlisting}

/* structs, not arrays, allows for comparions */

//comparison of unnamed structs
x: bool = {1,2} == {1,2}
print(x) //true
x = {1, 3} == {2, 4}
print(x)

//comparison of named structs
struct point = {x: float, y: float}
p1: point = {1.0, 2.0}
p2: point = {1.0, 2.0}
p3: point = {0.0, 0.0}
x = p1 == p2
print(x) //true
x = p1 == p2
print(x) //true
x = p1 == p3
print(x) //false


\end{lstlisting}

\subsubsection{tests/test-dot-product.out}

\begin{lstlisting}

11
\end{lstlisting}

\subsubsection{tests/test-dot-product.sos}

\begin{lstlisting}

struct point = {x: int, y: int}

p1: point = {1, 2}
p2: point = {3, 4}

dotted: int = p1*p2
print(dotted) //1*3 + 2*4 = 11
\end{lstlisting}

\subsubsection{tests/test-fibb.out}

\begin{lstlisting}

5
\end{lstlisting}

\subsubsection{tests/test-fibb.sos}

\begin{lstlisting}

fib: (n: int) -> int = 
if n <= 1
then n
else fib(n-1) + fib(n-2)

x: int = 5
x = fib(5)
print(x) //5
\end{lstlisting}

\subsubsection{tests/test-func-of-struct.out}

\begin{lstlisting}

5
\end{lstlisting}

\subsubsection{tests/test-func-of-struct.sos}

\begin{lstlisting}

//predefining struct in outer scope
struct normal = {field: int}

func1: (x: int) -> normal = 
    temp: normal  = {x}; temp

a: normal = func1(5)
print(a.field) //5
\end{lstlisting}

\subsubsection{tests/test-func1.out}

\begin{lstlisting}

5
\end{lstlisting}

\subsubsection{tests/test-func1.sos}

\begin{lstlisting}

//basic func, no parameters

foo: () -> int= 5

print(foo()) //5
\end{lstlisting}

\subsubsection{tests/test-func2.out}

\begin{lstlisting}

5
\end{lstlisting}

\subsubsection{tests/test-func2.sos}

\begin{lstlisting}

//function with parameter

foo: (x: int) -> int = 5

a: int = foo(1)
print(a) //5
\end{lstlisting}

\subsubsection{tests/test-func3.out}

\begin{lstlisting}

5
\end{lstlisting}

\subsubsection{tests/test-func3.sos}

\begin{lstlisting}

//function with body dependent on parameter

foo: (x: int) -> int = x

a: int = foo(5)
print(a) //5
\end{lstlisting}

\subsubsection{tests/test-func4.out}

\begin{lstlisting}

1
1
\end{lstlisting}

\subsubsection{tests/test-func4.sos}

\begin{lstlisting}

//nested function application

first: () -> int = 1
second: (x: int) -> int = x

a: int = second(first())
print(a) //1

b: int = second(second(second(first())))
print(b) //1
\end{lstlisting}

\subsubsection{tests/test-func5.out}

\begin{lstlisting}

5
\end{lstlisting}

\subsubsection{tests/test-func5.sos}

\begin{lstlisting}

//multiple parameters

add: (x: int, y: int) -> int = x+y

var: int = add(2,3)
print(var) //8
\end{lstlisting}

\subsubsection{tests/test-func6.out}

\begin{lstlisting}

0
0
\end{lstlisting}

\subsubsection{tests/test-func6.sos}

\begin{lstlisting}

//more complex function bodies: conditionals

foo: (x: int) -> int = 
    if x == 0
    then x
    else (y: int = x); y-x

a: int = foo(0)
b: int = foo(5)
print(a) //0
print(b) //0
\end{lstlisting}

\subsubsection{tests/test-func7.out}

\begin{lstlisting}

5
\end{lstlisting}

\subsubsection{tests/test-func7.sos}

\begin{lstlisting}

//using func keyword to turn a function into a variable

bar: () -> int = 5
baruser: (f: func  -> int) -> int = f()

barvar: func -> int = bar
print(baruser(barvar))
\end{lstlisting}

\subsubsection{tests/test-helloworld.out}

\begin{lstlisting}

5
startRendering...
endRendering...
startRendering...
endRendering...
\end{lstlisting}

\subsubsection{tests/test-helloworld.sos}

\begin{lstlisting}

import renderer.sos

a: int = 5
print(a)

p1: point = {-0.9, -0.9}
p2: point = {-0.9, -0.7}
p3: point = {-0.7, -0.7}
p4: point = {-0.7, -0.9}
point_arr : path = [p1, p2, p3, p4]

c1 : color = {255.0, 0.0, 0.0, 0.8}
c2 : color = {0.0, 255.0, 0.0, 0.8}
c3 : color = {0.0, 0.0, 255.0, 0.8}
c4 : color = {100.0, 100.0, 0.0, 0.8}
color_arr : colors = [c1, c2, c3, c4]

canvas1 : canvas = {400, 400, 0}

startCanvas(canvas1)
drawShape(point_arr, color_arr, 0, 1)
endCanvas(canvas1)

canvas2 : canvas = {400, 400, 1}

startCanvas(canvas2)
drawShape(point_arr, color_arr, 0, 1)
endCanvas(canvas2)
\end{lstlisting}

\subsubsection{tests/test-if.out}

\begin{lstlisting}

1
0
\end{lstlisting}

\subsubsection{tests/test-if.sos}

\begin{lstlisting}

x: int = 1

//catch if-then
y: int = 
if (x == 1)
then 1
else 0

print(y) //1

//catch else
z: int = 
if (y != 1)
then 1
else 0

print(z) //0

\end{lstlisting}

\subsubsection{tests/test-import-protection.out}

\begin{lstlisting}

0
\end{lstlisting}

\subsubsection{tests/test-import-protection.sos}

\begin{lstlisting}

import point.sos
import point.sos 
p1: point = {0.0, 0.0} 
printf(p1.x)

/* can duplicate a file import.
Usage example: dragon.sos imports renderer.sos and transform.sos, and both import shape.sos
*/
\end{lstlisting}

\subsubsection{tests/test-import.out}

\begin{lstlisting}

0
\end{lstlisting}

\subsubsection{tests/test-import.sos}

\begin{lstlisting}

import point.sos

p1: point = {0.0, 0.0}

printf(p1.x)
\end{lstlisting}

\subsubsection{tests/test-logical-operators.out}

\begin{lstlisting}

1
1
0
0
1
1
0
0
1
\end{lstlisting}

\subsubsection{tests/test-logical-operators.sos}

\begin{lstlisting}

/* logical operators: AND, OR, NOT  */

t: bool = true
f: bool = false


print(t && t)


//and
and_ans: bool = t && t
print(and_ans)             //T AND T = T
and_ans = t && f
print(and_ans)             //T AND F = F
and_ans = f && f
print(and_ans)             //F AND F = F

//or
or_ans: bool = t || f
print(or_ans)               //T OR T = T
or_ans = t || f
print(or_ans)               //T OR F = T
or_ans = f || f
print(or_ans)               //F OR F = F

//not
not_ans: bool = !t
print(not_ans)              //NOT TRUE = F
not_ans = !f
print(not_ans)              //NOT FALSE = T
\end{lstlisting}

\subsubsection{tests/test-memory.out}

\begin{lstlisting}

5
\end{lstlisting}

\subsubsection{tests/test-memory.sos}

\begin{lstlisting}

struct point = {x: float, y: float}

add: (x: point, y: point) -> point = 
    x+y
dotafteradd: (x: point, y: point) -> float =
    temp: point = add(x, y);
    ret: float = temp*temp;
    free(temp);
    ret

print(dotafteradd({0.0, 1.0}, {1.0, 1.0})) //{1, 2}*{1, 2} = 1+4 = 5
\end{lstlisting}

\subsubsection{tests/test-precedence.out}

\begin{lstlisting}

16
20
\end{lstlisting}

\subsubsection{tests/test-precedence.sos}

\begin{lstlisting}

print(5*3+1) //16
print(5*(3+1)) //20
\end{lstlisting}

\subsubsection{tests/test-primitive-arith-operators.out}

\begin{lstlisting}

3
-1
2
0
2
3.2
1
2.31
1.90909
\end{lstlisting}

\subsubsection{tests/test-primitive-arith-operators.sos}

\begin{lstlisting}

/* arithmetic operators on primitives int, float */

add: (x: int, y: int) -> int = x + y
sub: (x: int, y: int) -> int = x - y
mult: (x: int, y: int) -> int = x * y
div: (x: int, y: int) -> int = x/y
mod: (x: int, y: int) -> int = x%y

addf: (x: float, y: float) -> float = x + y 
subf: (x: float, y: float) -> float = x - y 
multf:(x: float, y: float) -> float = x * y 
divf: (x: float, y: float) -> float = x/y 

x: int = add(1,2)
print(x) //3
x = sub(1,2)
print(x) //-1
x = mult(1,2)
print(x) //2
x = div(1,2)
print(x) //0
x = mod(10,8)
print(x) //2

y: float = addf(1.1, 2.1)
printf(y) //3.2
y = subf(2.1, 1.1)
printf(y) //1.0
y = multf(1.1, 2.1)
printf(y) //2.31
y = divf(2.1, 1.1)
printf(y) //1.9090...
\end{lstlisting}

\subsubsection{tests/test-primitive-comparison.out}

\begin{lstlisting}

1
1
0
1
1
1
1
1
1
1
0
1
1
1
1
1
\end{lstlisting}

\subsubsection{tests/test-primitive-comparison.sos}

\begin{lstlisting}

/* 
==, !=, <, <=, >, >=
primitives are passed by value
*/

/*int*/
a: int = 5
b: int = 5
c: int = 7

out1: bool  = a == a
print(out1) //True

out2: bool  = a == b
print(out2) //True

out3: bool = a == c
print(out3) //False


x: bool = 5 != 6
print(x)

x = 5 <= 6
print(x)

x = 6 >= 6
print(x)

x = 6 > 5
print(x)

x = 5 < 6
print(x)



/*float*/
d: float = 1.2
e: float = 1.2
f: float = 2.3

out4: bool = d == d
print(out4) //True

out5: bool = d == e
print(out5) //True

out6: bool = e == f
print(out6) //False


x = 5.0 != 6.0
print(x)

x = 5.0 <= 6.0
print(x)

x = 6.0 >= 6.0
print(x)

x = 6.0 > 5.0 
print(x)

x = 5.0 < 6.0
print(x)

\end{lstlisting}

\subsubsection{tests/test-print.out}

\begin{lstlisting}

1
1
1.1
1
\end{lstlisting}

\subsubsection{tests/test-print.sos}

\begin{lstlisting}

/* Int: print */
x: int = 1
print(1) //1
print(x) //1

/* Float: printf */
y: float = 1.0
printf(1.1) //1.1
printf(y) //1
\end{lstlisting}

\subsubsection{tests/test-recursion.out}

\begin{lstlisting}

15
\end{lstlisting}

\subsubsection{tests/test-recursion.sos}

\begin{lstlisting}

add: (n: int) -> int = 
if n == 0
then 0
else add(n-1) + n

x: int = 5
x = add(x)
print(x) //5+4+3+2+1+0=15

\end{lstlisting}

\subsubsection{tests/test-scoping-if.out}

\begin{lstlisting}

1
1
3
0
4
\end{lstlisting}

\subsubsection{tests/test-scoping-if.sos}

\begin{lstlisting}

y: int = 1
x: int = 1

print(x)
print(y)

x = 
if 1==1
then x = 3; y = 0; print(x); print(y); 4
else 0

print(x) //4
\end{lstlisting}

\subsubsection{tests/test-scoping.out}

\begin{lstlisting}

6
12
6
\end{lstlisting}

\subsubsection{tests/test-scoping.sos}

\begin{lstlisting}

//global x vs x in function

double: (x: int) -> int = x+x

x: int = double(3)
print(x) //6

x = double(x)
print(x) //12

func2: () -> int = x: int = 3; x+x

x = func2()
print(x) //6
\end{lstlisting}

\subsubsection{tests/test-struct-access.out}

\begin{lstlisting}

2
3
\end{lstlisting}

\subsubsection{tests/test-struct-access.sos}

\begin{lstlisting}

//construction
struct person = {identifier: int, age: int, isMarried: bool}
tom: person = {2, 20, false}

//accessing struct field
tom_id: int = tom.identifier
print(tom_id) //2

//changing struct field
tom.identifier=3
tom_id = tom.identifier
print(tom_id) //3

\end{lstlisting}

\subsubsection{tests/test-struct-arith-diff-names.out}

\begin{lstlisting}

4
6
\end{lstlisting}

\subsubsection{tests/test-struct-arith-diff-names.sos}

\begin{lstlisting}

struct s1 = {a: int, b: int}

struct s2 = {c: int, d: int}

first: s1 = {1, 2}

second: s2 = {3, 4}

first = first + second

print(first.a)
print(first.b)
\end{lstlisting}

\subsubsection{tests/test-struct-construction.out}

\begin{lstlisting}

2
20
\end{lstlisting}

\subsubsection{tests/test-struct-construction.sos}

\begin{lstlisting}

//construction
struct person = {identifier: int, age: int}

p1: person = {2, 20}
print(p1.identifier) //2
print(p1.age) //20
\end{lstlisting}

\subsubsection{tests/test-struct-dot-assoc.out}

\begin{lstlisting}

2
\end{lstlisting}

\subsubsection{tests/test-struct-dot-assoc.sos}

\begin{lstlisting}

struct more = {field1: int}

struct stuff = {field1: int, field2: more}

temp1: more = {2}
temp2: stuff = {1, temp1}

ans: int = temp2.field2.field1
print(ans) //2
\end{lstlisting}

\subsubsection{tests/test-struct-dot-product.out}

\begin{lstlisting}

11
\end{lstlisting}

\subsubsection{tests/test-struct-dot-product.sos}

\begin{lstlisting}

struct s1 = {a: int, b: int}

first: s1 = {1, 2}

second: s1 = {3, 4}

dotted: int = first*second
print(dotted)
\end{lstlisting}

\subsubsection{tests/test-struct-of-array.out}

\begin{lstlisting}

3
\end{lstlisting}

\subsubsection{tests/test-struct-of-array.sos}

\begin{lstlisting}

struct first = {arr1: array int}

struct second = {arr2: array first}

struct third = {arr3: array second}

s1: first = {[1,2]}
s1_2: first = {[3,4]}
s2: second = {[s1, s1_2]}
s2_2: second = {[{[-1, 0]}, {[-2, -1]}]}
s1: third = {[s2, s2_2]}

x: int = s1.arr3[0].arr2[1].arr1[0]
print(x) //3
\end{lstlisting}

\subsubsection{tests/test-struct-of-same-types.out}

\begin{lstlisting}

-2.5
-4
1
2
\end{lstlisting}

\subsubsection{tests/test-struct-of-same-types.sos}

\begin{lstlisting}

struct point = {x: float, y: float}
struct vector = {a: float, b: float}


p: point = {-2.5, -4.0}
v: vector = p
printf(v.a) //-2.5
printf(v.b) //-4.0

v2: vector = {1.0, 2.0}
p2: point = v2
printf(p2.x) //1.0
printf(p2.y) //2.0
\end{lstlisting}

\subsubsection{tests/test-struct-of-struct.out}

\begin{lstlisting}

2
\end{lstlisting}

\subsubsection{tests/test-struct-of-struct.sos}

\begin{lstlisting}

//predefining field struct in outer scope
struct a = {field: int}
struct b = { 
    field: int,
    field2: a}

test1: b = {1, temp: a = {2}; temp}
print(test1.field2.field)

/*
//field struct defined in outer struct
struct c = {
    int field,
    struct d = {int field} field2
}

c test2 = {1, d temp2 = {2}; temp2}
int temp2 = test2.field2.field
*/
\end{lstlisting}

\subsubsection{tests/test-struct-scaling.out}

\begin{lstlisting}

6
-9
1
-1
\end{lstlisting}

\subsubsection{tests/test-struct-scaling.sos}

\begin{lstlisting}

struct point = {a: int, b: int}

p1: point = {2, -3}

p_mult: point = 3*p1

print(p_mult.a) //6
print(p_mult.b) //-9

p_div: point = p1/2

print(p_div.a) //1
print(p_div.b) //-1

\end{lstlisting}

\subsubsection{tests/test-type-conversions-comp.out}

\begin{lstlisting}

1
1
1
\end{lstlisting}

\subsubsection{tests/test-type-conversions-comp.sos}

\begin{lstlisting}

/*float over int*/
b: bool = 3.5 > 3
print(b) //True

b = 3 < 3.5
print(b) //True

/*int over bool*/
b = 2 > true
print(b) //True
\end{lstlisting}

\subsubsection{tests/test-type-conversions-func.out}

\begin{lstlisting}

4
\end{lstlisting}

\subsubsection{tests/test-type-conversions-func.sos}

\begin{lstlisting}

add_int: (x: int, y: int) -> int = x + y

x: float = 1.0
y: float = 3.0

sum: int = add_int(x, y)
print(sum)
\end{lstlisting}

\subsubsection{tests/test-type-conversions.out}

\begin{lstlisting}

10
1
1
0
0
0
0
2
-2
1
1
1
0
4
2
3
\end{lstlisting}

\subsubsection{tests/test-type-conversions.sos}

\begin{lstlisting}

/*bool to int*/
//Type conversion during variable assignment
x: bool = true
y: int = x
y = x + 9
print(y) //10

//Type conversion during function application
sum: (x: int, y: int) -> int = x + y
summed: int = sum(true, false)
print(summed) //1


/*int to bool*/
//Type conversion during variable assignment
m: int = 1
n: int = 0
o: bool = m 
p: bool = n
print(o) //True
print(p) //False

//Type conversion based on unary operator
p = !1
print(p) //False

//Type conversion during function application
not: (x: bool) -> bool = !x
notted: bool  = not(1)
print(notted) //False


/*float to int (via truncation) */
a: float = -0.5
b: int = a
print(b) //0

a = 2.9
b = a
print(b) //2

a = -2.9
b = a
print(b) //-2


/*float to bool: anything other than 0.0 is true */
b: bool = 1.1 //1
print(b)
b = -1.1 //1
print(b)
b = 0.1
print(b) //1
b = 0.0
print(b) // 0

//Type conversion during function application (use sum() on line 15)
summed2: int = sum(1.2, 3.5)
print(summed2) //4


/*int to float (via injection) */
c: int = 2
d: float = c
printf(d) //2.0

//Type conversion during function application
floatsum: (x: float, y: float) -> float = x + y
e: float = floatsum(1, 2)
printf(e) //3.0
\end{lstlisting}

\subsubsection{tests/test-unop.out}

\begin{lstlisting}

5
-5
-3
5
-5
2
1
0
0
\end{lstlisting}

\subsubsection{tests/test-unop.sos}

\begin{lstlisting}

/*negation of int and float: neg*/
//int
a: int = 5
b: int = -a
print(a) //5
print(b) //-5
print(-1+-2) //-3

//float
c: float = 5.0
d: float = -c
printf(c) //5.0
printf(d) //-5.0
printf(-1.0*-2.0) //2.0


/*negation of bool: not*/
e: bool = true
f: bool = !e
print(e) //True
print(f) //False
print(!true) //False
\end{lstlisting}

\subsubsection{tests/test-value-access-modifier.out}

\begin{lstlisting}

255
\end{lstlisting}

\subsubsection{tests/test-value-access-modifier.sos}

\begin{lstlisting}

import color.sos
printf($(c : color = {0.0, 255.0, 255.0, 0.8}).g)\end{lstlisting}

\subsubsection{tests/test-var-assign-as-expr.out}

\begin{lstlisting}

6
6
\end{lstlisting}

\subsubsection{tests/test-var-assign-as-expr.sos}

\begin{lstlisting}

a: int = 2

print(a = 3 + 3) //6
print(a) //6
\end{lstlisting}

\subsubsection{tests/test-vardecl-as-expr.out}

\begin{lstlisting}

10
\end{lstlisting}

\subsubsection{tests/test-vardecl-as-expr.sos}

\begin{lstlisting}

//variable declaration for x is an expression for assignment of y
y: int = 5
y= (x: int = 10)

print(y) //10
\end{lstlisting}

\subsubsection{tests/test-vardecl-twice.out}

\begin{lstlisting}

5
1
\end{lstlisting}

\subsubsection{tests/test-vardecl-twice.sos}

\begin{lstlisting}

/* SOS allows variables to be redeclared to different types */

a: int = 5
print(a) //5

a: bool = true
print(a) //1
\end{lstlisting}

\subsubsection{tests/test-vardecl.out}

\begin{lstlisting}

0
0
1
\end{lstlisting}

\subsubsection{tests/test-vardecl.sos}

\begin{lstlisting}

/*variable definitions must have a type id and an expression.
This test case is for primitives only. Reference type definitions
are tested in "construction" test cases.*/

x: int = 0
print(x)

y: float = 0.0
printf(y)

z: bool = true
print(z)
\end{lstlisting}

\subsubsection{tests/test-vardecl2.out}

\begin{lstlisting}

0
0
\end{lstlisting}

\subsubsection{tests/test-vardecl2.sos}

\begin{lstlisting}

/*variables must start with a letters, then can have any
number of letters, underscores, and numbers*/

a_d1: int = 0
a000___: float = 0.0
print(a_d1)
print(a000___)
\end{lstlisting}

\subsection{Example LL Outputs}

\subsubsection{helloworld.ll}
{\small	
\colorbox{green!30}{test-helloworld.ll}
\begin{lstlisting}
; ModuleID = 'SOS'
source_filename = "SOS"

declare i32 @printf(i8*, ...)

define i32 @main() {
entry:
  %p1 = alloca { float, float }*
  %malloccall = tail call i8* @malloc(i32 trunc (i64 mul nuw (i64 
    ptrtoint (float* getelementptr (float, float* null, i32 1) to 
    i64), i64 2) to i32))
  %anon = bitcast i8* %malloccall to { float, float }*
  %fieldaddr = getelementptr { float, float }, { float, float }* 
    %anon, i32 0, i32 0
  store float -5.000000e-01, float* %fieldaddr
  %fieldaddr1 = getelementptr { float, float }, { float, float }* 
    %anon, i32 0, i32 1
  store float -5.000000e-01, float* %fieldaddr1
  store { float, float }* %anon, { float, float }** %p1
  %p2 = alloca { float, float }*
  %malloccall2 = tail call i8* @malloc(i32 trunc (i64 mul nuw (i64 
    ptrtoint (float* getelementptr (float, float* null, i32 1) to 
    i64), i64 2) to i32))
  %anon3 = bitcast i8* %malloccall2 to { float, float }*
  %fieldaddr4 = getelementptr { float, float }, { float, float }* 
    %anon3, i32 0, i32 0
  store float -5.000000e-01, float* %fieldaddr4
  %fieldaddr5 = getelementptr { float, float }, { float, float }* 
    %anon3, i32 0, i32 1
  store float 5.000000e-01, float* %fieldaddr5
  store { float, float }* %anon3, { float, float }** %p2
  %p3 = alloca { float, float }*
  %malloccall6 = tail call i8* @malloc(i32 trunc (i64 mul nuw (i64 
    ptrtoint (float* getelementptr (float, float* null, i32 1) to 
    i64), i64 2) to i32))
  %anon7 = bitcast i8* %malloccall6 to { float, float }*
  %fieldaddr8 = getelementptr { float, float }, { float, float }* 
    %anon7, i32 0, i32 0
  store float 5.000000e-01, float* %fieldaddr8
  %fieldaddr9 = getelementptr { float, float }, { float, float }* 
    %anon7, i32 0, i32 1
  store float 5.000000e-01, float* %fieldaddr9
  store { float, float }* %anon7, { float, float }** %p3
  %p4 = alloca { float, float }*
  %malloccall10 = tail call i8* @malloc(i32 trunc (i64 mul nuw 
    (i64 ptrtoint (float* getelementptr (float, float* null, 
    i32 1) to i64), i64 2) to i32))
  %anon11 = bitcast i8* %malloccall10 to { float, float }*
  %fieldaddr12 = getelementptr { float, float }, { float, float }* 
    %anon11, i32 0, i32 0
  store float 5.000000e-01, float* %fieldaddr12
  %fieldaddr13 = getelementptr { float, float }, { float, float }* 
    %anon11, i32 0, i32 1
  store float -5.000000e-01, float* %fieldaddr13
  store { float, float }* %anon11, { float, float }** %p4
  %point_arr = alloca { { float, float }**, i32 }*
  %malloccall14 = tail call i8* @malloc(i32 mul (i32 ptrtoint (i1** 
    getelementptr (i1*, i1** null, i32 1) to i32), i32 4))
  %arrdata = bitcast i8* %malloccall14 to { float, float }**
  %p115 = load { float, float }*, { float, float }** %p1
  %storeref = getelementptr { float, float }*, { float, float }** 
    %arrdata, i32 0
  store { float, float }* %p115, { float, float }** %storeref
  %p216 = load { float, float }*, { float, float }** %p2
  %storeref17 = getelementptr { float, float }*, { float, float }** 
    %arrdata, i32 1
  store { float, float }* %p216, { float, float }** %storeref17
  %p318 = load { float, float }*, { float, float }** %p3
  %storeref19 = getelementptr { float, float }*, { float, float }** 
    %arrdata, i32 2
  store { float, float }* %p318, { float, float }** %storeref19
  %p420 = load { float, float }*, { float, float }** %p4
  %storeref21 = getelementptr { float, float }*, { float, float }** 
    %arrdata, i32 3
  store { float, float }* %p420, { float, float }** %storeref21
  %malloccall22 = tail call i8* @malloc(i32 ptrtoint ({ { float, float 
    }**, i32 }* getelementptr ({ { float, float }**, i32 }, { { float, 
    float }**, i32 }* null, i32 1) to i32))
  %arr = bitcast i8* %malloccall22 to { { float, float }**, i32 }*
  %arrdata23 = getelementptr { { float, float }**, i32 }, { { float, 
    float }**, i32 }* %arr, i32 0, i32 0
  %arrlen = getelementptr { { float, float }**, i32 }, { { float, float }**, 
    i32 }* %arr, i32 0, i32 1
  store { float, float }** %arrdata, { float, float }*** %arrdata23
  store i32 4, i32* %arrlen
  store { { float, float }**, i32 }* %arr, { { float, float }**, i32 }** 
    %point_arr
  %c1 = alloca { float, float, float, float }*
  %malloccall24 = tail call i8* @malloc(i32 trunc (i64 mul nuw (i64 ptrtoint 
    (float* getelementptr (float, float* null, i32 1) to i64), i64 4) to i32))
  %anon25 = bitcast i8* %malloccall24 to { float, float, float, float }*
  %fieldaddr26 = getelementptr { float, float, float, float }, { float, float, 
    float, float }* %anon25, i32 0, i32 0
  store float 2.550000e+02, float* %fieldaddr26
  %fieldaddr27 = getelementptr { float, float, float, float }, { float, float, 
    float, float }* %anon25, i32 0, i32 1
  store float 0.000000e+00, float* %fieldaddr27
  %fieldaddr28 = getelementptr { float, float, float, float }, { float, float, 
    float, float }* %anon25, i32 0, i32 2
  store float 0.000000e+00, float* %fieldaddr28
  %fieldaddr29 = getelementptr { float, float, float, float }, { float, float, 
    float, float }* %anon25, i32 0, i32 3
  store float 0x3FE99999A0000000, float* %fieldaddr29
  store { float, float, float, float }* %anon25, { float, float, float, float 
    }** %c1
  %c2 = alloca { float, float, float, float }*
  %malloccall30 = tail call i8* @malloc(i32 trunc (i64 mul nuw (i64 ptrtoint 
    (float* getelementptr (float, float* null, i32 1) to i64), i64 4) to i32))
  %anon31 = bitcast i8* %malloccall30 to { float, float, float, float }*
  %fieldaddr32 = getelementptr { float, float, float, float }, { float, float, 
    float, float }* %anon31, i32 0, i32 0
  store float 0.000000e+00, float* %fieldaddr32
  %fieldaddr33 = getelementptr { float, float, float, float }, { float, float, 
    float, float }* %anon31, i32 0, i32 1
  store float 2.550000e+02, float* %fieldaddr33
  %fieldaddr34 = getelementptr { float, float, float, float }, { float, float, 
    float, float }* %anon31, i32 0, i32 2
  store float 0.000000e+00, float* %fieldaddr34
  %fieldaddr35 = getelementptr { float, float, float, float }, { float, float, 
    float, float }* %anon31, i32 0, i32 3
  store float 0x3FE99999A0000000, float* %fieldaddr35
  store { float, float, float, float }* %anon31, { float, float, float, float 
    }** %c2
  %c3 = alloca { float, float, float, float }*
  %malloccall36 = tail call i8* @malloc(i32 trunc (i64 mul nuw (i64 ptrtoint 
    (float* getelementptr (float, float* null, i32 1) to i64), i64 4) to i32))
  %anon37 = bitcast i8* %malloccall36 to { float, float, float, float }*
  %fieldaddr38 = getelementptr { float, float, float, float }, { float, float, 
    float, float }* %anon37, i32 0, i32 0
  store float 0.000000e+00, float* %fieldaddr38
  %fieldaddr39 = getelementptr { float, float, float, float }, { float, float, 
    float, float }* %anon37, i32 0, i32 1
  store float 0.000000e+00, float* %fieldaddr39
  %fieldaddr40 = getelementptr { float, float, float, float }, { float, float, 
    float, float }* %anon37, i32 0, i32 2
  store float 2.550000e+02, float* %fieldaddr40
  %fieldaddr41 = getelementptr { float, float, float, float }, { float, float, 
    float, float }* %anon37, i32 0, i32 3
  store float 0x3FE99999A0000000, float* %fieldaddr41
  store { float, float, float, float }* %anon37, { float, float, float, float 
    }** %c3
  %c4 = alloca { float, float, float, float }*
  %malloccall42 = tail call i8* @malloc(i32 trunc (i64 mul nuw (i64 ptrtoint 
    (float* getelementptr (float, float* null, i32 1) to i64), i64 4) to i32))
  %anon43 = bitcast i8* %malloccall42 to { float, float, float, float }*
  %fieldaddr44 = getelementptr { float, float, float, float }, { float, float,
    float, float }* %anon43, i32 0, i32 0
  store float 1.000000e+02, float* %fieldaddr44
  %fieldaddr45 = getelementptr { float, float, float, float }, { float, float, 
    float, float }* %anon43, i32 0, i32 1
  store float 1.000000e+02, float* %fieldaddr45
  %fieldaddr46 = getelementptr { float, float, float, float }, { float, float, 
    float, float }* %anon43, i32 0, i32 2
  store float 0.000000e+00, float* %fieldaddr46
  %fieldaddr47 = getelementptr { float, float, float, float }, { float, float, 
    float, float }* %anon43, i32 0, i32 3
  store float 0x3FE99999A0000000, float* %fieldaddr47
  store { float, float, float, float }* %anon43, { float, float, float, float
    }** %c4
  %color_arr = alloca { { float, float, float, float }**, i32 }*
  %malloccall48 = tail call i8* @malloc(i32 mul (i32 ptrtoint (i1** 
    getelementptr (i1*, i1** null, i32 1) to i32), i32 4))
  %arrdata49 = bitcast i8* %malloccall48 to { float, float, float, float }**
  %c150 = load { float, float, float, float }*, { float, float, float, float
    }** %c1
  %storeref51 = getelementptr { float, float, float, float }*, { float, float, 
    float, float }** %arrdata49, i32 0
  store { float, float, float, float }* %c150, { float, float, float, float
    }** %storeref51
  %c252 = load { float, float, float, float }*, { float, float, float, 
    float }** %c2
  %storeref53 = getelementptr { float, float, float, float }*, { float,
    float, 
    float, float }** %arrdata49, i32 1
  store { float, float, float, float }* %c252, { float, float, float,
    float }** %storeref53
  %c354 = load { float, float, float, float }*, { float, float, float,
    float }** %c3
  %storeref55 = getelementptr { float, float, float, float }*, { float,
  float, float, float }** %arrdata49, i32 2
  store { float, float, float, float }* %c354, { float, float, float,
    float }** %storeref55
  %c456 = load { float, float, float, float }*, { float, float, float,
    float }** %c4
  %storeref57 = getelementptr { float, float, float, float }*, { float,
    float, 
    float, float }** %arrdata49, i32 3
  store { float, float, float, float }* %c456, { float, float, float,
    float }** %storeref57
  %malloccall58 = tail call i8* @malloc(i32 ptrtoint ({ { float, float, 
    float, float }**, i32 }* getelementptr ({ { float, float, float,
    float }**, i32 }, { { float, float, float, float }**, i32 }* null,
    i32 1) to i32))
  %arr59 = bitcast i8* %malloccall58 to { { float, float, float, 
    float }**, i32 }*
  %arrdata60 = getelementptr { { float, float, float, float }**,
    i32 }, { { float, float, float, float }**, i32 }* %arr59,
    i32 0, i32 0
  %arrlen61 = getelementptr { { float, float, float, float }**, i32 }, { 
    { float, float, float, float }**, i32 }* %arr59, i32 0, i32 1
  store { float, float, float, float }** %arrdata49, { float, 
    float, float, float }*** %arrdata60
  store i32 4, i32* %arrlen61
  store { { float, float, float, float }**, i32 }* %arr59, { { float,
    float, float, float }**, i32 }** %color_arr
  %canvas1 = alloca { i32, i32, i32 }*
  %malloccall62 = tail call i8* @malloc(i32 trunc (i64 mul nuw (i64 ptrtoint
    (i32* getelementptr (i32, i32* null, i32 1) to i64), i64 3) to i32))
  %anon63 = bitcast i8* %malloccall62 to { i32, i32, i32 }*
  %fieldaddr64 = getelementptr { i32, i32, i32 }, { i32, i32, i32 }*
    %anon63, i32 0, i32 0
  store i32 400, i32* %fieldaddr64
  %fieldaddr65 = getelementptr { i32, i32, i32 }, { i32, i32, i32 }*
    %anon63, i32 0, i32 1
  store i32 400, i32* %fieldaddr65
  %fieldaddr66 = getelementptr { i32, i32, i32 }, { i32, i32, i32 }*
    %anon63, i32 0, i32 2
  store i32 0, i32* %fieldaddr66
  store { i32, i32, i32 }* %anon63, { i32, i32, i32 }** %canvas1
  %canvas167 = load { i32, i32, i32 }*, { i32, i32, i32 }** %canvas1
  call void @startCanvas({ i32, i32, i32 }* %canvas167)
  %point_arr68 = load { { float, float }**, i32 }*, { { float, float
    }**, i32 }** %point_arr
  %color_arr69 = load { { float, float, float, float }**, i32 }*,
    { { float, float, float, float }**, i32 }** %color_arr
  call void @drawShape({ { float, float }**, i32 }* %point_arr68,
    { { float, float, float, float }**, i32 }* %color_arr69,
    i32 0, i32 1)
  %canvas170 = load { i32, i32, i32 }*, { i32, i32, i32 }** %canvas1
  call void @endCanvas({ i32, i32, i32 }* %canvas170)
  %canvas2 = alloca { i32, i32, i32 }*
  %malloccall71 = tail call i8* @malloc(i32 trunc (i64 mul nuw
    (i64 ptrtoint (i32* getelementptr (i32, i32* null, i32 1)
    to i64), i64 3) to i32))
  %anon72 = bitcast i8* %malloccall71 to { i32, i32, i32 }*
  %fieldaddr73 = getelementptr { i32, i32, i32 }, { i32, i32,
    i32 }* %anon72, i32 0, i32 0
  store i32 400, i32* %fieldaddr73
  %fieldaddr74 = getelementptr { i32, i32, i32 }, { i32, i32,
    i32 }* %anon72, i32 0, i32 1
  store i32 400, i32* %fieldaddr74
  %fieldaddr75 = getelementptr { i32, i32, i32 }, { i32, i32,
    i32 }* %anon72, i32 0, i32 2
  store i32 1, i32* %fieldaddr75
  store { i32, i32, i32 }* %anon72, { i32, i32, i32 }** %canvas2
  %canvas276 = load { i32, i32, i32 }*, { i32, i32, i32 }** %canvas2
  call void @startCanvas({ i32, i32, i32 }* %canvas276)
  %point_arr77 = load { { float, float }**, i32 }*, { { float,
    float }**, i32 }** %point_arr
  %color_arr78 = load { { float, float, float, float }**, i32 }*,
    { { float, float, float, float }**, i32 }** %color_arr
  call void @drawShape({ { float, float }**, i32 }* %point_arr77,
    { { float, float, float, float }**, i32 }* %color_arr78,
    i32 0, i32 1)
  %canvas279 = load { i32, i32, i32 }*, { i32, i32, i32 }** %canvas2
  call void @endCanvas({ i32, i32, i32 }* %canvas279)
  ret i32 0
}

declare float @sqrtf(float)

declare float @sinf(float)

declare float @cosf(float)

declare float @tanf(float)

declare float @asinf(float)

declare float @acosf(float)

declare float @atanf(float)

declare float @toradiansf(float)

declare void @gl_startRendering(i32, i32)

declare void @gl_endRendering(i32, i32, i32)

declare void @gl_drawCurve({ float*, i32 }*, { float*, i32 }*, i32)

declare void @gl_drawShape({ float*, i32 }*, { float*, i32 }*, i32, i32)

declare void @gl_drawPoint({ float*, i32 }*, { float*, i32 }*, i32)

define float @floor(float %x) {
entry:
  %x1 = alloca float
  store float %x, float* %x1
  %z = alloca float
  %y = alloca i32
  %x2 = load float, float* %x1
  %cast = fptosi float %x2 to i32
  store i32 %cast, i32* %y
  %cast3 = sitofp i32 %cast to float
  store float %cast3, float* %z
  %z4 = load float, float* %z
  %x5 = load float, float* %x1
  %tmp = fcmp ole float %z4, %x5
  %if_tmp = alloca float
  br i1 %tmp, label %then, label %else

merge:                                            ; preds = %else, %then
  %if_tmp9 = load float, float* %if_tmp
  ret float %if_tmp9

then:                                             ; preds = %entry
  %z6 = load float, float* %z
  store float %z6, float* %if_tmp
  br label %merge

else:                                             ; preds = %entry
  %z7 = load float, float* %z
  %tmp8 = fsub float %z7, 1.000000e+00
  store float %tmp8, float* %if_tmp
  br label %merge
}

define float @ceil(float %x) {
entry:
  %x1 = alloca float
  store float %x, float* %x1
  %x2 = load float, float* %x1
  %tmp = fneg float %x2
  %fxn_result = call float @floor(float %tmp)
  %tmp3 = fneg float %fxn_result
  ret float %tmp3
}

define float @frac(float %x) {
entry:
  %x1 = alloca float
  store float %x, float* %x1
  %x2 = load float, float* %x1
  %x3 = load float, float* %x1
  %fxn_result = call float @floor(float %x3)
  %tmp = fsub float %x2, %fxn_result
  ret float %tmp
}

define float @max(float %a, float %b) {
entry:
  %a1 = alloca float
  store float %a, float* %a1
  %b2 = alloca float
  store float %b, float* %b2
  %a3 = load float, float* %a1
  %b4 = load float, float* %b2
  %tmp = fcmp olt float %a3, %b4
  %if_tmp = alloca float
  br i1 %tmp, label %then, label %else

merge:                                            ; preds = %else, %then
  %if_tmp7 = load float, float* %if_tmp
  ret float %if_tmp7

then:                                             ; preds = %entry
  %b5 = load float, float* %b2
  store float %b5, float* %if_tmp
  br label %merge

else:                                             ; preds = %entry
  %a6 = load float, float* %a1
  store float %a6, float* %if_tmp
  br label %merge
}

define float @min(float %a, float %b) {
entry:
  %a1 = alloca float
  store float %a, float* %a1
  %b2 = alloca float
  store float %b, float* %b2
  %a3 = load float, float* %a1
  %b4 = load float, float* %b2
  %tmp = fcmp olt float %a3, %b4
  %if_tmp = alloca float
  br i1 %tmp, label %then, label %else

merge:                                            ; preds = %else, %then
  %if_tmp7 = load float, float* %if_tmp
  ret float %if_tmp7

then:                                             ; preds = %entry
  %a5 = load float, float* %a1
  store float %a5, float* %if_tmp
  br label %merge

else:                                             ; preds = %entry
  %b6 = load float, float* %b2
  store float %b6, float* %if_tmp
  br label %merge
}

define float @clamp(float %x, float %m, float %M) {
entry:
  %x1 = alloca float
  store float %x, float* %x1
  %m2 = alloca float
  store float %m, float* %m2
  %M3 = alloca float
  store float %M, float* %M3
  %M4 = load float, float* %M3
  %x5 = load float, float* %x1
  %m6 = load float, float* %m2
  %fxn_result = call float @max(float %x5, float %m6)
  %fxn_result7 = call float @min(float %M4, float %fxn_result)
  ret float %fxn_result7
}

define float @abs(float %x) {
entry:
  %x1 = alloca float
  store float %x, float* %x1
  %x2 = load float, float* %x1
  %tmp = fcmp olt float %x2, 0.000000e+00
  %if_tmp = alloca float
  br i1 %tmp, label %then, label %else

merge:                                            ; preds = %else, %then
  %if_tmp6 = load float, float* %if_tmp
  ret float %if_tmp6

then:                                             ; preds = %entry
  %x3 = load float, float* %x1
  %tmp4 = fneg float %x3
  store float %tmp4, float* %if_tmp
  br label %merge

else:                                             ; preds = %entry
  %x5 = load float, float* %x1
  store float %x5, float* %if_tmp
  br label %merge
}

define float @modf(float %x, float %m) {
entry:
  %x1 = alloca float
  store float %x, float* %x1
  %m2 = alloca float
  store float %m, float* %m2
  %m3 = load float, float* %m2
  %x4 = load float, float* %x1
  %m5 = load float, float* %m2
  %tmp = fdiv float %x4, %m5
  %fxn_result = call float @frac(float %tmp)
  %tmp6 = fmul float %m3, %fxn_result
  ret float %tmp6
}

define float @sin(float %x) {
entry:
  %x1 = alloca float
  store float %x, float* %x1
  %x2 = load float, float* %x1
  %fxn_result = call float @sinf(float %x2)
  ret float %fxn_result
}

define float @cos(float %x) {
entry:
  %x1 = alloca float
  store float %x, float* %x1
  %x2 = load float, float* %x1
  %fxn_result = call float @cosf(float %x2)
  ret float %fxn_result
}

define float @tan(float %x) {
entry:
  %x1 = alloca float
  store float %x, float* %x1
  %x2 = load float, float* %x1
  %fxn_result = call float @tanf(float %x2)
  ret float %fxn_result
}

define float @asin(float %x) {
entry:
  %x1 = alloca float
  store float %x, float* %x1
  %x2 = load float, float* %x1
  %fxn_result = call float @asinf(float %x2)
  ret float %fxn_result
}

define float @acos(float %x) {
entry:
  %x1 = alloca float
  store float %x, float* %x1
  %x2 = load float, float* %x1
  %fxn_result = call float @acosf(float %x2)
  ret float %fxn_result
}

define float @atan(float %x) {
entry:
  %x1 = alloca float
  store float %x, float* %x1
  %x2 = load float, float* %x1
  %fxn_result = call float @atanf(float %x2)
  ret float %fxn_result
}

define float @sqrt(float %x) {
entry:
  %x1 = alloca float
  store float %x, float* %x1
  %x2 = load float, float* %x1
  %fxn_result = call float @sqrtf(float %x2)
  ret float %fxn_result
}

define float @toradians(float %x) {
entry:
  %x1 = alloca float
  store float %x, float* %x1
  %x2 = load float, float* %x1
  %fxn_result = call float @toradiansf(float %x2)
  ret float %fxn_result
}

define float @sqrMagnitude({ float, float }* %p) {
entry:
  %p1 = alloca { float, float }*
  store { float, float }* %p, { float, float }** %p1
  %p2 = load { float, float }*, { float, float }** %p1
  %p3 = load { float, float }*, { float, float }** %p1
  %result = call float @__dotf2({ float, float }* %p2,
    { float, float }* %p3)
  ret float %result
}

define float @__dotf2({ float, float }* %a, { float, float }* %b) {
entry:
  %a1 = alloca { float, float }*
  store { float, float }* %a, { float, float }** %a1
  %a2 = load { float, float }*, { float, float }** %a1
  %b3 = alloca { float, float }*
  store { float, float }* %b, { float, float }** %b3
  %b4 = load { float, float }*, { float, float }** %b3
  %dot = alloca float
  %tmp = alloca float
  store float 0.000000e+00, float* %dot
  %avalref = getelementptr { float, float }, { float,
    float }* %a2, i32 0, i32 0
  %aval = load float, float* %avalref
  %bvalref = getelementptr { float, float }, { float,float
    }* %b4, i32 0, i32 0
  %bval = load float, float* %bvalref
  %tmp5 = fmul float %aval, %bval
  store float %tmp5, float* %tmp
  %tmp6 = load float, float* %tmp
  %res = load float, float* %dot
  %tmp7 = fadd float %tmp6, %res
  store float %tmp7, float* %dot
  %avalref8 = getelementptr { float, float }, { float, float
    }* %a2, i32 0, i32 1
  %aval9 = load float, float* %avalref8
  %bvalref10 = getelementptr { float, float }, { float, float
    }* %b4, i32 0, i32 1
  %bval11 = load float, float* %bvalref10
  %tmp12 = fmul float %aval9, %bval11
  store float %tmp12, float* %tmp
  %tmp13 = load float, float* %tmp
  %res14 = load float, float* %dot
  %tmp15 = fadd float %tmp13, %res14
  store float %tmp15, float* %dot
  %res16 = load float, float* %dot
  ret float %res16
}

define float @magnitude({ float, float }* %p) {
entry:
  %p1 = alloca { float, float }*
  store { float, float }* %p, { float, float }** %p1
  %p2 = load { float, float }*, { float, float }** %p1
  %fxn_result = call float @sqrMagnitude({ float, float }*
    %p2)
  %fxn_result3 = call float @sqrt(float %fxn_result)
  ret float %fxn_result3
}

define float @sqrDistance({ float, float }* %a, { float,
    float }* %b) {
entry:
  %a1 = alloca { float, float }*
  store { float, float }* %a, { float, float }** %a1
  %b2 = alloca { float, float }*
  store { float, float }* %b, { float, float }** %b2
  %p = alloca { float, float }*
  %a3 = load { float, float }*, { float, float }** %a1
  %b4 = load { float, float }*, { float, float }** %b2
  %result = call { float, float }* @__subf2({ float, float
    }* %a3, { float, float }* %b4)
  store { float, float }* %result, { float, float }** %p
  %d = alloca float
  %p5 = load { float, float }*, { float, float }** %p
  %fxn_result = call float @sqrMagnitude({ float, float }* %p5)
  store float %fxn_result, float* %d
  %p6 = load { float, float }*, { float, float }** %p
  %0 = bitcast { float, float }* %p6 to i8*
  tail call void @free(i8* %0)
  %d7 = load float, float* %d
  ret float %d7
}

define { float, float }* @__subf2({ float, float }* %a,
    { float, float }* %b) {
entry:
  %a1 = alloca { float, float }*
  store { float, float }* %a, { float, float }** %a1
  %a2 = load { float, float }*, { float, float }** %a1
  %b3 = alloca { float, float }*
  store { float, float }* %b, { float, float }** %b3
  %b4 = load { float, float }*, { float, float }** %b3
  %malloccall = tail call i8* @malloc(i32 trunc (i64 mul
    nuw (i64 ptrtoint (float* 
    getelementptr (float, float* null, i32 1) to i64), i64 2) to i32))
  %ret = bitcast i8* %malloccall to { float, float }*
  %avalref = getelementptr { float, float }, { float,
    float }* %a2, i32 0, i32 0
  %aval = load float, float* %avalref
  %bvalref = getelementptr { float, float }, { float,
    float }* %b4, i32 0, i32 0
  %bval = load float, float* %bvalref
  %tmp = fsub float %aval, %bval
  %ref = getelementptr { float, float }, { float, float
    }* %ret, i32 0, i32 0
  store float %tmp, float* %ref
  %avalref5 = getelementptr { float, float }, { float,
    float }* %a2, i32 0, i32 1
  %aval6 = load float, float* %avalref5
  %bvalref7 = getelementptr { float, float }, { float,
    float }* %b4, i32 0, i32 1
  %bval8 = load float, float* %bvalref7
  %tmp9 = fsub float %aval6, %bval8
  %ref10 = getelementptr { float, float }, { float, 
    float }* %ret, i32 0, i32 1
  store float %tmp9, float* %ref10
  ret { float, float }* %ret
}

declare noalias i8* @malloc(i32)

declare void @free(i8*)

define float @distance({ float, float }* %a, { float,
    float }* %b) {
entry:
  %a1 = alloca { float, float }*
  store { float, float }* %a, { float, float }** %a1
  %b2 = alloca { float, float }*
  store { float, float }* %b, { float, float }** %b2
  %a3 = load { float, float }*, { float, float }** %a1
  %b4 = load { float, float }*, { float, float }** %b2
  %fxn_result = call float @sqrDistance({ float, float
    }* %a3, { float, float }* %b4)
  %fxn_result5 = call float @sqrt(float %fxn_result)
  ret float %fxn_result5
}

define { float, float }* @copy_point({ float, float }* %p) {
entry:
  %p1 = alloca { float, float }*
  store { float, float }* %p, { float, float }** %p1
  %p2 = load { float, float }*, { float, float }** %p1
  %copied = call { float, float }* @__copy2({ float,
    float }* %p2)
  ret { float, float }* %copied
}

define { float, float }* @__copy2({ float, float }* %to_copy) {
entry:
  %to_copy1 = alloca { float, float }*
  store { float, float }* %to_copy, { float, float
    }** %to_copy1
  %to_copy2 = load { float, float }*, { float, float
    }** %to_copy1
  %malloccall = tail call i8* @malloc(i32 trunc (i64
    mul nuw (i64 ptrtoint (float* 
    getelementptr (float, float* null, i32 1) to i64)
        , i64 2) to i32))
  %struct = bitcast i8* %malloccall to { float, float }*
  %flref = getelementptr { float, float }, { float, 
    float }* %to_copy2, i32 0, i32 0
  %fl = load float, float* %flref
  %ref = getelementptr { float, float }, { float, float
    }* %struct, i32 0, i32 0
  store float %fl, float* %ref
  %flref3 = getelementptr { float, float }, { float, float
    }* %to_copy2, i32 0, i32 1
  %fl4 = load float, float* %flref3
  %ref5 = getelementptr { float, float }, { float, float
    }* %struct, i32 0, i32 1
  store float %fl4, float* %ref5
  ret { float, float }* %struct
}

define void @free_point({ float, float }* %p) {
entry:
  %p1 = alloca { float, float }*
  store { float, float }* %p, { float, float }** %p1
  %p2 = load { float, float }*, { float, float }** %p1
  %0 = bitcast { float, float }* %p2 to i8*
  tail call void @free(i8* %0)
  ret void
}

define { { float, float }**, i32 }* @copy_path({ { float,
    float }**, i32 }* %p) {
entry:
  %p1 = alloca { { float, float }**, i32 }*
  store { { float, float }**, i32 }* %p, { { float, 
    float }**, i32 }** %p1
  %p2 = load { { float, float }**, i32 }*, { { float,
    float }**, i32 }** %p1
  %lenref = getelementptr { { float, float }**, i32 },
    { { float, float }**, i32 }* %p2, i32 0, i32 1
  %len = load i32, i32* %lenref
  %dataref = getelementptr { { float, float }**, i32 },
    { { float, float }**, i32 }* %p2, i32 0, i32 0
  %data = load { float, float }**, { float, float }***
    %dataref
  %mallocsize = mul i32 %len, ptrtoint (i1** getelementptr
    (i1*, i1** null, i32 1) to i32)
  %malloccall = tail call i8* @malloc(i32 %mallocsize)
  %arrdata = bitcast i8* %malloccall to { float, float }**
  %i = alloca i32
  store i32 0, i32* %i
  br label %loop

loop:                                             ; preds = %loop, %entry
  %i3 = load i32, i32* %i
  %elref = getelementptr { float, float }*, { float,
    float }** %data, i32 %i3
  %el = load { float, float }*, { float, float }** %elref
  %fxn_result = call { float, float }* @copy_point({
    float, float }* %el)
  %storeref = getelementptr { float, float }*, { float,
    float }** %arrdata, i32 %i3
  store { float, float }* %fxn_result, { float, float
    }** %storeref
  %i4 = add i32 %i3, 1
  store i32 %i4, i32* %i
  %i5 = load i32, i32* %i
  %tmp = icmp slt i32 %i5, %len
  br i1 %tmp, label %loop, label %continue

continue:                                         ; preds = %loop
  %malloccall6 = tail call i8* @malloc(i32 ptrtoint
    ({ { float, float }**, i32 }* 
    getelementptr ({ { float, float }**, i32 }, { {
        float, float }**, i32 }* null, i32 1) to i32))
  %arr = bitcast i8* %malloccall6 to { { float, float
    }**, i32 }*
  %arrdata7 = getelementptr { { float, float }**, i32
    }, { { float, float }**, i32 }* %arr, i32 0, i32 0
  %arrlen = getelementptr { { float, float }**, i32 },
    { { float, float }**, i32 }* %arr, i32 0, i32 1
  store { float, float }** %arrdata, { float, float
    }*** %arrdata7
  store i32 %len, i32* %arrlen
  ret { { float, float }**, i32 }* %arr
}

define void @free_path({ { float, float }**, i32 }* %p) {
entry:
  %p1 = alloca { { float, float }**, i32 }*
  store { { float, float }**, i32 }* %p, { { float,
    float }**, i32 }** %p1
  %p2 = load { { float, float }**, i32 }*, { {
    float, float }**, i32 }** %p1
  %lenref = getelementptr { { float, float }**,
    i32 }, { { float, float }**, i32 }* %p2, i32 0, i32 1
  %len = load i32, i32* %lenref
  %dataref = getelementptr { { float, float }**,
    i32 }, { { float, float }**, 
    i32 }* %p2, i32 0, i32 0
  %data = load { float, float }**, { float, 
    float }*** %dataref
  %i = alloca i32
  store i32 0, i32* %i
  br label %loop

loop:                                             ; preds = %loop, %entry
  %i3 = load i32, i32* %i
  %elref = getelementptr { float, float }*,
    { float, float }** %data, i32 %i3
  %el = load { float, float }*, { float, 
    float }** %elref
  call void @free_point({ float, float }* %el)
  %i4 = add i32 %i3, 1
  store i32 %i4, i32* %i
  %i5 = load i32, i32* %i
  %tmp = icmp slt i32 %i5, %len
  br i1 %tmp, label %loop, label %continue

continue:                                         ; preds = %loop
  ret void
}

define void @appendhelp_copyin({ { float, float }**,
    i32 }* %in, { { float, float }**, i32 }*
    %from, i32 %i) {
entry:
  %in1 = alloca { { float, float }**, i32 }*
  store { { float, float }**, i32 }* %in, { { 
    float, float }**, i32 }** %in1
  %from2 = alloca { { float, float }**, i32 }*
  store { { float, float }**, i32 }* %from, { {
    float, float }**, i32 }** %from2
  %i3 = alloca i32
  store i32 %i, i32* %i3
  %i4 = load i32, i32* %i3
  %in5 = load { { float, float }**, i32 }*, { { 
    float, float }**, i32 }** %in1
  %lenref = getelementptr { { float, float }**,
    i32 }, { { float, float }**, i32 }* %in5, i32 0,
    i32 1
  %len = load i32, i32* %lenref
  %tmp = icmp slt i32 %i4, %len
  br i1 %tmp, label %then, label %else

merge:                                            ; preds = %else, %then
  ret void

then:                                             ; preds = %entry
  %in6 = load { { float, float }**, i32 }*, { { 
    float, float }**, i32 }** %in1
  %datarefref = getelementptr { { float, float 
    }**, i32 }, { { float, float 
    }**, i32 }* %in6, i32 0, i32 0
  %dataref = load { float, float }**, { float, 
    float }*** %datarefref
  %i7 = load i32, i32* %i3
  %from8 = load { { float, float }**, i32 }*, {
    { float, float }**, i32 }** %from2
  %i9 = load i32, i32* %i3
  %tmp10 = add i32 %i9, 1
  %dataref11 = getelementptr { { float, float }**, 
    i32 }, { { float, float }**, i32 }* %from8, i32 0, i32 0
  %data = load { float, float }**, { float, float }*** %dataref11
  %elref = getelementptr { float, float }*, { float, 
    float }** %data, i32 %tmp10
  %el = load { float, float }*, { float, float }** %elref
  %copied = call { float, float }* @__copy2.1({ float,
    float }* %el)
  %storeref = getelementptr { float, float }*, { float,
    float }** %dataref, i32 %i7
  store { float, float }* %copied, { float, float }** %storeref
  %in12 = load { { float, float }**, i32 }*, { { float,
    float }**, i32 }** %in1
  %from13 = load { { float, float }**, i32 }*, { { float,
    float }**, i32 }** %from2
  %i14 = load i32, i32* %i3
  %tmp15 = add i32 %i14, 1
  call void @appendhelp_copyin({ { float, float }**,
    i32 }* %in12, { { float, float }**, i32 }* %from13, i32 %tmp15)
  br label %merge

else:                                             ; preds = %entry
  br label %merge
}

define { float, float }* @__copy2.1({ float, float }*
    %to_copy) {
entry:
  %to_copy1 = alloca { float, float }*
  store { float, float }* %to_copy, { float, float 
    }** %to_copy1
  %to_copy2 = load { float, float }*, { float, float 
    }** %to_copy1
  %malloccall = tail call i8* @malloc(i32 trunc (i64 
    mul nuw (i64 ptrtoint (float* getelementptr (float, 
    float* null, i32 1) to i64), i64 2) to i32))
  %struct = bitcast i8* %malloccall to { float, float }*
  %flref = getelementptr { float, float }, { float,
    float }* %to_copy2, i32 0, i32 0
  %fl = load float, float* %flref
  %ref = getelementptr { float, float }, { float,
    float }* %struct, i32 0, i32 0
  store float %fl, float* %ref
  %flref3 = getelementptr { float, float }, { 
    float, float }* %to_copy2, i32 0, i32 1
  %fl4 = load float, float* %flref3
  %ref5 = getelementptr { float, float }, { 
    float, float }* %struct, i32 0, i32 1
  store float %fl4, float* %ref5
  ret { float, float }* %struct
}

define { { float, float }**, i32 }* 
    @appendhelp_tail({ { float, float }**, i32 }* %p) {
entry:
  %p1 = alloca { { float, float }**, i32 }*
  store { { float, float }**, i32 }* %p,
    { { float, float }**, i32 }** %p1
  %tail = alloca { { float, float }**, i32 }*
  %p2 = load { { float, float }**, i32 }*,
    { { float, float }**, i32 }** %p1
  %lenref = getelementptr { { float, float 
    }**, i32 }, { { float, float }**, i32 }* 
    %p2, i32 0, i32 1
  %len = load i32, i32* %lenref
  %tmp = sub i32 %len, 1
  %malloccall = tail call i8* @malloc(i32 ptrtoint
    (i1** getelementptr (i1*,i1** null, i32 1) to i32))
  %arrdata = bitcast i8* %malloccall to { float, float }**
  %malloccall3 = tail call i8* @malloc(i32 trunc (i64 mul 
    nuw (i64 ptrtoint (float* getelementptr (float, float* 
    null, i32 1) to i64), i64 2) to i32))
  %anon = bitcast i8* %malloccall3 to { float, float }*
  %fieldaddr = getelementptr { float, float }, { float, 
    float }* %anon, i32 0, i32 0
  store float 0.000000e+00, float* %fieldaddr
  %fieldaddr4 = getelementptr { float, float }, { float,
    float }* %anon, i32 0, i32 1
  store float 0.000000e+00, float* %fieldaddr4
  %storeref = getelementptr { float, float }*, { float,
    float }** %arrdata, i32 0
  store { float, float }* %anon, { float, float }** %storeref
  %malloccall5 = tail call i8* @malloc(i32 ptrtoint
    ({ { float, float }**, i32}* getelementptr 
    ({ { float, float }**, i32 }, { { float, float }**, i32 }* 
    null, i32 1) to i32))
  %arr = bitcast i8* %malloccall5 to { { float, float }**, i32 }*
  %arrdata6 = getelementptr { { float, float
    }**, i32 }, { { float, float }**, i32 }* %arr, i32 0, i32 0
  %arrlen = getelementptr { { float, float }**, i32 },
    { { float, float }**, i32 }* %arr, i32 0, i32 1
  store { float, float }** %arrdata, { float,
    float }*** %arrdata6
  store i32 1, i32* %arrlen
  %lenref7 = getelementptr { { float, float }**,
    i32 }, { { float, float }**, i32 }* %arr, i32 0, i32 1
  %len8 = load i32, i32* %lenref7
  %oflen = mul i32 %tmp, %len8
  %olddataref = getelementptr { { float, float }**,
    i32 }, { { float, float }**, i32 }* %arr, i32 0, i32 0
  %olddata = load { float, float }**, { float, float }*** %olddataref
  %mallocsize = mul i32 %oflen, ptrtoint (i1** 
    getelementptr (i1*, i1** null, i32 1) to i32)
  %malloccall9 = tail call i8* @malloc(i32 %mallocsize)
  %arrdata10 = bitcast i8* %malloccall9 to 
    { float, float }**
  %i = alloca i32
  store i32 0, i32* %i
  %j = alloca i32
  store i32 0, i32* %j
  br label %inner

loop:                                             ; preds = %inner
  %i18 = load i32, i32* %i
  store i32 0, i32* %j
  %tmp19 = icmp slt i32 %i18, %oflen
  br i1 %tmp19, label %inner, label %continue

inner:                                            
    ; preds = %loop, %inner, %entry
  %i11 = load i32, i32* %j
  %i12 = load i32, i32* %i
  %elref = getelementptr { float, float }*, { float, 
    float }** %olddata, i32 %i11
  %el = load { float, float }*, { float, float }** 
    %elref
  %storeref13 = getelementptr { float, float }*, 
    { float, float }** %arrdata10, i32 %i12
  store { float, float }* %el, { float, float }** 
    %storeref13
  %i14 = add i32 %i12, 1
  store i32 %i14, i32* %i
  %j15 = add i32 %i11, 1
  store i32 %j15, i32* %j
  %j16 = load i32, i32* %j
  %tmp17 = icmp slt i32 %j16, %len8
  br i1 %tmp17, label %inner, label %loop

continue:                                         ; preds = %loop
  %malloccall20 = tail call i8* @malloc(i32 ptrtoint 
    ({ { float, float }**, i32 }* getelementptr ({ 
        { float, float }**, i32 }, { { float, float }**,
        i32 }* null, i32 1) to i32))
  %arr21 = bitcast i8* %malloccall20 to { { float, 
    float }**, i32 }*
  %arrdata22 = getelementptr { { float, float }**, 
    i32 }, { { float, float }**, i32 }* %arr21, 
    i32 0, i32 0
  %arrlen23 = getelementptr { { float, float }**, 
    i32 }, { { float, float }**, 
    i32 }* %arr21, i32 0, i32 1
  store { float, float }** %arrdata10, { float, 
    float }*** %arrdata22
  store i32 %oflen, i32* %arrlen23
  store { { float, float }**, i32 }* %arr21, { 
    { float, float }**, i32 }** %tail
  %tail24 = load { { float, float }**, i32 }*, {
    { float, float }**, i32 }** %tail
  %p25 = load { { float, float }**, i32 }*, { { 
    float, float }**, i32 }** %p1
  call void @appendhelp_copyin({ { float, float 
    }**, i32 }* %tail24, { { float, 
    float }**, i32 }* %p25, i32 0)
  %tail26 = load { { float, float }**, i32 }*, 
    { { float, float }**, i32 }** %tail
  ret { { float, float }**, i32 }* %tail26
}

define { { float, float }**, i32 }* @append({ 
    { float, float }**, i32 }* %p1, 
    { { float, float }**, i32 }* %p2, float %epsilon) {
entry:
  %p11 = alloca { { float, float }**, i32 }*
  store { { float, float }**, i32 }* %p1, { { 
    float, float }**, i32 }** %p11
  %p22 = alloca { { float, float }**, i32 }*
  store { { float, float }**, i32 }* %p2, { { 
    float, float }**, i32 }** %p22
  %epsilon3 = alloca float
  store float %epsilon, float* %epsilon3
  %p14 = load { { float, float }**, i32 }*, { { 
    float, float }**, i32 }** %p11
  %lenref = getelementptr { { float, float }**,
    i32 }, { { float, float }**, i32 }* %p14, i32 0, i32 1
  %len = load i32, i32* %lenref
  %tmp = icmp eq i32 %len, 0
  %if_tmp = alloca { { float, float }**, i32 }*
  br i1 %tmp, label %then, label %else

merge:                                            
    ; preds = %merge11, %then
  %if_tmp66 = load { { float, float }**, i32 }*, 
    { { float, float }**, i32 }** 
    %if_tmp
  ret { { float, float }**, i32 }* %if_tmp66

then:                                             
    ; preds = %entry
  %p25 = load { { float, float }**, i32 }*, { { 
    float, float }**, i32 }** %p22
  %fxn_result = call { { float, float }**, i32 }*
    @copy_path({ { float, float }**, i32 }* %p25)
  store { { float, float }**, i32 }* %fxn_result,
    { { float, float }**, i32 }** %if_tmp
  br label %merge

else:                                            
    ; preds = %entry
  %p26 = load { { float, float }**, i32 }*, {
    { float, float }**, i32 }** %p22
  %lenref7 = getelementptr { { float, float }**, 
    i32 }, { { float, float }**, i32 }* %p26, i32 0, i32 1
  %len8 = load i32, i32* %lenref7
  %tmp9 = icmp eq i32 %len8, 0
  %if_tmp10 = alloca { { float, float }**, i32 }*
  br i1 %tmp9, label %then12, label %else13

merge11:                                          
    ; preds = %contb, %then12
  %if_tmp65 = load { { float, float }**, i32 }*,
    { { float, float }**, i32 }** %if_tmp10
  store { { float, float }**, i32 }* %if_tmp65,
    { { float, float }**, i32}** %if_tmp
  br label %merge

then12:                                           ; preds = %else
  %p114 = load { { float, float }**, i32 }*,
    { { float, float }**, i32 }** %p11
  %fxn_result15 = call { { float, float }**,
    i32 }* @copy_path({ { float, float }**, i32 }* %p114)
  store { { float, float }**, i32 }* %fxn_result15,
    { { float, float }**, i32 }** %if_tmp10
  br label %merge11

else13:                                           ; preds = %else
  %merge16 = alloca i1
  %p117 = load { { float, float }**, i32 }*,
    { { float, float }**, i32 }** %p11
  %p118 = load { { float, float }**, i32 }*,
    { { float, float }**, i32 }** %p11
  %lenref19 = getelementptr { { float, float }**, i32 }, { { float, float
    }**, i32 }* %p118, i32 0, i32 1
  %len20 = load i32, i32* %lenref19
  %tmp21 = sub i32 %len20, 1
  %dataref = getelementptr { { float, float }**, i32 }, { { float, float 
    }**, i32 }* %p117, i32 0, i32 0
  %data = load { float, float }**, { float, float }*** %dataref
  %elref = getelementptr { float, float }*, { float, float }** %data, i32 
    %tmp21
  %el = load { float, float }*, { float, float }** %elref
  %p222 = load { { float, float }**, i32 }*, { { float, float }**, i32 }**
    %p22
  %dataref23 = getelementptr { { float, float }**, i32 }, { { float, float 
    }**, i32 }* %p222, i32 0, i32 0
  %data24 = load { float, float }**, { float, float }*** %dataref23
  %elref25 = getelementptr { float, float }*, { float, float }** %data24,
    i32 0
  %el26 = load { float, float }*, { float, float }** %elref25
  %fxn_result27 = call float @sqrDistance({ float, float }* %el, { float, 
    float }* %el26)
  %epsilon28 = load float, float* %epsilon3
  %epsilon29 = load float, float* %epsilon3
  %tmp30 = fmul float %epsilon28, %epsilon29
  %tmp31 = fcmp olt float %fxn_result27, %tmp30
  store i1 %tmp31, i1* %merge16
  %p2c = alloca { { float, float }**, i32 }*
  %merge32 = load i1, i1* %merge16
  %if_tmp33 = alloca { { float, float }**, i32 }*
  br i1 %merge32, label %then35, label %else36

merge34:                                          
    ; preds = %else36, %then35
  %if_tmp40 = load { { float, float }**, i32 }*, { { float, float
    }**, i32 }** %if_tmp33
  store { { float, float }**, i32 }* %if_tmp40, { { float, 
    float }**, i32 }** %p2c
  %ret = alloca { { float, float }**, i32 }*
  %p141 = load { { float, float }**, i32 }*, { { float, float }**,
    i32 }** %p11
  %fxn_result42 = call { { float, float }**, i32 }* @copy_path({
    { float, float }**, i32 }* %p141)
  %p2c43 = load { { float, float }**, i32 }*, { { float, 
    float }**, i32 }** %p2c
  %fxn_result44 = call { { float, float }**, i32 }* 
    @copy_path({ { float, float }**, i32 }* %p2c43)
  %len1ref = getelementptr { { float, float }**, i32 },
    { { float, float }**, i32 }* %fxn_result42, i32 0, i32 1
  %len1 = load i32, i32* %len1ref
  %len2ref = getelementptr { { float, float }**, i32 },
    { { float, float }**, i32 }* %fxn_result44, i32 0, i32 1
  %len2 = load i32, i32* %len2ref
  %n = add i32 %len1, %len2
  %data1ref = getelementptr { { float, float }**, i32
    }, { { float, float }**, i32 }* %fxn_result42, i32
    0, i32 0
  %data1 = load { float, float }**, { float, float
    }*** %data1ref
  %data2ref = getelementptr { { float, float }**,
    i32 }, { { float, float 
    }**, i32 }* %fxn_result44, i32 0, i32 0
  %data2 = load { float, float }**, { float, float }*** %data2ref
  %mallocsize = mul i32 %n, ptrtoint (i1** getelementptr
    (i1*, i1** null, i32 1) to i32)
  %malloccall = tail call i8* @malloc(i32 %mallocsize)
  %data45 = bitcast i8* %malloccall to { float, float }**
  %i = alloca i32
  store i32 0, i32* %i
  %j = alloca i32
  store i32 0, i32* %j
  br label %loop1

then35:                                           ; preds = %else13
  %p237 = load { { float, float }**, i32 }*, { { float, float }**, i32 
    }** %p22
  %fxn_result38 = call { { float, float }**, i32 }* @appendhelp_tail({
    { float, float }**, i32 }* %p237)
  store { { float, float }**, i32 }* %fxn_result38, { { float, float 
    }**, i32 }** %if_tmp33
  br label %merge34

else36:                                           ; preds = %else13
  %p239 = load { { float, float }**, i32 }*, { { float, float }**, i32 }**
    %p22
  store { { float, float }**, i32 }* %p239, { { float, float }**, i32 }**
    %if_tmp33
  br label %merge34

loop1:                                           
    ; preds = %loop1, %merge34
  %i46 = load i32, i32* %j
  %i47 = load i32, i32* %i
  %elref48 = getelementptr { float, float }*, { float, float }**
    %data1, i32 %i46
  %el49 = load { float, float }*, { float, float }** %elref48
  %storeref = getelementptr { float, float }*, { float, float }**
    %data45, i32 %i47
  store { float, float }* %el49, { float, float }** %storeref
  %tmp50 = add i32 %i47, 1
  store i32 %tmp50, i32* %i
  %j51 = add i32 %i46, 1
  store i32 %j51, i32* %j
  %j52 = load i32, i32* %j
  %tmp53 = icmp slt i32 %j52, %len1
  br i1 %tmp53, label %loop1, label %inbtw

inbtw:                                            ; preds = %loop1
  store i32 0, i32* %j
  br label %loop2

loop2:                                            ; preds = %loop2, %inbtw
  %i54 = load i32, i32* %j
  %i55 = load i32, i32* %i
  %elref56 = getelementptr { float, float }*, { float, float }** %data2,
    i32 %i54
  %el57 = load { float, float }*, { float, float }** %elref56
  %storeref58 = getelementptr { float, float }*, { float, float }** 
    %data45, i32 %i55
  store { float, float }* %el57, { float, float }** %storeref58
  %tmp59 = add i32 %i55, 1
  store i32 %tmp59, i32* %i
  %j60 = add i32 %i54, 1
  store i32 %j60, i32* %j
  %j61 = load i32, i32* %j
  %tmp62 = icmp slt i32 %j61, %len2
  br i1 %tmp62, label %loop2, label %contb

contb:                                            ; preds = %loop2
  %malloccall63 = tail call i8* @malloc(i32 ptrtoint ({ { float, float
    }**, i32 }* getelementptr ({ { float, float }**, i32 }, { { float,
    float }**, i32 }* null, i32 1) to i32))
  %arr = bitcast i8* %malloccall63 to { { float, float }**, i32 }*
  %arrdata = getelementptr { { float, float }**, i32 }, { { float, 
    float }**, i32 }* %arr, i32 0, i32 0
  %arrlen = getelementptr { { float, float }**, i32 }, { { float,
    float }**, i32 }* %arr, i32 0, i32 1
  store { float, float }** %data45, { float, float }*** %arrdata
  store i32 %n, i32* %arrlen
  store { { float, float }**, i32 }* %arr, { { float, float }**,
    i32 }** %ret
  %ret64 = load { { float, float }**, i32 }*, { { float, float }**,
    i32 }** %ret
  store { { float, float }**, i32 }* %ret64, { { float, float }**,
    i32 }** %if_tmp10
  br label %merge11
}

define void @reversedhelp({ { float, float }**, i32 }* %in, { {
    float, float }**, i32 }* %from, i32 %i) {
entry:
  %in1 = alloca { { float, float }**, i32 }*
  store { { float, float }**, i32 }* %in, { { float, float 
    }**, i32 }** %in1
  %from2 = alloca { { float, float }**, i32 }*
  store { { float, float }**, i32 }* %from, { { float, float
    }**, i32 }** %from2
  %i3 = alloca i32
  store i32 %i, i32* %i3
  %i4 = load i32, i32* %i3
  %in5 = load { { float, float }**, i32 }*, { { float, 
    float }**, i32 }** %in1
  %lenref = getelementptr { { float, float }**, i32 }, { { float, float 
    }**, i32 }* %in5, i32 0, i32 1
  %len = load i32, i32* %lenref
  %tmp = icmp slt i32 %i4, %len
  br i1 %tmp, label %then, label %else

merge:                                            ; preds = %else, %then
  ret void

then:                                             ; preds = %entry
  %in6 = load { { float, float }**, i32 }*, { { float,
    float }**, i32 }** %in1
  %i7 = load i32, i32* %i3
  %dataref = getelementptr { { float, float }**, i32 },
    { { float, float 
    }**, i32 }* %in6, i32 0, i32 0
  %data = load { float, float }**, { float, float }***
    %dataref
  %elref = getelementptr { float, float }*, { float, 
    float }** %data, i32 %i7
  %el = load { float, float }*, { float, float }** %elref
  %from8 = load { { float, float }**, i32 }*, { { float,
    float }**, i32 }** %from2
  %in9 = load { { float, float }**, i32 }*, { { float,
    float }**, i32 }** %in1
  %lenref10 = getelementptr { { float, float }**, i32 },
    { { float, float}**, i32 }* %in9, i32 0, i32 1
  %len11 = load i32, i32* %lenref10
  %tmp12 = sub i32 %len11, 1
  %i13 = load i32, i32* %i3
  %tmp14 = sub i32 %tmp12, %i13
  %dataref15 = getelementptr { { float, float }**, i32 }, { { float, 
    float }**, i32 }* %from8, i32 0, i32 0
  %data16 = load { float, float }**, { float, float }***
    %dataref15
  %elref17 = getelementptr { float, float }*, { float, 
    float }** %data16, i32 %tmp14
  %el18 = load { float, float }*, { float, float }** %elref17
  %fieldadr = getelementptr { float, float }, { float, 
    float }* %el18, i32 0, i32 0
  %x = load float, float* %fieldadr
  %ref = getelementptr { float, float }, { float, float }*
    %el, i32 0, i32 0
  store float %x, float* %ref
  %in19 = load { { float, float }**, i32 }*, { { float, 
    float }**, i32 }** %in1
  %i20 = load i32, i32* %i3
  %dataref21 = getelementptr { { float, float }**, i32 },
    { { float, 
    float }**, i32 }* %in19, i32 0, i32 0
  %data22 = load { float, float }**, { float, float }***
    %dataref21
  %elref23 = getelementptr { float, float }*, { float, 
    float }** %data22, i32 %i20
  %el24 = load { float, float }*, { float, float }** 
    %elref23
  %from25 = load { { float, float }**, i32 }*, { { float, 
    float }**, i32 }** %from2
  %in26 = load { { float, float }**, i32 }*, { { float,
    float }**, i32 }** %in1
  %lenref27 = getelementptr { { float, float }**, i32 },
    { { float, 
    float }**, i32 }* %in26, i32 0, i32 1
  %len28 = load i32, i32* %lenref27
  %tmp29 = sub i32 %len28, 1
  %i30 = load i32, i32* %i3
  %tmp31 = sub i32 %tmp29, %i30
  %dataref32 = getelementptr { { float, float }**, i32 }, { { float, 
    float }**, i32 }* %from25, i32 0, i32 0
  %data33 = load { float, float }**, { float, float }*** %dataref32
  %elref34 = getelementptr { float, float }*, { float, float }** %data33, 
    i32 %tmp31
  %el35 = load { float, float }*, { float, float }** %elref34
  %fieldadr36 = getelementptr { float, float }, { float, float }* %el35, 
    i32 0, i32 1
  %y = load float, float* %fieldadr36
  %ref37 = getelementptr { float, float }, { float, float 
    }* %el24, i32 0, i32 1
  store float %y, float* %ref37
  %in38 = load { { float, float }**, i32 }*, { { float, 
    float }**, i32 }** %in1
  %from39 = load { { float, float }**, i32 }*, { { float, 
    float }**, i32 }** %from2
  %i40 = load i32, i32* %i3
  %tmp41 = add i32 %i40, 1
  call void @reversedhelp({ { float, float }**, i32 }* %in38, { { float, 
    float }**, i32 }* %from39, i32 %tmp41)
  br label %merge

else:                                             ; preds = %entry
  br label %merge
}

define { { float, float }**, i32 }* @reversed({ { float, 
    float }**, i32 }* %p) {
entry:
  %p1 = alloca { { float, float }**, i32 }*
  store { { float, float }**, i32 }* %p, { { float, float }**,
    i32 }** %p1
  %newpath = alloca { { float, float }**, i32 }*
  %p2 = load { { float, float }**, i32 }*, { { float, float }**,
    i32 }** %p1
  %lenref = getelementptr { { float, float }**, i32 }, { { float, float 
    }**, i32 }* %p2, i32 0, i32 1
  %len = load i32, i32* %lenref
  %malloccall = tail call i8* @malloc(i32 ptrtoint (i1**
    getelementptr (i1*, i1** null, i32 1) to i32))
  %arrdata = bitcast i8* %malloccall to { float, float }**
  %malloccall3 = tail call i8* @malloc(i32 trunc (i64 mul nuw 
    (i64 ptrtoint(float* getelementptr (float, float* null,
    i32 1) to i64), i64 2) to i32))
  %anon = bitcast i8* %malloccall3 to { float, float }*
  %fieldaddr = getelementptr { float, float }, { float, float
    }* %anon, i32 0, i32 0
  store float 0.000000e+00, float* %fieldaddr
  %fieldaddr4 = getelementptr { float, float }, { float, float
    }* %anon, i32 0, i32 1
  store float 0.000000e+00, float* %fieldaddr4
  %storeref = getelementptr { float, float }*, { float, float
    }** %arrdata, i32 0
  store { float, float }* %anon, { float, float }** %storeref
  %malloccall5 = tail call i8* @malloc(i32 ptrtoint ({ { float, float }**, 
    i32 }* getelementptr ({ { float, float }**, i32 }, { { float, float }**, 
    i32 }* null, i32 1) to i32))
  %arr = bitcast i8* %malloccall5 to { { float, float }**, i32 }*
  %arrdata6 = getelementptr { { float, float }**, i32 }, { { float, float }**, 
    i32 }* %arr, i32 0, i32 0
  %arrlen = getelementptr { { float, float }**, i32 }, { { float, float }**, 
    i32 }* %arr, i32 0, i32 1
  store { float, float }** %arrdata, { float, float }*** %arrdata6
  store i32 1, i32* %arrlen
  %lenref7 = getelementptr { { float, float }**, i32 }, { { float, float }**, 
    i32 }* %arr, i32 0, i32 1
  %len8 = load i32, i32* %lenref7
  %oflen = mul i32 %len, %len8
  %olddataref = getelementptr { { float, float }**, i32 },
    { { float, float }**, i32 }* %arr, i32 0, i32 0
  %olddata = load { float, float }**, { float, float 
    }*** %olddataref
  %mallocsize = mul i32 %oflen, ptrtoint (i1** 
    getelementptr (i1*, i1** null, i32 1) to i32)
  %malloccall9 = tail call i8* @malloc(i32 %mallocsize)
  %arrdata10 = bitcast i8* %malloccall9 to { float, float }**
  %i = alloca i32
  store i32 0, i32* %i
  %j = alloca i32
  store i32 0, i32* %j
  br label %inner

loop:                                             ; preds = %inner
  %i17 = load i32, i32* %i
  store i32 0, i32* %j
  %tmp18 = icmp slt i32 %i17, %oflen
  br i1 %tmp18, label %inner, label %continue

inner:                                            ; preds = %loop, %inner, %entry
  %i11 = load i32, i32* %j
  %i12 = load i32, i32* %i
  %elref = getelementptr { float, float }*, { float, float 
    }** %olddata, i32 %i11
  %el = load { float, float }*, { float, float }** %elref
  %storeref13 = getelementptr { float, float }*, { float, float
    }** %arrdata10, i32 %i12
  store { float, float }* %el, { float, float }** %storeref13
  %i14 = add i32 %i12, 1
  store i32 %i14, i32* %i
  %j15 = add i32 %i11, 1
  store i32 %j15, i32* %j
  %j16 = load i32, i32* %j
  %tmp = icmp slt i32 %j16, %len8
  br i1 %tmp, label %inner, label %loop

continue:                                         ; preds = %loop
  %malloccall19 = tail call i8* @malloc(i32 ptrtoint 
    ({ { float, float }**, i32}* getelementptr ({ { float,
    float }**, i32 }, { { float, float }**, i32 
    }* null, i32 1) to i32))
  %arr20 = bitcast i8* %malloccall19 to { { float,
    float }**, i32 }*
  %arrdata21 = getelementptr { { float, float }**,
    i32 }, { { float, float }**, i32 }* %arr20, i32 0, i32 0
  %arrlen22 = getelementptr { { float, float }**, i32 
    }, { { float, float }**, i32 }* %arr20, i32 0, i32 1
  store { float, float }** %arrdata10, { float, float
    }*** %arrdata21
  store i32 %oflen, i32* %arrlen22
  store { { float, float }**, i32 }* %arr20, 
    { { float, float }**, i32 }** %newpath
  %newpath23 = load { { float, float }**, i32 }*,
    { { float, float }**, i32 }** %newpath
  %p24 = load { { float, float }**, i32 }*, { { float,
    float }**, i32 }** %p1
  call void @reversedhelp({ { float, float }**, i32 }*
    %newpath23, { { float, float }**, i32 }* %p24, i32 0)
  %newpath25 = load { { float, float }**, i32 }*, 
    { { float, float }**, i32 }** %newpath
  ret { { float, float }**, i32 }* %newpath25
}

define void @reversehelp({ { float, float }**, i32 }* %p, i32 %i) {
entry:
  %p1 = alloca { { float, float }**, i32 }*
  store { { float, float }**, i32 }* %p, { { float, float }**, i32 }** %p1
  %i2 = alloca i32
  store i32 %i, i32* %i2
  %i3 = load i32, i32* %i2
  %p4 = load { { float, float }**, i32 }*, { { float, float }**, i32 }** %p1
  %lenref = getelementptr { { float, float }**, i32 }, { { float, float }**, 
    i32 }* %p4, i32 0, i32 1
  %len = load i32, i32* %lenref
  %tmp = sdiv i32 %len, 2
  %tmp5 = icmp slt i32 %i3, %tmp
  br i1 %tmp5, label %then, label %else

merge:                                            ; preds = %else, %then
  ret void

then:                                             ; preds = %entry
  %q = alloca { float, float }*
  %p6 = load { { float, float }**, i32 }*, { { float, float }**, i32 }** %p1
  %i7 = load i32, i32* %i2
  %dataref = getelementptr { { float, float }**, i32 }, { { float, float }**,
    i32 }* %p6, i32 0, i32 0
  %data = load { float, float }**, { float, float }*** %dataref
  %elref = getelementptr { float, float }*, { float, float }** %data, i32 %i7
  %el = load { float, float }*, { float, float }** %elref
  store { float, float }* %el, { float, float }** %q
  %p8 = load { { float, float }**, i32 }*, { { float, float }**, i32 }** %p1
  %datarefref = getelementptr { { float, float }**, i32 }, { { float,
    float }**,i32 }* %p8, i32 0, i32 0
  %dataref9 = load { float, float }**, { float, float }*** %datarefref
  %i10 = load i32, i32* %i2
  %p11 = load { { float, float }**, i32 }*, { { float, float }**, i32 }** %p1
  %p12 = load { { float, float }**, i32 }*, { { float, float }**, i32 }** %p1
  %lenref13 = getelementptr { { float, float }**, i32 }, { { float, float }**,
    i32 }* %p12, i32 0, i32 1
  %len14 = load i32, i32* %lenref13
  %tmp15 = sub i32 %len14, 1
  %i16 = load i32, i32* %i2
  %tmp17 = sub i32 %tmp15, %i16
  %dataref18 = getelementptr { { float, float }**, i32 }, { { float,
    float }**,i32 }* %p11, i32 0, i32 0
  %data19 = load { float, float }**, { float, float }*** %dataref18
  %elref20 = getelementptr { float, float }*, { float, float }** 
    %data19, i32 %tmp17
  %el21 = load { float, float }*, { float, float }** %elref20
  %storeref = getelementptr { float, float }*, { float, float }**
    %dataref9, i32 %i10
  store { float, float }* %el21, { float, float }** %storeref
  %p22 = load { { float, float }**, i32 }*, { { float, float }**,
    i32 }** %p1
  %datarefref23 = getelementptr { { float, float }**, i32 }, { { 
    float, float}**, i32 }* %p22, i32 0, i32 0
  %dataref24 = load { float, float }**, { float, float }***
    %datarefref23
  %p25 = load { { float, float }**, i32 }*, { { float, float
    }**, i32 }** %p1
  %lenref26 = getelementptr { { float, float }**, i32 }, {
    { float, float }**,i32 }* %p25, i32 0, i32 1
  %len27 = load i32, i32* %lenref26
  %tmp28 = sub i32 %len27, 1
  %i29 = load i32, i32* %i2
  %tmp30 = sub i32 %tmp28, %i29
  %q31 = load { float, float }*, { float, float }** %q
  %storeref32 = getelementptr { float, float }*,
    { float, float }** %dataref24,i32 %tmp30
  store { float, float }* %q31, { float, float }** %storeref32
  %p33 = load { { float, float }**, i32 }*, {
    { float, float }**, i32 }** %p1
  %i34 = load i32, i32* %i2
  %tmp35 = add i32 %i34, 1
  call void @reversehelp({ { float, float }**, 
    i32 }* %p33, i32 %tmp35)
  br label %merge

else:                                             ; preds = %entry
  br label %merge
}

define void @reverse({ { float, float }**, i32 }* %p) {
entry:
  %p1 = alloca { { float, float }**, i32 }*
  store { { float, float }**, i32 }* %p, { { 
    float, float }**, i32 }** %p1
  %p2 = load { { float, float }**, i32 }*,
    { { float, float }**, i32 }** %p1
  call void @reversehelp({ { float, float }**, 
    i32 }* %p2, i32 0)
  ret void
}

define { float, float, float, float }* @rgb(float
    %r, float %g, float %b) {
entry:
  %r1 = alloca float
  store float %r, float* %r1
  %g2 = alloca float
  store float %g, float* %g2
  %b3 = alloca float
  store float %b, float* %b3
  %malloccall = tail call i8* @malloc(i32 trunc
    (i64 mul nuw (i64 ptrtoint (float* getelementptr 
    (float, float* null, i32 1) to i64), i64 4) to i32))
  %anon = bitcast i8* %malloccall to { float, float,
    float, float }*
  %fieldaddr = getelementptr { float, float, float, 
    float }, { float, float,
    float, float }* %anon, i32 0, i32 0
  %r4 = load float, float* %r1
  store float %r4, float* %fieldaddr
  %fieldaddr5 = getelementptr { float, float, float,
    float }, { float, float,
    float, float }* %anon, i32 0, i32 1
  %g6 = load float, float* %g2
  store float %g6, float* %fieldaddr5
  %fieldaddr7 = getelementptr { float, float,float, float 
    }, { float, float, float, float }* %anon, i32 0, i32 2
  %b8 = load float, float* %b3
  store float %b8, float* %fieldaddr7
  %fieldaddr9 = getelementptr { float, float, float, float 
    }, { float, float,float, float }* %anon, i32 0, i32 3
  store float 1.000000e+00, float* %fieldaddr9
  ret { float, float, float, float }* %anon
}

define { float, float, float, float }* @hsv(float %h,
    float %s, float %v) {
entry:
  %h1 = alloca float
  store float %h, float* %h1
  %s2 = alloca float
  store float %s, float* %s2
  %v3 = alloca float
  store float %v, float* %v3
  %c = alloca float
  %v4 = load float, float* %v3
  %s5 = load float, float* %s2
  %tmp = fmul float %v4, %s5
  store float %tmp, float* %c
  %hfac = alloca float
  %h6 = load float, float* %h1
  %tmp7 = fmul float %h6, 6.000000e+00
  %fxn_result = call float @modf(float %tmp7, float 2.000000e+00)
  store float %fxn_result, float* %hfac
  %x = alloca float
  %c8 = load float, float* %c
  %hfac9 = load float, float* %hfac
  %tmp10 = fsub float %hfac9, 1.000000e+00
  %fxn_result11 = call float @abs(float %tmp10)
  %tmp12 = fsub float 1.000000e+00, %fxn_result11
  %tmp13 = fmul float %c8, %tmp12
  store float %tmp13, float* %x
  %m = alloca float
  %v14 = load float, float* %v3
  %c15 = load float, float* %c
  %tmp16 = fsub float %v14, %c15
  store float %tmp16, float* %m
  %hh = alloca float
  %h17 = load float, float* %h1
  %tmp18 = fmul float %h17, 6.000000e+00
  store float %tmp18, float* %hh
  %hh19 = load float, float* %hh
  %tmp20 = fcmp olt float %hh19, 1.000000e+00
  %if_tmp = alloca { float, float, float, float }*
  br i1 %tmp20, label %then, label %else

merge:                                            ; preds = %merge30, %then
  %if_tmp85 = load { float, float, float, float }*, { float, float, float, 
    float }** %if_tmp
  ret { float, float, float, float }* %if_tmp85

then:                                             ; preds = %entry
  %v21 = load float, float* %v3
  %x22 = load float, float* %x
  %m23 = load float, float* %m
  %tmp24 = fadd float %x22, %m23
  %m25 = load float, float* %m
  %fxn_result26 = call { float, float, float, float }* @rgb(float %v21,
    float %tmp24, float %m25)
  store { float, float, float, float }* %fxn_result26, { float, float,
    float, float }** %if_tmp
  br label %merge

else:                                             ; preds = %entry
  %hh27 = load float, float* %hh
  %tmp28 = fcmp olt float %hh27, 2.000000e+00
  %if_tmp29 = alloca { float, float, float, float }*
  br i1 %tmp28, label %then31, label %else32

merge30:                                          
    ; preds = %merge42,%then31
  %if_tmp84 = load { float, float, float, float }*, { float,
    float, float,float }** %if_tmp29
  store { float, float, float, float }* %if_tmp84, { float,
    float, float,float }** %if_tmp
  br label %merge

then31:                                           ; preds = %else
  %x33 = load float, float* %x
  %m34 = load float, float* %m
  %tmp35 = fadd float %x33, %m34
  %v36 = load float, float* %v3
  %m37 = load float, float* %m
  %fxn_result38 = call { float, float, float, float }* @rgb(float %tmp35,
    float %v36, float %m37)
  store { float, float, float, float }* %fxn_result38, { float, float,
    float, float }** %if_tmp29
  br label %merge30

else32:                                           ; preds = %else
  %hh39 = load float, float* %hh
  %tmp40 = fcmp olt float %hh39, 3.000000e+00
  %if_tmp41 = alloca { float, float, float, float }*
  br i1 %tmp40, label %then43, label %else44

merge42:                                          ; preds = %merge54, %then43
  %if_tmp83 = load { float, float, float, float }*, { float, float,
    float, float }** %if_tmp41
  store { float, float, float, float }* %if_tmp83, { float, float,
    float, float }** %if_tmp29
  br label %merge30

then43:                                           ; preds = %else32
  %m45 = load float, float* %m
  %v46 = load float, float* %v3
  %x47 = load float, float* %x
  %m48 = load float, float* %m
  %tmp49 = fadd float %x47, %m48
  %fxn_result50 = call { float, float, float, float }* @rgb(float %m45,
    float %v46, float %tmp49)
  store { float, float, float, float }* %fxn_result50, { float, float,
    float, float }** %if_tmp41
  br label %merge42

else44:                                           ; preds = %else32
  %hh51 = load float, float* %hh
  %tmp52 = fcmp olt float %hh51, 4.000000e+00
  %if_tmp53 = alloca { float, float, float, float }*
  br i1 %tmp52, label %then55, label %else56

merge54:                                          ; preds = %merge66, %then55
  %if_tmp82 = load { float, float, float, float }*, { float, float,
    float, float }** %if_tmp53
  store { float, float, float, float }* %if_tmp82, { float, float,
    float, float }** %if_tmp41
  br label %merge42

then55:                                           ; preds = %else44
  %m57 = load float, float* %m
  %x58 = load float, float* %x
  %m59 = load float, float* %m
  %tmp60 = fadd float %x58, %m59
  %v61 = load float, float* %v3
  %fxn_result62 = call { float, float, float, float }* @rgb(float %m57,
    float %tmp60, float %v61)
  store { float, float, float, float }* %fxn_result62, { float, float,
    float, float }** %if_tmp53
  br label %merge54

else56:                                           ; preds = %else44
  %hh63 = load float, float* %hh
  %tmp64 = fcmp olt float %hh63, 5.000000e+00
  %if_tmp65 = alloca { float, float, float, float }*
  br i1 %tmp64, label %then67, label %else68

merge66:                                          ; preds = %else68, %then67
  %if_tmp81 = load { float, float, float, float }*, { float, float,
    float, float }** %if_tmp65
  store { float, float, float, float }* %if_tmp81, { float, float,
    float, float }** %if_tmp53
  br label %merge54

then67:                                           ; preds = %else56
  %x69 = load float, float* %x
  %m70 = load float, float* %m
  %tmp71 = fadd float %x69, %m70
  %m72 = load float, float* %m
  %v73 = load float, float* %v3
  %fxn_result74 = call { float, float, float, float }* @rgb(float
    %tmp71, float %m72, float %v73)
  store { float, float, float, float }* %fxn_result74, { float,
    float, float, float }** %if_tmp65
  br label %merge66

else68:                                           ; preds = %else56
  %v75 = load float, float* %v3
  %m76 = load float, float* %m
  %x77 = load float, float* %x
  %m78 = load float, float* %m
  %tmp79 = fadd float %x77, %m78
  %fxn_result80 = call { float, float, float, float }* @rgb(float
    %v75, float %m76, float %tmp79)
  store { float, float, float, float }* %fxn_result80, { float, 
    float, float, float }** %if_tmp65
  br label %merge66
}

define void @startCanvas({ i32, i32, i32 }* %c) {
entry:
  %c1 = alloca { i32, i32, i32 }*
  store { i32, i32, i32 }* %c, { i32, i32, i32 }** %c1
  %c2 = load { i32, i32, i32 }*, { i32, i32, i32 }** %c1
  %fieldadr = getelementptr { i32, i32, i32 }, { i32, i32, i32 }* %c2, i32
    0, i32 0
  %width = load i32, i32* %fieldadr
  %c3 = load { i32, i32, i32 }*, { i32, i32, i32 }** %c1
  %fieldadr4 = getelementptr { i32, i32, i32 }, { i32, i32, i32 }* %c3, i32
    0, i32 1
  %height = load i32, i32* %fieldadr4
  call void @gl_startRendering(i32 %width, i32 %height)
  ret void
}

define void @cvoid() {
entry:
  ret void
}

define void @drawHelper({ { float, float }**, i32 }* %point_structs,
    { { float, float, float, float }**, i32 }* %color_structs, 
    i32 %numOfPoints, i32 %i, { float*, i32 }* %points, { 
    float*, i32 }* %colors) {
entry:
  %point_structs1 = alloca { { float, float }**, i32 }*
  store { { float, float }**, i32 }* %point_structs, { { float, 
    float }**, i32 }** %point_structs1
  %color_structs2 = alloca { { float, float, float, float }**, i32 }*
  store { { float, float, float, float }**, i32 }* %color_structs, 
    { { float, float, float, float }**, i32 }** %color_structs2
  %numOfPoints3 = alloca i32
  store i32 %numOfPoints, i32* %numOfPoints3
  %i4 = alloca i32
  store i32 %i, i32* %i4
  %points5 = alloca { float*, i32 }*
  store { float*, i32 }* %points, { float*, i32 }** %points5
  %colors6 = alloca { float*, i32 }*
  store { float*, i32 }* %colors, { float*, i32 }** %colors6
  %i7 = load i32, i32* %i4
  %numOfPoints8 = load i32, i32* %numOfPoints3
  %tmp = icmp sge i32 %i7, %numOfPoints8
  br i1 %tmp, label %then, label %else

merge:                                            ; preds = %else, %then
  ret void

then:                                             ; preds = %entry
  call void @cvoid()
  br label %merge

else:                                             ; preds = %entry
  %px = alloca float
  %point_structs9 = load { { float, float }**, i32 }*, { { float,
    float }**, i32 }** %point_structs1
  %i10 = load i32, i32* %i4
  %dataref = getelementptr { { float, float }**, i32 }, { { float,
    float }**, i32 }* %point_structs9, i32 0, i32 0
  %data = load { float, float }**, { float, float }*** %dataref
  %elref = getelementptr { float, float }*, { float, float 
    }** %data, i32 %i10
  %el = load { float, float }*, { float, float }** %elref
  %fieldadr = getelementptr { float, float }, { float, float
    }* %el, i32 0, i32 0
  %x = load float, float* %fieldadr
  store float %x, float* %px
  %py = alloca float
  %point_structs11 = load { { float, float }**, i32 }*, { { float, 
    float }**, i32 }** %point_structs1
  %i12 = load i32, i32* %i4
  %dataref13 = getelementptr { { float, float }**, i32 }, { { float, 
    float }**, i32 }* %point_structs11, i32 0, i32 0
  %data14 = load { float, float }**, { float, float }*** %dataref13
  %elref15 = getelementptr { float, float }*, { float, float }**
    %data14, i32 %i12
  %el16 = load { float, float }*, { float, float }** %elref15
  %fieldadr17 = getelementptr { float, float }, { float, float }*
    %el16, i32 0, i32 1
  %y = load float, float* %fieldadr17
  store float %y, float* %py
  %points18 = load { float*, i32 }*, { float*, i32 }** %points5
  %datarefref = getelementptr { float*, i32 }, { float*, i32 }* 
    %points18, i32 0, i32 0
  %dataref19 = load float*, float** %datarefref
  %i20 = load i32, i32* %i4
  %tmp21 = mul i32 2, %i20
  %px22 = load float, float* %px
  %storeref = getelementptr float, float* %dataref19, i32 %tmp21
  store float %px22, float* %storeref
  %points23 = load { float*, i32 }*, { float*, i32 }** %points5
  %datarefref24 = getelementptr { float*, i32 }, { float*, i32 }*
    %points23, i32 0, i32 0
  %dataref25 = load float*, float** %datarefref24
  %i26 = load i32, i32* %i4
  %tmp27 = mul i32 2, %i26
  %tmp28 = add i32 %tmp27, 1
  %py29 = load float, float* %py
  %storeref30 = getelementptr float, float* %dataref25, i32 %tmp28
  store float %py29, float* %storeref30
  %cr = alloca float
  %color_structs31 = load { { float, float, float, float }**, i32 }*,
    { { float, float, float, float }**, i32 }** %color_structs2
  %i32 = load i32, i32* %i4
  %dataref33 = getelementptr { { float, float, float, float }**,i32 },
    { { float, float, float, float }**, i32 }* %color_structs31, i32 0,
    i32 0
  %data34 = load { float, float, float, float }**, { float, float, 
    float, float }*** %dataref33
  %elref35 = getelementptr { float, float, float, float }*, { float,
    float, float, float }** %data34, i32 %i32
  %el36 = load { float, float, float, float }*, { float, float, 
    float, float }** %elref35
  %fieldadr37 = getelementptr { float, float, float, float }, { float, 
    float, float, float }* %el36, i32 0, i32 0
  %r = load float, float* %fieldadr37
  store float %r, float* %cr
  %cg = alloca float
  %color_structs38 = load { { float, float, float, float }**, i32 }*, 
    { { float, float, float, float }**, i32 }** %color_structs2
  %i39 = load i32, i32* %i4
  %dataref40 = getelementptr { { float, float, float, float }**, i32 }, 
    { { float, float, float, float }**, i32 }* %color_structs38, i32
    0, i32 0
  %data41 = load { float, float, float, float }**, { float, float, 
    float, float }*** %dataref40
  %elref42 = getelementptr { float, float, float, float }*, { float, 
    float, float, float }** %data41, i32 %i39
  %el43 = load { float, float, float, float }*, { float, float, float, 
    float }** %elref42
  %fieldadr44 = getelementptr { float, float, float, float }, { float, 
    float, float, float }* %el43, i32 0, i32 1
  %g = load float, float* %fieldadr44
  store float %g, float* %cg
  %cb = alloca float
  %color_structs45 = load { { float, float, float, float }**, i32 }*, 
    { { float, float, float, float }**, i32 }** %color_structs2
  %i46 = load i32, i32* %i4
  %dataref47 = getelementptr { { float, float, float, float }**, i32 },
    { { float, float, float, float }**, i32 }* %color_structs45, 
    i32 0, i32 0
  %data48 = load { float, float, float, float }**, { float, float, 
    float, float }*** %dataref47
  %elref49 = getelementptr { float, float, float, float }*, { float, 
    float, float, float }** %data48, i32 %i46
  %el50 = load { float, float, float, float }*, { float, float, float,
    float }** %elref49
  %fieldadr51 = getelementptr { float, float, float, float }, { float, 
    float, float, float }* %el50, i32 0, i32 2
  %b = load float, float* %fieldadr51
  store float %b, float* %cb
  %ca = alloca float
  %color_structs52 = load { { float, float, float, float }**, i32 }*, 
    { { float, float, float, float }**, i32 }** %color_structs2
  %i53 = load i32, i32* %i4
  %dataref54 = getelementptr { { float, float, float, float }**, i32 }, 
    { { float, float, float, float }**, i32 }* %color_structs52,
    i32 0, i32 0
  %data55 = load { float, float, float, float }**, { float, float, 
    float, float }*** %dataref54
  %elref56 = getelementptr { float, float, float, float }*, { float,
    float, float, float }** %data55, i32 %i53
  %el57 = load { float, float, float, float }*, { float, float, float,
    float }** %elref56
  %fieldadr58 = getelementptr { float, float, float, float }, { float,
    float, float, float }* %el57, i32 0, i32 3
  %a = load float, float* %fieldadr58
  store float %a, float* %ca
  %colors59 = load { float*, i32 }*, { float*, i32 }** %colors6
  %datarefref60 = getelementptr { float*, i32 }, { float*, i32 }*
    %colors59, i32 0, i32 0
  %dataref61 = load float*, float** %datarefref60
  %i62 = load i32, i32* %i4
  %tmp63 = mul i32 4, %i62
  %cr64 = load float, float* %cr
  %storeref65 = getelementptr float, float* %dataref61, i32 %tmp63
  store float %cr64, float* %storeref65
  %colors66 = load { float*, i32 }*, { float*, i32 }** %colors6
  %datarefref67 = getelementptr { float*, i32 }, { float*, i32 }*
    %colors66, i32 0, i32 0
  %dataref68 = load float*, float** %datarefref67
  %i69 = load i32, i32* %i4
  %tmp70 = mul i32 4, %i69
  %tmp71 = add i32 %tmp70, 1
  %cg72 = load float, float* %cg
  %storeref73 = getelementptr float, float* %dataref68, i32 %tmp71
  store float %cg72, float* %storeref73
  %colors74 = load { float*, i32 }*, { float*, i32 }** %colors6
  %datarefref75 = getelementptr { float*, i32 }, { float*, i32 }*
    %colors74, i32 0, i32 0
  %dataref76 = load float*, float** %datarefref75
  %i77 = load i32, i32* %i4
  %tmp78 = mul i32 4, %i77
  %tmp79 = add i32 %tmp78, 2
  %cb80 = load float, float* %cb
  %storeref81 = getelementptr float, float* %dataref76, i32 %tmp79
  store float %cb80, float* %storeref81
  %colors82 = load { float*, i32 }*, { float*, i32 }** %colors6
  %datarefref83 = getelementptr { float*, i32 }, { float*, i32 }*
    %colors82, i32 0, i32 0
  %dataref84 = load float*, float** %datarefref83
  %i85 = load i32, i32* %i4
  %tmp86 = mul i32 4, %i85
  %tmp87 = add i32 %tmp86, 3
  %ca88 = load float, float* %ca
  %storeref89 = getelementptr float, float* %dataref84, i32 %tmp87
  store float %ca88, float* %storeref89
  %point_structs90 = load { { float, float }**, i32 }*, { { float, 
    float }**, i32 }** %point_structs1
  %color_structs91 = load { { float, float, float, float }**, i32 }*,
    { { float, float, float, float }**, i32 }** %color_structs2
  %numOfPoints92 = load i32, i32* %numOfPoints3
  %i93 = load i32, i32* %i4
  %tmp94 = add i32 %i93, 1
  %points95 = load { float*, i32 }*, { float*, i32 }** %points5
  %colors96 = load { float*, i32 }*, { float*, i32 }** %colors6
  call void @drawHelper({ { float, float }**, i32 }* %point_structs90, 
    { { float, float, float, float }**, i32 }* %color_structs91, i32 
        %numOfPoints92, i32 %tmp94, { float*, i32 }* %points95, 
        { float*, i32 }* %colors96)
  br label %merge
}

define void @drawPoints({ { float, float }**, i32 }* %point_structs,
    { { float, float, float, float }**, i32 }* %color_structs) {
entry:
  %point_structs1 = alloca { { float, float }**, i32 }*
  store { { float, float }**, i32 }* %point_structs, { { float, 
    float }**, i32 }** %point_structs1
  %color_structs2 = alloca { { float, float, float, float }**, i32 }*
  store { { float, float, float, float }**, i32 }* %color_structs, 
    { { float, float, float, float }**, i32 }** %color_structs2
  %numOfPoints = alloca i32
  %point_structs3 = load { { float, float }**, i32 }*, {
    { float, float }**, i32 }** %point_structs1
  %lenref = getelementptr { { float, float }**, i32 }, {
    { float, float }**, i32 }* %point_structs3, i32 0, i32 1
  %len = load i32, i32* %lenref
  store i32 %len, i32* %numOfPoints
  %points = alloca { float*, i32 }*
  %numOfPoints4 = load i32, i32* %numOfPoints
  %tmp = mul i32 %numOfPoints4, 2
  %malloccall = tail call i8* @malloc(i32 ptrtoint (float*
    getelementptr(float, float* null, i32 1) to i32))
  %arrdata = bitcast i8* %malloccall to float*
  %storeref = getelementptr float, float* %arrdata, i32 0
  store float 0.000000e+00, float* %storeref
  %malloccall5 = tail call i8* @malloc(i32 ptrtoint
    ({ float*, i32 }* getelementptr ({ float*, i32 }, 
    { float*, i32 }* null, i32 1) to i32))
  %arr = bitcast i8* %malloccall5 to { float*, i32 }*
  %arrdata6 = getelementptr { float*, i32 }, { float*,
    i32 }* %arr, i32 0, i32 0
  %arrlen = getelementptr { float*, i32 }, { float*,
    i32 }* %arr, i32 0, i32 1
  store float* %arrdata, float** %arrdata6
  store i32 1, i32* %arrlen
  %lenref7 = getelementptr { float*, i32 }, { float*, i32 }*
    %arr, i32 0, i32 1
  %len8 = load i32, i32* %lenref7
  %oflen = mul i32 %tmp, %len8
  %olddataref = getelementptr { float*, i32 }, { float*,
    i32 }* %arr, i32 0, i32 0
  %olddata = load float*, float** %olddataref
  %mallocsize = mul i32 %oflen, ptrtoint (float* getelementptr
    (float,float* null, i32 1) to i32)
  %malloccall9 = tail call i8* @malloc(i32 %mallocsize)
  %arrdata10 = bitcast i8* %malloccall9 to float*
  %i = alloca i32
  store i32 0, i32* %i
  %j = alloca i32
  store i32 0, i32* %j
  br label %inner

loop:                                             
    ; preds = %inner
  %i18 = load i32, i32* %i
  store i32 0, i32* %j
  %tmp19 = icmp slt i32 %i18, %oflen
  br i1 %tmp19, label %inner, label %continue

inner:                                            
    ; preds = %loop, %inner, %entry
  %i11 = load i32, i32* %j
  %i12 = load i32, i32* %i
  %elref = getelementptr float, float* %olddata, i32 %i11
  %el = load float, float* %elref
  %storeref13 = getelementptr float, float* %arrdata10, i32 %i12
  store float %el, float* %storeref13
  %i14 = add i32 %i12, 1
  store i32 %i14, i32* %i
  %j15 = add i32 %i11, 1
  store i32 %j15, i32* %j
  %j16 = load i32, i32* %j
  %tmp17 = icmp slt i32 %j16, %len8
  br i1 %tmp17, label %inner, label %loop

continue:                                         
    ; preds = %loop
  %malloccall20 = tail call i8* @malloc(i32 ptrtoint
    ({ float*, i32 }* getelementptr ({ float*, i32 },
    { float*, i32 }* null, i32 1) to i32))
  %arr21 = bitcast i8* %malloccall20 to { float*,
    i32 }*
  %arrdata22 = getelementptr { float*, i32 }, { float*,
    i32 }* %arr21, i32 0, i32 0
  %arrlen23 = getelementptr { float*, i32 }, { float*,
    i32 }* %arr21, i32 0, i32 1
  store float* %arrdata10, float** %arrdata22
  store i32 %oflen, i32* %arrlen23
  store { float*, i32 }* %arr21, { float*, i32 }**
    %points
  %colors = alloca { float*, i32 }*
  %numOfPoints24 = load i32, i32* %numOfPoints
  %tmp25 = mul i32 %numOfPoints24, 4
  %malloccall26 = tail call i8* @malloc(i32 ptrtoint 
    (float* getelementptr (float, float* null, i32 1)
    to i32))
  %arrdata27 = bitcast i8* %malloccall26 to float*
  %storeref28 = getelementptr float, float* %arrdata27, i32 0
  store float 0.000000e+00, float* %storeref28
  %malloccall29 = tail call i8* @malloc(i32 ptrtoint 
    ({ float*, i32 }* getelementptr ({ float*, i32 },
    { float*, i32 }* null, i32 1) to i32))
  %arr30 = bitcast i8* %malloccall29 to { float*, i32 }*
  %arrdata31 = getelementptr { float*, i32 }, { float*,
    i32 }* %arr30, i32 0, i32 0
  %arrlen32 = getelementptr { float*, i32 }, { float*,
    i32 }* %arr30, i32 0, i32 1
  store float* %arrdata27, float** %arrdata31
  store i32 1, i32* %arrlen32
  %lenref33 = getelementptr { float*, i32 }, { float*,
    i32 }* %arr30, i32 0, i32 1
  %len34 = load i32, i32* %lenref33
  %oflen35 = mul i32 %tmp25, %len34
  %olddataref36 = getelementptr { float*, i32 }, { float*,
    i32 }* %arr30, i32 0, i32 0
  %olddata37 = load float*, float** %olddataref36
  %mallocsize38 = mul i32 %oflen35, ptrtoint (float* 
    getelementptr (float,
    float* null, i32 1) to i32)
  %malloccall39 = tail call i8* @malloc(i32 %mallocsize38)
  %arrdata40 = bitcast i8* %malloccall39 to float*
  %i41 = alloca i32
  store i32 0, i32* %i41
  %j42 = alloca i32
  store i32 0, i32* %j42
  br label %inner44

loop43:                                           
    ; preds = %inner44
  %i55 = load i32, i32* %i41
  store i32 0, i32* %j42
  %tmp56 = icmp slt i32 %i55, %oflen35
  br i1 %tmp56, label %inner44, label %continue45

inner44:                                          
    ; preds = %loop43, %inner44, %continue
  %i46 = load i32, i32* %j42
  %i47 = load i32, i32* %i41
  %elref48 = getelementptr float, float* %olddata37, i32 %i46
  %el49 = load float, float* %elref48
  %storeref50 = getelementptr float, float* %arrdata40, i32 %i47
  store float %el49, float* %storeref50
  %i51 = add i32 %i47, 1
  store i32 %i51, i32* %i41
  %j52 = add i32 %i46, 1
  store i32 %j52, i32* %j42
  %j53 = load i32, i32* %j42
  %tmp54 = icmp slt i32 %j53, %len34
  br i1 %tmp54, label %inner44, label %loop43

continue45:                                       
    ; preds = %loop43
  %malloccall57 = tail call i8* @malloc(i32
    ptrtoint ({ float*, i32 }*getelementptr
    ({ float*, i32 }, { float*, i32}* null, i32 1) to i32))
  %arr58 = bitcast i8* %malloccall57 to { float*, i32 }*
  %arrdata59 = getelementptr { float*, i32 }, { float*,
    i32 }* %arr58, i32 0, i32 0
  %arrlen60 = getelementptr { float*, i32 }, { float*,
    i32 }* %arr58, i32 0, i32 1
  store float* %arrdata40, float** %arrdata59
  store i32 %oflen35, i32* %arrlen60
  store { float*, i32 }* %arr58, { float*, i32 }** %colors
  %point_structs61 = load { { float, float }**, i32 }*, { { float, 
    float }**, i32 }** %point_structs1
  %color_structs62 = load { { float, float, float, float }**, i32 }*, 
    { { float, float, float, float }**, i32 }** %color_structs2
  %numOfPoints63 = load i32, i32* %numOfPoints
  %points64 = load { float*, i32 }*, { float*, i32 }** %points
  %colors65 = load { float*, i32 }*, { float*, i32 }** %colors
  call void @drawHelper({ { float, float }**, i32 }* %point_structs61, 
    { { float, float, float, float }**, i32 }* %color_structs62, i32 
    %numOfPoints63, i32 0, { float*, i32 }* %points64,
    { float*, i32 }* %colors65)
  %points66 = load { float*, i32 }*, { float*, i32 }** %points
  %colors67 = load { float*, i32 }*, { float*, i32 }** %colors
  call void @gl_drawPoint({ float*, i32 }* %points66, { float*, i32 
    }* %colors67, i32 2)
  ret void
}

define void @drawPath({ { float, float }**, i32 }* %point_structs, 
    { { float, float, float, float }**, i32 }* %color_structs,
    i32 %colorMode) {
entry:
  %point_structs1 = alloca { { float, float }**, i32 }*
  store { { float, float }**, i32 }* %point_structs, { { float, 
    float }**, i32 }** %point_structs1
  %color_structs2 = alloca { { float, float, float, float }**, i32 }*
  store { { float, float, float, float }**, i32 }* %color_structs,
    { { float, float, float, float }**, i32 }** %color_structs2
  %colorMode3 = alloca i32
  store i32 %colorMode, i32* %colorMode3
  %numOfPoints = alloca i32
  %point_structs4 = load { { float, float }**, i32 }*, { { float,
    float }**, i32 }** %point_structs1
  %lenref = getelementptr { { float, float }**, i32 }, { { float,
    float }**, i32 }* %point_structs4, i32 0, i32 1
  %len = load i32, i32* %lenref
  store i32 %len, i32* %numOfPoints
  %points = alloca { float*, i32 }*
  %numOfPoints5 = load i32, i32* %numOfPoints
  %tmp = mul i32 %numOfPoints5, 2
  %malloccall = tail call i8* @malloc(i32 ptrtoint (float*
    getelementptr (float, float* null, i32 1) to i32))
  %arrdata = bitcast i8* %malloccall to float*
  %storeref = getelementptr float, float* %arrdata, i32 0
  store float 0.000000e+00, float* %storeref
  %malloccall6 = tail call i8* @malloc(i32 ptrtoint ({ 
    float*, i32 }* getelementptr ({ float*, i32 }, { float*,
    i32 }* null, i32 1) to i32))
  %arr = bitcast i8* %malloccall6 to { float*, i32 }*
  %arrdata7 = getelementptr { float*, i32 }, { float*, i32
    }* %arr, i32 0, i32 0
  %arrlen = getelementptr { float*, i32 }, { float*,
    i32 }* %arr, i32 0, i32 1
  store float* %arrdata, float** %arrdata7
  store i32 1, i32* %arrlen
  %lenref8 = getelementptr { float*, i32 },
    { float*, i32 }* %arr, i32 0, i32 1
  %len9 = load i32, i32* %lenref8
  %oflen = mul i32 %tmp, %len9
  %olddataref = getelementptr { float*, i32 },
    { float*, i32 }* %arr, i32 0, i32 0
  %olddata = load float*, float** %olddataref
  %mallocsize = mul i32 %oflen, ptrtoint (float*
    getelementptr (float, float* null, i32 1) to i32)
  %malloccall10 = tail call i8* @malloc(i32 %mallocsize)
  %arrdata11 = bitcast i8* %malloccall10 to float*
  %i = alloca i32
  store i32 0, i32* %i
  %j = alloca i32
  store i32 0, i32* %j
  br label %inner

loop:                                             
    ; preds = %inner
  %i19 = load i32, i32* %i
  store i32 0, i32* %j
  %tmp20 = icmp slt i32 %i19, %oflen
  br i1 %tmp20, label %inner, label %continue

inner:                                            
    ; preds = %loop, %inner, %entry
  %i12 = load i32, i32* %j
  %i13 = load i32, i32* %i
  %elref = getelementptr float, float* %olddata, i32 %i12
  %el = load float, float* %elref
  %storeref14 = getelementptr float, float* %arrdata11, i32 %i13
  store float %el, float* %storeref14
  %i15 = add i32 %i13, 1
  store i32 %i15, i32* %i
  %j16 = add i32 %i12, 1
  store i32 %j16, i32* %j
  %j17 = load i32, i32* %j
  %tmp18 = icmp slt i32 %j17, %len9
  br i1 %tmp18, label %inner, label %loop

continue:                                         ; preds = %loop
  %malloccall21 = tail call i8* @malloc(i32 
    ptrtoint ({ float*, i32 }* 
    getelementptr ({ float*, i32 }, { float*, 
    i32 }* null, i32 1) to i32))
  %arr22 = bitcast i8* %malloccall21 to { float*, i32 }*
  %arrdata23 = getelementptr { float*, i32 },
    { float*, i32 }* %arr22, i32 0, i32 0
  %arrlen24 = getelementptr { float*, i32 },
    { float*, i32 }* %arr22, i32 0, i32 1
  store float* %arrdata11, float** %arrdata23
  store i32 %oflen, i32* %arrlen24
  store { float*, i32 }* %arr22, { float*, i32
    }** %points
  %colors = alloca { float*, i32 }*
  %numOfPoints25 = load i32, i32* %numOfPoints
  %tmp26 = mul i32 %numOfPoints25, 4
  %malloccall27 = tail call i8* @malloc(i32
    ptrtoint (float* 
    getelementptr (float, float* null, i32 1) to i32))
  %arrdata28 = bitcast i8* %malloccall27 to float*
  %storeref29 = getelementptr float, float* %arrdata28, i32 0
  store float 0.000000e+00, float* %storeref29
  %malloccall30 = tail call i8* @malloc(i32 ptrtoint
    ({ float*,i32 }* getelementptr ({ float*, i32 },
    { float*, i32 }* null, i32 1) to i32))
  %arr31 = bitcast i8* %malloccall30 to { float*, i32 }*
  %arrdata32 = getelementptr { float*, i32 }, { float*,
    i32 }* %arr31, i32 0, i32 0
  %arrlen33 = getelementptr { float*, i32 }, { float*,
    i32 }* %arr31, i32 0, i32 1
  store float* %arrdata28, float** %arrdata32
  store i32 1, i32* %arrlen33
  %lenref34 = getelementptr { float*, i32 }, { float*,
    i32 }* %arr31, i32 0, i32 1
  %len35 = load i32, i32* %lenref34
  %oflen36 = mul i32 %tmp26, %len35
  %olddataref37 = getelementptr { float*, i32 }, { float*,
    i32 }* %arr31, i32 0, i32 0
  %olddata38 = load float*, float** %olddataref37
  %mallocsize39 = mul i32 %oflen36, ptrtoint (float*
    getelementptr(float, float* null, i32 1) to i32)
  %malloccall40 = tail call i8* @malloc(i32 %mallocsize39)
  %arrdata41 = bitcast i8* %malloccall40 to float*
  %i42 = alloca i32
  store i32 0, i32* %i42
  %j43 = alloca i32
  store i32 0, i32* %j43
  br label %inner45

loop44:                                           ; preds = %inner45
  %i56 = load i32, i32* %i42
  store i32 0, i32* %j43
  %tmp57 = icmp slt i32 %i56, %oflen36
  br i1 %tmp57, label %inner45, label %continue46

inner45:                                          
    ; preds = %loop44, %inner45, %continue
  %i47 = load i32, i32* %j43
  %i48 = load i32, i32* %i42
  %elref49 = getelementptr float, float* %olddata38, i32 %i47
  %el50 = load float, float* %elref49
  %storeref51 = getelementptr float, float* %arrdata41, i32 %i48
  store float %el50, float* %storeref51
  %i52 = add i32 %i48, 1
  store i32 %i52, i32* %i42
  %j53 = add i32 %i47, 1
  store i32 %j53, i32* %j43
  %j54 = load i32, i32* %j43
  %tmp55 = icmp slt i32 %j54, %len35
  br i1 %tmp55, label %inner45, label %loop44

continue46:                                       ; preds = %loop44
  %malloccall58 = tail call i8* @malloc(i32 ptrtoint
    ({ float*, i32 }*
    getelementptr ({ float*, i32 }, { float*, i32 }* null, i32 1) to i32))
  %arr59 = bitcast i8* %malloccall58 to { float*, i32 }*
  %arrdata60 = getelementptr { float*, i32 }, { float*,
    i32 }* %arr59, i32 0, i32 0
  %arrlen61 = getelementptr { float*, i32 }, { float*,
    i32 }* %arr59, i32 0, i32 1
  store float* %arrdata41, float** %arrdata60
  store i32 %oflen36, i32* %arrlen61
  store { float*, i32 }* %arr59, { float*, i32 }** %colors
  %point_structs62 = load { { float, float }**, i32 }*,
    { { float, float}**, i32 }** %point_structs1
  %color_structs63 = load { { float, float, float, float }**, i32 }*, 
    { { float, float, float, float }**, i32 }** %color_structs2
  %numOfPoints64 = load i32, i32* %numOfPoints
  %points65 = load { float*, i32 }*, { float*, i32 }** %points
  %colors66 = load { float*, i32 }*, { float*, i32 }** %colors
  call void @drawHelper({ { float, float }**, i32 }* %point_structs62, 
    { { float, float, float, float }**, i32 }* %color_structs63, i32 
    %numOfPoints64, i32 0, { float*, i32 }* %points65, { float*, i32 }
    * %colors66)
  %points67 = load { float*, i32 }*, { float*, i32 }** %points
  %colors68 = load { float*, i32 }*, { float*, i32 }** %colors
  %colorMode69 = load i32, i32* %colorMode3
  call void @gl_drawCurve({ float*, i32 }* %points67, { float*, i32 }*
    %colors68, i32 %colorMode69)
  ret void
}

define void @drawShape({ { float, float }**, i32 }* %point_structs, { 
    { float, float, float, float }**, i32 }* %color_structs,
    i32 %colorMode, i32 %filled) {
entry:
  %point_structs1 = alloca { { float, float }**, i32 }*
  store { { float, float }**, i32 }* %point_structs,
    { { float, float }**, i32 }** %point_structs1
  %color_structs2 = alloca { { float, float, float, float }**, i32 }*
  store { { float, float, float, float }**, i32 }* %color_structs, { 
    { float, float, float, float }**, i32 }** %color_structs2
  %colorMode3 = alloca i32
  store i32 %colorMode, i32* %colorMode3
  %filled4 = alloca i32
  store i32 %filled, i32* %filled4
  %numOfPoints = alloca i32
  %point_structs5 = load { { float, float }**, i32 }*, { { float, 
    float }**, i32 }** %point_structs1
  %lenref = getelementptr { { float, float }**, i32 }, { { float, 
    float }**, i32 }* %point_structs5, i32 0, i32 1
  %len = load i32, i32* %lenref
  store i32 %len, i32* %numOfPoints
  %points = alloca { float*, i32 }*
  %numOfPoints6 = load i32, i32* %numOfPoints
  %tmp = mul i32 %numOfPoints6, 2
  %malloccall = tail call i8* @malloc(i32 ptrtoint (float* 
    getelementptr (float, float* null, i32 1) to i32))
  %arrdata = bitcast i8* %malloccall to float*
  %storeref = getelementptr float, float* %arrdata, i32 0
  store float 0.000000e+00, float* %storeref
  %malloccall7 = tail call i8* @malloc(i32 ptrtoint ({ float*, 
    i32 }* getelementptr ({ float*, i32 }, { float*, i32 }* 
    null, i32 1) to i32))
  %arr = bitcast i8* %malloccall7 to { float*, i32 }*
  %arrdata8 = getelementptr { float*, i32 }, { float*, i32 }*
    %arr, i32 0, i32 0
  %arrlen = getelementptr { float*, i32 }, { float*, i32 }*
    %arr, i32 0, i32 1
  store float* %arrdata, float** %arrdata8
  store i32 1, i32* %arrlen
  %lenref9 = getelementptr { float*, i32 }, { float*, i32 }*
    %arr, i32 0, i32 1
  %len10 = load i32, i32* %lenref9
  %oflen = mul i32 %tmp, %len10
  %olddataref = getelementptr { float*, i32 }, { float*, i32
    }* %arr, i32 0, i32 0
  %olddata = load float*, float** %olddataref
  %mallocsize = mul i32 %oflen, ptrtoint (float* getelementptr 
    (float, float* null, i32 1) to i32)
  %malloccall11 = tail call i8* @malloc(i32 %mallocsize)
  %arrdata12 = bitcast i8* %malloccall11 to float*
  %i = alloca i32
  store i32 0, i32* %i
  %j = alloca i32
  store i32 0, i32* %j
  br label %inner

loop:                                             
    ; preds = %inner
  %i20 = load i32, i32* %i
  store i32 0, i32* %j
  %tmp21 = icmp slt i32 %i20, %oflen
  br i1 %tmp21, label %inner, label %continue

inner:                                            
    ; preds = %loop, %inner, %entry
  %i13 = load i32, i32* %j
  %i14 = load i32, i32* %i
  %elref = getelementptr float, float* %olddata, i32 %i13
  %el = load float, float* %elref
  %storeref15 = getelementptr float, float* %arrdata12, i32 %i14
  store float %el, float* %storeref15
  %i16 = add i32 %i14, 1
  store i32 %i16, i32* %i
  %j17 = add i32 %i13, 1
  store i32 %j17, i32* %j
  %j18 = load i32, i32* %j
  %tmp19 = icmp slt i32 %j18, %len10
  br i1 %tmp19, label %inner, label %loop

continue:                                         ; preds = %loop
  %malloccall22 = tail call i8* @malloc(i32 ptrtoint
    ({ float*, 
    i32 }* getelementptr ({ float*, i32 }, { float*,
    i32 }* null, i32 1) to i32))
  %arr23 = bitcast i8* %malloccall22 to { float*, i32 }*
  %arrdata24 = getelementptr { float*, i32 }, { 
    float*, i32 }* %arr23, i32 0, i32 0
  %arrlen25 = getelementptr { float*, i32 },
    { float*, i32 }* %arr23, i32 0, i32 1
  store float* %arrdata12, float** %arrdata24
  store i32 %oflen, i32* %arrlen25
  store { float*, i32 }* %arr23, { float*,
    i32 }** %points
  %colors = alloca { float*, i32 }*
  %numOfPoints26 = load i32, i32* %numOfPoints
  %tmp27 = mul i32 %numOfPoints26, 4
  %malloccall28 = tail call i8* @malloc(i32 ptrtoint (float* 
    getelementptr (float, float* null, i32 1) to i32))
  %arrdata29 = bitcast i8* %malloccall28 to float*
  %storeref30 = getelementptr float, float* %arrdata29, i32 0
  store float 0.000000e+00, float* %storeref30
  %malloccall31 = tail call i8* @malloc(i32
    ptrtoint ({ float*, i32 }* getelementptr
    ({ float*, i32 }, { float*, i32 }* null, i32 1) to i32))
  %arr32 = bitcast i8* %malloccall31 to { float*, i32 }*
  %arrdata33 = getelementptr { float*, i32 }, 
    { float*, i32 }* %arr32, i32 0, i32 0
  %arrlen34 = getelementptr { float*, i32 }, 
    { float*, i32 }* %arr32, i32 0, i32 1
  store float* %arrdata29, float** %arrdata33
  store i32 1, i32* %arrlen34
  %lenref35 = getelementptr { float*, i32 },
    { float*, i32 }* %arr32, i32 0, i32 1
  %len36 = load i32, i32* %lenref35
  %oflen37 = mul i32 %tmp27, %len36
  %olddataref38 = getelementptr { float*, i32 }, 
    { float*, i32 }* %arr32, i32 0, i32 0
  %olddata39 = load float*, float** %olddataref38
  %mallocsize40 = mul i32 %oflen37, ptrtoint (float* getelementptr 
    (float, float* null, i32 1) to i32)
  %malloccall41 = tail call i8* @malloc(i32 %mallocsize40)
  %arrdata42 = bitcast i8* %malloccall41 to float*
  %i43 = alloca i32
  store i32 0, i32* %i43
  %j44 = alloca i32
  store i32 0, i32* %j44
  br label %inner46

loop45:                                           ; preds = %inner46
  %i57 = load i32, i32* %i43
  store i32 0, i32* %j44
  %tmp58 = icmp slt i32 %i57, %oflen37
  br i1 %tmp58, label %inner46, label %continue47

inner46:                                          
    ; preds = %loop45, %inner46, %continue
  %i48 = load i32, i32* %j44
  %i49 = load i32, i32* %i43
  %elref50 = getelementptr float, float* %olddata39, i32 %i48
  %el51 = load float, float* %elref50
  %storeref52 = getelementptr float, float* %arrdata42, i32 %i49
  store float %el51, float* %storeref52
  %i53 = add i32 %i49, 1
  store i32 %i53, i32* %i43
  %j54 = add i32 %i48, 1
  store i32 %j54, i32* %j44
  %j55 = load i32, i32* %j44
  %tmp56 = icmp slt i32 %j55, %len36
  br i1 %tmp56, label %inner46, label %loop45

continue47:                                       
    ; preds = %loop45
  %malloccall59 = tail call i8* @malloc(i32 
    ptrtoint ({ float*, i32 }*getelementptr 
    ({ float*, i32 }, { float*, i32 }* null, i32 1) to i32))
  %arr60 = bitcast i8* %malloccall59 to { float*, i32 }*
  %arrdata61 = getelementptr { float*, i32 },
    { float*, i32 }* %arr60, i32 0, i32 0
  %arrlen62 = getelementptr { float*, i32 },
    { float*, i32 }* %arr60, i32 0, i32 1
  store float* %arrdata42, float** %arrdata61
  store i32 %oflen37, i32* %arrlen62
  store { float*, i32 }* %arr60, { float*, 
    i32 }** %colors
  %point_structs63 = load { { float, float }**,
    i32 }*, { { float, float }**, i32 }** %point_structs1
  %color_structs64 = load { { float, float, float,
    float }**, i32 }*, { 
    { float, float, float, float }**, i32 }** %color_structs2
  %numOfPoints65 = load i32, i32* %numOfPoints
  %points66 = load { float*, i32 }*, { float*,
    i32 }** %points
  %colors67 = load { float*, i32 }*, { float*,
    i32 }** %colors
  call void @drawHelper({ { float, float }**, i32 }* %point_structs63, { 
    { float, float, float, float }**, i32 }*
        %color_structs64, i32 
    %numOfPoints65, i32 0, { float*, i32 }*
        %points66, { float*, i32 }* %colors67)
  %points68 = load { float*, i32 }*, { float*,
    i32 }** %points
  %colors69 = load { float*, i32 }*, { float*,
    i32 }** %colors
  %colorMode70 = load i32, i32* %colorMode3
  %filled71 = load i32, i32* %filled4
  call void @gl_drawShape({ float*, i32 }* 
    %points68, { float*, i32 }* %colors69, i32 %colorMode70, i32 %filled71)
  ret void
}

define void @endCanvas({ i32, i32, i32 }* %c) {
entry:
  %c1 = alloca { i32, i32, i32 }*
  store { i32, i32, i32 }* %c, { i32, i32, i32 }** %c1
  %c2 = load { i32, i32, i32 }*, { i32, i32, i32 }** %c1
  %fieldadr = getelementptr { i32, i32, i32 }, { i32, i32, i32 }* %c2, i32 
    0, i32 0
  %width = load i32, i32* %fieldadr
  %c3 = load { i32, i32, i32 }*, { i32, i32, i32 }** %c1
  %fieldadr4 = getelementptr { i32, i32, i32 }, { i32, i32, i32 }* %c3, i32 
    0, i32 1
  %height = load i32, i32* %fieldadr4
  %c5 = load { i32, i32, i32 }*, { i32, i32, i32 }** %c1
  %fieldadr6 = getelementptr { i32, i32, i32 }, { i32, i32, i32 }* %c5, i32 
    0, i32 2
  %file_number = load i32, i32* %fieldadr6
  call void @gl_endRendering(i32 %width, i32 %height, i32 %file_number)
  ret void
}
\end{lstlisting}}

\subsubsection{dragon.ll}
\colorbox{blue!30}{dragon.sos}
	\begin{lstlisting}
import renderer.sos
import vector.sos
import transform.sos
import array.sos
import math.sos

// Creates a dragon curve of depth n
dragon: (n: int) -> path =
    if n == 0 // Base case
    then [point{0.0, 0.0}, point{1.0, 0.0}]
    else
    // Create two copies of the previous depths
    d1: path = dragon(n-1) ;
    d2: path = copy_path(d1) ;

    // Position d1
    s: float = sqrt(2.0)/2.0 ;
    rotate(d1, toradians(45.0), -1, {0.0, 0.0}) ;
    scale(d1, s, s) ;

    // Position d2
    rotate(d2, toradians(135.0), -1, {0., 0.}) ;
    scale(d2,s,s) ;
    trans(d2, {1., 0.}) ;
    reverse(d2) ;

    // Merge the paths
    r: path = append(d1, d2, 1.0) ;
    free_path(d1); free_path(d2); r

// Creates a rainbow color effect
rainbow: (r: int, len: int) -> color =
    h: float = (1.0*r)/len ;
    hsv(h, 0.8, 0.8)

// Render a 400px by 400px canvas, name the image pic0
my_canvas: canvas = {400, 400, 0}

// Start render
startCanvas(my_canvas)
d: path = dragon(7)
// Position the curve (0.4, 0.2 is approximately the center of mass of the 
//curve for large n)
trans(d, {-0.4, -0.2})

// Draw it
drawPath(d, rainbow(ints(d.length), d.length), 0)
endCanvas(my_canvas)
\end{lstlisting}

{\small
\colorbox{green!30}{dragon.ll}
\begin{lstlisting}
 ModuleID = 'SOS'
source_filename = "SOS"

declare i32 @printf(i8*, ...)

define i32 @main() {
entry:
  %my_canvas = alloca { i32, i32, i32 }*
  %malloccall = tail call i8* @malloc(i32 trunc (i64 mul nuw (i64 ptrtoint 
    (i32* getelementptr (i32, i32* null, i32 1) to i64), i64 3) to i32))
  %anon = bitcast i8* %malloccall to { i32, i32, i32 }*
  %fieldaddr = getelementptr { i32, i32, i32 }, { i32, i32, i32
    }* %anon, i32 0, i32 0
  store i32 400, i32* %fieldaddr
  %fieldaddr1 = getelementptr { i32, i32, i32 }, { i32, i32, i32
    }* %anon, i32 0, i32 1
  store i32 400, i32* %fieldaddr1
  %fieldaddr2 = getelementptr { i32, i32, i32 }, { i32, i32, i32
    }* %anon, i32 0, i32 2
  store i32 0, i32* %fieldaddr2
  store { i32, i32, i32 }* %anon, { i32, i32, i32 }** %my_canvas
  %my_canvas3 = load { i32, i32, i32 }*, { i32, i32, i32 }** %my_canvas
  call void @startCanvas({ i32, i32, i32 }* %my_canvas3)
  %d = alloca { { float, float }**, i32 }*
  %fxn_result = call { { float, float }**, i32 }* @dragon(i32 7)
  store { { float, float }**, i32 }* %fxn_result, { { float, 
    float }**, i32 }** %d
  %d4 = load { { float, float }**, i32 }*, { { float, float }**, 
    i32 }** %d
  %malloccall5 = tail call i8* @malloc(i32 trunc (i64 mul nuw 
    (i64 ptrtoint(float* getelementptr (float, float* null,
    i32 1) to i64), i64 2) to i32))
  %anon6 = bitcast i8* %malloccall5 to { float, float }*
  %fieldaddr7 = getelementptr { float, float }, { float, 
    float }* %anon6, i32 0, i32 0
  store float 0xBFD99999A0000000, float* %fieldaddr7
  %fieldaddr8 = getelementptr { float, float }, { float, 
    float }* %anon6, i32 0, i32 1
  store float 0xBFC99999A0000000, float* %fieldaddr8
  %lenref = getelementptr { { float, float }**, i32 }, { {
    float, float }**, i32 }* %d4, i32 0, i32 1
  %len = load i32, i32* %lenref
  %dataref = getelementptr { { float, float }**, i32 }, {
    { float, float }**, i32 }* %d4, i32 0, i32 0
  %data = load { float, float }**, { float, float }*** %dataref
  %i = alloca i32
  store i32 0, i32* %i
  br label %loop

loop:                                             ; preds = %loop, %entry
  %i9 = load i32, i32* %i
  %elref = getelementptr { float, float }*, { float, float }** %data,
    i32 %i9
  %el = load { float, float }*, { float, float }** %elref
  call void @trans({ float, float }* %el, { float, float }* %anon6)
  %i10 = add i32 %i9, 1
  store i32 %i10, i32* %i
  %i11 = load i32, i32* %i
  %tmp = icmp slt i32 %i11, %len
  br i1 %tmp, label %loop, label %continue

continue:                                         ; preds = %loop
  %d12 = load { { float, float }**, i32 }*, { { float,
    float }**, i32 }** %d
  %d13 = load { { float, float }**, i32 }*, { { float, 
    float }**, i32 }** %d
  %lenref14 = getelementptr { { float, float }**, i32 }, 
    { { float, float }**i32 }* %d13, i32 0, i32 1
  %len15 = load i32, i32* %lenref14
  %fxn_result16 = call { i32*, i32 }* @ints(i32 %len15)
  %d17 = load { { float, float }**, i32 }*, { { float, float }**, i32 }** %d
  %lenref18 = getelementptr { { float, float }**, i32 },
    { { float, float }**,i32 }* %d17, i32 0, i32 1
  %len19 = load i32, i32* %lenref18
  %lenref20 = getelementptr { i32*, i32 }, { i32*, i32 }*
    %fxn_result16, i32 0, i32 1
  %len21 = load i32, i32* %lenref20
  %dataref22 = getelementptr { i32*, i32 }, { i32*, i32 }*
    %fxn_result16, i32 0, i32 0
  %data23 = load i32*, i32** %dataref22
  %mallocsize = mul i32 %len21, ptrtoint (i1** 
    getelementptr (i1*, i1** null,i32 1) to i32)
  %malloccall24 = tail call i8* @malloc(i32 %mallocsize)
  %arrdata = bitcast i8* %malloccall24 to { float, float,
    float, float }**
  %i25 = alloca i32
  store i32 0, i32* %i25
  br label %loop26

loop26:                                           
    ; preds = %loop26, %continue
  %i28 = load i32, i32* %i25
  %elref29 = getelementptr i32, i32* %data23, i32 %i28
  %el30 = load i32, i32* %elref29
  %fxn_result31 = call { float, float, float, float 
    }* @rainbow(i32 %el30, i32 %len19)
  %storeref = getelementptr { float, float, float,
    float }*, { float, 
    float, float, float }** %arrdata, i32 %i28
  store { float, float, float, float }* %fxn_result31,
    { float, float, float, float }** %storeref
  %i32 = add i32 %i28, 1
  store i32 %i32, i32* %i25
  %i33 = load i32, i32* %i25
  %tmp34 = icmp slt i32 %i33, %len21
  br i1 %tmp34, label %loop26, label %continue27

continue27:                                       ; preds = %loop26
  %malloccall35 = tail call i8* @malloc(i32 ptrtoint
    ({ { float, float, float, float }**, i32 }* getelementptr
    ({ { float, float, float, float }**, i32 }, { 
    { float, float, float, float }**, i32 }* null, i32 1) to i32))
  %arr = bitcast i8* %malloccall35 to { { float, float,
    float, float 
    }**, i32 }*
  %arrdata36 = getelementptr { { float, float, float, 
    float }**, i32 }, { { float, float, float, 
    float }**, i32 }* %arr, i32 0, i32 0
  %arrlen = getelementptr { { float, float, float,float }**, i32 }, 
    { { float, float, float, float }**, i32 }* %arr, i32 0, i32 1
  store { float, float, float, float }** %arrdata, { float, float, 
    float, float }*** %arrdata36
  store i32 %len21, i32* %arrlen
  call void @drawPath({ { float, float }**, i32 }* %d12, { { float, 
    float, float, float }**, i32 }* %arr, i32 0)
  %my_canvas37 = load { i32, i32, i32 }*, { i32, i32, i32 }** %my_canvas
  call void @endCanvas({ i32, i32, i32 }* %my_canvas37)
  ret i32 0
}

declare float @sqrtf(float)

declare float @sinf(float)

declare float @cosf(float)

declare float @tanf(float)

declare float @asinf(float)

declare float @acosf(float)

declare float @atanf(float)

declare float @toradiansf(float)

declare void @gl_startRendering(i32, i32)

declare void @gl_endRendering(i32, i32, i32)

declare void @gl_drawCurve({ float*, i32 }*, { float*, i32 }*, i32)

declare void @gl_drawShape({ float*, i32 }*, { float*, i32 }*, i32, i32)

declare void @gl_drawPoint({ float*, i32 }*, { float*, i32 }*, i32)

define float @floor(float %x) {
entry:
  %x1 = alloca float
  store float %x, float* %x1
  %z = alloca float
  %y = alloca i32
  %x2 = load float, float* %x1
  %cast = fptosi float %x2 to i32
  store i32 %cast, i32* %y
  %cast3 = sitofp i32 %cast to float
  store float %cast3, float* %z
  %z4 = load float, float* %z
  %x5 = load float, float* %x1
  %tmp = fcmp ole float %z4, %x5
  %if_tmp = alloca float
  br i1 %tmp, label %then, label %else

merge:                                            ; preds = %else, %then
  %if_tmp9 = load float, float* %if_tmp
  ret float %if_tmp9

then:                                             ; preds = %entry
  %z6 = load float, float* %z
  store float %z6, float* %if_tmp
  br label %merge

else:                                             ; preds = %entry
  %z7 = load float, float* %z
  %tmp8 = fsub float %z7, 1.000000e+00
  store float %tmp8, float* %if_tmp
  br label %merge
}

define float @ceil(float %x) {
entry:
  %x1 = alloca float
  store float %x, float* %x1
  %x2 = load float, float* %x1
  %tmp = fneg float %x2
  %fxn_result = call float @floor(float %tmp)
  %tmp3 = fneg float %fxn_result
  ret float %tmp3
}

define float @frac(float %x) {
entry:
  %x1 = alloca float
  store float %x, float* %x1
  %x2 = load float, float* %x1
  %x3 = load float, float* %x1
  %fxn_result = call float @floor(float %x3)
  %tmp = fsub float %x2, %fxn_result
  ret float %tmp
}

define float @max(float %a, float %b) {
entry:
  %a1 = alloca float
  store float %a, float* %a1
  %b2 = alloca float
  store float %b, float* %b2
  %a3 = load float, float* %a1
  %b4 = load float, float* %b2
  %tmp = fcmp olt float %a3, %b4
  %if_tmp = alloca float
  br i1 %tmp, label %then, label %else

merge:                                            ; preds = %else, %then
  %if_tmp7 = load float, float* %if_tmp
  ret float %if_tmp7

then:                                             ; preds = %entry
  %b5 = load float, float* %b2
  store float %b5, float* %if_tmp
  br label %merge

else:                                             ; preds = %entry
  %a6 = load float, float* %a1
  store float %a6, float* %if_tmp
  br label %merge
}

define float @min(float %a, float %b) {
entry:
  %a1 = alloca float
  store float %a, float* %a1
  %b2 = alloca float
  store float %b, float* %b2
  %a3 = load float, float* %a1
  %b4 = load float, float* %b2
  %tmp = fcmp olt float %a3, %b4
  %if_tmp = alloca float
  br i1 %tmp, label %then, label %else

merge:                                            ; preds = %else, %then
  %if_tmp7 = load float, float* %if_tmp
  ret float %if_tmp7

then:                                             ; preds = %entry
  %a5 = load float, float* %a1
  store float %a5, float* %if_tmp
  br label %merge

else:                                             ; preds = %entry
  %b6 = load float, float* %b2
  store float %b6, float* %if_tmp
  br label %merge
}

define float @clamp(float %x, float %m, float %M) {
entry:
  %x1 = alloca float
  store float %x, float* %x1
  %m2 = alloca float
  store float %m, float* %m2
  %M3 = alloca float
  store float %M, float* %M3
  %M4 = load float, float* %M3
  %x5 = load float, float* %x1
  %m6 = load float, float* %m2
  %fxn_result = call float @max(float %x5, float %m6)
  %fxn_result7 = call float @min(float %M4, float %fxn_result)
  ret float %fxn_result7
}

define float @abs(float %x) {
entry:
  %x1 = alloca float
  store float %x, float* %x1
  %x2 = load float, float* %x1
  %tmp = fcmp olt float %x2, 0.000000e+00
  %if_tmp = alloca float
  br i1 %tmp, label %then, label %else

merge:                                            ; preds = %else, %then
  %if_tmp6 = load float, float* %if_tmp
  ret float %if_tmp6

then:                                             ; preds = %entry
  %x3 = load float, float* %x1
  %tmp4 = fneg float %x3
  store float %tmp4, float* %if_tmp
  br label %merge

else:                                             ; preds = %entry
  %x5 = load float, float* %x1
  store float %x5, float* %if_tmp
  br label %merge
}

define float @modf(float %x, float %m) {
entry:
  %x1 = alloca float
  store float %x, float* %x1
  %m2 = alloca float
  store float %m, float* %m2
  %m3 = load float, float* %m2
  %x4 = load float, float* %x1
  %m5 = load float, float* %m2
  %tmp = fdiv float %x4, %m5
  %fxn_result = call float @frac(float %tmp)
  %tmp6 = fmul float %m3, %fxn_result
  ret float %tmp6
}

define float @sin(float %x) {
entry:
  %x1 = alloca float
  store float %x, float* %x1
  %x2 = load float, float* %x1
  %fxn_result = call float @sinf(float %x2)
  ret float %fxn_result
}

define float @cos(float %x) {
entry:
  %x1 = alloca float
  store float %x, float* %x1
  %x2 = load float, float* %x1
  %fxn_result = call float @cosf(float %x2)
  ret float %fxn_result
}

define float @tan(float %x) {
entry:
  %x1 = alloca float
  store float %x, float* %x1
  %x2 = load float, float* %x1
  %fxn_result = call float @tanf(float %x2)
  ret float %fxn_result
}

define float @asin(float %x) {
entry:
  %x1 = alloca float
  store float %x, float* %x1
  %x2 = load float, float* %x1
  %fxn_result = call float @asinf(float %x2)
  ret float %fxn_result
}

define float @acos(float %x) {
entry:
  %x1 = alloca float
  store float %x, float* %x1
  %x2 = load float, float* %x1
  %fxn_result = call float @acosf(float %x2)
  ret float %fxn_result
}

define float @atan(float %x) {
entry:
  %x1 = alloca float
  store float %x, float* %x1
  %x2 = load float, float* %x1
  %fxn_result = call float @atanf(float %x2)
  ret float %fxn_result
}

define float @sqrt(float %x) {
entry:
  %x1 = alloca float
  store float %x, float* %x1
  %x2 = load float, float* %x1
  %fxn_result = call float @sqrtf(float %x2)
  ret float %fxn_result
}

define float @toradians(float %x) {
entry:
  %x1 = alloca float
  store float %x, float* %x1
  %x2 = load float, float* %x1
  %fxn_result = call float @toradiansf(float %x2)
  ret float %fxn_result
}

define float @sqrMagnitude({ float, float }* %p) {
entry:
  %p1 = alloca { float, float }*
  store { float, float }* %p, { float, float }** %p1
  %p2 = load { float, float }*, { float, float }** %p1
  %p3 = load { float, float }*, { float, float }** %p1
  %result = call float @__dotf2({ float, float }* %p2, { float, float }* %p3)
  ret float %result
}

define float @__dotf2({ float, float }* %a, { float, float }* %b) {
entry:
  %a1 = alloca { float, float }*
  store { float, float }* %a, { float, float }** %a1
  %a2 = load { float, float }*, { float, float }** %a1
  %b3 = alloca { float, float }*
  store { float, float }* %b, { float, float }** %b3
  %b4 = load { float, float }*, { float, float }** %b3
  %dot = alloca float
  %tmp = alloca float
  store float 0.000000e+00, float* %dot
  %avalref = getelementptr { float, float }, { float, float }*
    %a2, i32 0, i32 0
  %aval = load float, float* %avalref
  %bvalref = getelementptr { float, float }, { float, float }*
    %b4, i32 0, i32 0
  %bval = load float, float* %bvalref
  %tmp5 = fmul float %aval, %bval
  store float %tmp5, float* %tmp
  %tmp6 = load float, float* %tmp
  %res = load float, float* %dot
  %tmp7 = fadd float %tmp6, %res
  store float %tmp7, float* %dot
  %avalref8 = getelementptr { float, float }, { float, float }*
    %a2, i32 0, i32 1
  %aval9 = load float, float* %avalref8
  %bvalref10 = getelementptr { float, float }, { float, float }*
    %b4, i32 0, i32 1
  %bval11 = load float, float* %bvalref10
  %tmp12 = fmul float %aval9, %bval11
  store float %tmp12, float* %tmp
  %tmp13 = load float, float* %tmp
  %res14 = load float, float* %dot
  %tmp15 = fadd float %tmp13, %res14
  store float %tmp15, float* %dot
  %res16 = load float, float* %dot
  ret float %res16
}

define float @magnitude({ float, float }* %p) {
entry:
  %p1 = alloca { float, float }*
  store { float, float }* %p, { float, float }** %p1
  %p2 = load { float, float }*, { float, float }** %p1
  %fxn_result = call float @sqrMagnitude({ float, float }* %p2)
  %fxn_result3 = call float @sqrt(float %fxn_result)
  ret float %fxn_result3
}

define float @sqrDistance({ float, float }* %a, { float, 
    float }* %b) {
entry:
  %a1 = alloca { float, float }*
  store { float, float }* %a, { float, float }** %a1
  %b2 = alloca { float, float }*
  store { float, float }* %b, { float, float }** %b2
  %p = alloca { float, float }*
  %a3 = load { float, float }*, { float, float }** %a1
  %b4 = load { float, float }*, { float, float }** %b2
  %result = call { float, float }* @__subf2({ float, float }* 
    %a3, { float, float }* %b4)
  store { float, float }* %result, { float, float }** %p
  %d = alloca float
  %p5 = load { float, float }*, { float, float }** %p
  %fxn_result = call float @sqrMagnitude({ float, float }* %p5)
  store float %fxn_result, float* %d
  %p6 = load { float, float }*, { float, float }** %p
  %0 = bitcast { float, float }* %p6 to i8*
  tail call void @free(i8* %0)
  %d7 = load float, float* %d
  ret float %d7
}

define { float, float }* @__subf2({ float, float }* %a, { float, float }* %b) {
entry:
  %a1 = alloca { float, float }*
  store { float, float }* %a, { float, float }** %a1
  %a2 = load { float, float }*, { float, float }** %a1
  %b3 = alloca { float, float }*
  store { float, float }* %b, { float, float }** %b3
  %b4 = load { float, float }*, { float, float }** %b3
  %malloccall = tail call i8* @malloc(i32 trunc (i64 mul nuw (i64 
    ptrtoint (float* getelementptr (float, float* null, i32 1) to 
    i64), i64 2) to i32))
  %ret = bitcast i8* %malloccall to { float, float }*
  %avalref = getelementptr { float, float }, { float, float }*
    %a2, i32 0, i32 0
  %aval = load float, float* %avalref
  %bvalref = getelementptr { float, float }, { float, float }*
    %b4, i32 0, i32 0
  %bval = load float, float* %bvalref
  %tmp = fsub float %aval, %bval
  %ref = getelementptr { float, float }, { float, float }* %ret,
    i32 0, i32 0
  store float %tmp, float* %ref
  %avalref5 = getelementptr { float, float }, { float, float }*
    %a2, i32 0, i32 1
  %aval6 = load float, float* %avalref5
  %bvalref7 = getelementptr { float, float }, { float, float }*
    %b4, i32 0, i32 1
  %bval8 = load float, float* %bvalref7
  %tmp9 = fsub float %aval6, %bval8
  %ref10 = getelementptr { float, float }, { float, float }* %ret,
    i32 0, i32 1
  store float %tmp9, float* %ref10
  ret { float, float }* %ret
}

declare noalias i8* @malloc(i32)

declare void @free(i8*)

define float @distance({ float, float }* %a, { float, float }* %b) {
entry:
  %a1 = alloca { float, float }*
  store { float, float }* %a, { float, float }** %a1
  %b2 = alloca { float, float }*
  store { float, float }* %b, { float, float }** %b2
  %a3 = load { float, float }*, { float, float }** %a1
  %b4 = load { float, float }*, { float, float }** %b2
  %fxn_result = call float @sqrDistance({ float, float }* %a3, 
    { float, float }* %b4)
  %fxn_result5 = call float @sqrt(float %fxn_result)
  ret float %fxn_result5
}

define { float, float }* @copy_point({ float, float }* %p) {
entry:
  %p1 = alloca { float, float }*
  store { float, float }* %p, { float, float }** %p1
  %p2 = load { float, float }*, { float, float }** %p1
  %copied = call { float, float }* @__copy2({ float, float }* %p2)
  ret { float, float }* %copied
}

define { float, float }* @__copy2({ float, float }* %to_copy) {
entry:
  %to_copy1 = alloca { float, float }*
  store { float, float }* %to_copy, { float, float }** %to_copy1
  %to_copy2 = load { float, float }*, { float, float }** %to_copy1
  %malloccall = tail call i8* @malloc(i32 trunc (i64 mul nuw (i64 
    ptrtoint (float* getelementptr (float, float* null, i32 1) to 
    i64), i64 2) to i32))
  %struct = bitcast i8* %malloccall to { float, float }*
  %flref = getelementptr { float, float }, { float, float }* %to_copy2,
    i32 0, i32 0
  %fl = load float, float* %flref
  %ref = getelementptr { float, float }, { float, float }*
    %struct, i32 0, i32 0
  store float %fl, float* %ref
  %flref3 = getelementptr { float, float }, { float, float }*
    %to_copy2, i32 0, i32 1
  %fl4 = load float, float* %flref3
  %ref5 = getelementptr { float, float }, { float, float }*
    %struct, i32 0, i32 1
  store float %fl4, float* %ref5
  ret { float, float }* %struct
}

define void @free_point({ float, float }* %p) {
entry:
  %p1 = alloca { float, float }*
  store { float, float }* %p, { float, float }** %p1
  %p2 = load { float, float }*, { float, float }** %p1
  %0 = bitcast { float, float }* %p2 to i8*
  tail call void @free(i8* %0)
  ret void
}

define { { float, float }**, i32 }* @copy_path({ { float, 
    float }**, i32 }* %p) {
entry:
  %p1 = alloca { { float, float }**, i32 }*
  store { { float, float }**, i32 }* %p, { { float, float }**, i32 }** %p1
  %p2 = load { { float, float }**, i32 }*, { { float, float }**, i32 }** %p1
  %lenref = getelementptr { { float, float }**, i32 }, { { float, float
    }**, i32 }* %p2, i32 0, i32 1
  %len = load i32, i32* %lenref
  %dataref = getelementptr { { float, float }**, i32 }, { { float, float 
    }**, i32 }* %p2, i32 0, i32 0
  %data = load { float, float }**, { float, float }*** %dataref
  %mallocsize = mul i32 %len, ptrtoint (i1** getelementptr (i1*, i1** 
    null, i32 1) to i32)
  %malloccall = tail call i8* @malloc(i32 %mallocsize)
  %arrdata = bitcast i8* %malloccall to { float, float }**
  %i = alloca i32
  store i32 0, i32* %i
  br label %loop

loop:                                             ; preds = %loop, %entry
  %i3 = load i32, i32* %i
  %elref = getelementptr { float, float }*, { float, 
    float }** %data, i32 %i3
  %el = load { float, float }*, { float, float }** %elref
  %fxn_result = call { float, float }* @copy_point({ float,
    float }* %el)
  %storeref = getelementptr { float, float }*, { float, float
    }** %arrdata, i32 %i3
  store { float, float }* %fxn_result, { float, float }** %storeref
  %i4 = add i32 %i3, 1
  store i32 %i4, i32* %i
  %i5 = load i32, i32* %i
  %tmp = icmp slt i32 %i5, %len
  br i1 %tmp, label %loop, label %continue

continue:                                         ; preds = %loop
  %malloccall6 = tail call i8* @malloc(i32 ptrtoint ({ { 
    float, float }**,
    i32 }* getelementptr ({ { float, float }**, i32 }, { { float, float }**,
    i32 }* null, i32 1) to i32))
  %arr = bitcast i8* %malloccall6 to { { float, float }**, i32 }*
  %arrdata7 = getelementptr { { float, float }**, i32 }, 
    { { float, float }**,i32 }* %arr, i32 0, i32 0
  %arrlen = getelementptr { { float, float }**, i32 }, 
    { { float, float }**, i32 }* %arr, i32 0, i32 1
  store { float, float }** %arrdata, { float,
    float }*** %arrdata7
  store i32 %len, i32* %arrlen
  ret { { float, float }**, i32 }* %arr
}

define void @free_path({ { float, float }**, i32 }* %p) {
entry:
  %p1 = alloca { { float, float }**, i32 }*
  store { { float, float }**, i32 }* %p, { { float, 
    float }**, i32 }** %p1
  %p2 = load { { float, float }**, i32 }*, { { float,
    float }**, i32 }** %p1
  %lenref = getelementptr { { float, float }**, i32 }, 
    { { float, float }**, 
    i32 }* %p2, i32 0, i32 1
  %len = load i32, i32* %lenref
  %dataref = getelementptr { { float, float }**, i32 },
    { { float, float }**, 
    i32 }* %p2, i32 0, i32 0
  %data = load { float, float }**, { float, float }*** %dataref
  %i = alloca i32
  store i32 0, i32* %i
  br label %loop

loop:                                             ; preds = %loop, %entry
  %i3 = load i32, i32* %i
  %elref = getelementptr { float, float }*, { float,
    float }** %data, i32 %i3
  %el = load { float, float }*, { float, float }** %elref
  call void @free_point({ float, float }* %el)
  %i4 = add i32 %i3, 1
  store i32 %i4, i32* %i
  %i5 = load i32, i32* %i
  %tmp = icmp slt i32 %i5, %len
  br i1 %tmp, label %loop, label %continue

continue:                                         ; preds = %loop
  ret void
}

define void @appendhelp_copyin({ { float, float }**, i32 }* %in, { { float,
    float }**, i32 }* %from, i32 %i) {
entry:
  %in1 = alloca { { float, float }**, i32 }*
  store { { float, float }**, i32 }* %in, { { float, float }**, i32 }** %in1
  %from2 = alloca { { float, float }**, i32 }*
  store { { float, float }**, i32 }* %from, { { float, float }**, i32 }** %from2
  %i3 = alloca i32
  store i32 %i, i32* %i3
  %i4 = load i32, i32* %i3
  %in5 = load { { float, float }**, i32 }*, { { float, float }**, i32 }** %in1
  %lenref = getelementptr { { float, float }**, i32 }, { { float, float }**,
    i32 }* %in5, i32 0, i32 1
  %len = load i32, i32* %lenref
  %tmp = icmp slt i32 %i4, %len
  br i1 %tmp, label %then, label %else

merge:                                            ; preds = %else, %then
  ret void

then:                                             ; preds = %entry
  %in6 = load { { float, float }**, i32 }*, { { float, float }**, i32 }** %in1
  %datarefref = getelementptr { { float, float }**, i32 }, { { float, float
    }**, i32 }* %in6, i32 0, i32 0
  %dataref = load { float, float }**, { float, float }*** %datarefref
  %i7 = load i32, i32* %i3
  %from8 = load { { float, float }**, i32 }*, { { float, float }**, i32
    }** %from2
  %i9 = load i32, i32* %i3
  %tmp10 = add i32 %i9, 1
  %dataref11 = getelementptr { { float, float }**, i32 }, { { float, float
    }**, i32 }* %from8, i32 0, i32 0
  %data = load { float, float }**, { float, float }*** %dataref11
  %elref = getelementptr { float, float }*, { float, float }** %data, i32 %tmp10
  %el = load { float, float }*, { float, float }** %elref
  %copied = call { float, float }* @__copy2.1({ float, float }* %el)
  %storeref = getelementptr { float, float }*, { float, float }** %dataref, 
    i32 %i7
  store { float, float }* %copied, { float, float }** %storeref
  %in12 = load { { float, float }**, i32 }*, { { float, float }**, i32 }**
    %in1
  %from13 = load { { float, float }**, i32 }*, { { float, float }**, i32 }**
    %from2
  %i14 = load i32, i32* %i3
  %tmp15 = add i32 %i14, 1
  call void @appendhelp_copyin({ { float, float }**, i32 }* %in12, { 
    { float, float }**, i32 }* %from13, i32 %tmp15)
  br label %merge

else:                                             ; preds = %entry
  br label %merge
}

define { float, float }* @__copy2.1({ float, float }* %to_copy) {
entry:
  %to_copy1 = alloca { float, float }*
  store { float, float }* %to_copy, { float, float }** %to_copy1
  %to_copy2 = load { float, float }*, { float, float }** %to_copy1
  %malloccall = tail call i8* @malloc(i32 trunc (i64 mul nuw (i64 ptrtoint
    (float* getelementptr (float, float* null, i32 1) to i64), i64 2) to i32))
  %struct = bitcast i8* %malloccall to { float, float }*
  %flref = getelementptr { float, float }, { float, float }* %to_copy2, i32 0, i32 0
  %fl = load float, float* %flref
  %ref = getelementptr { float, float }, { float, float }* %struct, i32 0, i32 0
  store float %fl, float* %ref
  %flref3 = getelementptr { float, float }, { float, float }* %to_copy2, i32 0, i32 1
  %fl4 = load float, float* %flref3
  %ref5 = getelementptr { float, float }, { float, float }* %struct, i32 0, i32 1
  store float %fl4, float* %ref5
  ret { float, float }* %struct
}

define { { float, float }**, i32 }* @appendhelp_tail({ { float, float }**, i32
    }* %p) {
entry:
  %p1 = alloca { { float, float }**, i32 }*
  store { { float, float }**, i32 }* %p, { { float, float }**, i32 }** %p1
  %tail = alloca { { float, float }**, i32 }*
  %p2 = load { { float, float }**, i32 }*, { { float, float }**, i32 }** %p1
  %lenref = getelementptr { { float, float }**, i32 }, { { float, float }**, 
    i32}* %p2, i32 0, i32 1
  %len = load i32, i32* %lenref
  %tmp = sub i32 %len, 1
  %malloccall = tail call i8* @malloc(i32 ptrtoint (i1** getelementptr
    (i1*, i1** null, i32 1) to i32))
  %arrdata = bitcast i8* %malloccall to { float, float }**
  %malloccall3 = tail call i8* @malloc(i32 trunc (i64 mul nuw (i64 ptrtoint
    (float* getelementptr (float, float* null, i32 1) to i64), i64 2)to i32))
  %anon = bitcast i8* %malloccall3 to { float, float }*
  %fieldaddr = getelementptr { float, float }, { float, float }* %anon,
    i32 0, i32 0
  store float 0.000000e+00, float* %fieldaddr
  %fieldaddr4 = getelementptr { float, float }, { float, float }* %anon,
    i32 0, i32 1
  store float 0.000000e+00, float* %fieldaddr4
  %storeref = getelementptr { float, float }*, { float, float }** %arrdata, i32 0
  store { float, float }* %anon, { float, float }** %storeref
  %malloccall5 = tail call i8* @malloc(i32 ptrtoint ({ { float, float 
    }**, i32 }* getelementptr ({ { float, float }**, i32 }, { { float, 
    float }**, i32 }* null, i32 1) to i32))
  %arr = bitcast i8* %malloccall5 to { { float, float }**, i32 }*
  %arrdata6 = getelementptr { { float, float }**, i32 }, { { float, 
    float }**, i32 }* %arr, i32 0, i32 0
  %arrlen = getelementptr { { float, float }**, i32 }, { { float, 
    float }**, i32 }* %arr, i32 0, i32 1
  store { float, float }** %arrdata, { float, float }*** %arrdata6
  store i32 1, i32* %arrlen
  %lenref7 = getelementptr { { float, float }**, i32 }, { { float, 
    float }**, i32 }* %arr, i32 0, i32 1
  %len8 = load i32, i32* %lenref7
  %oflen = mul i32 %tmp, %len8
  %olddataref = getelementptr { { float, float }**, i32 }, { { float, 
    float }**, i32 }* %arr, i32 0, i32 0
  %olddata = load { float, float }**, { float, float }*** %olddataref
  %mallocsize = mul i32 %oflen, ptrtoint (i1** getelementptr (i1*, 
    i1** null, i32 1) to i32)
  %malloccall9 = tail call i8* @malloc(i32 %mallocsize)
  %arrdata10 = bitcast i8* %malloccall9 to { float, float }**
  %i = alloca i32
  store i32 0, i32* %i
  %j = alloca i32
  store i32 0, i32* %j
  br label %inner

loop:                                             ; preds = %inner
  %i18 = load i32, i32* %i
  store i32 0, i32* %j
  %tmp19 = icmp slt i32 %i18, %oflen
  br i1 %tmp19, label %inner, label %continue

inner:                                            
    ; preds = %loop, %inner, %entry
  %i11 = load i32, i32* %j
  %i12 = load i32, i32* %i
  %elref = getelementptr { float, float }*, { float, float }** %olddata,
    i32 %i11
  %el = load { float, float }*, { float, float }** %elref
  %storeref13 = getelementptr { float, float }*, { float, float }** 
    %arrdata10, i32 %i12
  store { float, float }* %el, { float, float }** %storeref13
  %i14 = add i32 %i12, 1
  store i32 %i14, i32* %i
  %j15 = add i32 %i11, 1
  store i32 %j15, i32* %j
  %j16 = load i32, i32* %j
  %tmp17 = icmp slt i32 %j16, %len8
  br i1 %tmp17, label %inner, label %loop

continue:                                         ; preds = %loop
  %malloccall20 = tail call i8* @malloc(i32 ptrtoint ({ { float, 
    float }**, i32 }* getelementptr ({ { float, float }**, i32 }, 
    { { float, float }**, i32 }* null, i32 1) to i32))
  %arr21 = bitcast i8* %malloccall20 to { { float, float }**, i32 }*
  %arrdata22 = getelementptr { { float, float }**, i32 }, { { float,
    float }**, i32 }* %arr21, i32 0, i32 0
  %arrlen23 = getelementptr { { float, float }**, i32 }, { { float, 
    float }**, i32 }* %arr21, i32 0, i32 1
  store { float, float }** %arrdata10, { float, float }*** %arrdata22
  store i32 %oflen, i32* %arrlen23
  store { { float, float }**, i32 }* %arr21, { { float, float }**,
    i32 }** %tail
  %tail24 = load { { float, float }**, i32 }*, { { float, float }**,
    i32 }** %tail
  %p25 = load { { float, float }**, i32 }*, { { float, float }**, 
    i32 }** %p1
  call void @appendhelp_copyin({ { float, float }**, i32 }* %tail24,
    { { float, float }**, i32 }* %p25, i32 0)
  %tail26 = load { { float, float }**, i32 }*, { { float, float }**,
    i32 }** %tail
  ret { { float, float }**, i32 }* %tail26
}

define { { float, float }**, i32 }* @append({ { float, float }**, i32
    }* %p1, { { float, float }**, i32 }* %p2, float %epsilon) {
entry:
  %p11 = alloca { { float, float }**, i32 }*
  store { { float, float }**, i32 }* %p1, { { float, float }**, i32
    }** %p11
  %p22 = alloca { { float, float }**, i32 }*
  store { { float, float }**, i32 }* %p2, { { float, float }**, i32
    }** %p22
  %epsilon3 = alloca float
  store float %epsilon, float* %epsilon3
  %p14 = load { { float, float }**, i32 }*, { { float, float }**,
    i32 }** %p11
  %lenref = getelementptr { { float, float }**, i32 }, { { float,
    float }**, i32 }* %p14, i32 0, i32 1
  %len = load i32, i32* %lenref
  %tmp = icmp eq i32 %len, 0
  %if_tmp = alloca { { float, float }**, i32 }*
  br i1 %tmp, label %then, label %else

merge:                                            
    ; preds = %merge11, %then
  %if_tmp66 = load { { float, float }**, i32 }*, { { float, 
    float }**, i32 }** %if_tmp
  ret { { float, float }**, i32 }* %if_tmp66

then:                                             
    ; preds = %entry
  %p25 = load { { float, float }**, i32 }*, { { float, 
    float }**, i32 }** %p22
  %fxn_result = call { { float, float }**, i32 }* @copy_path({ 
    { float, float }**, i32 }* %p25)
  store { { float, float }**, i32 }* %fxn_result, { { float, 
    float }**, i32 }** %if_tmp
  br label %merge

else:                                             ; preds = %entry
  %p26 = load { { float, float }**, i32 }*, { { float, float 
    }**, i32 }** %p22
  %lenref7 = getelementptr { { float, float }**, i32 }, { { 
    float, float }**, i32 }* %p26, i32 0, i32 1
  %len8 = load i32, i32* %lenref7
  %tmp9 = icmp eq i32 %len8, 0
  %if_tmp10 = alloca { { float, float }**, i32 }*
  br i1 %tmp9, label %then12, label %else13

merge11:                                          
    ; preds = %contb, %then12
  %if_tmp65 = load { { float, float }**, i32 }*, { { float,
    float }**, i32 }** %if_tmp10
  store { { float, float }**, i32 }* %if_tmp65, { { float,
    float }**, i32 }** %if_tmp
  br label %merge

then12:                                           ; preds = %else
  %p114 = load { { float, float }**, i32 }*, { { float, 
    float }**, i32 }** %p11
  %fxn_result15 = call { { float, float }**, i32 }* @copy_path({
    { float, float }**, i32 }* %p114)
  store { { float, float }**, i32 }* %fxn_result15, { { float,
    float }**, i32 }** %if_tmp10
  br label %merge11

else13:                                           ; preds = %else
  %merge16 = alloca i1
  %p117 = load { { float, float }**, i32 }*, { { float,
    float }**, i32 }** %p11
  %p118 = load { { float, float }**, i32 }*, { { float, 
    float }**, i32 }** %p11
  %lenref19 = getelementptr { { float, float }**, i32 }, { { 
    float, float }**, i32 }* %p118, i32 0, i32 1
  %len20 = load i32, i32* %lenref19
  %tmp21 = sub i32 %len20, 1
  %dataref = getelementptr { { float, float }**, i32 }, { { float,
    float }**, i32 }* %p117, i32 0, i32 0
  %data = load { float, float }**, { float, float }***
    %dataref
  %elref = getelementptr { float, float }*, { float, float
    }** %data, i32 %tmp21
  %el = load { float, float }*, { float, float }** %elref
  %p222 = load { { float, float }**, i32 }*, { { float, 
    float }**, i32 }** %p22
  %dataref23 = getelementptr { { float, float }**, i32 }, 
    { { float,float }**, i32 }* %p222, i32 0, i32 0
  %data24 = load { float, float }**, { float, float 
    }*** %dataref23
  %elref25 = getelementptr { float, float }*, { float, float }** %data24, i32 0
  %el26 = load { float, float }*, { float, float }** %elref25
  %fxn_result27 = call float @sqrDistance({ float,
    float }* %el, { float, float }* %el26)
  %epsilon28 = load float, float* %epsilon3
  %epsilon29 = load float, float* %epsilon3
  %tmp30 = fmul float %epsilon28, %epsilon29
  %tmp31 = fcmp olt float %fxn_result27, %tmp30
  store i1 %tmp31, i1* %merge16
  %p2c = alloca { { float, float }**, i32 }*
  %merge32 = load i1, i1* %merge16
  %if_tmp33 = alloca { { float, float }**, i32 }*
  br i1 %merge32, label %then35, label %else36

merge34:                                          
    ; preds = %else36, %then35
  %if_tmp40 = load { { float, float }**, i32 }*,
    { { float, float }**,i32 }** %if_tmp33
  store { { float, float }**, i32 }* %if_tmp40, {
    { float, float }**, 
    i32 }** %p2c
  %ret = alloca { { float, float }**, i32 }*
  %p141 = load { { float, float }**, i32 }*, { { float, float }**, i32 }** %p11
  %fxn_result42 = call { { float, float }**, i32 }* @copy_path({ { float, 
    float }**, i32 }* %p141)
  %p2c43 = load { { float, float }**, i32 }*, { { float, float }**, i32 }** %p2c
  %fxn_result44 = call { { float, float }**, i32 }* @copy_path({ { float, 
    float }**, i32 }* %p2c43)
  %len1ref = getelementptr { { float, float }**, i32 }, { { float, float 
    }**, i32 }* %fxn_result42, i32 0, i32 1
  %len1 = load i32, i32* %len1ref
  %len2ref = getelementptr { { float, float }**, i32 }, { { float, float
    }**, i32 }* %fxn_result44, i32 0, i32 1
  %len2 = load i32, i32* %len2ref
  %n = add i32 %len1, %len2
  %data1ref = getelementptr { { float, float }**, i32 }, { { float, float
    }**, i32 }* %fxn_result42, i32 0, i32 0
  %data1 = load { float, float }**, { float, float }*** %data1ref
  %data2ref = getelementptr { { float, float }**, i32 }, { { float, float
    }**, i32 }* %fxn_result44, i32 0, i32 0
  %data2 = load { float, float }**, { float, float }*** %data2ref
  %mallocsize = mul i32 %n, ptrtoint (i1** getelementptr (i1*, i1** null, 
    i32 1) to i32)
  %malloccall = tail call i8* @malloc(i32 %mallocsize)
  %data45 = bitcast i8* %malloccall to { float, float }**
  %i = alloca i32
  store i32 0, i32* %i
  %j = alloca i32
  store i32 0, i32* %j
  br label %loop1

then35:                                           ; preds = %else13
  %p237 = load { { float, float }**, i32 }*, { { float, float }**, i32 }** %p22
  %fxn_result38 = call { { float, float }**, i32 }* @appendhelp_tail({
    { float, float }**, i32 }* %p237)
  store { { float, float }**, i32 }* %fxn_result38, { { float, float
    }**, i32 }** %if_tmp33
  br label %merge34

else36:                                           ; preds = %else13
  %p239 = load { { float, float }**, i32 }*, { { float, float }**,
    i32 }** %p22
  store { { float, float }**, i32 }* %p239, { { float, float }**, 
    i32 }** %if_tmp33
  br label %merge34

loop1:                                            ; preds = %loop1, %merge34
  %i46 = load i32, i32* %j
  %i47 = load i32, i32* %i
  %elref48 = getelementptr { float, float }*, { float, float }** 
    %data1, i32 %i46
  %el49 = load { float, float }*, { float, float }** %elref48
  %storeref = getelementptr { float, float }*, { float, float }**
    %data45, i32 %i47
  store { float, float }* %el49, { float, float }** %storeref
  %tmp50 = add i32 %i47, 1
  store i32 %tmp50, i32* %i
  %j51 = add i32 %i46, 1
  store i32 %j51, i32* %j
  %j52 = load i32, i32* %j
  %tmp53 = icmp slt i32 %j52, %len1
  br i1 %tmp53, label %loop1, label %inbtw

inbtw:                                            ; preds = %loop1
  store i32 0, i32* %j
  br label %loop2

loop2:                                            ; preds = %loop2, %inbtw
  %i54 = load i32, i32* %j
  %i55 = load i32, i32* %i
  %elref56 = getelementptr { float, float }*, { float, float }**
    %data2, i32 %i54
  %el57 = load { float, float }*, { float, float }** %elref56
  %storeref58 = getelementptr { float, float }*, { float, float }**
    %data45, i32 %i55
  store { float, float }* %el57, { float, float }** %storeref58
  %tmp59 = add i32 %i55, 1
  store i32 %tmp59, i32* %i
  %j60 = add i32 %i54, 1
  store i32 %j60, i32* %j
  %j61 = load i32, i32* %j
  %tmp62 = icmp slt i32 %j61, %len2
  br i1 %tmp62, label %loop2, label %contb

contb:                                            ; preds = %loop2
  %malloccall63 = tail call i8* @malloc(i32 ptrtoint ({ { float, float
    }**, i32 }* getelementptr ({ { float, float }**, i32 }, { { 
    float, float }**, i32 }* null, i32 1) to i32))
  %arr = bitcast i8* %malloccall63 to { { float, float }**, i32 }*
  %arrdata = getelementptr { { float, float }**, i32 }, { { float,
    float }**, i32 }* %arr, i32 0, i32 0
  %arrlen = getelementptr { { float, float }**, i32 }, { { float, 
    float }**, i32 }* %arr, i32 0, i32 1
  store { float, float }** %data45, { float, float }*** %arrdata
  store i32 %n, i32* %arrlen
  store { { float, float }**, i32 }* %arr, { { float, float }**, 
    i32 }** %ret
  %ret64 = load { { float, float }**, i32 }*, { { float, float }**,
    i32 }** %ret
  store { { float, float }**, i32 }* %ret64, { { float, float }**,
    i32 }** %if_tmp10
  br label %merge11
}

define void @reversedhelp({ { float, float }**, i32 }* %in, { {
    float, float }**, i32 }* %from, i32 %i) {
entry:
  %in1 = alloca { { float, float }**, i32 }*
  store { { float, float }**, i32 }* %in, { { float, float }**,
    i32 }** %in1
  %from2 = alloca { { float, float }**, i32 }*
  store { { float, float }**, i32 }* %from, { { float, float }**, i32 }** %from2
  %i3 = alloca i32
  store i32 %i, i32* %i3
  %i4 = load i32, i32* %i3
  %in5 = load { { float, float }**, i32 }*, { { float, float }**, i32 }** %in1
  %lenref = getelementptr { { float, float }**, i32 }, { { float,
    float }**, i32 }* %in5, i32 0, i32 1
  %len = load i32, i32* %lenref
  %tmp = icmp slt i32 %i4, %len
  br i1 %tmp, label %then, label %else

merge:                                            ; preds = %else, %then
  ret void

then:                                             ; preds = %entry
  %in6 = load { { float, float }**, i32 }*, { { float, float }**, i32 }** %in1
  %i7 = load i32, i32* %i3
  %dataref = getelementptr { { float, float }**, i32 }, { { float,
    float }**, i32 }* %in6, i32 0, i32 0
  %data = load { float, float }**, { float, float }*** %dataref
  %elref = getelementptr { float, float }*, { float, float }** %data, i32 %i7
  %el = load { float, float }*, { float, float }** %elref
  %from8 = load { { float, float }**, i32 }*, { { float, float }**,
    i32 }** %from2
  %in9 = load { { float, float }**, i32 }*, { { float, float }**,
    i32 }** %in1
  %lenref10 = getelementptr { { float, float }**, i32 }, { { float, 
    float }**, i32 }* %in9, i32 0, i32 1
  %len11 = load i32, i32* %lenref10
  %tmp12 = sub i32 %len11, 1
  %i13 = load i32, i32* %i3
  %tmp14 = sub i32 %tmp12, %i13
  %dataref15 = getelementptr { { float, float }**, i32 }, { { float,
    float }**, i32 }* %from8, i32 0, i32 0
  %data16 = load { float, float }**, { float, float }*** %dataref15
  %elref17 = getelementptr { float, float }*, { float, float }** %data16, i32
    %tmp14
  %el18 = load { float, float }*, { float, float }** %elref17
  %fieldadr = getelementptr { float, float }, { float, float }* %el18, i32 0, i32
    0
  %x = load float, float* %fieldadr
  %ref = getelementptr { float, float }, { float, float }* %el, i32 0, i32 0
  store float %x, float* %ref
  %in19 = load { { float, float }**, i32 }*, { { float, float }**, i32 }** %in1
  %i20 = load i32, i32* %i3
  %dataref21 = getelementptr { { float, float }**, i32 }, { { float,
    float }**, i32 }* %in19, i32 0, i32 0
  %data22 = load { float, float }**, { float, float }*** %dataref21
  %elref23 = getelementptr { float, float }*, { float, float }** %data22, i32 %i20
  %el24 = load { float, float }*, { float, float }** %elref23
  %from25 = load { { float, float }**, i32 }*, { { float, float }**,
    i32 }** %from2
  %in26 = load { { float, float }**, i32 }*, { { float, float }**, i32 }** %in1
  %lenref27 = getelementptr { { float, float }**, i32 }, { { float,
    float }**, i32 }* %in26, i32 0, i32 1
  %len28 = load i32, i32* %lenref27
  %tmp29 = sub i32 %len28, 1
  %i30 = load i32, i32* %i3
  %tmp31 = sub i32 %tmp29, %i30
  %dataref32 = getelementptr { { float, float }**, i32 }, { { float,
    float }**, i32 }* %from25, i32 0, i32 0
  %data33 = load { float, float }**, { float, float }*** %dataref32
  %elref34 = getelementptr { float, float }*, { float, float }** 
    %data33, i32 %tmp31
  %el35 = load { float, float }*, { float, float }** %elref34
  %fieldadr36 = getelementptr { float, float }, { float, float }* 
    %el35, i32 0, i32 1
  %y = load float, float* %fieldadr36
  %ref37 = getelementptr { float, float }, { float, float }* %el24, 
    i32 0, i32 1
  store float %y, float* %ref37
  %in38 = load { { float, float }**, i32 }*, { { float, float }**, 
    i32 }** %in1
  %from39 = load { { float, float }**, i32 }*, { { float, float }**,
    i32 }** %from2
  %i40 = load i32, i32* %i3
  %tmp41 = add i32 %i40, 1
  call void @reversedhelp({ { float, float }**, i32 }* %in38, { { 
    float, float }**, i32 }* %from39, i32 %tmp41)
  br label %merge

else:                                             ; preds = %entry
  br label %merge
}

define { { float, float }**, i32 }* @reversed({ { float, float }**, i32 }* %p) {
entry:
  %p1 = alloca { { float, float }**, i32 }*
  store { { float, float }**, i32 }* %p, { { float, float }**, i32 }** %p1
  %newpath = alloca { { float, float }**, i32 }*
  %p2 = load { { float, float }**, i32 }*, { { float, float }**, i32 }** %p1
  %lenref = getelementptr { { float, float }**, i32 }, { { float, 
    float }**, i32 }* %p2, i32 0, i32 1
  %len = load i32, i32* %lenref
  %malloccall = tail call i8* @malloc(i32 ptrtoint (i1** getelementptr
    (i1*, i1** null, i32 1) to i32))
  %arrdata = bitcast i8* %malloccall to { float, float }**
  %malloccall3 = tail call i8* @malloc(i32 trunc (i64 mul nuw (i64 
    ptrtoint (float* getelementptr (float, float* null, i32 1) to i64), i64 2) 
    to i32))
  %anon = bitcast i8* %malloccall3 to { float, float }*
  %fieldaddr = getelementptr { float, float }, { float, float }* %anon, i32 0, 
    i32 0
  store float 0.000000e+00, float* %fieldaddr
  %fieldaddr4 = getelementptr { float, float }, { float, float }* %anon, i32 0,
    i32 1
  store float 0.000000e+00, float* %fieldaddr4
  %storeref = getelementptr { float, float }*, { float, float }** %arrdata, i32
    0
  store { float, float }* %anon, { float, float }** %storeref
  %malloccall5 = tail call i8* @malloc(i32 ptrtoint ({ { float, 
    float }**, i32 }* getelementptr ({ { float, float }**, i32 }, 
    { { float, float }**, i32 }* null, i32 1) to i32))
  %arr = bitcast i8* %malloccall5 to { { float, float }**, i32 }*
  %arrdata6 = getelementptr { { float, float }**, i32 }, { { float,
    float }**, i32 }* %arr, i32 0, i32 0
  %arrlen = getelementptr { { float, float }**, i32 }, { { 
    float, float }**, i32 }* %arr, i32 0, i32 1
  store { float, float }** %arrdata, { float, float }*** %arrdata6
  store i32 1, i32* %arrlen
  %lenref7 = getelementptr { { float, float }**, i32 }, { { 
    float, float }**, i32 }* %arr, i32 0, i32 1
  %len8 = load i32, i32* %lenref7
  %oflen = mul i32 %len, %len8
  %olddataref = getelementptr { { float, float }**, i32 }, { { 
    float, float }**, i32 }* %arr, i32 0, i32 0
  %olddata = load { float, float }**, { float, float }*** %olddataref
  %mallocsize = mul i32 %oflen, ptrtoint (i1** getelementptr (i1*,
    i1** null, i32 1) to i32)
  %malloccall9 = tail call i8* @malloc(i32 %mallocsize)
  %arrdata10 = bitcast i8* %malloccall9 to { float, float }**
  %i = alloca i32
  store i32 0, i32* %i
  %j = alloca i32
  store i32 0, i32* %j
  br label %inner

loop:                                             ; preds = %inner
  %i17 = load i32, i32* %i
  store i32 0, i32* %j
  %tmp18 = icmp slt i32 %i17, %oflen
  br i1 %tmp18, label %inner, label %continue

inner:                                            ; preds = %loop, %inner, %entry
  %i11 = load i32, i32* %j
  %i12 = load i32, i32* %i
  %elref = getelementptr { float, float }*, { float, float }** %olddata, i32 %i11
  %el = load { float, float }*, { float, float }** %elref
  %storeref13 = getelementptr { float, float }*, { float, float }** %arrdata10, 
    i32 %i12
  store { float, float }* %el, { float, float }** %storeref13
  %i14 = add i32 %i12, 1
  store i32 %i14, i32* %i
  %j15 = add i32 %i11, 1
  store i32 %j15, i32* %j
  %j16 = load i32, i32* %j
  %tmp = icmp slt i32 %j16, %len8
  br i1 %tmp, label %inner, label %loop

continue:                                         ; preds = %loop
  %malloccall19 = tail call i8* @malloc(i32 ptrtoint ({ { float, float }**,
    i32 }* getelementptr ({ { float, float }**, i32 }, { { float, float }**,
    i32 }* null, i32 1) to i32))
  %arr20 = bitcast i8* %malloccall19 to { { float, float }**, i32 }*
  %arrdata21 = getelementptr { { float, float }**, i32 }, { { float, float 
    }**, i32 }* %arr20, i32 0, i32 0
  %arrlen22 = getelementptr { { float, float }**, i32 }, { { float, float 
    }**, i32 }* %arr20, i32 0, i32 1
  store { float, float }** %arrdata10, { float, float }*** %arrdata21
  store i32 %oflen, i32* %arrlen22
  store { { float, float }**, i32 }* %arr20, { { float, float }**, i32 }** 
    %newpath
  %newpath23 = load { { float, float }**, i32 }*, { { float, float }**, i32 }** 
    %newpath
  %p24 = load { { float, float }**, i32 }*, { { float, float }**, i32 }** %p1
  call void @reversedhelp({ { float, float }**, i32 }* %newpath23, { { float,
    float }**, i32 }* %p24, i32 0)
  %newpath25 = load { { float, float }**, i32 }*, { { float, float }**, i32 }** 
    %newpath
  ret { { float, float }**, i32 }* %newpath25
}

define void @reversehelp({ { float, float }**, i32 }* %p, i32 %i) {
entry:
  %p1 = alloca { { float, float }**, i32 }*
  store { { float, float }**, i32 }* %p, { { float, float }**, i32 }** %p1
  %i2 = alloca i32
  store i32 %i, i32* %i2
  %i3 = load i32, i32* %i2
  %p4 = load { { float, float }**, i32 }*, { { float, float }**, i32 }** %p1
  %lenref = getelementptr { { float, float }**, i32 }, { { float, float }**, 
    i32 }* %p4, i32 0, i32 1
  %len = load i32, i32* %lenref
  %tmp = sdiv i32 %len, 2
  %tmp5 = icmp slt i32 %i3, %tmp
  br i1 %tmp5, label %then, label %else

merge:                                            ; preds = %else, %then
  ret void

then:                                             ; preds = %entry
  %q = alloca { float, float }*
  %p6 = load { { float, float }**, i32 }*, { { float, float }**, i32 }** %p1
  %i7 = load i32, i32* %i2
  %dataref = getelementptr { { float, float }**, i32 }, { { float, float }**, 
    i32 }* %p6, i32 0, i32 0
  %data = load { float, float }**, { float, float }*** %dataref
  %elref = getelementptr { float, float }*, { float, float }** %data, i32 %i7
  %el = load { float, float }*, { float, float }** %elref
  store { float, float }* %el, { float, float }** %q
  %p8 = load { { float, float }**, i32 }*, { { float, float }**, i32 }** %p1
  %datarefref = getelementptr { { float, float }**, i32 }, { { float, float }**,
    i32 }* %p8, i32 0, i32 0
  %dataref9 = load { float, float }**, { float, float }*** %datarefref
  %i10 = load i32, i32* %i2
  %p11 = load { { float, float }**, i32 }*, { { float, float }**, i32 }** %p1
  %p12 = load { { float, float }**, i32 }*, { { float, float }**, i32 }** %p1
  %lenref13 = getelementptr { { float, float }**, i32 }, { { float, float }**,
    i32 }* %p12, i32 0, i32 1
  %len14 = load i32, i32* %lenref13
  %tmp15 = sub i32 %len14, 1
  %i16 = load i32, i32* %i2
  %tmp17 = sub i32 %tmp15, %i16
  %dataref18 = getelementptr { { float, float }**, i32 }, { { float, float }**,
    i32 }* %p11, i32 0, i32 0
  %data19 = load { float, float }**, { float, float }*** %dataref18
  %elref20 = getelementptr { float, float }*, { float, float }** %data19, i32
    %tmp17
  %el21 = load { float, float }*, { float, float }** %elref20
  %storeref = getelementptr { float, float }*, { float, float }** %dataref9, 
    i32 %i10
  store { float, float }* %el21, { float, float }** %storeref
  %p22 = load { { float, float }**, i32 }*, { { float, float }**, i32 }** %p1
  %datarefref23 = getelementptr { { float, float }**, i32 }, { { float, float 
    }**, i32 }* %p22, i32 0, i32 0
  %dataref24 = load { float, float }**, { float, float }*** %datarefref23
  %p25 = load { { float, float }**, i32 }*, { { float, float }**, i32 }** %p1
  %lenref26 = getelementptr { { float, float }**, i32 }, { { float, float }**,
    i32 }* %p25, i32 0, i32 1
  %len27 = load i32, i32* %lenref26
  %tmp28 = sub i32 %len27, 1
  %i29 = load i32, i32* %i2
  %tmp30 = sub i32 %tmp28, %i29
  %q31 = load { float, float }*, { float, float }** %q
  %storeref32 = getelementptr { float, float }*, { float, float }** %dataref24,
    i32 %tmp30
  store { float, float }* %q31, { float, float }** %storeref32
  %p33 = load { { float, float }**, i32 }*, { { float, float }**, i32 }** %p1
  %i34 = load i32, i32* %i2
  %tmp35 = add i32 %i34, 1
  call void @reversehelp({ { float, float }**, i32 }* %p33, i32 %tmp35)
  br label %merge

else:                                             ; preds = %entry
  br label %merge
}

define void @reverse({ { float, float }**, i32 }* %p) {
entry:
  %p1 = alloca { { float, float }**, i32 }*
  store { { float, float }**, i32 }* %p, { { float, float }**, i32 }** %p1
  %p2 = load { { float, float }**, i32 }*, { { float, float }**, i32 }** %p1
  call void @reversehelp({ { float, float }**, i32 }* %p2, i32 0)
  ret void
}

define { float, float, float, float }* @rgb(float %r, float %g, float %b) {
entry:
  %r1 = alloca float
  store float %r, float* %r1
  %g2 = alloca float
  store float %g, float* %g2
  %b3 = alloca float
  store float %b, float* %b3
  %malloccall = tail call i8* @malloc(i32 trunc (i64 mul nuw (i64 ptrtoint (float* 
    getelementptr (float, float* null, i32 1) to i64), i64 4) to i32))
  %anon = bitcast i8* %malloccall to { float, float, float, float }*
  %fieldaddr = getelementptr { float, float, float, float }, { float, float, float,
    float }* %anon, i32 0, i32 0
  %r4 = load float, float* %r1
  store float %r4, float* %fieldaddr
  %fieldaddr5 = getelementptr { float, float, float, float }, { float, float, float,
    float }* %anon, i32 0, i32 1
  %g6 = load float, float* %g2
  store float %g6, float* %fieldaddr5
  %fieldaddr7 = getelementptr { float, float, float, float }, { float, float, float,
    float }* %anon, i32 0, i32 2
  %b8 = load float, float* %b3
  store float %b8, float* %fieldaddr7
  %fieldaddr9 = getelementptr { float, float, float, float }, { float, float, float,
    float }* %anon, i32 0, i32 3
  store float 1.000000e+00, float* %fieldaddr9
  ret { float, float, float, float }* %anon
}

define { float, float, float, float }* @hsv(float %h, float %s, float %v) {
entry:
  %h1 = alloca float
  store float %h, float* %h1
  %s2 = alloca float
  store float %s, float* %s2
  %v3 = alloca float
  store float %v, float* %v3
  %c = alloca float
  %v4 = load float, float* %v3
  %s5 = load float, float* %s2
  %tmp = fmul float %v4, %s5
  store float %tmp, float* %c
  %hfac = alloca float
  %h6 = load float, float* %h1
  %tmp7 = fmul float %h6, 6.000000e+00
  %fxn_result = call float @modf(float %tmp7, float 2.000000e+00)
  store float %fxn_result, float* %hfac
  %x = alloca float
  %c8 = load float, float* %c
  %hfac9 = load float, float* %hfac
  %tmp10 = fsub float %hfac9, 1.000000e+00
  %fxn_result11 = call float @abs(float %tmp10)
  %tmp12 = fsub float 1.000000e+00, %fxn_result11
  %tmp13 = fmul float %c8, %tmp12
  store float %tmp13, float* %x
  %m = alloca float
  %v14 = load float, float* %v3
  %c15 = load float, float* %c
  %tmp16 = fsub float %v14, %c15
  store float %tmp16, float* %m
  %hh = alloca float
  %h17 = load float, float* %h1
  %tmp18 = fmul float %h17, 6.000000e+00
  store float %tmp18, float* %hh
  %hh19 = load float, float* %hh
  %tmp20 = fcmp olt float %hh19, 1.000000e+00
  %if_tmp = alloca { float, float, float, float }*
  br i1 %tmp20, label %then, label %else

merge:                                            ; preds = %merge30, %then
  %if_tmp85 = load { float, float, float, float }*, { float, float, float, float
    }** %if_tmp
  ret { float, float, float, float }* %if_tmp85

then:                                             ; preds = %entry
  %v21 = load float, float* %v3
  %x22 = load float, float* %x
  %m23 = load float, float* %m
  %tmp24 = fadd float %x22, %m23
  %m25 = load float, float* %m
  %fxn_result26 = call { float, float, float, float }* @rgb(float %v21, float %tmp24
    , float %m25)
  store { float, float, float, float }* %fxn_result26, { float, float, float, float
    }** %if_tmp
  br label %merge

else:                                             ; preds = %entry
  %hh27 = load float, float* %hh
  %tmp28 = fcmp olt float %hh27, 2.000000e+00
  %if_tmp29 = alloca { float, float, float, float }*
  br i1 %tmp28, label %then31, label %else32

merge30:                                          ; preds = %merge42, %then31
  %if_tmp84 = load { float, float, float, float }*, { float, float, float, float
    }** %if_tmp29
  store { float, float, float, float }* %if_tmp84, { float, float, float, float
    }** %if_tmp
  br label %merge

then31:                                           ; preds = %else
  %x33 = load float, float* %x
  %m34 = load float, float* %m
  %tmp35 = fadd float %x33, %m34
  %v36 = load float, float* %v3
  %m37 = load float, float* %m
  %fxn_result38 = call { float, float, float, float }* @rgb(float %tmp35, float
    %v36, float %m37)
  store { float, float, float, float }* %fxn_result38, { float, float, float,
    float }** %if_tmp29
  br label %merge30

else32:                                           ; preds = %else
  %hh39 = load float, float* %hh
  %tmp40 = fcmp olt float %hh39, 3.000000e+00
  %if_tmp41 = alloca { float, float, float, float }*
  br i1 %tmp40, label %then43, label %else44

merge42:                                          ; preds = %merge54, %then43
  %if_tmp83 = load { float, float, float, float }*, { float, float, float,
    float }** %if_tmp41
  store { float, float, float, float }* %if_tmp83, { float, float, float,
    float }** %if_tmp29
  br label %merge30

then43:                                           ; preds = %else32
  %m45 = load float, float* %m
  %v46 = load float, float* %v3
  %x47 = load float, float* %x
  %m48 = load float, float* %m
  %tmp49 = fadd float %x47, %m48
  %fxn_result50 = call { float, float, float, float }* @rgb(float %m45, float
    %v46, float %tmp49)
  store { float, float, float, float }* %fxn_result50, { float, float, float,
    float }** %if_tmp41
  br label %merge42

else44:                                           ; preds = %else32
  %hh51 = load float, float* %hh
  %tmp52 = fcmp olt float %hh51, 4.000000e+00
  %if_tmp53 = alloca { float, float, float, float }*
  br i1 %tmp52, label %then55, label %else56

merge54:                                          ; preds = %merge66, %then55
  %if_tmp82 = load { float, float, float, float }*, { float, float, float,
    float }** %if_tmp53
  store { float, float, float, float }* %if_tmp82, { float, float, float,
    float }** %if_tmp41
  br label %merge42

then55:                                           ; preds = %else44
  %m57 = load float, float* %m
  %x58 = load float, float* %x
  %m59 = load float, float* %m
  %tmp60 = fadd float %x58, %m59
  %v61 = load float, float* %v3
  %fxn_result62 = call { float, float, float, float }* @rgb(float %m57,
    float %tmp60, float %v61)
  store { float, float, float, float }* %fxn_result62, { float, float,
    float, float }** %if_tmp53
  br label %merge54

else56:                                           ; preds = %else44
  %hh63 = load float, float* %hh
  %tmp64 = fcmp olt float %hh63, 5.000000e+00
  %if_tmp65 = alloca { float, float, float, float }*
  br i1 %tmp64, label %then67, label %else68

merge66:                                          ; preds = %else68, %then67
  %if_tmp81 = load { float, float, float, float }*, { float, float,
    float, float }** %if_tmp65
  store { float, float, float, float }* %if_tmp81, { float, float,
    float, float }** %if_tmp53
  br label %merge54

then67:                                           ; preds = %else56
  %x69 = load float, float* %x
  %m70 = load float, float* %m
  %tmp71 = fadd float %x69, %m70
  %m72 = load float, float* %m
  %v73 = load float, float* %v3
  %fxn_result74 = call { float, float, float, float }* @rgb(float
    %tmp71, float %m72, float %v73)
  store { float, float, float, float }* %fxn_result74, { float, 
    float, float, float }** %if_tmp65
  br label %merge66

else68:                                           ; preds = %else56
  %v75 = load float, float* %v3
  %m76 = load float, float* %m
  %x77 = load float, float* %x
  %m78 = load float, float* %m
  %tmp79 = fadd float %x77, %m78
  %fxn_result80 = call { float, float, float, float }* @rgb(float %v75,
    float %m76, float %tmp79)
  store { float, float, float, float }* %fxn_result80, { float, float,
    float, float }** %if_tmp65
  br label %merge66
}

define void @startCanvas({ i32, i32, i32 }* %c) {
entry:
  %c1 = alloca { i32, i32, i32 }*
  store { i32, i32, i32 }* %c, { i32, i32, i32 }** %c1
  %c2 = load { i32, i32, i32 }*, { i32, i32, i32 }** %c1
  %fieldadr = getelementptr { i32, i32, i32 }, { i32, i32, i32 }* %c2, 
    i32 0, i32 0
  %width = load i32, i32* %fieldadr
  %c3 = load { i32, i32, i32 }*, { i32, i32, i32 }** %c1
  %fieldadr4 = getelementptr { i32, i32, i32 }, { i32, i32, i32 }* %c3,
    i32 0, i32 1
  %height = load i32, i32* %fieldadr4
  call void @gl_startRendering(i32 %width, i32 %height)
  ret void
}

define void @cvoid() {
entry:
  ret void
}

define void @drawHelper({ { float, float }**, i32 }* %point_structs, { { 
    float, float, float, float }**, i32 }* %color_structs, i32 %numOfPoints,
    i32 %i, { float*, i32 }* %points, { float*, i32 }* %colors) {
entry:
  %point_structs1 = alloca { { float, float }**, i32 }*
  store { { float, float }**, i32 }* %point_structs, { { float, float }**, i32
    }** %point_structs1
  %color_structs2 = alloca { { float, float, float, float }**, i32 }*
  store { { float, float, float, float }**, i32 }* %color_structs, { { 
    float, float, float, float }**, i32 }** %color_structs2
  %numOfPoints3 = alloca i32
  store i32 %numOfPoints, i32* %numOfPoints3
  %i4 = alloca i32
  store i32 %i, i32* %i4
  %points5 = alloca { float*, i32 }*
  store { float*, i32 }* %points, { float*, i32 }** %points5
  %colors6 = alloca { float*, i32 }*
  store { float*, i32 }* %colors, { float*, i32 }** %colors6
  %i7 = load i32, i32* %i4
  %numOfPoints8 = load i32, i32* %numOfPoints3
  %tmp = icmp sge i32 %i7, %numOfPoints8
  br i1 %tmp, label %then, label %else

merge:                                            ; preds = %else, %then
  ret void

then:                                             ; preds = %entry
  call void @cvoid()
  br label %merge

else:                                             ; preds = %entry
  %px = alloca float
  %point_structs9 = load { { float, float }**, i32 }*, { { float, float 
    }**, i32 }** %point_structs1
  %i10 = load i32, i32* %i4
  %dataref = getelementptr { { float, float }**, i32 }, { { float, float 
    }**, i32 }* %point_structs9, i32 0, i32 0
  %data = load { float, float }**, { float, float }*** %dataref
  %elref = getelementptr { float, float }*, { float, float }** %data, i32 %i10
  %el = load { float, float }*, { float, float }** %elref
  %fieldadr = getelementptr { float, float }, { float, float }* %el, i32 0, i32 0
  %x = load float, float* %fieldadr
  store float %x, float* %px
  %py = alloca float
  %point_structs11 = load { { float, float }**, i32 }*, { { float, float
    }**, i32 }** %point_structs1
  %i12 = load i32, i32* %i4
  %dataref13 = getelementptr { { float, float }**, i32 }, { { float, float
    }**, i32 }* %point_structs11, i32 0, i32 0
  %data14 = load { float, float }**, { float, float }*** %dataref13
  %elref15 = getelementptr { float, float }*, { float, float }** %data14,
    i32 %i12
  %el16 = load { float, float }*, { float, float }** %elref15
  %fieldadr17 = getelementptr { float, float }, { float, float }* %el16, 
    i32 0, i32 1
  %y = load float, float* %fieldadr17
  store float %y, float* %py
  %points18 = load { float*, i32 }*, { float*, i32 }** %points5
  %datarefref = getelementptr { float*, i32 }, { float*, i32 }* %points18,
    i32 0, i32 0
  %dataref19 = load float*, float** %datarefref
  %i20 = load i32, i32* %i4
  %tmp21 = mul i32 2, %i20
  %px22 = load float, float* %px
  %storeref = getelementptr float, float* %dataref19, i32 %tmp21
  store float %px22, float* %storeref
  %points23 = load { float*, i32 }*, { float*, i32 }** %points5
  %datarefref24 = getelementptr { float*, i32 }, { float*, i32 }* %points23,
    i32 0, i32 0
  %dataref25 = load float*, float** %datarefref24
  %i26 = load i32, i32* %i4
  %tmp27 = mul i32 2, %i26
  %tmp28 = add i32 %tmp27, 1
  %py29 = load float, float* %py
  %storeref30 = getelementptr float, float* %dataref25, i32 %tmp28
  store float %py29, float* %storeref30
  %cr = alloca float
  %color_structs31 = load { { float, float, float, float }**, i32 }*, { { 
    float, float, float, float }**, i32 }** %color_structs2
  %i32 = load i32, i32* %i4
  %dataref33 = getelementptr { { float, float, float, float }**, i32 }, { {
    float, float, float, float }**, i32 }* %color_structs31, i32 0, i32 0
  %data34 = load { float, float, float, float }**, { float, float, float, 
    float }*** %dataref33
  %elref35 = getelementptr { float, float, float, float }*, { float, float, 
    float, float }** %data34, i32 %i32
  %el36 = load { float, float, float, float }*, { float, float, float, float 
    }** %elref35
  %fieldadr37 = getelementptr { float, float, float, float }, { float, float, 
    float, float }* %el36, i32 0, i32 0
  %r = load float, float* %fieldadr37
  store float %r, float* %cr
  %cg = alloca float
  %color_structs38 = load { { float, float, float, float }**, i32 }*, { { 
    float, float, float, float }**, i32 }** %color_structs2
  %i39 = load i32, i32* %i4
  %dataref40 = getelementptr { { float, float, float, float }**, i32 }, { { 
    float, float, float, float }**, i32 }* %color_structs38, i32 0, i32 0
  %data41 = load { float, float, float, float }**, { float, float, float, 
    float }*** %dataref40
  %elref42 = getelementptr { float, float, float, float }*, { float, float,
    float, float }** %data41, i32 %i39
  %el43 = load { float, float, float, float }*, { float, float, float, float 
    }** %elref42
  %fieldadr44 = getelementptr { float, float, float, float }, { float, float,
    float, float }* %el43, i32 0, i32 1
  %g = load float, float* %fieldadr44
  store float %g, float* %cg
  %cb = alloca float
  %color_structs45 = load { { float, float, float, float }**, i32 }*, { { float,
    float, float, float }**, i32 }** %color_structs2
  %i46 = load i32, i32* %i4
  %dataref47 = getelementptr { { float, float, float, float }**, i32 }, { { float,
    float, float, float }**, i32 }* %color_structs45, i32 0, i32 0
  %data48 = load { float, float, float, float }**, { float, float, float, float
    }*** %dataref47
  %elref49 = getelementptr { float, float, float, float }*, { float, float, float,
    float }** %data48, i32 %i46
  %el50 = load { float, float, float, float }*, { float, float, float, float }** 
    %elref49
  %fieldadr51 = getelementptr { float, float, float, float }, { float, float, 
    float, float }* %el50, i32 0, i32 2
  %b = load float, float* %fieldadr51
  store float %b, float* %cb
  %ca = alloca float
  %color_structs52 = load { { float, float, float, float }**, i32 }*, { { float,
    float, float, float }**, i32 }** %color_structs2
  %i53 = load i32, i32* %i4
  %dataref54 = getelementptr { { float, float, float, float }**, i32 }, { { float,
    float, float, float }**, i32 }* %color_structs52, i32 0, i32 0
  %data55 = load { float, float, float, float }**, { float, float, float, float 
    }*** %dataref54
  %elref56 = getelementptr { float, float, float, float }*, { float, float, float, 
    float }** %data55, i32 %i53
  %el57 = load { float, float, float, float }*, { float, float, float, float }** 
    %elref56
  %fieldadr58 = getelementptr { float, float, float, float }, { float, float,
    float,float }* %el57, i32 0, i32 3
  %a = load float, float* %fieldadr58
  store float %a, float* %ca
  %colors59 = load { float*, i32 }*, { float*, i32 }** %colors6
  %datarefref60 = getelementptr { float*, i32 }, { float*, i32 }* %colors59, 
    i32 0, i32 0
  %dataref61 = load float*, float** %datarefref60
  %i62 = load i32, i32* %i4
  %tmp63 = mul i32 4, %i62
  %cr64 = load float, float* %cr
  %storeref65 = getelementptr float, float* %dataref61, i32 %tmp63
  store float %cr64, float* %storeref65
  %colors66 = load { float*, i32 }*, { float*, i32 }** %colors6
  %datarefref67 = getelementptr { float*, i32 }, { float*, i32 }* %colors66, 
    i32 0, i32 0
  %dataref68 = load float*, float** %datarefref67
  %i69 = load i32, i32* %i4
  %tmp70 = mul i32 4, %i69
  %tmp71 = add i32 %tmp70, 1
  %cg72 = load float, float* %cg
  %storeref73 = getelementptr float, float* %dataref68, i32 %tmp71
  store float %cg72, float* %storeref73
  %colors74 = load { float*, i32 }*, { float*, i32 }** %colors6
  %datarefref75 = getelementptr { float*, i32 }, { float*, i32 }* %colors74,
    i32 0, i32 0
  %dataref76 = load float*, float** %datarefref75
  %i77 = load i32, i32* %i4
  %tmp78 = mul i32 4, %i77
  %tmp79 = add i32 %tmp78, 2
  %cb80 = load float, float* %cb
  %storeref81 = getelementptr float, float* %dataref76, i32 %tmp79
  store float %cb80, float* %storeref81
  %colors82 = load { float*, i32 }*, { float*, i32 }** %colors6
  %datarefref83 = getelementptr { float*, i32 }, { float*, i32 }* %colors82,
    i32 0, i32 0
  %dataref84 = load float*, float** %datarefref83
  %i85 = load i32, i32* %i4
  %tmp86 = mul i32 4, %i85
  %tmp87 = add i32 %tmp86, 3
  %ca88 = load float, float* %ca
  %storeref89 = getelementptr float, float* %dataref84, i32 %tmp87
  store float %ca88, float* %storeref89
  %point_structs90 = load { { float, float }**, i32 }*, { { float, float }**,
    i32 }** %point_structs1
  %color_structs91 = load { { float, float, float, float }**, i32 }*, { { float,
    float, float, float }**, i32 }** %color_structs2
  %numOfPoints92 = load i32, i32* %numOfPoints3
  %i93 = load i32, i32* %i4
  %tmp94 = add i32 %i93, 1
  %points95 = load { float*, i32 }*, { float*, i32 }** %points5
  %colors96 = load { float*, i32 }*, { float*, i32 }** %colors6
  call void @drawHelper({ { float, float }**, i32 }* %point_structs90, { { float,
    float, float, float }**, i32 }* %color_structs91, i32 %numOfPoints92, i32 %tmp94,
    { float*, i32 }* %points95, { float*, i32 }* %colors96)
  br label %merge
}

define void @drawPoints({ { float, float }**, i32 }* %point_structs, { { float, 
    float, float, float }**, i32 }* %color_structs) {
entry:
  %point_structs1 = alloca { { float, float }**, i32 }*
  store { { float, float }**, i32 }* %point_structs, { { float, float }**, 
    i32 }** %point_structs1
  %color_structs2 = alloca { { float, float, float, float }**, i32 }*
  store { { float, float, float, float }**, i32 }* %color_structs, { { float,
    float, float, float }**, i32 }** %color_structs2
  %numOfPoints = alloca i32
  %point_structs3 = load { { float, float }**, i32 }*, { { float, float }**, i32 
    }** %point_structs1
  %lenref = getelementptr { { float, float }**, i32 }, { { float, float }**, i32
    }* %point_structs3, i32 0, i32 1
  %len = load i32, i32* %lenref
  store i32 %len, i32* %numOfPoints
  %points = alloca { float*, i32 }*
  %numOfPoints4 = load i32, i32* %numOfPoints
  %tmp = mul i32 %numOfPoints4, 2
  %malloccall = tail call i8* @malloc(i32 ptrtoint (float* getelementptr (float,
    float* null, i32 1) to i32))
  %arrdata = bitcast i8* %malloccall to float*
  %storeref = getelementptr float, float* %arrdata, i32 0
  store float 0.000000e+00, float* %storeref
  %malloccall5 = tail call i8* @malloc(i32 ptrtoint ({ float*, i32 }* getelementptr
    ({ float*, i32 }, { float*, i32 }* null, i32 1) to i32))
  %arr = bitcast i8* %malloccall5 to { float*, i32 }*
  %arrdata6 = getelementptr { float*, i32 }, { float*, i32 }* %arr, i32 0, i32 0
  %arrlen = getelementptr { float*, i32 }, { float*, i32 }* %arr, i32 0, i32 1
  store float* %arrdata, float** %arrdata6
  store i32 1, i32* %arrlen
  %lenref7 = getelementptr { float*, i32 }, { float*, i32 }* %arr, i32 0, i32 1
  %len8 = load i32, i32* %lenref7
  %oflen = mul i32 %tmp, %len8
  %olddataref = getelementptr { float*, i32 }, { float*, i32 }* %arr, i32 0, i32 0
  %olddata = load float*, float** %olddataref
  %mallocsize = mul i32 %oflen, ptrtoint (float* getelementptr (float, float* null,
    i32 1) to i32)
  %malloccall9 = tail call i8* @malloc(i32 %mallocsize)
  %arrdata10 = bitcast i8* %malloccall9 to float*
  %i = alloca i32
  store i32 0, i32* %i
  %j = alloca i32
  store i32 0, i32* %j
  br label %inner

loop:                                             ; preds = %inner
  %i18 = load i32, i32* %i
  store i32 0, i32* %j
  %tmp19 = icmp slt i32 %i18, %oflen
  br i1 %tmp19, label %inner, label %continue

inner:                                            ; preds = %loop, %inner, %entry
  %i11 = load i32, i32* %j
  %i12 = load i32, i32* %i
  %elref = getelementptr float, float* %olddata, i32 %i11
  %el = load float, float* %elref
  %storeref13 = getelementptr float, float* %arrdata10, i32 %i12
  store float %el, float* %storeref13
  %i14 = add i32 %i12, 1
  store i32 %i14, i32* %i
  %j15 = add i32 %i11, 1
  store i32 %j15, i32* %j
  %j16 = load i32, i32* %j
  %tmp17 = icmp slt i32 %j16, %len8
  br i1 %tmp17, label %inner, label %loop

continue:                                         ; preds = %loop
  %malloccall20 = tail call i8* @malloc(i32 ptrtoint ({ float*, i32 }* getelementptr
    ({ float*, i32 }, { float*, i32 }* null, i32 1) to i32))
  %arr21 = bitcast i8* %malloccall20 to { float*, i32 }*
  %arrdata22 = getelementptr { float*, i32 }, { float*, i32 }* %arr21, i32 0, i32 0
  %arrlen23 = getelementptr { float*, i32 }, { float*, i32 }* %arr21, i32 0, i32 1
  store float* %arrdata10, float** %arrdata22
  store i32 %oflen, i32* %arrlen23
  store { float*, i32 }* %arr21, { float*, i32 }** %points
  %colors = alloca { float*, i32 }*
  %numOfPoints24 = load i32, i32* %numOfPoints
  %tmp25 = mul i32 %numOfPoints24, 4
  %malloccall26 = tail call i8* @malloc(i32 ptrtoint (float* getelementptr (float,
    float* null, i32 1) to i32))
  %arrdata27 = bitcast i8* %malloccall26 to float*
  %storeref28 = getelementptr float, float* %arrdata27, i32 0
  store float 0.000000e+00, float* %storeref28
  %malloccall29 = tail call i8* @malloc(i32 ptrtoint ({ float*, i32 }* getelementptr
    ({ float*, i32 }, { float*, i32 }* null, i32 1) to i32))
  %arr30 = bitcast i8* %malloccall29 to { float*, i32 }*
  %arrdata31 = getelementptr { float*, i32 }, { float*, i32 }* %arr30, i32 0, i32 0
  %arrlen32 = getelementptr { float*, i32 }, { float*, i32 }* %arr30, i32 0, i32 1
  store float* %arrdata27, float** %arrdata31
  store i32 1, i32* %arrlen32
  %lenref33 = getelementptr { float*, i32 }, { float*, i32 }* %arr30, i32 0, i32 1
  %len34 = load i32, i32* %lenref33
  %oflen35 = mul i32 %tmp25, %len34
  %olddataref36 = getelementptr { float*, i32 }, { float*, i32 }* %arr30, i32 0, i32 0
  %olddata37 = load float*, float** %olddataref36
  %mallocsize38 = mul i32 %oflen35, ptrtoint (float* getelementptr (float, float* null,
    i32 1) to i32)
  %malloccall39 = tail call i8* @malloc(i32 %mallocsize38)
  %arrdata40 = bitcast i8* %malloccall39 to float*
  %i41 = alloca i32
  store i32 0, i32* %i41
  %j42 = alloca i32
  store i32 0, i32* %j42
  br label %inner44

loop43:                                           ; preds = %inner44
  %i55 = load i32, i32* %i41
  store i32 0, i32* %j42
  %tmp56 = icmp slt i32 %i55, %oflen35
  br i1 %tmp56, label %inner44, label %continue45

inner44:                                          
    ; preds = %loop43, %inner44, %continue
  %i46 = load i32, i32* %j42
  %i47 = load i32, i32* %i41
  %elref48 = getelementptr float, float* %olddata37, i32 %i46
  %el49 = load float, float* %elref48
  %storeref50 = getelementptr float, float* %arrdata40, i32 %i47
  store float %el49, float* %storeref50
  %i51 = add i32 %i47, 1
  store i32 %i51, i32* %i41
  %j52 = add i32 %i46, 1
  store i32 %j52, i32* %j42
  %j53 = load i32, i32* %j42
  %tmp54 = icmp slt i32 %j53, %len34
  br i1 %tmp54, label %inner44, label %loop43

continue45:                                       ; preds = %loop43
  %malloccall57 = tail call i8* @malloc(i32 ptrtoint ({ float*, i32 }* 
    getelementptr ({ float*, i32 }, { float*, i32 }* null, i32 1) to i32))
  %arr58 = bitcast i8* %malloccall57 to { float*, i32 }*
  %arrdata59 = getelementptr { float*, i32 }, { float*, i32 }* %arr58, i32 0, i32 0
  %arrlen60 = getelementptr { float*, i32 }, { float*, i32 }* %arr58, i32 0, i32 1
  store float* %arrdata40, float** %arrdata59
  store i32 %oflen35, i32* %arrlen60
  store { float*, i32 }* %arr58, { float*, i32 }** %colors
  %point_structs61 = load { { float, float }**, i32 }*, { { float, float }**,
    i32 }** %point_structs1
  %color_structs62 = load { { float, float, float, float }**, i32 }*, { { float,
    float, float, float }**, i32 }** %color_structs2
  %numOfPoints63 = load i32, i32* %numOfPoints
  %points64 = load { float*, i32 }*, { float*, i32 }** %points
  %colors65 = load { float*, i32 }*, { float*, i32 }** %colors
  call void @drawHelper({ { float, float }**, i32 }* %point_structs61, { { float,
    float, float, float }**, i32 }* %color_structs62, i32 %numOfPoints63, i32 0,
    { float*, i32 }* %points64, { float*, i32 }* %colors65)
  %points66 = load { float*, i32 }*, { float*, i32 }** %points
  %colors67 = load { float*, i32 }*, { float*, i32 }** %colors
  call void @gl_drawPoint({ float*, i32 }* %points66, { float*, i32 }* %colors67,
    i32 2)
  ret void
}

define void @drawPath({ { float, float }**, i32 }* %point_structs, { { float,
    float, float, float }**, i32 }* %color_structs, i32 %colorMode) {
entry:
  %point_structs1 = alloca { { float, float }**, i32 }*
  store { { float, float }**, i32 }* %point_structs, { { float, float }**, i32 }**
    %point_structs1
  %color_structs2 = alloca { { float, float, float, float }**, i32 }*
  store { { float, float, float, float }**, i32 }* %color_structs, { { float, 
    float, float, float }**, i32 }** %color_structs2
  %colorMode3 = alloca i32
  store i32 %colorMode, i32* %colorMode3
  %numOfPoints = alloca i32
  %point_structs4 = load { { float, float }**, i32 }*, { { float, float }**,
    i32 }** %point_structs1
  %lenref = getelementptr { { float, float }**, i32 }, { { float, float }**,
    i32 }* %point_structs4, i32 0, i32 1
  %len = load i32, i32* %lenref
  store i32 %len, i32* %numOfPoints
  %points = alloca { float*, i32 }*
  %numOfPoints5 = load i32, i32* %numOfPoints
  %tmp = mul i32 %numOfPoints5, 2
  %malloccall = tail call i8* @malloc(i32 ptrtoint (float* getelementptr
    (float, float* null, i32 1) to i32))
  %arrdata = bitcast i8* %malloccall to float*
  %storeref = getelementptr float, float* %arrdata, i32 0
  store float 0.000000e+00, float* %storeref
  %malloccall6 = tail call i8* @malloc(i32 ptrtoint ({ float*, i32 }* getelementptr
    ({ float*, i32 }, { float*, i32 }* null, i32 1) to i32))
  %arr = bitcast i8* %malloccall6 to { float*, i32 }*
  %arrdata7 = getelementptr { float*, i32 }, { float*, i32 }* %arr, i32 0, i32 0
  %arrlen = getelementptr { float*, i32 }, { float*, i32 }* %arr, i32 0, i32 1
  store float* %arrdata, float** %arrdata7
  store i32 1, i32* %arrlen
  %lenref8 = getelementptr { float*, i32 }, { float*, i32 }* %arr, i32 0, i32 1
  %len9 = load i32, i32* %lenref8
  %oflen = mul i32 %tmp, %len9
  %olddataref = getelementptr { float*, i32 }, { float*, i32 }* %arr, i32 0, i32 0
  %olddata = load float*, float** %olddataref
  %mallocsize = mul i32 %oflen, ptrtoint (float* getelementptr (float, float* null,
    i32 1) to i32)
  %malloccall10 = tail call i8* @malloc(i32 %mallocsize)
  %arrdata11 = bitcast i8* %malloccall10 to float*
  %i = alloca i32
  store i32 0, i32* %i
  %j = alloca i32
  store i32 0, i32* %j
  br label %inner

loop:                                             ; preds = %inner
  %i19 = load i32, i32* %i
  store i32 0, i32* %j
  %tmp20 = icmp slt i32 %i19, %oflen
  br i1 %tmp20, label %inner, label %continue

inner:                                            ; preds = %loop, %inner, %entry
  %i12 = load i32, i32* %j
  %i13 = load i32, i32* %i
  %elref = getelementptr float, float* %olddata, i32 %i12
  %el = load float, float* %elref
  %storeref14 = getelementptr float, float* %arrdata11, i32 %i13
  store float %el, float* %storeref14
  %i15 = add i32 %i13, 1
  store i32 %i15, i32* %i
  %j16 = add i32 %i12, 1
  store i32 %j16, i32* %j
  %j17 = load i32, i32* %j
  %tmp18 = icmp slt i32 %j17, %len9
  br i1 %tmp18, label %inner, label %loop

continue:                                         ; preds = %loop
  %malloccall21 = tail call i8* @malloc(i32 ptrtoint ({ float*, i32 }* 
    getelementptr({ float*, i32 }, { float*, i32 }* null, i32 1) to i32))
  %arr22 = bitcast i8* %malloccall21 to { float*, i32 }*
  %arrdata23 = getelementptr { float*, i32 }, { float*, i32 }* %arr22, i32 0,
    i32 0
  %arrlen24 = getelementptr { float*, i32 }, { float*, i32 }* %arr22, i32 0, 
    i32 1
  store float* %arrdata11, float** %arrdata23
  store i32 %oflen, i32* %arrlen24
  store { float*, i32 }* %arr22, { float*, i32 }** %points
  %colors = alloca { float*, i32 }*
  %numOfPoints25 = load i32, i32* %numOfPoints
  %tmp26 = mul i32 %numOfPoints25, 4
  %malloccall27 = tail call i8* @malloc(i32 ptrtoint (float* getelementptr 
    (float,float* null, i32 1) to i32))
  %arrdata28 = bitcast i8* %malloccall27 to float*
  %storeref29 = getelementptr float, float* %arrdata28, i32 0
  store float 0.000000e+00, float* %storeref29
  %malloccall30 = tail call i8* @malloc(i32 ptrtoint ({ float*, i32 }* 
    getelementptr({ float*, i32 }, { float*, i32 }* null, i32 1) to i32))
  %arr31 = bitcast i8* %malloccall30 to { float*, i32 }*
  %arrdata32 = getelementptr { float*, i32 }, { float*, i32 }* %arr31, i32 0,
    i32 0
  %arrlen33 = getelementptr { float*, i32 }, { float*, i32 }* %arr31, i32 0, 
    i32 1
  store float* %arrdata28, float** %arrdata32
  store i32 1, i32* %arrlen33
  %lenref34 = getelementptr { float*, i32 }, { float*, i32 }* %arr31, i32 0, 
    i32 1
  %len35 = load i32, i32* %lenref34
  %oflen36 = mul i32 %tmp26, %len35
  %olddataref37 = getelementptr { float*, i32 }, { float*, i32 }* %arr31, i32 0,
    i32 0
  %olddata38 = load float*, float** %olddataref37
  %mallocsize39 = mul i32 %oflen36, ptrtoint (float* getelementptr (float, float*
    null,i32 1) to i32)
  %malloccall40 = tail call i8* @malloc(i32 %mallocsize39)
  %arrdata41 = bitcast i8* %malloccall40 to float*
  %i42 = alloca i32
  store i32 0, i32* %i42
  %j43 = alloca i32
  store i32 0, i32* %j43
  br label %inner45

loop44:                                           ; preds = %inner45
  %i56 = load i32, i32* %i42
  store i32 0, i32* %j43
  %tmp57 = icmp slt i32 %i56, %oflen36
  br i1 %tmp57, label %inner45, label %continue46

inner45:                                          
    ; preds = %loop44, %inner45, %continue
  %i47 = load i32, i32* %j43
  %i48 = load i32, i32* %i42
  %elref49 = getelementptr float, float* %olddata38, i32 %i47
  %el50 = load float, float* %elref49
  %storeref51 = getelementptr float, float* %arrdata41, i32 %i48
  store float %el50, float* %storeref51
  %i52 = add i32 %i48, 1
  store i32 %i52, i32* %i42
  %j53 = add i32 %i47, 1
  store i32 %j53, i32* %j43
  %j54 = load i32, i32* %j43
  %tmp55 = icmp slt i32 %j54, %len35
  br i1 %tmp55, label %inner45, label %loop44

continue46:                                       ; preds = %loop44
  %malloccall58 = tail call i8* @malloc(i32 ptrtoint ({ float*, i32 }* 
    getelementptr ({ float*, i32 }, { float*, i32 }* null, i32 1) to i32))
  %arr59 = bitcast i8* %malloccall58 to { float*, i32 }*
  %arrdata60 = getelementptr { float*, i32 }, { float*, i32 }* %arr59, i32 0, i32 0
  %arrlen61 = getelementptr { float*, i32 }, { float*, i32 }* %arr59, i32 0, i32 1
  store float* %arrdata41, float** %arrdata60
  store i32 %oflen36, i32* %arrlen61
  store { float*, i32 }* %arr59, { float*, i32 }** %colors
  %point_structs62 = load { { float, float }**, i32 }*, { { float, float }**, 
    i32 }** %point_structs1
  %color_structs63 = load { { float, float, float, float }**, i32 }*, { { float,
    float, float, float }**, i32 }** %color_structs2
  %numOfPoints64 = load i32, i32* %numOfPoints
  %points65 = load { float*, i32 }*, { float*, i32 }** %points
  %colors66 = load { float*, i32 }*, { float*, i32 }** %colors
  call void @drawHelper({ { float, float }**, i32 }* %point_structs62, { { float,
    float, float, float }**, i32 }* %color_structs63, i32 %numOfPoints64, i32 0,
    { float*, i32 }* %points65, { float*, i32 }* %colors66)
  %points67 = load { float*, i32 }*, { float*, i32 }** %points
  %colors68 = load { float*, i32 }*, { float*, i32 }** %colors
  %colorMode69 = load i32, i32* %colorMode3
  call void @gl_drawCurve({ float*, i32 }* %points67, { float*, i32 }* %colors68,
    i32 %colorMode69)
  ret void
}

define void @drawShape({ { float, float }**, i32 }* %point_structs, { { float, 
    float, float, float }**, i32 }* %color_structs, i32 %colorMode, i32 %filled) {
entry:
  %point_structs1 = alloca { { float, float }**, i32 }*
  store { { float, float }**, i32 }* %point_structs, { { float, float }**, i32 
    }** %point_structs1
  %color_structs2 = alloca { { float, float, float, float }**, i32 }*
  store { { float, float, float, float }**, i32 }* %color_structs, { { float, 
    float, float, float }**, i32 }** %color_structs2
  %colorMode3 = alloca i32
  store i32 %colorMode, i32* %colorMode3
  %filled4 = alloca i32
  store i32 %filled, i32* %filled4
  %numOfPoints = alloca i32
  %point_structs5 = load { { float, float }**, i32 }*, { { float, float }**, 
    i32 }** %point_structs1
  %lenref = getelementptr { { float, float }**, i32 }, { { float, float }**, 
    i32 }* %point_structs5, i32 0, i32 1
  %len = load i32, i32* %lenref
  store i32 %len, i32* %numOfPoints
  %points = alloca { float*, i32 }*
  %numOfPoints6 = load i32, i32* %numOfPoints
  %tmp = mul i32 %numOfPoints6, 2
  %malloccall = tail call i8* @malloc(i32 ptrtoint (float* getelementptr 
    (float, float* null, i32 1) to i32))
  %arrdata = bitcast i8* %malloccall to float*
  %storeref = getelementptr float, float* %arrdata, i32 0
  store float 0.000000e+00, float* %storeref
  %malloccall7 = tail call i8* @malloc(i32 ptrtoint ({ float*, i32 }* 
    getelementptr ({ float*, i32 }, { float*, i32 }* null, i32 1) to i32))
  %arr = bitcast i8* %malloccall7 to { float*, i32 }*
  %arrdata8 = getelementptr { float*, i32 }, { float*, i32 }* %arr, i32 0, i32 0
  %arrlen = getelementptr { float*, i32 }, { float*, i32 }* %arr, i32 0, i32 1
  store float* %arrdata, float** %arrdata8
  store i32 1, i32* %arrlen
  %lenref9 = getelementptr { float*, i32 }, { float*, i32 }* %arr, i32 0, i32 1
  %len10 = load i32, i32* %lenref9
  %oflen = mul i32 %tmp, %len10
  %olddataref = getelementptr { float*, i32 }, { float*, i32 }* %arr, i32 0, i32 0
  %olddata = load float*, float** %olddataref
  %mallocsize = mul i32 %oflen, ptrtoint (float* getelementptr (float, float* null,
    i32 1) to i32)
  %malloccall11 = tail call i8* @malloc(i32 %mallocsize)
  %arrdata12 = bitcast i8* %malloccall11 to float*
  %i = alloca i32
  store i32 0, i32* %i
  %j = alloca i32
  store i32 0, i32* %j
  br label %inner

loop:                                             ; preds = %inner
  %i20 = load i32, i32* %i
  store i32 0, i32* %j
  %tmp21 = icmp slt i32 %i20, %oflen
  br i1 %tmp21, label %inner, label %continue

inner:                                            ; preds = %loop, %inner, %entry
  %i13 = load i32, i32* %j
  %i14 = load i32, i32* %i
  %elref = getelementptr float, float* %olddata, i32 %i13
  %el = load float, float* %elref
  %storeref15 = getelementptr float, float* %arrdata12, i32 %i14
  store float %el, float* %storeref15
  %i16 = add i32 %i14, 1
  store i32 %i16, i32* %i
  %j17 = add i32 %i13, 1
  store i32 %j17, i32* %j
  %j18 = load i32, i32* %j
  %tmp19 = icmp slt i32 %j18, %len10
  br i1 %tmp19, label %inner, label %loop

continue:                                         ; preds = %loop
  %malloccall22 = tail call i8* @malloc(i32 ptrtoint ({ float*, i32 }* getelementptr
    ({ float*, i32 }, { float*, i32 }* null, i32 1) to i32))
  %arr23 = bitcast i8* %malloccall22 to { float*, i32 }*
  %arrdata24 = getelementptr { float*, i32 }, { float*, i32 }* %arr23, i32 0, i32 0
  %arrlen25 = getelementptr { float*, i32 }, { float*, i32 }* %arr23, i32 0, i32 1
  store float* %arrdata12, float** %arrdata24
  store i32 %oflen, i32* %arrlen25
  store { float*, i32 }* %arr23, { float*, i32 }** %points
  %colors = alloca { float*, i32 }*
  %numOfPoints26 = load i32, i32* %numOfPoints
  %tmp27 = mul i32 %numOfPoints26, 4
  %malloccall28 = tail call i8* @malloc(i32 ptrtoint (float* getelementptr
    (float,float* null, i32 1) to i32))
  %arrdata29 = bitcast i8* %malloccall28 to float*
  %storeref30 = getelementptr float, float* %arrdata29, i32 0
  store float 0.000000e+00, float* %storeref30
  %malloccall31 = tail call i8* @malloc(i32 ptrtoint ({ float*, i32 }*
    getelementptr({ float*, i32 }, { float*, i32 }* null, i32 1) to i32))
  %arr32 = bitcast i8* %malloccall31 to { float*, i32 }*
  %arrdata33 = getelementptr { float*, i32 }, { float*, i32 }* %arr32,
    i32 0, i32 0
  %arrlen34 = getelementptr { float*, i32 }, { float*, i32 }* %arr32,
    i32 0, i32 1
  store float* %arrdata29, float** %arrdata33
  store i32 1, i32* %arrlen34
  %lenref35 = getelementptr { float*, i32 }, { float*, i32 }* %arr32,
    i32 0, i32 1
  %len36 = load i32, i32* %lenref35
  %oflen37 = mul i32 %tmp27, %len36
  %olddataref38 = getelementptr { float*, i32 }, { float*, i32 }* 
    %arr32, i32 0, i32 0
  %olddata39 = load float*, float** %olddataref38
  %mallocsize40 = mul i32 %oflen37, ptrtoint (float* getelementptr
    (float, float*null,i32 1) to i32)
  %malloccall41 = tail call i8* @malloc(i32 %mallocsize40)
  %arrdata42 = bitcast i8* %malloccall41 to float*
  %i43 = alloca i32
  store i32 0, i32* %i43
  %j44 = alloca i32
  store i32 0, i32* %j44
  br label %inner46

loop45:                                           
    ; preds = %inner46
  %i57 = load i32, i32* %i43
  store i32 0, i32* %j44
  %tmp58 = icmp slt i32 %i57, %oflen37
  br i1 %tmp58, label %inner46, label %continue47

inner46:                                          
    ; preds = %loop45, %inner46, %continue
  %i48 = load i32, i32* %j44
  %i49 = load i32, i32* %i43
  %elref50 = getelementptr float, float* %olddata39, i32 %i48
  %el51 = load float, float* %elref50
  %storeref52 = getelementptr float, float* %arrdata42, i32 %i49
  store float %el51, float* %storeref52
  %i53 = add i32 %i49, 1
  store i32 %i53, i32* %i43
  %j54 = add i32 %i48, 1
  store i32 %j54, i32* %j44
  %j55 = load i32, i32* %j44
  %tmp56 = icmp slt i32 %j55, %len36
  br i1 %tmp56, label %inner46, label %loop45

continue47:                                       ; preds = %loop45
  %malloccall59 = tail call i8* @malloc(i32 ptrtoint ({ float*, i32
    }* getelementptr
    ({ float*, i32 }, { float*, i32 }* null, i32 1) to i32))
  %arr60 = bitcast i8* %malloccall59 to { float*, i32 }*
  %arrdata61 = getelementptr { float*, i32 }, { float*, i32 }* %arr60,
    i32 0, i32 0
  %arrlen62 = getelementptr { float*, i32 }, { float*, i32 }* %arr60,
    i32 0, i32 1
  store float* %arrdata42, float** %arrdata61
  store i32 %oflen37, i32* %arrlen62
  store { float*, i32 }* %arr60, { float*, i32 }** %colors
  %point_structs63 = load { { float, float }**, i32 }*, { { float, 
    float }**, i32}** %point_structs1
  %color_structs64 = load { { float, float, float, float }**, i32 }*,
    { { float,float, float, float }**, i32 }** %color_structs2
  %numOfPoints65 = load i32, i32* %numOfPoints
  %points66 = load { float*, i32 }*, { float*, i32 }** %points
  %colors67 = load { float*, i32 }*, { float*, i32 }** %colors
  call void @drawHelper({ { float, float }**, i32 }* %point_structs63, { { float,
    float, float, float }**, i32 }* %color_structs64, i32 %numOfPoints65,
    i32 0, { float*, i32 }* %points66, { float*, i32 }* %colors67)
  %points68 = load { float*, i32 }*, { float*, i32 }** %points
  %colors69 = load { float*, i32 }*, { float*, i32 }** %colors
  %colorMode70 = load i32, i32* %colorMode3
  %filled71 = load i32, i32* %filled4
  call void @gl_drawShape({ float*, i32 }* %points68, { float*, i32 }* %colors69,
    i32 %colorMode70, i32 %filled71)
  ret void
}

define void @endCanvas({ i32, i32, i32 }* %c) {
entry:
  %c1 = alloca { i32, i32, i32 }*
  store { i32, i32, i32 }* %c, { i32, i32, i32 }** %c1
  %c2 = load { i32, i32, i32 }*, { i32, i32, i32 }** %c1
  %fieldadr = getelementptr { i32, i32, i32 }, { i32, i32, i32 }* %c2, i32 0, i32 0
  %width = load i32, i32* %fieldadr
  %c3 = load { i32, i32, i32 }*, { i32, i32, i32 }** %c1
  %fieldadr4 = getelementptr { i32, i32, i32 }, { i32, i32, i32 }* %c3, i32 0, i32 1
  %height = load i32, i32* %fieldadr4
  %c5 = load { i32, i32, i32 }*, { i32, i32, i32 }** %c1
  %fieldadr6 = getelementptr { i32, i32, i32 }, { i32, i32, i32 }* %c5, i32 0, i32 2
  %file_number = load i32, i32* %fieldadr6
  call void @gl_endRendering(i32 %width, i32 %height, i32 %file_number)
  ret void
}

define void @rotate({ float, float }* %p, float %angle, i32 %direction, { float,
    float }* %about) {
entry:
  %p1 = alloca { float, float }*
  store { float, float }* %p, { float, float }** %p1
  %angle2 = alloca float
  store float %angle, float* %angle2
  %direction3 = alloca i32
  store i32 %direction, i32* %direction3
  %about4 = alloca { float, float }*
  store { float, float }* %about, { float, float }** %about4
  %px = alloca float
  %p5 = load { float, float }*, { float, float }** %p1
  %fieldadr = getelementptr { float, float }, { float, float }* %p5, i32 0, i32 0
  %x = load float, float* %fieldadr
  %about6 = load { float, float }*, { float, float }** %about4
  %fieldadr7 = getelementptr { float, float }, { float, float }* %about6, i32 0, 
    i32 0
  %x8 = load float, float* %fieldadr7
  %tmp = fsub float %x, %x8
  store float %tmp, float* %px
  %py = alloca float
  %p9 = load { float, float }*, { float, float }** %p1
  %fieldadr10 = getelementptr { float, float }, { float, float }* %p9, i32 0, i32 1
  %y = load float, float* %fieldadr10
  %about11 = load { float, float }*, { float, float }** %about4
  %fieldadr12 = getelementptr { float, float }, { float, float }* %about11, i32
    0, i32 1
  %y13 = load float, float* %fieldadr12
  %tmp14 = fsub float %y, %y13
  store float %tmp14, float* %py
  %direction15 = load i32, i32* %direction3
  %tmp16 = icmp eq i32 %direction15, -1
  %if_tmp = alloca float
  br i1 %tmp16, label %then, label %else

merge:                                            ; preds = %merge48, %then
  %if_tmp93 = load float, float* %if_tmp
  ret void

then:                                             ; preds = %entry
  %p17 = load { float, float }*, { float, float }** %p1
  %px18 = load float, float* %px
  %angle19 = load float, float* %angle2
  %fxn_result = call float @cos(float %angle19)
  %tmp20 = fmul float %px18, %fxn_result
  %py21 = load float, float* %py
  %angle22 = load float, float* %angle2
  %fxn_result23 = call float @sin(float %angle22)
  %tmp24 = fmul float %py21, %fxn_result23
  %tmp25 = fsub float %tmp20, %tmp24
  %about26 = load { float, float }*, { float, float }** %about4
  %fieldadr27 = getelementptr { float, float }, { float, float }* %about26, i32 0,
    i32 0
  %x28 = load float, float* %fieldadr27
  %tmp29 = fadd float %tmp25, %x28
  %ref = getelementptr { float, float }, { float, float }* %p17, i32 0, i32 0
  store float %tmp29, float* %ref
  %p30 = load { float, float }*, { float, float }** %p1
  %px31 = load float, float* %px
  %angle32 = load float, float* %angle2
  %fxn_result33 = call float @sin(float %angle32)
  %tmp34 = fmul float %px31, %fxn_result33
  %py35 = load float, float* %py
  %angle36 = load float, float* %angle2
  %fxn_result37 = call float @cos(float %angle36)
  %tmp38 = fmul float %py35, %fxn_result37
  %tmp39 = fadd float %tmp34, %tmp38
  %about40 = load { float, float }*, { float, float }** %about4
  %fieldadr41 = getelementptr { float, float }, { float, float }* %about40, i32 0,
    i32 1
  %y42 = load float, float* %fieldadr41
  %tmp43 = fadd float %tmp39, %y42
  %ref44 = getelementptr { float, float }, { float, float }* %p30, i32 0, i32 1
  store float %tmp43, float* %ref44
  store float %tmp43, float* %if_tmp
  br label %merge

else:                                             ; preds = %entry
  %direction45 = load i32, i32* %direction3
  %tmp46 = icmp eq i32 %direction45, 1
  %if_tmp47 = alloca float
  br i1 %tmp46, label %then49, label %else50

merge48:                                          ; preds = %else50, %then49
  %if_tmp92 = load float, float* %if_tmp47
  store float %if_tmp92, float* %if_tmp
  br label %merge

then49:                                           ; preds = %else
  %p51 = load { float, float }*, { float, float }** %p1
  %px52 = load float, float* %px
  %angle53 = load float, float* %angle2
  %fxn_result54 = call float @cos(float %angle53)
  %tmp55 = fmul float %px52, %fxn_result54
  %py56 = load float, float* %py
  %angle57 = load float, float* %angle2
  %fxn_result58 = call float @sin(float %angle57)
  %tmp59 = fmul float %py56, %fxn_result58
  %tmp60 = fadd float %tmp55, %tmp59
  %about61 = load { float, float }*, { float, float }** %about4
  %fieldadr62 = getelementptr { float, float }, { float, float }* %about61, i32 0,
    i32 0
  %x63 = load float, float* %fieldadr62
  %tmp64 = fadd float %tmp60, %x63
  %ref65 = getelementptr { float, float }, { float, float }* %p51, i32 0, i32 0
  store float %tmp64, float* %ref65
  %p66 = load { float, float }*, { float, float }** %p1
  %px67 = load float, float* %px
  %angle68 = load float, float* %angle2
  %fxn_result69 = call float @sin(float %angle68)
  %tmp70 = fmul float %px67, %fxn_result69
  %tmp71 = fneg float %tmp70
  %py72 = load float, float* %py
  %angle73 = load float, float* %angle2
  %fxn_result74 = call float @cos(float %angle73)
  %tmp75 = fmul float %py72, %fxn_result74
  %tmp76 = fadd float %tmp71, %tmp75
  %about77 = load { float, float }*, { float, float }** %about4
  %fieldadr78 = getelementptr { float, float }, { float, float }* %about77, i32 0,
    i32 1
  %y79 = load float, float* %fieldadr78
  %tmp80 = fadd float %tmp76, %y79
  %ref81 = getelementptr { float, float }, { float, float }* %p66, i32 0, i32 1
  store float %tmp80, float* %ref81
  store float %tmp80, float* %if_tmp47
  br label %merge48

else50:                                           ; preds = %else
  %p82 = load { float, float }*, { float, float }** %p1
  %p83 = load { float, float }*, { float, float }** %p1
  %fieldadr84 = getelementptr { float, float }, { float, float }* %p83, i32 0, i32 0
  %x85 = load float, float* %fieldadr84
  %ref86 = getelementptr { float, float }, { float, float }* %p82, i32 0, i32 0
  store float %x85, float* %ref86
  %p87 = load { float, float }*, { float, float }** %p1
  %p88 = load { float, float }*, { float, float }** %p1
  %fieldadr89 = getelementptr { float, float }, { float, float }* %p88, i32 0, i32 1
  %y90 = load float, float* %fieldadr89
  %ref91 = getelementptr { float, float }, { float, float }* %p87, i32 0, i32 1
  store float %y90, float* %ref91
  store float %y90, float* %if_tmp47
  br label %merge48
}

define void @trans({ float, float }* %p, { float, float }* %direction) {
entry:
  %p1 = alloca { float, float }*
  store { float, float }* %p, { float, float }** %p1
  %direction2 = alloca { float, float }*
  store { float, float }* %direction, { float, float }** %direction2
  %p3 = load { float, float }*, { float, float }** %p1
  %p4 = load { float, float }*, { float, float }** %p1
  %fieldadr = getelementptr { float, float }, { float, float }* %p4, i32 0, i32 0
  %x = load float, float* %fieldadr
  %direction5 = load { float, float }*, { float, float }** %direction2
  %fieldadr6 = getelementptr { float, float }, { float, float }* %direction5, i32 0,
    i32 0
  %x7 = load float, float* %fieldadr6
  %tmp = fadd float %x, %x7
  %ref = getelementptr { float, float }, { float, float }* %p3, i32 0, i32 0
  store float %tmp, float* %ref
  %p8 = load { float, float }*, { float, float }** %p1
  %p9 = load { float, float }*, { float, float }** %p1
  %fieldadr10 = getelementptr { float, float }, { float, float }* %p9, i32 0, i32 1
  %y = load float, float* %fieldadr10
  %direction11 = load { float, float }*, { float, float }** %direction2
  %fieldadr12 = getelementptr { float, float }, { float, float }* %direction11, i32 0,
    i32 1
  %y13 = load float, float* %fieldadr12
  %tmp14 = fadd float %y, %y13
  %ref15 = getelementptr { float, float }, { float, float }* %p8, i32 0, i32 1
  store float %tmp14, float* %ref15
  ret void
}

define void @scale({ float, float }* %p, float %sx, float %sy) {
entry:
  %p1 = alloca { float, float }*
  store { float, float }* %p, { float, float }** %p1
  %sx2 = alloca float
  store float %sx, float* %sx2
  %sy3 = alloca float
  store float %sy, float* %sy3
  %p4 = load { float, float }*, { float, float }** %p1
  %p5 = load { float, float }*, { float, float }** %p1
  %fieldadr = getelementptr { float, float }, { float, float }* %p5, i32 0, i32 0
  %x = load float, float* %fieldadr
  %sx6 = load float, float* %sx2
  %tmp = fmul float %x, %sx6
  %ref = getelementptr { float, float }, { float, float }* %p4, i32 0, i32 0
  store float %tmp, float* %ref
  %p7 = load { float, float }*, { float, float }** %p1
  %p8 = load { float, float }*, { float, float }** %p1
  %fieldadr9 = getelementptr { float, float }, { float, float }* %p8, i32 0, i32 1
  %y = load float, float* %fieldadr9
  %sy10 = load float, float* %sy3
  %tmp11 = fmul float %y, %sy10
  %ref12 = getelementptr { float, float }, { float, float }* %p7, i32 0, i32 1
  store float %tmp11, float* %ref12
  ret void
}

define { float, float }* @rotated({ float, float }* %p, float %angle, i32 %direction,
    { float, float }* %about) {
entry:
  %p1 = alloca { float, float }*
  store { float, float }* %p, { float, float }** %p1
  %angle2 = alloca float
  store float %angle, float* %angle2
  %direction3 = alloca i32
  store i32 %direction, i32* %direction3
  %about4 = alloca { float, float }*
  store { float, float }* %about, { float, float }** %about4
  %q = alloca { float, float }*
  %p5 = load { float, float }*, { float, float }** %p1
  %copied = call { float, float }* @__copy2.2({ float, float }* %p5)
  store { float, float }* %copied, { float, float }** %q
  %q6 = load { float, float }*, { float, float }** %q
  %angle7 = load float, float* %angle2
  %direction8 = load i32, i32* %direction3
  %about9 = load { float, float }*, { float, float }** %about4
  call void @rotate({ float, float }* %q6, float %angle7, i32 %direction8, { float,
    float }* %about9)
  %q10 = load { float, float }*, { float, float }** %q
  ret { float, float }* %q10
}

define { float, float }* @__copy2.2({ float, float }* %to_copy) {
entry:
  %to_copy1 = alloca { float, float }*
  store { float, float }* %to_copy, { float, float }** %to_copy1
  %to_copy2 = load { float, float }*, { float, float }** %to_copy1
  %malloccall = tail call i8* @malloc(i32 trunc (i64 mul nuw (i64 ptrtoint (float* 
    getelementptr(float, float* null, i32 1) to i64), i64 2) to i32))
  %struct = bitcast i8* %malloccall to { float, float }*
  %flref = getelementptr { float, float }, { float, float }* %to_copy2, i32 0, i32
    0
  %fl = load float, float* %flref
  %ref = getelementptr { float, float }, { float, float }* %struct, i32 0, i32 0
  store float %fl, float* %ref
  %flref3 = getelementptr { float, float }, { float, float }* %to_copy2, i32 0,
    i32 1
  %fl4 = load float, float* %flref3
  %ref5 = getelementptr { float, float }, { float, float }* %struct, i32 0, 
    i32 1
  store float %fl4, float* %ref5
  ret { float, float }* %struct
}

define { float, float }* @translated({ float, float }* %p, { float, float }* 
    %direction) {
entry:
  %p1 = alloca { float, float }*
  store { float, float }* %p, { float, float }** %p1
  %direction2 = alloca { float, float }*
  store { float, float }* %direction, { float, float }** %direction2
  %q = alloca { float, float }*
  %p3 = load { float, float }*, { float, float }** %p1
  %copied = call { float, float }* @__copy2.3({ float, float }* %p3)
  store { float, float }* %copied, { float, float }** %q
  %q4 = load { float, float }*, { float, float }** %q
  %direction5 = load { float, float }*, { float, float }** %direction2
  call void @trans({ float, float }* %q4, { float, float }* %direction5)
  %q6 = load { float, float }*, { float, float }** %q
  ret { float, float }* %q6
}

define { float, float }* @__copy2.3({ float, float }* %to_copy) {
entry:
  %to_copy1 = alloca { float, float }*
  store { float, float }* %to_copy, { float, float }** %to_copy1
  %to_copy2 = load { float, float }*, { float, float }** %to_copy1
  %malloccall = tail call i8* @malloc(i32 trunc (i64 mul nuw (i64 ptrtoint 
    (float* getelementptr(float, float* null, i32 1) to i64), i64 2) to i32))
  %struct = bitcast i8* %malloccall to { float, float }*
  %flref = getelementptr { float, float }, { float, float }* %to_copy2, 
    i32 0, i32 0
  %fl = load float, float* %flref
  %ref = getelementptr { float, float }, { float, float }* %struct, i32 0,
    i32 0
  store float %fl, float* %ref
  %flref3 = getelementptr { float, float }, { float, float }* %to_copy2, 
    i32 0, i32 1
  %fl4 = load float, float* %flref3
  %ref5 = getelementptr { float, float }, { float, float }* %struct, i32 0,
    i32 1
  store float %fl4, float* %ref5
  ret { float, float }* %struct
}

define { float, float }* @scaled({ float, float }* %p, float %sx, float %sy) {
entry:
  %p1 = alloca { float, float }*
  store { float, float }* %p, { float, float }** %p1
  %sx2 = alloca float
  store float %sx, float* %sx2
  %sy3 = alloca float
  store float %sy, float* %sy3
  %q = alloca { float, float }*
  %p4 = load { float, float }*, { float, float }** %p1
  %copied = call { float, float }* @__copy2.4({ float, float }* %p4)
  store { float, float }* %copied, { float, float }** %q
  %p5 = load { float, float }*, { float, float }** %p1
  %sx6 = load float, float* %sx2
  %sy7 = load float, float* %sy3
  call void @scale({ float, float }* %p5, float %sx6, float %sy7)
  %q8 = load { float, float }*, { float, float }** %q
  ret { float, float }* %q8
}

define { float, float }* @__copy2.4({ float, float }* %to_copy) {
entry:
  %to_copy1 = alloca { float, float }*
  store { float, float }* %to_copy, { float, float }** %to_copy1
  %to_copy2 = load { float, float }*, { float, float }** %to_copy1
  %malloccall = tail call i8* @malloc(i32 trunc (i64 mul nuw (i64 ptrtoint 
    (float* getelementptr(float, float* null, i32 1) to i64), i64 2) to i32))
  %struct = bitcast i8* %malloccall to { float, float }*
  %flref = getelementptr { float, float }, { float, float }* %to_copy2, 
    i32 0, i32 0
  %fl = load float, float* %flref
  %ref = getelementptr { float, float }, { float, float }* %struct, i32 0, 
    i32 0
  store float %fl, float* %ref
  %flref3 = getelementptr { float, float }, { float, float }* %to_copy2, 
    i32 0, i32 1
  %fl4 = load float, float* %flref3
  %ref5 = getelementptr { float, float }, { float, float }* %struct, i32 0,
    i32 1
  store float %fl4, float* %ref5
  ret { float, float }* %struct
}

define void @fill_ints({ i32*, i32 }* %a, i32 %i) {
entry:
  %a1 = alloca { i32*, i32 }*
  store { i32*, i32 }* %a, { i32*, i32 }** %a1
  %i2 = alloca i32
  store i32 %i, i32* %i2
  %i3 = load i32, i32* %i2
  %a4 = load { i32*, i32 }*, { i32*, i32 }** %a1
  %lenref = getelementptr { i32*, i32 }, { i32*, i32 }* %a4, i32 0, i32 1
  %len = load i32, i32* %lenref
  %tmp = icmp slt i32 %i3, %len
  br i1 %tmp, label %then, label %else

merge:                                            ; preds = %else, %then
  ret void

then:                                             ; preds = %entry
  %a5 = load { i32*, i32 }*, { i32*, i32 }** %a1
  %datarefref = getelementptr { i32*, i32 }, { i32*, i32 }* %a5, i32 0, i32 0
  %dataref = load i32*, i32** %datarefref
  %i6 = load i32, i32* %i2
  %i7 = load i32, i32* %i2
  %storeref = getelementptr i32, i32* %dataref, i32 %i6
  store i32 %i7, i32* %storeref
  %a8 = load { i32*, i32 }*, { i32*, i32 }** %a1
  %i9 = load i32, i32* %i2
  %tmp10 = add i32 %i9, 1
  call void @fill_ints({ i32*, i32 }* %a8, i32 %tmp10)
  br label %merge

else:                                             ; preds = %entry
  br label %merge
}

define { i32*, i32 }* @ints(i32 %n) {
entry:
  %n1 = alloca i32
  store i32 %n, i32* %n1
  %n2 = load i32, i32* %n1
  %tmp = icmp sle i32 %n2, 0
  %if_tmp = alloca { i32*, i32 }*
  br i1 %tmp, label %then, label %else

merge:                                            ; preds = %continue, %then
  %if_tmp30 = load { i32*, i32 }*, { i32*, i32 }** %if_tmp
  ret { i32*, i32 }* %if_tmp30

then:                                             ; preds = %entry
  %a = alloca { i32*, i32 }*
  %malloccall = tail call i8* @malloc(i32 0)
  %arrdata = bitcast i8* %malloccall to i32*
  %malloccall3 = tail call i8* @malloc(i32 ptrtoint ({ i32*, i32 }* getelementptr 
    ({ i32*, i32 }, { i32*, i32 }* null, i32 1) to i32))
  %arr = bitcast i8* %malloccall3 to { i32*, i32 }*
  %arrdata4 = getelementptr { i32*, i32 }, { i32*, i32 }* %arr, i32 0, i32 0
  %arrlen = getelementptr { i32*, i32 }, { i32*, i32 }* %arr, i32 0, i32 1
  store i32* %arrdata, i32** %arrdata4
  store i32 0, i32* %arrlen
  store { i32*, i32 }* %arr, { i32*, i32 }** %a
  store { i32*, i32 }* %arr, { i32*, i32 }** %if_tmp
  br label %merge

else:                                             ; preds = %entry
  %arr5 = alloca { i32*, i32 }*
  %n6 = load i32, i32* %n1
  %malloccall7 = tail call i8* @malloc(i32 ptrtoint (i32* getelementptr (i32, i32* 
    null, i32 1) to i32))
  %arrdata8 = bitcast i8* %malloccall7 to i32*
  %storeref = getelementptr i32, i32* %arrdata8, i32 0
  store i32 0, i32* %storeref
  %malloccall9 = tail call i8* @malloc(i32 ptrtoint ({ i32*, i32 }* getelementptr ({ 
    i32*, i32 }, { i32*, i32 }* null, i32 1) to i32))
  %arr10 = bitcast i8* %malloccall9 to { i32*, i32 }*
  %arrdata11 = getelementptr { i32*, i32 }, { i32*, i32 }* %arr10, i32 0, i32 0
  %arrlen12 = getelementptr { i32*, i32 }, { i32*, i32 }* %arr10, i32 0, i32 1
  store i32* %arrdata8, i32** %arrdata11
  store i32 1, i32* %arrlen12
  %lenref = getelementptr { i32*, i32 }, { i32*, i32 }* %arr10, i32 0, i32 1
  %len = load i32, i32* %lenref
  %oflen = mul i32 %n6, %len
  %olddataref = getelementptr { i32*, i32 }, { i32*, i32 }* %arr10, i32 0, i32 0
  %olddata = load i32*, i32** %olddataref
  %mallocsize = mul i32 %oflen, ptrtoint (i32* getelementptr (i32, i32* null, i32 1) 
    to i32)
  %malloccall13 = tail call i8* @malloc(i32 %mallocsize)
  %arrdata14 = bitcast i8* %malloccall13 to i32*
  %i = alloca i32
  store i32 0, i32* %i
  %j = alloca i32
  store i32 0, i32* %j
  br label %inner

loop:                                             ; preds = %inner
  %i22 = load i32, i32* %i
  store i32 0, i32* %j
  %tmp23 = icmp slt i32 %i22, %oflen
  br i1 %tmp23, label %inner, label %continue

inner:                                            ; preds = %loop, %inner, %else
  %i15 = load i32, i32* %j
  %i16 = load i32, i32* %i
  %elref = getelementptr i32, i32* %olddata, i32 %i15
  %el = load i32, i32* %elref
  %storeref17 = getelementptr i32, i32* %arrdata14, i32 %i16
  store i32 %el, i32* %storeref17
  %i18 = add i32 %i16, 1
  store i32 %i18, i32* %i
  %j19 = add i32 %i15, 1
  store i32 %j19, i32* %j
  %j20 = load i32, i32* %j
  %tmp21 = icmp slt i32 %j20, %len
  br i1 %tmp21, label %inner, label %loop

continue:                                         ; preds = %loop
  %malloccall24 = tail call i8* @malloc(i32 ptrtoint ({ i32*, i32 }* getelementptr 
    ({ i32*, i32 }, { i32*, i32 }* null, i32 1) to i32))
  %arr25 = bitcast i8* %malloccall24 to { i32*, i32 }*
  %arrdata26 = getelementptr { i32*, i32 }, { i32*, i32 }* %arr25, i32 0, i32 0
  %arrlen27 = getelementptr { i32*, i32 }, { i32*, i32 }* %arr25, i32 0, i32 1
  store i32* %arrdata14, i32** %arrdata26
  store i32 %oflen, i32* %arrlen27
  store { i32*, i32 }* %arr25, { i32*, i32 }** %arr5
  %arr28 = load { i32*, i32 }*, { i32*, i32 }** %arr5
  call void @fill_ints({ i32*, i32 }* %arr28, i32 0)
  %arr29 = load { i32*, i32 }*, { i32*, i32 }** %arr5
  store { i32*, i32 }* %arr29, { i32*, i32 }** %if_tmp
  br label %merge
}

define { { float, float }**, i32 }* @dragon(i32 %n) {
entry:
  %n1 = alloca i32
  store i32 %n, i32* %n1
  %n2 = load i32, i32* %n1
  %tmp = icmp eq i32 %n2, 0
  %if_tmp = alloca { { float, float }**, i32 }*
  br i1 %tmp, label %then, label %else

merge:                                            ; preds = %continue89, %then
  %if_tmp103 = load { { float, float }**, i32 }*, { { float, float }**, i32 }**
    %if_tmp
  ret { { float, float }**, i32 }* %if_tmp103

then:                                             ; preds = %entry
  %malloccall = tail call i8* @malloc(i32 mul (i32 ptrtoint (i1** getelementptr 
    (i1*, i1** null, i32 1) to i32), i32 2))
  %arrdata = bitcast i8* %malloccall to { float, float }**
  %malloccall3 = tail call i8* @malloc(i32 trunc (i64 mul nuw (i64 ptrtoint (float*
    getelementptr (float, float* null, i32 1) to i64), i64 2) to i32))
  %point = bitcast i8* %malloccall3 to { float, float }*
  %fieldaddr = getelementptr { float, float }, { float, float }* %point, 
    i32 0, i32 0
  store float 0.000000e+00, float* %fieldaddr
  %fieldaddr4 = getelementptr { float, float }, { float, float }* %point, 
    i32 0, i32 1
  store float 0.000000e+00, float* %fieldaddr4
  %storeref = getelementptr { float, float }*, { float, float }** %arrdata,
    i32 0
  store { float, float }* %point, { float, float }** %storeref
  %malloccall5 = tail call i8* @malloc(i32 trunc (i64 mul nuw (i64 ptrtoint
    (float*getelementptr (float, float* null, i32 1) to i64), i64 2) to i32))
  %point6 = bitcast i8* %malloccall5 to { float, float }*
  %fieldaddr7 = getelementptr { float, float }, { float, float }* %point6,
    i32 0, i32 0
  store float 1.000000e+00, float* %fieldaddr7
  %fieldaddr8 = getelementptr { float, float }, { float, float }* %point6,
    i32 0, i32 1
  store float 0.000000e+00, float* %fieldaddr8
  %storeref9 = getelementptr { float, float }*, { float, float }** %arrdata,
    i32 1
  store { float, float }* %point6, { float, float }** %storeref9
  %malloccall10 = tail call i8* @malloc(i32 ptrtoint ({ { float, float }**,
    i32 }* getelementptr ({ { float, float }**, i32 }, { { float, float }**,
    i32 }* null, i32 1) to i32))
  %arr = bitcast i8* %malloccall10 to { { float, float }**, i32 }*
  %arrdata11 = getelementptr { { float, float }**, i32 }, { { float, float
    }**, 
    i32 }* %arr, i32 0, i32 0
  %arrlen = getelementptr { { float, float }**, i32 }, { { float, float }**,
    i32 }* %arr, i32 0, i32 1
  store { float, float }** %arrdata, { float, float }*** %arrdata11
  store i32 2, i32* %arrlen
  store { { float, float }**, i32 }* %arr, { { float, float }**, i32 }** 
    %if_tmp
  br label %merge

else:                                             ; preds = %entry
  %d1 = alloca { { float, float }**, i32 }*
  %n12 = load i32, i32* %n1
  %tmp13 = sub i32 %n12, 1
  %fxn_result = call { { float, float }**, i32 }* @dragon(i32 %tmp13)
  store { { float, float }**, i32 }* %fxn_result, { { float, float }**, i32 
    }** %d1
  %d2 = alloca { { float, float }**, i32 }*
  %d114 = load { { float, float }**, i32 }*, { { float, float }**, i32 }** %d1
  %fxn_result15 = call { { float, float }**, i32 }* @copy_path({ { float, float
    }**, i32 }* %d114)
  store { { float, float }**, i32 }* %fxn_result15, { { float, float }**, i32 
    }** %d2
  %s = alloca float
  %fxn_result16 = call float @sqrt(float 2.000000e+00)
  %tmp17 = fdiv float %fxn_result16, 2.000000e+00
  store float %tmp17, float* %s
  %d118 = load { { float, float }**, i32 }*, { { float, float }**, i32 }** %d1
  %fxn_result19 = call float @toradians(float 4.500000e+01)
  %malloccall20 = tail call i8* @malloc(i32 trunc (i64 mul nuw (i64 ptrtoint 
    (float* getelementptr (float, float* null, i32 1) to i64), i64 2) to i32))
  %anon = bitcast i8* %malloccall20 to { float, float }*
  %fieldaddr21 = getelementptr { float, float }, { float, float }* %anon, 
    i32 0, i32 0
  store float 0.000000e+00, float* %fieldaddr21
  %fieldaddr22 = getelementptr { float, float }, { float, float }* %anon, 
    i32 0, i32 1
  store float 0.000000e+00, float* %fieldaddr22
  %lenref = getelementptr { { float, float }**, i32 }, { { float, float }**,
    i32 }* %d118, i32 0, i32 1
  %len = load i32, i32* %lenref
  %dataref = getelementptr { { float, float }**, i32 }, { { float, float }**,
    i32 }* %d118, i32 0, i32 0
  %data = load { float, float }**, { float, float }*** %dataref
  %i = alloca i32
  store i32 0, i32* %i
  br label %loop

loop:                                             ; preds = %loop, %else
  %i23 = load i32, i32* %i
  %elref = getelementptr { float, float }*, { float, float }** %data, i32 %i23
  %el = load { float, float }*, { float, float }** %elref
  call void @rotate({ float, float }* %el, float %fxn_result19, i32 -1, { 
    float, float }* %anon)
  %i24 = add i32 %i23, 1
  store i32 %i24, i32* %i
  %i25 = load i32, i32* %i
  %tmp26 = icmp slt i32 %i25, %len
  br i1 %tmp26, label %loop, label %continue

continue:                                         ; preds = %loop
  %d127 = load { { float, float }**, i32 }*, { { float, float }**, 
    i32 }** %d1
  %s28 = load float, float* %s
  %s29 = load float, float* %s
  %lenref30 = getelementptr { { float, float }**, i32 }, { { float, float 
    }**, i32 }* %d127, i32 0, i32 1
  %len31 = load i32, i32* %lenref30
  %dataref32 = getelementptr { { float, float }**, i32 }, { { float, float 
    }**, i32 }* %d127, i32 0, i32 0
  %data33 = load { float, float }**, { float, float }*** %dataref32
  %i34 = alloca i32
  store i32 0, i32* %i34
  br label %loop35

loop35:                                           ; preds = %loop35, %continue
  %i37 = load i32, i32* %i34
  %elref38 = getelementptr { float, float }*, { float, float }** %data33, 
    i32 %i37
  %el39 = load { float, float }*, { float, float }** %elref38
  call void @scale({ float, float }* %el39, float %s28, float %s29)
  %i40 = add i32 %i37, 1
  store i32 %i40, i32* %i34
  %i41 = load i32, i32* %i34
  %tmp42 = icmp slt i32 %i41, %len31
  br i1 %tmp42, label %loop35, label %continue36

continue36:                                       ; preds = %loop35
  %d243 = load { { float, float }**, i32 }*, { { float, float }**, i32 }** %d2
  %fxn_result44 = call float @toradians(float 1.350000e+02)
  %malloccall45 = tail call i8* @malloc(i32 trunc (i64 mul nuw (i64 ptrtoint
    (float* getelementptr (float, float* null, i32 1) to i64), i64 2) to i32))
  %anon46 = bitcast i8* %malloccall45 to { float, float }*
  %fieldaddr47 = getelementptr { float, float }, { float, float }* %anon46,
    i32 0, i32 0
  store float 0.000000e+00, float* %fieldaddr47
  %fieldaddr48 = getelementptr { float, float }, { float, float }* %anon46,
    i32 0, i32 1
  store float 0.000000e+00, float* %fieldaddr48
  %lenref49 = getelementptr { { float, float }**, i32 }, { { float, float }**,
    i32 }* %d243, i32 0, i32 1
  %len50 = load i32, i32* %lenref49
  %dataref51 = getelementptr { { float, float }**, i32 }, { { float, float }**,
    i32 }* %d243, i32 0, i32 0
  %data52 = load { float, float }**, { float, float }*** %dataref51
  %i53 = alloca i32
  store i32 0, i32* %i53
  br label %loop54

loop54:                                           ; preds = %loop54, %continue36
  %i56 = load i32, i32* %i53
  %elref57 = getelementptr { float, float }*, { float, float }** %data52,
    i32 %i56
  %el58 = load { float, float }*, { float, float }** %elref57
  call void @rotate({ float, float }* %el58, float %fxn_result44, i32 -1,
    { float, float }* %anon46)
  %i59 = add i32 %i56, 1
  store i32 %i59, i32* %i53
  %i60 = load i32, i32* %i53
  %tmp61 = icmp slt i32 %i60, %len50
  br i1 %tmp61, label %loop54, label %continue55

continue55:                                       ; preds = %loop54
  %d262 = load { { float, float }**, i32 }*, { { float, float }**, i32 
    }** %d2
  %s63 = load float, float* %s
  %s64 = load float, float* %s
  %lenref65 = getelementptr { { float, float }**, i32 }, { { float, float
    }**, i32 }* %d262, i32 0, i32 1
  %len66 = load i32, i32* %lenref65
  %dataref67 = getelementptr { { float, float }**, i32 }, { { float, float
    }**, i32 }* %d262, i32 0, i32 0
  %data68 = load { float, float }**, { float, float }*** %dataref67
  %i69 = alloca i32
  store i32 0, i32* %i69
  br label %loop70

loop70:                                           ; preds = %loop70, %continue55
  %i72 = load i32, i32* %i69
  %elref73 = getelementptr { float, float }*, { float, float }** %data68,
    i32 %i72
  %el74 = load { float, float }*, { float, float }** %elref73
  call void @scale({ float, float }* %el74, float %s63, float %s64)
  %i75 = add i32 %i72, 1
  store i32 %i75, i32* %i69
  %i76 = load i32, i32* %i69
  %tmp77 = icmp slt i32 %i76, %len66
  br i1 %tmp77, label %loop70, label %continue71

continue71:                                       ; preds = %loop70
  %d278 = load { { float, float }**, i32 }*, { { float, float }**, i32 }** %d2
  %malloccall79 = tail call i8* @malloc(i32 trunc (i64 mul nuw (i64 ptrtoint
    (float* getelementptr (float, float* null, i32 1) to i64), i64 2) to i32))
  %anon80 = bitcast i8* %malloccall79 to { float, float }*
  %fieldaddr81 = getelementptr { float, float }, { float, float }* %anon80,
    i32 0, i32 0
  store float 1.000000e+00, float* %fieldaddr81
  %fieldaddr82 = getelementptr { float, float }, { float, float }* %anon80,
    i32 0, i32 1
  store float 0.000000e+00, float* %fieldaddr82
  %lenref83 = getelementptr { { float, float }**, i32 }, { { float, float }**,
    i32 }* %d278, i32 0, i32 1
  %len84 = load i32, i32* %lenref83
  %dataref85 = getelementptr { { float, float }**, i32 }, { { float, float }**,
    i32 }* %d278, i32 0, i32 0
  %data86 = load { float, float }**, { float, float }*** %dataref85
  %i87 = alloca i32
  store i32 0, i32* %i87
  br label %loop88

loop88:                                           ; preds = %loop88, %continue71
  %i90 = load i32, i32* %i87
  %elref91 = getelementptr { float, float }*, { float, float }** %data86, i32 %i90
  %el92 = load { float, float }*, { float, float }** %elref91
  call void @trans({ float, float }* %el92, { float, float }* %anon80)
  %i93 = add i32 %i90, 1
  store i32 %i93, i32* %i87
  %i94 = load i32, i32* %i87
  %tmp95 = icmp slt i32 %i94, %len84
  br i1 %tmp95, label %loop88, label %continue89

continue89:                                       ; preds = %loop88
  %d296 = load { { float, float }**, i32 }*, { { float, float }**, i32 }** %d2
  call void @reverse({ { float, float }**, i32 }* %d296)
  %r = alloca { { float, float }**, i32 }*
  %d197 = load { { float, float }**, i32 }*, { { float, float }**, i32 }** %d1
  %d298 = load { { float, float }**, i32 }*, { { float, float }**, i32 }** %d2
  %fxn_result99 = call { { float, float }**, i32 }* @append({ { float, float }**,
    i32 }* %d197, { { float, float }**, i32 }* %d298, float 1.000000e+00)
  store { { float, float }**, i32 }* %fxn_result99, { { float, float }**, i32 }** %r
  %d1100 = load { { float, float }**, i32 }*, { { float, float }**, i32 }** %d1
  call void @free_path({ { float, float }**, i32 }* %d1100)
  %d2101 = load { { float, float }**, i32 }*, { { float, float }**, i32 }** %d2
  call void @free_path({ { float, float }**, i32 }* %d2101)
  %r102 = load { { float, float }**, i32 }*, { { float, float }**, i32 }** %r
  store { { float, float }**, i32 }* %r102, { { float, float }**, i32 }** %if_tmp
  br label %merge
}

define { float, float, float, float }* @rainbow(i32 %r, i32 %len) {
entry:
  %r1 = alloca i32
  store i32 %r, i32* %r1
  %len2 = alloca i32
  store i32 %len, i32* %len2
  %h = alloca float
  %r3 = load i32, i32* %r1
  %cast = sitofp i32 %r3 to float
  %tmp = fmul float 1.000000e+00, %cast
  %len4 = load i32, i32* %len2
  %cast5 = sitofp i32 %len4 to float
  %tmp6 = fdiv float %tmp, %cast5
  store float %tmp6, float* %h
  %h7 = load float, float* %h
  %fxn_result = call { float, float, float, float }* @hsv(float %h7, float 
    0x3FE99999A0000000, float 0x3FE99999A0000000)
  ret { float, float, float, float }* %fxn_result
}
\end{lstlisting}
}

\end{document}
