\documentclass[main.tex]{subfiles}
%\usepackage{parskip}
%\usepackage{listings}

\begin{document}
	\section{Test Plan}
	
	The end-to-end integration tests can be found in the test/ directory. An an end-to-end integration test suite, it was used to rigorously test the functionality of SOS, including lower level lexical, syntactic, and semantic checks.
	
	A complete set of tests can be found in Appendix 9.4 Integration Tests Files (Negative tests) and Appendix 9.4 Integration Tests Files (Positive tests).
	
	Tojo, Sheron, and Sitong created the tests/test-* and tests/fail-* test cases to test specific features of SOS. G, Sitong, and Tojo created sample programs as end-to-end tests.
	
	\subsection{Test Automation}
	
	While in the docker environment, running ./testall in the SOS directory will run all of the tests and say whether each test passed with “OK” or failed with “FAILED”. Each test case that should successfully pass and produce output follow the naming pattern of using test-*.sos for the test program and a corresponding (in file name represented by *) test-*.out for the expected output of the program. Similarly, each negative test case used fail-*.sos for the test program that should fail and fail-*.err (instead of .out) for the expected error message.

	For our integration tests, we modified the testall.sh script provided by Professor Edwards for the MicroC compiler. The script works by compiling and executing all of the test-*.sos and fail-*.sos programs. For each test case (test and fail), the script compares the produced outputs (.out) or errors (.err) with their corresponding references made by us, to see if what was produced  matches their expected values. If the produced output or error files fail to match what is expected, the script will print out 'FAILED' next to the test name along with an error message. A more detailed error message can be found by looking in testall.log. One can refer to either the produced .out/.err file and compare it to the reference, or the .diff file, which summarizes the differences between the two files. If the test is successful, the script prints an ‘OK’, and the compiled .out/.err files are removed from the directory for cleanliness.
	
	For larger end-to-end tests, (i.e. sample programs), we created a separate directory called "sample\_programs" to hold these programs. These tests can be run using the compile\_exec.sh script, which takes the file path to the program to run as an argument.
	
	Example: ./compile\_exec.sh sample\_programs/dragon.sos
	
	testall.sh and compile\_exec.sh can be found in Appendix 9.4 testall.sh and Appendix 9.4 compile\_exec.sh, respectively. 
	
    \subsection{Testing Suite}
    
    See Appendices 9.4 for the testing suite.
    
	\subsection{Sample Programs}

	\subsubsection{Square: test-helloworld.sos}
	\begin{lstlisting}
import renderer.sos

p1: point = {-0.5, -0.5}
p2: point = {-0.5, 0.5}
p3: point = {0.5, 0.5}
p4: point = {0.5, -0.5}
point_arr : path = [p1, p2, p3, p4] 

c1 : color = {255.0, 0.0, 0.0, 0.8}
c2 : color = {0.0, 255.0, 0.0, 0.8}
c3 : color = {0.0, 0.0, 255.0, 0.8}
c4 : color = {100.0, 100.0, 0.0, 0.8}
color_arr : colors = [c1, c2, c3, c4] 

//create first canvas, draw square, and save and end canvas
canvas1 : canvas = {400, 400, 0}

startCanvas(canvas1)
drawShape(point_arr, color_arr, 0, 1)
endCanvas(canvas1)

//create a second canvas, draw same shape, and close this canvas
//there should now be two images of the same drawing saved in root directory
canvas2 : canvas = {400, 400, 1}

startCanvas(canvas2)
drawShape(point_arr, color_arr, 0, 1)
endCanvas(canvas2)
\end{lstlisting}

See Appendix 9.5.1 for test-helloworld.ll

\subsubsection{Dragon: dragon.sos}

\begin{lstlisting}
import renderer.sos
import vector.sos
import transform.sos
import array.sos
import math.sos

// Creates a dragon curve of depth n
dragon: (n: int) -> path =
    if n == 0 // Base case
    then [point{0.0, 0.0}, point{1.0, 0.0}]
    else
    // Create two copies of the previous depths
    d1: path = dragon(n-1) ;
    d2: path = copy_path(d1) ;

    // Position d1
    s: float = sqrt(2.0)/2.0 ;
    rotate(d1, toradians(45.0), -1, {0.0, 0.0}) ;
    scale(d1, s, s) ;

    // Position d2
    rotate(d2, toradians(135.0), -1, {0., 0.}) ;
    scale(d2,s,s) ;
    trans(d2, {1., 0.}) ;
    reverse(d2) ;

    // Merge the paths
    r: path = append(d1, d2, 1.0) ;
    free_path(d1); free_path(d2); r

// Creates a rainbow color effect
rainbow: (r: int, len: int) -> color =
    h: float = (1.0*r)/len ;
    hsv(h, 0.8, 0.8)

// Render a 400px by 400px canvas, name the image pic0
my_canvas: canvas = {400, 400, 0}

// Start render
startCanvas(my_canvas)
d: path = dragon(7)
// Position the curve (0.4, 0.2 is approximately the center of mass of the 
//curve for large n)
trans(d, {-0.4, -0.2})

// Draw it
drawPath(d, rainbow(ints(d.length), d.length), 0)
endCanvas(my_canvas)
\end{lstlisting}

See Appendix 9.5.2 for dragon.ll

\subsubsection{Drunk walk: drunk.sos}
\begin{lstlisting}

import renderer.sos
import random.sos

n: int = 100 
p: path = n of [{0.0, 0.0}]
c: colors = n of [{0.5, 0.5, 0.5, 0.5}]
r: rng = {1,2,3}

drunk_walk : (i: int, p: path, c: colors, r: rng) -> void = 
    if i < p.length then
      theta: float = randf(r) * 6.28319;
      d: float = randf(r)*0.1 + 0.02;
      dx: float = cos(theta)*d ; dy: float = sin(theta)*d ;
      p[i] = {p[i-1].x+dx, p[i-1].y+dy} ;
      dc: float = 0.1 ; 
      dr: float = (randf(r) - 0.5) * dc ;
      dg: float = (randf(r) - 0.5) * dc ;
      db: float = (randf(r) - 0.5) * dc ;
      c[i] = rgb(c[i-1].r + dr, c[i-1].g+dg, c[i-1].b+db) ;
      drunk_walk(i+1,p,c,r)

    else void


my_canvas: canvas = {400, 400, 0}
startCanvas(my_canvas)

draw_walks : (count: int, p: path, c: colors, r: rng) -> void =
    if count > 0 then
    p[0] = {randf(r)*0.5-0.25, randf(r)*0.5-0.25};
    c[0] = hsv(randf(r), 0.8, 0.8) ;
    drunk_walk(1, p, c, r) ;
    drawPath(p,c,0) ;
    draw_walks(count-1,p,c,r)
    else void

draw_walks(20, p, c, r)

endCanvas(my_canvas)
\end{lstlisting}

See Appendix 9.5 for drunk.ll

\end{document}
